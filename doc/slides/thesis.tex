\documentclass{beamer}
\usetheme{Warsaw}
\usefonttheme[onlymath]{serif}
%\usetheme{CambridgeUS}
%\usecolortheme{beaver}

% Add frame number and total frame number in footline
\defbeamertemplate*{footline}{shadow theme}{%
    \leavevmode%
    \hbox{\begin{beamercolorbox}[wd=.5\paperwidth,ht=2.5ex,dp=1.125ex,leftskip=.3cm plus1fil,rightskip=.3cm]{author in head/foot}%
            \usebeamerfont{author in head/foot}\hfill\insertshortauthor
        \end{beamercolorbox}%
        \begin{beamercolorbox}[wd=.4\paperwidth,ht=2.5ex,dp=1.125ex,leftskip=.3cm,rightskip=.3cm plus1fil]{title in head/foot}%
            \usebeamerfont{title in head/foot}\insertshorttitle\hfill%
        \end{beamercolorbox}%
        \begin{beamercolorbox}[wd=.1\paperwidth,ht=2.5ex,dp=1.125ex,leftskip=.3cm,rightskip=.3cm plus1fil]{title in head/foot}%
            \hfill\insertframenumber\,/\,\inserttotalframenumber
    \end{beamercolorbox}}%
    \vskip0pt%
}

% Tikz related
\usepackage{tikz}
\usetikzlibrary{calc}
\usetikzlibrary{positioning}
\usetikzlibrary{shapes,arrows}

% Use pfgplots to generate plots and labeled axes
\usepackage{pgfplots}

% Number the figures
\setbeamertemplate{caption}[numbered]

% Add outline page at begining of each section
\AtBeginSection[]
{
    \begin{frame}<beamer>
        \frametitle{Outline}
        \tableofcontents[currentsection, hideallsubsections]
    \end{frame}
}

% Insert assembly code
\usepackage{listings}

% lslisting settings
\input{plugins/lslisting/settings.tex}

% Mathematics
\usepackage{amsmath}

% Appendix
\usepackage{appendixnumberbeamer}

% Table
\usepackage{array}

%%%%%%%%%%%%%%%%%%%%%%%%%%%%%%%%%%%%%%%%%%%%%
\title{Parallel \\ Variable Gapped Longest Common Sequence}
\author{Shiang-Yun Yang}
\institute{Department of Computer Science \& Information Engineering\\
National Taiwan University}
\date{Master's Thesis Defense\\\today}

\begin{document}
\begin{frame}
    \titlepage
\end{frame}

\begin{frame}
    \frametitle{Outline}
    \tableofcontents[hideallsubsections]
\end{frame}
\section{介紹} %Introduction
\label{sec:Introduction}

最長共同子序列 (\emph{longest common subsequence}, LCS)廣泛地使用在各個應用上。
在多核心平台下,大多數的研究專注於如何高效率地在波前平行 (wavefront parallelism),
而 Jiaoyun Yang ~\cite{jiaoyun} 提出的論文中改變一般的 LCS 遞迴定義以得到更好快取使用率。
在這篇論文中,針對在 Iliopoulos ~\cite{iliopoulos} 提及的約束條件下的 LCS 問題使用相關的想法來改善效能。

在約束條件下的 LCS 中,如 \emph{fixed gap LCS } (FGLCS)要求任兩個挑選的距離在相對應的另一個字串中相等,
同時距離最大為 $k+1$,可在時間複雜度在 $O(nm)$ 內解決,其中 $n$, $m$ 分別為兩個輸入的字串長度。
我們將在這篇論文針對 \emph{variable gap LCS} (VGLCS) 進行探討。
在 VGLCS 中,對各個不同的位置提供約束限制,如目前給定兩個字串 $A = \tt{GCGCAATG}$, 
$B = \tt{GCCCTAGCG}$,各自的約束限制為 $G_A = [3, 1, 1, 2, 0, 0, 2, 1]$ 
和 $G_B = [2, 0, 3, 2, 0, 1, 2, 0, 1]$,其中 $G_A(i)$ 表示當挑選第 $i$ 個位置時,與前一個挑選的位置最多差 $G_A(i)+1$,同理 $G_B(i)$;
我們可以得到兩組 VGLCS 的解 $\tt{GCCA}$ 和 $\tt{GCCT}$,挑選的方式如圖 ~\ref{fig:VGLCSex}。
在 Yung-Hsing Peng ~\cite{yunghsing} 的論文已對 VGLCS 提出易於實作的 $O(nm \alpha(n))$ 和理論 $O(nm)$ 的解法。

這一篇論文,我們將在第二 \ref{sec:parallelSerial} 節部分將 Yung-Hsing Peng ~\cite{yunghsing} 提出的算法進行平行化。
在第三節 ~\ref{sec:parallelRMQ},在理論分析上提供易平行且時間複雜度 $O(nm)$ 的設計。
在第四節 ~\ref{sec:Implementation},我們將藉由快取忘卻 (cache-oblivious) 技術,在實作上提供更好的效能。最後,我們總結實驗結果與理論實務上的差異。

\begin{figure}[!thb]
  \centering
  \includegraphics[width=\linewidth]{graphics/fig-VGLCSex.pdf}
  \includegraphics[width=\linewidth]{graphics/fig-VGLCSex2.pdf}
  \caption{VGLCS 於兩個序列 $A = \tt{GCGCAATG}$, $B = \tt{GCCCTAGCG}$,各自的約束限制為 $G_A = [3, 1, 1, 2, 0, 0, 2, 1]$ 和 $G_B = [2, 0, 3, 2, 0, 1, 2, 0, 1]$,的其中幾個可挑選的方案}
  \label{fig:VGLCSex}
\end{figure}
\section{Related Works}

\begin{frame}
    \frametitle{Related Works (TODO)}
    \begin{itemize}
    	\item Hirschberg's Algorithm
    \end{itemize}
\end{frame}
\section{Background}

\begin{frame}
    \frametitle{Background}
    \begin{itemize}
    	\setlength\itemsep{1em}
    	\item VGLCS can be solved in $O(n^2)$ time and $O(n^2)$ space.
    	\item Wavefront parallelism need to exploit data reuse effectively between adjacent tiles to ensure that locality is also enhanced.
    \end{itemize}
    \begin{figure}
		\includegraphics[scale=0.20]{figure/fig-wavefront-dp.png}
	\end{figure}
\end{frame}
\section{Parallel Design}

\begin{frame}
    \frametitle{Parallel Design}
\end{frame}
\section{Experiment}

\subsection{Environment Settings}
\begin{frame}
    \frametitle{Environment}
\end{frame}

\subsection{Parallel VGLCS}
\begin{frame}
	\frametitle{Runtime}
	\begin{figure}[!ht]
		\centering
		\subfigure[Runtime]{
			\begin{tikzpicture}[scale=0.4,font=\sffamily]
	\begin{axis}[
			xlabel={Length $n$},
			ylabel={Time (second)},
			xmin=0, xmax=10000,
			ymin=0, ymax=4.5,
			scaled ticks = false,
			tick label style={/pgf/number format/fixed},
			xtick={0, 1000, 2000, 3000, 4000, 5000, 6000, 7000, 8000, 9000, 10000},
			ytick={0, 0.5, 1, 1.5, 2, 2.5, 3, 3.5, 4, 4.5},
			legend pos=north east,
			legend cell align=left,
			ymajorgrids=true,
			grid style=dashed,
			,height=10cm,width=15cm,
		]
		\addplot [mark=*] file{figure/serial-n.dat};
		\addplot [mark=square*, mark options={fill=white}] file{figure/parallel-n.dat};
		\legend{serial, parallel}
	\end{axis}
\end{tikzpicture}

			\label{fig:fig-parallel}
		}
		%\subfigure[Little core cluster]{
		%	\input{./data/light_little}
		%	\label{fig:light_little}
		%}
		\caption{Serial Algorithm and Parallel Algorithm}
		\label{fig:light_weight}
	\end{figure}
\end{frame}

\begin{frame}
	\frametitle{Scalability}
	\begin{figure}[!ht]
		\centering
		\subfigure[Scalability]{
			\begin{tikzpicture}[scale=0.4,font=\sffamily]
	\begin{axis}[
			xlabel={Processor $p$},
			ylabel={Time (second)},
			xmin=1, xmax=16,
			ymin=0, ymax=1.2,
			scaled ticks = false,
			tick label style={/pgf/number format/fixed},
			xtick={1, 2, 4, 8, 16},
			ytick={0, 0.2, 0.4, 0.6, 0.8, 1, 1.2},
			legend pos=north east,
			legend cell align=left,
			ymajorgrids=true,
			grid style=dashed,
			,height=10cm,width=15cm,
		]
		\addplot [mark=square*, mark options={fill=white}] file{figure/parallel-p.dat};
		\legend{parallel $n=5000$}
	\end{axis}
\end{tikzpicture}

			\label{fig:fig-parallel}
		}
		%\subfigure[Little core cluster]{
		%	\input{./data/light_little}
		%	\label{fig:light_little}
		%}
		\caption{Parallel Algorithm}
		\label{fig:light_weight}
	\end{figure}
\end{frame}

\subsection{Parallel RMQ and Cache-Oblivious Cartesian Tree}
\begin{frame}
	\frametitle{Speedup}
	We have $N$ elements of an array and $N$ querys.
	\begin{itemize}
		\setlength\itemsep{1em}
		\item When $N = 30000$ and maximum length of interval $L \le 100$, \texttt{CORQM} can speedup $1.25\times$ 
		because of less cache-misses.
	\end{itemize}
\end{frame}
\section{Conclusion}

\subsection{Conclusion}
\begin{frame}
  \frametitle{Conclusion}
  \begin{itemize}
    \item The rightmost-pops encoding is easy to implement and runs in
      $O(n^2 s/p+n\;\max(\log n, s))$ with $p$ processors,
    \item Consequently we can solve VGLCS with the rightmost-pops
      encoding for blocked sparse table in $O(n^2/p+n\log n)$ time
      with $p$ processors.
    \item Our dynamic tree computation technique can answer
      incremental ranged maximum query in amortized $O(1)$ time.
    \item Asymtotically better solution is {\em not} necessarily the
      better solution in practice.
  \end{itemize}
\end{frame}

%\appendix
\section{Progress}

\subsection{Prerequisites}
\begin{frame}
    \frametitle{All Pair Least Common Ancestors}
    \begin{itemize}
    	\setlength\itemsep{1em}
    	\item Generating all binary search tree(BST) from $1$ to $n$.
    	\item Store $\textit{LCA}(p, q)$ with node $p$ and $q$ for each BST $T$.
		\item The size of table is $\theta(n^2 \frac{1}{n+1} \binom{2n}{n})$.
		\item We need real-time encoding-decoding. Therefore, 
			each step consumes $\mathcal{O}(1)$ time, 
			and total time is $\mathcal{\theta}(n)$.
    \end{itemize}
\end{frame}

\subsection{Parallel Generating All BST}
\begin{frame}
	\frametitle{Identity for BST}
		\tikzset{
  treeInode/.style = {align=center, inner sep=1pt, text centered,
    font=\sffamily},
  arn_n/.style = {treeInode, rectangle, rounded corners=1mm, black, draw=black, fill=black, text=white, font=\sffamily\bfseries, minimum width=1em, minimum height=1em}
    % arbre rouge noir, noeud noir
}
\usetikzlibrary{patterns}

\begin{tikzpicture}[-,>=stealth',level/.style={sibling distance = 0.5cm/#1,
  level distance = 0.6cm}]
\node[arn_n]{0}
; 
\node[below=0.2cm, align=flush center,text width=1cm]{$0$};
\end{tikzpicture}
		\begin{tikzpicture}[-,>=stealth',level/.style={sibling distance = 0.5cm/#1,
  level distance = 0.6cm}]
\node[arn_n]{1}
  child{
    node[arn_n]{0}
  }
  child[missing]{
  }
; 
\node[below=1cm, align=flush center,text width=1cm]{$0$};
\end{tikzpicture}
\begin{tikzpicture}[-,>=stealth',level/.style={sibling distance = 0.5cm/#1,
  level distance = 0.6cm}]
\node[arn_n]{0}
	child[missing]{
	}
	child{
		node[arn_n]{1}
	}
; 
\node[below=1cm, align=flush center,text width=1cm]{$1$};
\end{tikzpicture}
		\tikzset{
  treeInode/.style = {align=center, inner sep=0pt, text centered,
    font=\sffamily},
  treeEnode/.style = {align=center, inner sep=0pt, text centered,
    font=\sffamily},
  arn_n/.style = {treeInode, circle, black, draw=black, fill=black, text=white,
	  text width=0.2em,font=\small},% arbre rouge noir, noeud noir
  arn_x/.style = {treeEnode, rectangle, draw=black,
    minimum width=0.5em, minimum height=0.5em}% arbre rouge noir, nil
}
\usetikzlibrary{patterns}

\begin{tikzpicture}[-,>=stealth',level/.style={sibling distance = 0.5cm/#1,
  level distance = 0.4cm}] 
\node[arn_n]{2}
  child{
    node[arn_n]{1}
    child{
      node[arn_n]{0}
    }
    child[missing]{
    }
  }
  child[missing]{
  }
; 
\node[below=1.0cm, align=flush center,text width=0.5cm]{$0$};
\end{tikzpicture}
\begin{tikzpicture}[-,>=stealth',level/.style={sibling distance = 0.5cm/#1,
  level distance = 0.4cm}] 
\node[arn_n]{2}
  child{
    node[arn_n]{0}
    child[missing]{
    }
    child{
      node[arn_n]{1}
    }
  }
  child[missing]{
  }
; 
\node[below=1.0cm, align=flush center,text width=0.5cm]{$1$};
\end{tikzpicture}
\begin{tikzpicture}[-,>=stealth',level/.style={sibling distance = 0.5cm/#1,
  level distance = 0.4cm}] 
\node[arn_n]{1}
  child{
    node[arn_n]{0}
  }
  child{
    node[arn_n]{2}
  }
; 
\node[below=1.0cm, align=flush center,text width=0.5cm]{$2$};
\end{tikzpicture}
\begin{tikzpicture}[-,>=stealth',level/.style={sibling distance = 0.5cm/#1,
  level distance = 0.4cm}] 
\node[arn_n]{0}
  child[missing]{
  }
  child{
    node[arn_n]{2}
    child{
      node[arn_n]{1}
    }
    child[missing]{
    }
  }
; 
\node[below=1.0cm, align=flush center,text width=1cm]{$3$};
\end{tikzpicture}
\begin{tikzpicture}[-,>=stealth',level/.style={sibling distance = 0.5cm/#1,
  level distance = 0.4cm}] 
\node[arn_n]{0}
  child[missing]{
  }
  child{
    node[arn_n]{1}
    child[missing]{
    }
    child{
      node[arn_n]{2}
    }
  }
; 
\node[below=1.0cm, align=flush center,text width=0.5cm]{$4$};
\end{tikzpicture}
\end{frame}

\begin{frame}
	\frametitle{Definition}
	\begin{align*}
		& \mathit{LCA}(n, \mathit{tid}, p, q) \\
			&= \left\{\begin{matrix*}[l]
 				\mathit{LCA}(\mathit{lsz}, \mathit{lid}, p, q) &&, p \le q < \mathit{lsz}\\ 
 				\mathit{LCA}(\mathit{rsz}, \mathit{rid}, p-\mathit{lsz}-1, q-\mathit{lsz}-1)+\mathit{lsz}+1 &&, 
 						\mathit{lsz} \le p \le q < n \\ 
 				\mathit{lsz} && , 0 \le p \le \mathit{lsz}, p \le q \le i\\ 
 				-1 && ,\mathit{otherwise}
			\end{matrix*}\right.
	\end{align*}
\end{frame}

\end{document}
