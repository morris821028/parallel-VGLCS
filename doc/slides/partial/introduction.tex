\section{Introduction}

\subsection{Longest Common Subsequence}
\begin{frame}
    \frametitle{Longest Common Subsequence}
    %\setlength\itemsep{1em}
    Given two strings $X = x_1 \; x_2 \cdots x_n$ and $Y = y_1 \; y_2 \cdots y_m$.
    \begin{definition}
    	Subsequence $S = s_1 \; s_2 \cdots s_r$ of $X$, we define a 
    	\emph{correspondence sequence} of $X$ and $S$, $C(X, S) = c_1 \; c_2 \cdots c_r$ 
    	to be a strictly increasing sequence of integers such that 
    	$s_i = x_{c_i}$ $1 \le i \le r$
	\end{definition}
	\begin{definition}
		A common subsequence $S$ of $X$ and $Y$, if there exists correspondence sequence
		$C(X, S)$ and $C(Y, S)$.
	\end{definition}
\end{frame}

\subsection{Constrained LCS}
\begin{frame}
    \frametitle{Constrained Longest Common Subsequence}
    \begin{itemize}
    	\item Fixed Gapped LCS with respect to a given integer $K$
    		$$c_i - c_{i-1} \le K+1$$
    	\item Elastic Gapped LCS with respect to given integers $K_1$ and $K_2$
    		$$K_1 < c_i - c_{i-1} \le K_2 + 1$$
    \end{itemize}
\end{frame}

\begin{frame}
    \frametitle{Constrained Longest Common Subsequence}
    \begin{itemize}
        \item Rigid Fixed Gapped LCS
            $$C(X, S)[i] - C(X, S)[i-1] = C(Y, S)[i] - C(Y, S)[i-1]$$
        \item Variable Gapped LCS
            $$C(X, S)[i] - C(X, S)[i-1] \le G_{X}[c_i]$$
            $$C(Y, S)[i] - C(Y, S)[i-1] \le G_{Y}[c_i]$$
    \end{itemize}
\end{frame}

\begin{frame}
    \frametitle{Serial Algorithm Analysis}
    \begin{center}
        \begin{tabular}{>{\textsc}l >{\textsc}l >{\textrm}l >{\textsf}l}
            \hline
            Problem & Time  & Space     & Data Structure \\ \hline
            FIG     & $n^2$                 & $n^2$ & Monotone queue \\ \hline
            ELAG    & $n^2 + \mathcal{R} \log \log n$   & $\max(\mathcal{R}, n)$ 
                & van Emde Boas \footnote{$\mathcal{R} = |\{\;(i, j) \;|\; X[i] = Y[j]\}|$}\\ \hline
            RLCS    & $n^2$                 & $n^2$ & \\ \hline
            RIFIG   & $n^2$                 & $n^2$ & \\ \hline
            VG      & $n^2 \; \alpha(n)$    & $n^2$ 
                & Disjoint-set \footnote{$\alpha(2^{2^{2^{2^{16}}}}) = 4$}\\ \hline
        \end{tabular}
    \end{center}
\end{frame}

\begin{frame}
    \frametitle{VGLCS Example}
    \begin{figure}[!thb]
      \centering
      \subfigure[example 1]{
        \includegraphics[scale=0.5]{graphics/fig-VGLCSex.pdf}
      }
      \subfigure[example 2]{
        \includegraphics[scale=0.5]{graphics/fig-VGLCSex2.pdf}
      }
      \label{fig:VGLCSex}
    \end{figure}
\end{frame}

\subsection{Serial VGLCS Algorithm}
\begin{frame}
    \begin{center}
    \scalebox{0.6}{
    \begin{minipage}{1.5\linewidth}
      \begin{algorithm}[H]
        \caption{Algorithm for Finding VGLCS}
        \label{alg:serial}
          \begin{algorithmic}[1]
            \Require
              $A, B$: the input string;
              $G_A, G_B$: the array of variable gapped constraints;
            \Ensure Find the LCS with variable gapped constraints
            \State ISMQ $Q[n]$
            \State int $V[n][m]$
            \For{$i = 1$ to $n$}
              \State \tt{ISMQ} $RQ$
              \State $r = i - \min(GA[i]+1, i)$
              \For{$j = 1$ to $m$}
                \If{$A[i] = B[j]$}
                    \State $V[i][j] = RQ.get(j - \min(GB[j]+1, j))+1$
                    \State $t = Q[j].get(r)$ // the suffix maximum $A[r, i]$ in column $j$
                    \State $RQ.set(j, t)$ // set the value $t$ to position $j$
                    \State $Q[j].set(i, V[i][j])$
                \Else
                    \State $V[i][j] = 0$
                    \State $t = Q[j].get(r)$
                    \State $RQ.set(j, t)$
                \EndIf
              \EndFor
            \EndFor
            \State Retrieve the VGLCS by tracing $V[n][m]$
          \end{algorithmic}
      \end{algorithm}
    \end{minipage}%
    }
  \end{center}
\end{frame}

\subsection{Contributions}
\begin{frame}
    \frametitle{Contributions}
    \begin{itemize}
        \item Parallel VGLCS Algorithm in $O(n \log n)$
        \item Incremental Range Maximum Query in $O(n)$ -- amortized $O(1)$
    \end{itemize}
\end{frame}
