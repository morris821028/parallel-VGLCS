\section{Experiment}

\subsection{Environment Settings}
\begin{frame}
    \frametitle{Environment}
\end{frame}

\subsection{Parallel VGLCS}
\begin{frame}
	\frametitle{Runtime}
	\begin{figure}[!ht]
		\centering
		\subfigure[Runtime]{
			\documentclass[border=2pt]{standalone}
\usepackage{tikz}
\usetikzlibrary{calc} \usetikzlibrary{positioning} \usetikzlibrary{shapes,arrows} \usetikzlibrary{plotmarks}
\usepackage{pgfplots}

\begin{document}
	\begin{tikzpicture}
		\begin{axis}[
				xlabel={Length $n$},
				ylabel={Time (second)},
				xmin=0, xmax=10000,
				ymin=0, ymax=4.5,
				scaled ticks = false,
				tick label style={/pgf/number format/fixed},
				xtick={0, 1000, 2000, 3000, 4000, 5000, 6000, 7000, 8000, 9000, 10000},
				ytick={0, 0.5, 1, 1.5, 2, 2.5, 3, 3.5, 4, 4.5},
				legend pos=north east,
				legend cell align=left,
				ymajorgrids=true,
				grid style=dashed,
				line width=1pt,
				mark size=3pt,
				height=10cm,width=15cm,
			]
			\addplot [mark=*] file{./data/serial-n.dat};
			\addplot [mark=square*, mark options={fill=white}] file{./data/parallel-n.dat};
			\legend{serial, parallel}
		\end{axis}
	\end{tikzpicture}
\end{document}

			\label{fig:fig-parallel}
		}
		%\subfigure[Little core cluster]{
		%	\input{./data/light_little}
		%	\label{fig:light_little}
		%}
		\caption{Serial Algorithm and Parallel Algorithm}
		\label{fig:light_weight}
	\end{figure}
\end{frame}

\begin{frame}
	\frametitle{Scalability}
	\begin{figure}[!ht]
		\centering
		\subfigure[Scalability]{
			\documentclass[border=2pt]{standalone}
\usepackage{tikz}
\usetikzlibrary{calc} \usetikzlibrary{positioning} \usetikzlibrary{shapes,arrows} \usetikzlibrary{plotmarks}
\usepackage{pgfplots}

\begin{document}
	\begin{tikzpicture}
		\begin{axis}[
				xlabel={Processor $p$},
				ylabel={Time (second)},
				xmin=1, xmax=16,
				ymin=0, ymax=1.2,
				scaled ticks = false,
				tick label style={/pgf/number format/fixed},
				xtick={1, 2, 4, 8, 16},
				ytick={0, 0.2, 0.4, 0.6, 0.8, 1, 1.2},
				legend pos=north east,
				legend cell align=left,
				ymajorgrids=true,
				grid style=dashed,
				line width=1pt,
				mark size=3pt,
				height=10cm,width=15cm,
			]
			\addplot [mark=square*, mark options={fill=white}] file{./data/parallel-p.dat};
			\legend{parallel $n=5000$}
		\end{axis}
	\end{tikzpicture}
\end{document}

			\label{fig:fig-parallel}
		}
		%\subfigure[Little core cluster]{
		%	\input{./data/light_little}
		%	\label{fig:light_little}
		%}
		\caption{Parallel Algorithm}
		\label{fig:light_weight}
	\end{figure}
\end{frame}

\subsection{Parallel RMQ and Cache-Oblivious Cartesian Tree}
\begin{frame}
	\frametitle{Speedup}
	We have $N$ elements of an array and $N$ querys.
	\begin{itemize}
		\setlength\itemsep{1em}
		\item When $N = 30000$ and maximum length of interval $L \le 100$, \texttt{CORQM} can speedup $1.25\times$ 
		because of less cache-misses.
	\end{itemize}
\end{frame}