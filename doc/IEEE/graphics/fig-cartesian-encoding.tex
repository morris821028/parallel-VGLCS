\documentclass[border=2pt]{standalone}
\usepackage{tikz}
\usepackage{amsmath}
\usepackage{mathtools}
\usepackage{adjustbox}
\usetikzlibrary{patterns}
\usetikzlibrary{calc} \usetikzlibrary{positioning} \usetikzlibrary{shapes,arrows} \usetikzlibrary{plotmarks}
\usetikzlibrary{positioning,decorations.pathreplacing}
\tikzset{
itria/.style={
  draw,dashed,shape border uses incircle,
  isosceles triangle,shape border rotate=90,yshift=-0.8cm}
}

\begin{document}

\begin{tikzpicture}[level/.style={sibling distance=20mm, level distance=10mm}]
\node [circle,draw](inf){$\infty$}
  child[missing]{}
  child { node[circle,draw, dashed](start){$\phantom{A}$}
    child[missing]{}
    child[dashed] { node[circle,draw, dashed]{$\phantom{A}$}
      child[missing]{}
      child[dashed] { node[circle,draw, dashed](mid){$\phantom{A}$}
        child[missing]{}
        child[dashed] { node[circle,draw, dashed]{$\phantom{A}$}
          child[missing]{}
          child[dashed] { node[circle,draw, dashed](end){$\phantom{A}$}
          }
        }
      }
    }
  };
\draw[decorate,decoration={brace,amplitude=3mm}](start.45)--node[anchor=south west,above right=5pt and 5pt] {$n$ nodes}(end.60);

\coordinate (A) at ([yshift=-.4cm,xshift=-.4cm] start.north west);
\coordinate (B) at ([yshift=.4cm,xshift=.4cm] start.north east);
\coordinate (C) at ([yshift=.4cm,xshift=.4cm] end.south east);
\coordinate (D) at ([yshift=-.4cm,xshift=-.4cm] end.south west);

\draw[dashed, red] plot[smooth cycle] coordinates {(A) (B) (C) (D)};

%\node[below left=10pt and 10pt of D]{$t_{{\it root}_0}= C_n - 1$};

\end{tikzpicture}

\end{document}