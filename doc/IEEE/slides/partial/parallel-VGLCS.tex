\section{Parallel VGLCS Algorithm}

\subsection{Dynamic Programming}
\begin{frame}
    \frametitle{Basic Dynamic Programming}
	Given two strings $A$, $B$, and two gap values $G_{A}$, $G_{B}$
	\begin{align*}
        C(A, S)[i] - C(A, S)[i-1] \le G_{A}(c_i) \\
        C(B, S)[i] - C(B, S)[i-1] \le G_{B}(c_i)
    \end{align*}	

    Recursive formula as follows:

    \begin{align*}
    	V[i][j] = \left\{\begin{matrix}
 			(\max\limits_{
 				\substack{i-G_A(i)-1 \le x \le i-1 \\ 
 						  j-G_B(j)-1 \le y \le j-1}} 
 				V[x][y])+1
 				& \text{, if } A[i] = B[j] \\
 				0 & \text{, otherwise}\\
			\end{matrix}\right.
    \end{align*}
\end{frame}

\begin{frame}
    \frametitle{Illustration for Basic Dynamic Programming}
	\begin{figure}[!thb]
		\includegraphics[width=0.5\linewidth]{\GraphicPath/fig-VGLCS-dp-naive.pdf}
		\caption{How to compute $V$.}
		\label{fig:basic-dp-VGLCS}
	\end{figure}
\end{frame}

\subsection{Peng's Algorithm}
\begin{frame}
    \frametitle{Peng's Algorithm}
   	Computation of all $V[i][j]$'s with the same $i$ has the {\em same}
	gap constraint $G_A(i)$, so the maximum within the rectangle can be
	computed in two steps.

    \begin{enumerate}
    	\setlength\itemsep{1em}
    	\item 
    		First, we compute the maximum of {\em every column} of this
			rectangle, and place them into another array $R$.
		\item 
			Second, we compute the maximum of the {\em suffix} of length
			$G_B$ on $R$, which is exactly $V[i][j]$.
    \end{enumerate}
\end{frame}

\begin{frame}
    \frametitle{Illustration for Peng's Algorithm}
    \begin{figure}[!thb]
		\includegraphics[width=0.5\linewidth]{\GraphicPath/fig-VGLCS-dp.pdf}
		\caption{Compute $V$ with incremental suffix maximum queries.}
		\label{fig:basic-dp-VGLCS}
	\end{figure}
\end{frame}

\begin{withoutheadline}
\begin{frame}
	\frametitle{Pseudocode for Peng's Algorithm}
	\begin{center}
		\scalebox{.7} { \begin{minipage}{1.2\textwidth}
			\begin{algorithm}[t]
\SetAlgoNoLine
\KwIn{$A, B$: the input string; $G_A, G_B$: the array of variable gapped constraints;}
\KwOut{Find the LCS with variable gapped constraints.}

Create $m$ number of data structure $Q[m]$ to support ISMQ problem. \;
Create an empty table $V[n][m]$.

\For{$i \gets 1$ to $n$} {
  Create a data structure $RQ$ to support ISMQ problem.\;
  $r \gets i - (GA[i]+1)$ \;
  \For{$j \gets 1$ to $m$} {
    \uIf{$A[i] = B[j]$} {
        Get the suffix maximum value $t$ from position $j - (GB[j]+1)$ to the end in $RQ$.\;
        $V[i][j] \gets t + 1$;
        Get the suffix maximum value $t$ from position $r$ to $i$ in $Q[j]$.\;
        Append value $t$ into $RQ$. \;
        Append value $V[i][j]$ into $Q[j]$. \;
    }
    \Else {
        $V[i][j] \gets 0$ \;
        Get the suffix maximum value $t$ from position $r$ to $i$ in $Q[j]$.\;
        Append value $t$ into $RQ$. \;
    }
  }
}
Retrieve the VGLCS by tracing $V[n][m]$\;

  \caption{Algorithm for Finding VGLCS~\cite{Peng2011TheLC}}
  \label{alg:serial-VGLCS}
\end{algorithm}

			\end{minipage}
		}
	\end{center}
\end{frame}
\end{withoutheadline}

\subsection{Incremental Suffix Maximum Query}
\begin{frame}
    \frametitle{Incremental Suffix Maximum Query}

    We need to address the {\em incremental suffix maximum query} (ISMQ)
    problem to solve VGLCS problem efficiently.
    \\~\\
    It supports the three operations as follows:
    \begin{description}[align=right]
    	\setlength\itemsep{1em}
    	\item[{\sc Make}]
			creates an empty array $A$.
		\item[{\sc Append(V)}]
			appends a value $V$ to array $A$. 
		\item[{\sc Query(x)}]
			finds the {\em maximum} value among those from $x$ to the
			end of an array $A$.
    \end{description}
\end{frame}

\begin{frame}
    \frametitle{Incremental Suffix Maximum Query with Disjoint Set}

    Peng uses a {\em disjoint-set} data structure to answer incremental
	suffix maximum queries in his VGLCS algorithm.

	\begin{itemize}
		\setlength\itemsep{1em}
		\item 
			The set of data are stored in a sequence of disjoint sets,
			and the {\em maximum} of each disjoint set is at the root of
			the tree, and these maximum are in {\em decreasing} order.
		\item
			The amortized time per union/find operation is
			$O(\alpha(n))$.
		\item
			Union/find operation of {\em incremental tree set union}
			runs in amortized time $O(1)$ by constructing the answer
			table.
	\end{itemize}
\end{frame}

\begin{withoutheadline}
\begin{frame}
	\frametitle{Illustration for ISMQ with Disjoint Set}
	\begin{figure}[!thb]
		\centering \subfigure[Before] {
			\includegraphics[width=0.45\linewidth]{\GraphicPath/fig-ISMQ-DS.pdf}
		} \subfigure[After] {
			\includegraphics[width=0.45\linewidth]{\GraphicPath/fig-ISMQ-DS-after.pdf}
		}
		\caption{Answer ISMQ with disjoint set. $A$ is the input array,
			$W$ is the maximum weight array of each group, and the $I$
			is the index of the maximum weight array of each group. }
	\end{figure}
\end{frame}
\end{withoutheadline}

\subsection{Parallel VGLCS Algorithm}
\begin{frame}
    \frametitle{Parallel VGLCS Algorithm}
	Wavefront method is easy to parallel VGLCS algorithm, but row-by-row
	approach provides

    \begin{itemize}
    	\setlength\itemsep{1em}
		\item 
			Less space
		\item 
			Smaller the length of the critical path
    \end{itemize}
\end{frame}

\begin{frame}
	\frametitle{Wavefront Method for Parallel VGLCS Algorithm}
	\begin{figure}[!thb]
	  \centering \subfigure[The first stage]{
	    \includegraphics[width=0.45\linewidth]{\GraphicPath/fig-VGLCS-dp-wavefront-second.pdf}
	  } \subfigure[The second stage]{
	    \includegraphics[width=0.45\linewidth]{\GraphicPath/fig-VGLCS-dp-wavefront-first.pdf}
	  }
	  \caption{The book keeping data of the wavefront method}
	  \label{fig:fig-VGLCS-dp-wavefront}
	\end{figure}
\end{frame}

\begin{frame}
	\frametitle{The Advantage of Our Algorithm}
	Our optimized row-by-row approach
	\begin{itemize}
		\setlength\itemsep{1em}
		\item 
			Less space
		\item 
			Smaller the length of the critical path
		\item 
			More balanced workload
		\item 
			Smaller thread synchronization overhead
	\end{itemize}
\end{frame}

\begin{frame}
	\frametitle{A Sketch of Our Algorithm}
	Our algorithm computes $V$ one row at a time.  The computation of
	each row has two stages.
	\begin{center}
		\scalebox{.7} { \begin{minipage}{1.2\textwidth}
			\begin{figure}
			  \centering \subfigure[The first stage]{
			    \includegraphics[width=0.4\linewidth]{\GraphicPath/fig-VGLCS-dp-rmq-first.pdf}
			  } \subfigure[The second stage]{
			    \includegraphics[width=0.4\linewidth]{\GraphicPath/fig-VGLCS-dp-rmq-second.pdf}
			  }
			  \caption{Two stages of the computation of one row of $V$.}
			  \label{fig:fig-VGLCS-dp-rmq}
			\end{figure}
			\end{minipage}
		}
	\end{center}
\end{frame}

\subsection{Data Structure of Parallel VGLCS Algorithm}
\subsubsection{Disjoint Set}
\begin{frame}
    \frametitle{Disjoint Set}
    It is {\em not} feasible to parallelize the disjoint set
	implementation for three reasons when a large number of threads work
	together.

	\begin{itemize}
		\setlength\itemsep{1em}
		\item
			{\sc Query} will change the data structure because it will
			{\em compress} the path to the root and lead {\em
			inconsistent} view of the data structure.
		\item 
			The load of compressing different paths among threads could
			be very different, and this will incur {\em load imbalance}.
		\item 
			Different parts of the disjoint set will lead to synchronize
			them {\em inefficiently}.
	\end{itemize}
\end{frame}

\subsubsection{Sparse Table}
\begin{frame}
    \frametitle{Sparse Table}
	We use {\em sparse table} to support incremental suffix/range
	maximum queries in our VGLCS algorithm.

	\begin{itemize}
		\setlength\itemsep{1em}
		\item 
			$O(n \log n)$ preprocessing, and 
		\item 
			$O(1)$ time to answer queries on one dimensional data.
	\end{itemize}
\end{frame}

\begin{withoutheadline}
\begin{frame}
	\frametitle{An Example for Sparse Table}
	\begin{figure}[!thb]
	  \centering \subfigure[Array]{
	    \includegraphics[width=0.8\linewidth]{\GraphicPath/fig-interval-decomposition-origin.pdf}
	    \label{fig:fig-interval-decomposition}
	  } \subfigure[Sparse table]{
	    \includegraphics[width=0.8\linewidth]{\GraphicPath/fig-sparse-table-origin.pdf}
	    \label{fig:fig-sparse-table}
	  }
	  \caption{A sparse table example}
	  \label{fig:interval-decomposition}
	\end{figure}
\end{frame}
\end{withoutheadline}

\begin{withoutheadline}
\begin{frame}
	\frametitle{Parallel Sparse Table Building Algorithm}
	Algorithm~\ref{alg:parallel-sparse-table} runs in $O(n \log n / p
	+ \log n)$ time, where $p$ is the number of processors.
	\\~\\
	\begin{algorithm}[!thb]
\SetAlgoNoLine
\KwIn{$A[0 .. N-1]$: the input array}
\KwOut{$T$: a sparse table for the input array $A$}

Create a two-dimensional array $T[\log N][N]$ \;
Copy $A$ to $T[0]$ \;
\For{$j \gets 0$ to $\log N$} {
  \ForPar{$i \gets 2^j$ to $N$} {
    $T[j][i] \gets \max(T[j-1][i-2^{j}], T[j-1][i])$ \;
  }
}
  \caption{A Parallel Sparse Table Building Algorithm}
  \label{alg:parallel-sparse-table}
\end{algorithm}

\end{frame}
\end{withoutheadline}


\subsection{Parallel VGLCS Algorithm with Sparse Table}
\begin{withoutheadline}
\begin{frame}
	\frametitle{Parallel VGLCS Algorithm with Sparse Table}
	Algorithm~\ref{alg:parallel-VGLCS} runs in $O(n m (\log n + \log m)
	/ p + n \log m)$ time, where $p$ is the number of processors.
	\\~\\
	\begin{center}
		\scalebox{.7} { \begin{minipage}{1.4\textwidth}
			\begin{algorithm}[!thb]
\SetAlgoNoLine
\KwIn{$A, B$: the input string; $G_A, G_B$: the array of variable gapped constraints;}
\KwOut{Find the LCS with variable gapped constraints}
    
Create an array of $m$ data structure $C[m]$ to support ISMQ \;
Create an empty table $V[n][m]$ \;
\For{$i \gets 1$ to $n$} {
  \ForPar(\tcp*[f]{The first stage}){$j \gets 1$ to $m$} {
    $M[j] \gets$ suffix maximum of length $G_A(i) + 1$ from $C[j]$ \;
  }
  Build a sparse table $T$ with data of $M$ in parallel with Algorithm~\ref{alg:parallel-sparse-table}\; 
  \ForPar(\tcp*[f]{The second stage}){$j \gets 1$ to $m$} {
    \If{$A[i] = B[j]$} {
        $t \gets $ the range maximum among the $G_B(j) + 1$ elements before $M[j]$ by querying $T$ \;
        $V[i][j] \gets t + 1$ \;
        Append $V[i][j]$ to $C[j]$ \;
    }
  }
}
Retrieve the VGLCS by tracing $V[n][m]$ \;
  \caption{Parallel Algorithm for Finding VGLCS}
  \label{alg:parallel-VGLCS}
\end{algorithm}

			\end{minipage}
		}
	\end{center}
\end{frame}
\end{withoutheadline}