\section{RMQ on Incremental Data}

\subsection{Range Maximum Query on Incremental Data}
\begin{frame}
    \frametitle{Range Maximum Query on Incremental Data}
    \begin{itemize}
    	\setlength\itemsep{1em}
    	\item
    		Answer incremental {\em range} maximum queries on
			incrementally added data.
    	\item
    		This section describes our approach to address the
			challenges in the {\em first} stage of
			Algorithm~\ref{alg:parallel-VGLCS}.
    \end{itemize}
\end{frame}

\subsection{Parallel Building Least Common Ancestor Table}
\begin{frame}
    \frametitle{Parallel Building Least Common Ancestor Table}
    We need to address two issues in parallel environment:
    \begin{itemize}
    	\setlength\itemsep{1em}
    	\item 
    		How to label a binary tree into its {\em Catalan index}
		\item 
			How to find the least common ancestor of two nodes in a
			given Cartesian tree
	\end{itemize}
\end{frame}

\subsection{Catalan Tree Labeling}
\begin{frame}
    \frametitle{Catalan Tree Labeling}
    Our Cartesian tree labeling will enumerate all binary search trees
	in {\em lexicographical order} from $0$ to the $C_n-1$.
	\\~\\
	A binary tree $x$ appears {\em before} another binary tree $y$ if
	any of the the following condition is true.
	\begin{itemize}
		\setlength\itemsep{1em}
		\item 
			$x$ has more nodes than $y$ in the left subtree.
		\item 
			$x$ and $y$ have the same number of nodes in the left
  			subtree, and $x$'s left subtree appears before $y$'s left
  			subtree in lexicographical order.
		\item 
			$x$ and $y$ has the same left subtree, and $x$'s right
	  		subtree appears before $y$'s right subtree in
	  		lexicographical order.
	\end{itemize}
\end{frame}

\begin{frame}
	\frametitle{An Example for Catalan Tree Labeling}
	\begin{figure}[!thb]
	  \centering
	  \includegraphics[width=\linewidth]{\GraphicPath/fig-bst-encoding.pdf}
	  \caption{The labeling of binary search trees of sizes 1, 2, and 3.}
	  \label{fig:labelingBST}
	\end{figure}
\end{frame}

\subsection{Least Common Ancestor}
\begin{frame}
	\frametitle{Recursive Formula of Least Common Ancestor}
	Let ${\cal A}(s, t, p, q)$ denote the least common ancestor of the
	node $p$ and $q$ within a binary search tree of size $s$ that has a
	Catalan index $t$.
	\begin{center}
		\scalebox{.7} { \begin{minipage}{1.4\textwidth}
			\begin{equation*}
  \begin{split}
    &\mathit{LCA}(n, \mathit{tid}, p, q) \\
      &= \left\{\begin{matrix*}[l]
        \mathit{LCA}(\mathit{lsz}, \mathit{lid}, p, q) &&, p \le q < \mathit{lsz}\\ 
        \mathit{LCA}(\mathit{rsz}, \mathit{rid}, p-\mathit{lsz}-1, q-\mathit{lsz}-1)+\mathit{lsz}+1 &&, 
            \mathit{lsz} \le p \le q < n \\ 
        \mathit{lsz} && , 0 \le p \le \mathit{lsz}, \mathit{lsz} \le q \le i\\ 
        -1 && ,\mathit{otherwise}
      \end{matrix*}\right.
  \end{split}
\end{equation*}
			\end{minipage}
		}
	\end{center}
\end{frame}

\begin{frame}
	\frametitle{Parallel Building LCA Algorithm}
	The time complexity of Algorithm~\ref{alg:parallel-LCA} is
	$O(\frac{s^3}{s+1} \binom{2s}{s} / p + s^2)$.
	\\~\\
	We choose the block size $s = {{\frac{\log n}{4}}}$, so time
	complexity is $O(\sqrt{n} \; (\log^{1.5} n) / p + \log^2 n )$, where
	$p$ is the number of processors.
	\begin{center}
		\scalebox{.7} { \begin{minipage}{1.4\textwidth}
			\begin{algorithm}[!thb]
\SetAlgoNoLine
\KwIn{$s$: the maximum tree size}

\For{$n \gets 1$ to $s$} {
  \ForPar{$t \gets 0$ to $C_n - 1$} {
    \ForPar{$p \gets 0$ to $n-1$} {
      Compute $s_l$, $t_l$, $s_r$, and $t_r$ \;
      \For{$q \gets p$ to $n-1$} {
        Compute ${\cal A}[n][t][p][q]$ according to Equation~\ref{fun:LCA} \;
      }
    }
  }
}

\caption{A parallel algorithm that computes the least common ancestor
  table $\cal A$}
\label{alg:parallel-LCA}
\end{algorithm}

			\end{minipage}
		}
	\end{center}
\end{frame}

\subsection{Catalan Index Computation}

\begin{frame}
	\frametitle{Catalan Index Computation}
	Determine Catalan index $t$ efficiently when given a block of data.
	There are two possible approaches

	\begin{itemize}
		\setlength\itemsep{1em}
		\item Build the tree
		\item Keep the rightmost path
	\end{itemize}
\end{frame}

\subsubsection{Build the Tree}
\begin{frame}
    \frametitle{Build the Tree}
    \begin{enumerate}
    	\setlength\itemsep{1em}
    	\item 
    		Build the tree from the data of block.
    	\item
    		Compute Catalan index from the the sizes and indices of the
			left and right subtrees.  
		\item 
			This requires a recursive traversal on the tree and has a
			$O(n)$ time complexity.\footnote{precomputing the {\em
			prefix sum} of the Catalan number products.}
	\end{enumerate}

    \begin{center}
		\scalebox{1} { \begin{minipage}{\textwidth}
			\begin{eqnarray}  \label{fun:tid}
  {\cal T}({\mathit l}_t, {\mathit l}_s, {\mathit r}_t, {\mathit r}_s)
    = {\mathit l}_t \cdot C_{{\mathit r}_s} + {\mathit r}_t + 
          \sum_{i = 0}^{{\mathit l}_s - 1} C_i C_{{\mathit l}_s + {\mathit r}_s - i}
\end{eqnarray}

			\end{minipage}
		}
	\end{center}
\end{frame}

\subsubsection{Keep the Rightmost Path}
\begin{frame}
	\frametitle{Keep the Rightmost Path}
	We propose a more efficient method than the previous computation of
	building the tree.

	\begin{itemize}
		\setlength\itemsep{1em}
		\item 
			Determines the Catalan index by keeps only the {\em
			rightmost path} in a {\em stack} $D$.
		\item
			After knowing the Catalan index $t$ we can compute LCA and
			answer queries with Algorithm~\ref{alg:parallel-LCA} and
			Equation~\ref{fun:tid}.
	\end{itemize}
	\begin{center}
		\scalebox{0.8} { \begin{minipage}{0.8\textwidth}
			\begin{figure}[!thb]
			  \centering
			  \includegraphics[width=0.5\linewidth]{\GraphicPath/fig-cartesian-encoding-static.pdf}
			  \caption{Compute Catalan index for a tree.  $A_l$ and $B_l$ denote
			    the left subtrees of $A$ and $B$ respectively.}
			  \label{fig:fig-cartesian-encoding-static}
			\end{figure}
			\end{minipage}
		}
	\end{center}
\end{frame}

\begin{withoutheadline}
\begin{frame}
	\frametitle{Pseudocode of Catalan Index Computation}
	\begin{center}
		\scalebox{0.7} { \begin{minipage}{\textwidth}
			\begin{algorithm}
\SetAlgoNoLine
\KwIn{
  $A[1 .. n]$: input data block; $n$: the number of elements\;
}
\KwOut{
  $t$: The Catalan index of the input data block
}
Create a stack $D$ of $n+1$ elements.  Every element has $s$, $t$, and $v$ \;
$p \gets 0$, $D[0] \gets \langle 0,0,\infty \rangle$ \;
\For{$i \gets 1$ to $n$} {
  $v \gets A[i]$ \; $s \gets 0$, $t \gets 0$ \;
  \While{$D[p].{v} < v$} {
    $t \gets {\cal T}(D[p].s, D[p].t, s, t)$ \;
    $s \gets s + D[p].s + 1$ \;
    $p \gets p - 1$ \;
  }
  $p \gets p + 1$ \;
  $D[p] \gets \langle s,t,{v}\rangle$ \;
}

$s \gets 0$, $t \gets 0$ \;
\While{$p > 0$} {
  $t \gets {\cal T}(D[p].s, D[p].t, s, t)$ \;
  $s \gets s + D[p].s + 1$ \;
  $p \gets p - 1$ \;
}
return $t$ \;

\caption{Catalan index computation for a data block}
\label{alg:cartesian-encode-offline}
\end{algorithm}

			\end{minipage}
		}
	\end{center}
\end{frame}
\end{withoutheadline}

\subsection{Dynamic Catalan Index Computation}
\begin{frame}
	\frametitle{Background -- Dynamic Catalan Index Computation}
	\begin{itemize}
		\setlength\itemsep{1em}
	 	\item 
	 		Fischer introduced the first encoding method.
		\item
			Masud presents a new encoding method that reduces the number
			of instructions.
		\item 
			Unfortunately all these algorithms are {\em off-line}, or
			could not use the encoding, which is presented by parallel
			building algorithm.
	\end{itemize}
\end{frame}

\begin{frame}
	\frametitle{Dynamic Catalan Index Computation}
	\begin{itemize}
		\setlength\itemsep{1em}
	 	\item 
	 		Keeps the size of tree as a constant $n$.
	 	\item
	 		Append a path of $n-i$ {\em right-child-only} nodes to the
			right most path of the existing tree of size $i$.
		\item
			Compute the {\em difference} between the Catalan indices of
			these two different parts, we can compute the final $t^*$ by
			{\em patching} the Catalan index with this difference.
	\end{itemize}
\end{frame}

\begin{frame}
	\frametitle{Illustration for Dynamic Catalan Index Computation}
	\begin{center}
		\scalebox{0.8} { \begin{minipage}{1.3\textwidth}
			\begin{figure}[!thb]
			  \centering \subfigure[$t_{{\it root}_0}= C_n - 1$]{
			    \includegraphics[width=0.26\linewidth]{\GraphicPath/fig-cartesian-encoding.pdf}
			    \label{fig:cartesianEncoding-init}
			  } \subfigure[$t_{{\it root}_i}$]{
			    \includegraphics[width=0.3\linewidth]{\GraphicPath/fig-cartesian-encoding-before.pdf}
			    \label{fig:cartesianEncoding-before}
			  } \subfigure[$t_{{\it root}_{i+1}} = t_{{\it root}_i} + t_x - t_A$]{
			    \includegraphics[width=0.3\linewidth]{\GraphicPath/fig-cartesian-encoding-after.pdf}
			    \label{fig:cartesianEncoding-after}
			  }
			  \caption{Normalization of Cartesian trees of increasing sizes by
			    adding a virtual path.}
			  \label{fig:cartesianEncoding}
			\end{figure}
			\end{minipage}
		}
	\end{center}
\end{frame}

\begin{frame}
	\frametitle{Pseudocode of Dynamic Catalan Index Computation}
	\begin{center}
		\scalebox{0.7} { \begin{minipage}{\textwidth}
			\begin{algorithm}[!thb]
\SetAlgoNoLine
\KwIn{
      $\mathit{state}$: state of Cartesian Tree\;
      $v$: the value which append to array\;
}
\KwOut{
      $\mathit{tid}$: this label
}

$\textit{Dp} \gets \textit{state}.\textit{Dp}$, $\textit{lsz} \gets 0$, $\textit{lid} \gets 0$ \;
$\textit{bsz} \gets \textit{state}.\textit{s} - \textit{state}.\textit{i} + 1$ \;
$\textit{bid} \gets C[\textit{bsz}] - 1$ \;

\While{$\textit{state}.D[\textit{Dp}].\textit{value} < v$} {
  $\textit{lid} \gets \textit{tid}(\textit{state}.D[\textit{Dp}].\textit{lsz}, \textit{state}.D[\textit{Dp}].\textit{lid}, \textit{lsz}, \textit{lid})$ \;
  $\textit{bid} \gets \textit{tid}(\textit{state}.D[\textit{Dp}].\textit{lsz}, \textit{state}.D[\textit{Dp}].\textit{lid}, \textit{bsz}, \textit{bid})$ \;
  $\textit{lsz} \gets \textit{lsz} + \textit{state}.D[\textit{Dp}].\textit{lsz}+1$ \;
  $\textit{bsz} \gets \textit{bsz} + \textit{state}.D[\textit{Dp}].\textit{lsz}+1$ \;
  $\textit{Dp} \gets \textit{Dp} - 1$ \;
}
$\textit{Dp} \gets \textit{Dp} + 1$ \;
$\textit{state}.D[\textit{Dp}] \gets \left \langle \textit{lsz}, \textit{lid}, \textit{v} \right \rangle$ \;
$\textit{state}.\textit{Dp} \gets \textit{Dp}$ \;
$x.\textit{tid} \gets \textit{tid}(\textit{lsz}, \textit{lid}, \textit{state}.s-\textit{state}.i, C[\textit{state}.s-\textit{state}.i]-1)$ \;
$\textit{state}.\textit{tid} \gets \textit{state}.\textit{tid} + \textit{bid} - x.\textit{tid}$ \;
$\textit{state}.i \gets \textit{state}.i + 1$ \;
return $\textit{state}.\textit{tid}$ \;

  \caption{Online Type of Cartesian Tree}
  \label{alg:cartesian-encode-online}
\end{algorithm}
			\end{minipage}
		}
	\end{center}
\end{frame}