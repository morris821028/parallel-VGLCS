\section{Implementation}

\subsection{The Strategy of Disjoint Set}

\begin{frame}
	\frametitle{Cache Performance}
	\begin{itemize}
		\setlength\itemsep{1em}
		\item
			Traditional disjoint set merging techniques are {\em merge-by-rank}
			and {\em merge-by-size}. 
		\item
			Usually, an algorithm with a lower time complexity, e.g.,
			will have more ``long jumps'' than an algorithm with a
			higher time complexity.
	\end{itemize}
	\begin{center}
	\scalebox{0.7} { \begin{minipage}{1.5\textwidth}
				\begin{figure}[!thb]
			  \centering \subfigure[Lower time complexity] {
			    \includegraphics[width=0.42\linewidth]{\GraphicPath/fig-rem-long-jump.pdf}
			    \label{fig:long-short-jump-disjoint-long}
			  } \subfigure[Higher time complexity] {
			    \includegraphics[width=0.42\linewidth]{\GraphicPath/fig-rem-short-jump.pdf}
			    \label{fig:long-short-jump-disjoint-short}
			  }
			  \caption{The parent jump in disjoint set}
			  \label{fig:long-short-jump-disjoint}
			\end{figure}
			\end{minipage}
		}
	\end{center}
\end{frame}

\begin{frame}
	\begin{itemize}
		\setlength\itemsep{1em}
		\item
			The Rem's algorithm ({\sc Rem}) achieves better cache
			performance by a {\em merge-by-index} technique.
		\item
			Experiments show that the merge-by-index tie-breaking
			technique improves by 3\% over the approach that breaks the
			tie randomly.
	\end{itemize}
\end{frame}

\subsection{Parallel Range Maximum Query}
\begin{frame}
    \frametitle{Parallel Range Maximum Query}
    We further improve the performance of range query by maintaining two
	extra tables in the blocked sparse table approach.  
	\\~\\
	There are now three tables as follows:

	\begin{itemize}
		\setlength\itemsep{1em}
		\item
			A sparse table on the block maximum $T_S$, 
		\item
			A {\em prefix maximum table} $P$, and 
		\item
			A {\em suffix maximum table} $S$.
	\end{itemize}
\end{frame}

\begin{frame}
	\frametitle{The Order to Access these Tables}
	The {\em order} to access these maximum tables is important.
	\begin{itemize}
		\setlength\itemsep{1em}
		\item
			Our implementation improves performance by ``peeking'' into
			two neighboring elements in the $T_{S}$ table.
		\item 
			We access two elements in the sparse table $T_S$ in {\em the
  			same level} because accessing the first will bring in the
  			other by the hardware caching mechanism.
  		\item
  			Because of intermediate maximum value and neighboring
			elements, it can save unnecessary access to $P$ and $S$
			tables.
		\item
			It improves the overall performance of query operations by
			up to 35\% when $n = 10^7$.
	\end{itemize}
\end{frame}

\begin{withoutheadline}
\begin{frame}
	\frametitle{Illustration for the Access these Tables}
	\begin{center}
	\scalebox{0.7} { \begin{minipage}{1.5\textwidth}
				\begin{figure}[!thb]
				  \centering \subfigure[The prefix/suffix maximum tables for blocks] {
				    \includegraphics[width=0.9\linewidth]{\GraphicPath/fig-compressed-sp-prefix-suffix.pdf}
				  } \subfigure[A Sparse Table for $T_S$] {
				    \includegraphics[width=0.3\linewidth]{\GraphicPath/fig-sparse-table-opt.pdf}
				  } \caption{Block maximum $T_S$, prefix maximum $P$, and suffix
				    maximum $S$.}
				  \label{fig:compressed-sp-opt}
				\end{figure}
				\end{minipage}
		}
	\end{center}
\end{frame}
\end{withoutheadline}

\begin{withoutheadline}
\begin{frame}
	\frametitle{Pseudocode for Order to Access these Tables}
	\begin{center}
	\scalebox{0.65} { \begin{minipage}{1.5\textwidth}
				\begin{algorithm}
\SetAlgoNoLine
\KwIn{
  $A$: the input array\; 
  $P, \; S$: the prefix/suffix maximum array for each block of the compressed sparse table \;
  ${\cal T}$: the Catalan index array for each block of the compressed sparse table \;
  $T_s$: the compressed sparse table \; 
  $[l, r]$: ranged query \;
}
\KwOut{
  $v$: the maximum value in $A[l .. r]$ \;
}

\If{$l$ and $r$ in the same block} {
  $i \gets$ Query the ranged maximum query $[l, r]$ on ${\cal T}_\text{block(l)}$ \;
  return $A[i]$ \;
}
$v \gets - \infty$ \;
$l' \gets$ The lower bound of the next block by the position $l$ \;
$r' \gets$ The upper bound of the previous block by the position $r$ \;
\If{$l' \le r'$} {
  $t \gets \lfloor \log_2 (r'-l'+1) \rfloor$ \;
  $SQ_L \gets T_s[t][l' + 2^t - 1]$ \;
  $SQ_R \gets T_s[t][r']$ \;
  $v \gets \max(SQ_L, SQ_R)$ \;
}

\If(\tcc*[f]{$T_s[0]$ is a cache to avoid impossible event}){$T_s[0][\text{block}(r)] > v$} {
  $v \gets \max(v, P[r])$ \;
}

\If(\tcc*[f]{Place very low possibility event to the end}){$T_s[0][\text{block}(l)] > v$} {
  $v \gets \max(v, S[l])$ \;
}

return $v$ \;

\caption{Access order of ranged maximum query}
\label{alg:rmq-access-order-2e}
\end{algorithm}

				\end{minipage}
		}
	\end{center}
\end{frame}
\end{withoutheadline}