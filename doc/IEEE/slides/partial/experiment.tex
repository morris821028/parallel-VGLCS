\section{Experiment}

\begin{frame}
	\frametitle{Environment}
	Experiments on 
	\begin{itemize}
		\setlength\itemsep{1em}
		\item
			Intel Xeon E5-2620 v3
			\begin{itemize}
				\setlength\itemsep{1em}
				\item
					2.4 Ghz processor, hyper-threading, and each
					processor has six cores.
				\item
					384 KB of L1 cache, 1536 KB of L2 cache, and 15 MB
					of shared L3 cache.
			\end{itemize}
		\item 
			Ubuntu 14.04
		\item
			C++ and OpenMP \\
			Using gcc with {\tt -O2} and {\tt -fopenmp} flag.
	\end{itemize}
\end{frame}

\subsection{VGLCS}
\begin{frame}
    \frametitle{Variable Gapped Longest Common Subsequence}
	Four combinations of data structures in solving the VGLCS problem
	and evaluated their performance.

	\begin{description}
		\setlength\itemsep{1em}
		\item[Peng's algorithm] \hfill \\
			uses disjoint set on {\em both} the first stage and second stage.
		\item[{\sc Para-ST-DS}] \hfill \\
			uses disjoint set in the first stage and sparse table in the
			second stage.
		\item[{\sc Para-CoST-DS}] \hfill \\
			uses disjoint set in the first stage and compressed sparse
			table in the second stage.
		\item[{\sc Para-CoST}] \hfill \\
			uses compressed sparse table in both first and second stage.
	\end{description}
\end{frame}

\begin{frame}
	\begin{center}
		\scalebox{0.8} { \begin{minipage}{1.3\textwidth}
			\begin{figure}
			  \centering
			  \subfigure[Runtime]{
			    \includegraphics[width=0.45\linewidth]{\GraphicPath/fig-parallel-n.pdf}
			    \label{fig:fig-parallel}
			  }
			  \subfigure[Scalability]{
			    \includegraphics[width=0.45\linewidth]{\GraphicPath/fig-parallel-p.pdf}
			    \label{fig:fig-parallel-scala}
			  }
			  \caption{The execution time and scalability results of our parallel
			    implementations on an E5-2620 server with 6 cores and
			    hyper-threading}
			\end{figure}
			\end{minipage}
		}
	\end{center}
\end{frame}

\subsection{Incremental Suffix Maximum Query}
\begin{frame}
    \frametitle{Incremental Suffix Maximum Query}
    Compare the performance of {\em four} data structures for supporting
	incremental suffix maximum query only.

    \begin{description}
    	\setlength\itemsep{1em}
    	\item[{\sc Disjoint Set}] \hfill \\
    		described earlier
    	\item[{\sc Sparse Table}] \hfill \\
    		described earlier
    	\item[{\sc Encoding Sparse Table}] \hfill \\
    		It has an amortized insertion time $O(1)$ and query time
			$O(s)$, where $s$ is the size of the block and is set to
			$16$ in the experiments.
    	\item[{\sc LCA Blocked Sparse Table}]  \hfill \\
    		It has $O(1)$ for both amortized insertion time and the
			query time.
    \end{description}
\end{frame}

\begin{frame}
	When $n$ is greater than $10^7$, the {\sc Encoding Sparse Table}
	runs $1.8$ times faster than the {\sc Disjoint Set}.
	\begin{center}
	\scalebox{0.7} { \begin{minipage}{1.3\textwidth}
			\begin{figure}
			  \centering
			  \includegraphics[width=0.85\linewidth]{\GraphicPath/fig-ISMQ.pdf}
			  \caption{The performance of the different data structures for
			    supporting incremental suffix maximum query on an E5-2620 server}
			  \label{fig:fig-ISMQcmp}
			\end{figure}
			\end{minipage}
		}
	\end{center}
\end{frame}

\begin{frame}
	\frametitle{More Complex Scenario}
	\begin{description}
		\item[$p$]
			the probability of inserting a larger next element,
		\item[$q$]
			the probability of inserting a zero, and
		\item[$L$]
			the maximum interval being queried.
	\end{description}
	\begin{center}
	\scalebox{0.65} { \begin{minipage}{1.53\textwidth}
			\begin{table}[htbp]
  \caption{The timing (in seconds) of answering incremental suffix
    maximum query using rightmost-pops sparse table and the theoretically
    better LCA table sparse table (in bold
    font).} \label{tlb:ISMQcmp} \tiny
  \begin{tabular}{|r|rrrrr|rrrrr|rrrrr|r|} 
    \hline
      & \multicolumn{5}{c|}{$L = 4$} & \multicolumn{5}{c|}{$L=8$} & \multicolumn{5}{c|}{$L=16$} &  \\ 
      \hline 
      \diagbox{$q$}{$p$} & 0\% & 25\% & 50\% & 75\% & 100\% & 0\% & 25\% & 50\% & 75\% & 100\% & 0\% & 25\% & 50\% & 75\% & 100\% & speedup\\
      \hline
      $0\%$ &
            {\bf 1.15} & {\bf 0.89} & {\bf 0.86} & {\bf 0.88} & {\bf 0.91}
          & {\bf 0.88} & {\bf 0.87} & {\bf 0.87} & {\bf 0.85} & {\bf 0.87}   
          & {\bf 1.02} & {\bf 1.00} & {\bf 0.99} & {\bf 1.00} & {\bf 1.02} & 1.56 \\
        & 1.30 & 1.05 & 1.05 & 1.05 & 1.05   & 1.32 & 1.32 & 1.32 & 1.32 & 1.32   & 1.35 & 1.34 & 1.34 & 1.34 & 1.34 & \\ \hline
      $20\%$ & 
            {\bf 0.98} & {\bf 0.95} & {\bf 0.98} & {\bf 0.99} & {\bf 0.96}   
          & {\bf 1.16} & {\bf 1.16} & {\bf 1.19} & {\bf 1.19} & {\bf 1.18}   
          & {\bf 1.24} & {\bf 1.28} & {\bf 1.31} & {\bf 1.25} & {\bf 1.21} & 1.26 \\
        & 1.09 & 1.09 & 1.09 & 1.09 & 1.09   & 1.40 & 1.40 & 1.40 & 1.40 & 1.40   & 1.53 & 1.53 & 1.53 & 1.53 & 1.53 & \\ \hline
      $40\%$ & 
            {\bf 1.01} & {\bf 1.01} & {\bf 1.02} & {\bf 1.02} & {\bf 0.99}   
          & {\bf 1.23} & {\bf 1.24} & {\bf 1.25} & {\bf 1.24} & {\bf 1.21}   
          & {\bf 1.39} & {\bf 1.43} & {\bf 1.45} & {\bf 1.31} & {\bf 1.26} & 1.28 \\
        & 1.13 & 1.13 & 1.13 & 1.13 & 1.12   & 1.46 & 1.46 & 1.47 & 1.47 & 1.45   & 1.62 & 1.62 & 1.62 & 1.62 & 1.61 & \\ \hline
      $60\%$ & 
            {\bf 1.01} & {\bf 1.02} & {\bf 1.04} & {\bf 1.02} & {\bf 0.99}   
          & {\bf 1.23} & {\bf 1.25} & {\bf 1.27} & {\bf 1.26} & {\bf 1.20}   
          & {\bf 1.44} & {\bf 1.48} & {\bf 1.51} & {\bf 1.34} & {\bf 1.26} & 1.28 \\
        & 1.13 & 1.14 & 1.15 & 1.14 & 1.12   & 1.47 & 1.48 & 1.50 & 1.49 & 1.45   & 1.63 & 1.64 & 1.66 & 1.65 & 1.61 & \\ \hline
      $80\%$ & 
            {\bf 0.99} & {\bf 1.01} & {\bf 1.03} & {\bf 1.01} & {\bf 0.97}   
          & {\bf 1.20} & {\bf 1.22} & {\bf 1.25} & {\bf 1.23} & {\bf 1.15}   
          & {\bf 1.38} & {\bf 1.43} & {\bf 1.49} & {\bf 1.31} & {\bf 1.19} & 1.31 \\
        & 1.11 & 1.12 & 1.15 & 1.12 & 1.10   & 1.44 & 1.46 & 1.49 & 1.47 & 1.41   & 1.59 & 1.61 & 1.65 & 1.63 & 1.56 & \\ \hline
      $100\%$ &
            {\bf 0.94} & {\bf 0.96} & {\bf 1.00} & {\bf 0.97} & {\bf 0.91}   
          & {\bf 0.96} & {\bf 1.01} & {\bf 1.01} & {\bf 0.99} & {\bf 1.03}   
          & {\bf 1.09} & {\bf 1.12} & {\bf 1.16} & {\bf 1.34} & {\bf 1.20} & 1.39 \\
        & 1.04 & 1.06 & 1.10 & 1.07 & 1.03   & 1.34 & 1.36 & 1.39 & 1.36 & 1.33   & 1.39 & 1.41 & 1.44 & 1.41 & 1.39 & \\ \hline
  \end{tabular}
\end{table}

			\end{minipage}
		}
	\end{center}
	{\sc LCA Blocked Sparse Table} runs up to $1.5$ times faster than
	{\sc Encoding Sparse Table}.
\end{frame}

\subsection{Parallel Range Maximum Query}
\begin{frame}
    \frametitle{Parallel Range Maximum Query}
    Compare the performance of {\em two} data structures for supporting
	parallel range maximum query.

	\begin{itemize}
		\item 
			{\em Sparse table} ({\sc ST})
		\item 
			{\em Compressed sparse table} ({\sc CoST})
	\end{itemize}
	\vspace{1em}
	The experiments test the cases of all possible query range sizes,
	and build only the {\em necessary} blocks before answering the
	queries.
\end{frame}

\begin{frame}
	Compressed sparse table is $1.4$ times faster than sparse table when
	$N$ reaches $100000$.
	\begin{center}
	\scalebox{0.65} { \begin{minipage}{1.53\textwidth}
			\begin{table*}[!thb]
  %\tiny
  \caption{Total running time (ms) for finding RMQ of different sizes $N$ and maximum interval sizes $L$.}
  \label{tlb:CORMQ}
  \centering
  \begin{tabular}{l c c c c}
    \firsthline
      & \multicolumn{4}{c}{$N$} \\
      \cline{2-5}
        & \multicolumn{2}{c}{$30000$} & $50000$ & $100000$ \\
      $L$ & $2^{10}$ & $2^{15}$ & $2^{15}$ & $2^{15}$ \\
      \hline
      parallel-\tt{RMQ}     & $903$ & $1516$ & $1874$ & $4116$ \\
      parallel-\tt{CORMQ}   & $995$ & $1475$ & $1689$ & $2594$ \\
      parallel-\tt{CORMQ-opt} & $843$ & $1373$ & $1136$ & $1745$ \\
      \hline
      Speedup & $1.07\times$ & $1.10\times$ & $1.64\times$ & $2.35\times$\\
    \lasthline
  \end{tabular}
\end{table*}
			\end{minipage}
		}
	\end{center}
\end{frame}