\section{Range Maximum Query}

\subsection{Range Maximum Query}
\begin{frame}
    \frametitle{Range Maximum Query}
    \begin{itemize}
    	\setlength\itemsep{1em}
    	\item 
    		The suffix maximum query is a special case of the range
			maximum query.
		\item
			In this section, we will describe our approach to address
			the challenges in the {\em second} stage of
			Algorithm~\ref{alg:parallel-VGLCS}.
    \end{itemize}
\end{frame}

\subsection{Blocked Range Maximum Query}
\begin{frame}
    \frametitle{Blocked Range Maximum Query}
	%We improve the efficiency of our parallel VGLCS algorithm with a
	%{\em blocked} sparse table proposed by Fischer.
	Fischer proposed {\em blocked} sparse table as follows:
	\begin{enumerate}
		\setlength\itemsep{1em}
		\item
			Partitions the data into blocks of size $s$, then
		\item
			Computes the maximum of each block, then
		\item
			Compute a sparse table $T_s$ for these maximums.
	\end{enumerate}
	\vspace{1em}
	It answers a range maximum query as two types of queries 
	\begin{itemize}
		\setlength\itemsep{1em}
		\item
			{\em super block} query
		\item
			{\em in-block} query.
	\end{itemize}
\end{frame}

\begin{withoutheadline}
\begin{frame}
    \frametitle{An Super Block Query Example}
	\begin{figure}[!thb]
		\centering \subfigure[Array] {
	    	\includegraphics[width=0.85\linewidth]{\GraphicPath/fig-interval-decomposition.pdf}
	  	} \subfigure[Blocked sparse table] {
	    	\includegraphics[width=0.4\linewidth]{\GraphicPath/fig-sparse-table.pdf}
	  	}
	  	\caption{A Sparse Table}
	  	\label{fig:block-interval-decomposition}
	\end{figure}
\end{frame}
\end{withoutheadline}

\begin{frame}
	\frametitle{Answer In-Block Query with Least Common Ancestor}
	\begin{itemize}
		\setlength\itemsep{1em}
		\item
			Fischer's algorithm scans through the data within a block
			and places them into a {\em Cartesian tree}.
		\item
			Cartesian tree as a {\em heap} where the data are in heap
			order, and the indexes of the data are in {\em sorted binary
			search tree order}.
		\item 
			As a result, the answer to an in-block range maximum query
			from the $i$-th to the $j$-th element of a block is located
			at their {\em least common ancestor} in the Cartesian tree.
	\end{itemize}
\end{frame}

\begin{frame}
	\frametitle{An In-Block Query Example}
	\begin{figure}[htbp]   
	  \centering
	  \includegraphics[width=0.9\linewidth]{\GraphicPath/fig-interval-cartesian.pdf}
	  \caption{Least common ancestor tables}
	  \label{fig:ancesstor-cartesian}
	\end{figure}
\end{frame}

\begin{frame}
	\frametitle{Fischer's Build Least Common Ancestor Table}
	Fischer's algorithm chooses block size $s=\frac{\log n}{4}$.
	\begin{itemize}
		\setlength\itemsep{1em}
		\item 
			Catalan number $C_s$ is the number of different Cartesian
			trees, and also is the number of different binary trees of
			$s$ nodes.
		\item
			Space requirement is
			$O(s^2C_s)=O(s^2\frac{1}{s+1}\binom{2s}{s})=O(n)$
		\item
			Construct LCA tables for blocks {\em on-demand}.
	\end{itemize}
\end{frame}

\begin{withoutheadline}
\begin{frame}
	\frametitle{Illustration for Constructing On-demand}
	\begin{figure}[htbp]   
	  \centering
	  \includegraphics[width=0.9\linewidth]{\GraphicPath/fig-fischer-ondemand.pdf}
	  \caption{Construct LCA table for blocks on-demand}
	  \label{fig:ancesstor-cartesian}
	\end{figure}
\end{frame}
\end{withoutheadline}

\begin{frame}
	\frametitle{The Least Common Ancestor Table in Parallel VGLCS}
	There are two performance issues:
	\begin{itemize}
		\setlength\itemsep{1em}
		\item 
			Building process repeats for as many times as the number of
			blocks, and may cause serious cache misses.
		\item
			The LCA table will be cached and this may {\em evict} other
			LCA tables from cache.
	\end{itemize}
\end{frame}

\begin{frame}
	\frametitle{The Static Least Common Ancestor Table}
	Demaine's algorithm does not check if an LCA table is in memory --
	instead it builds LCA table for {\em every possible} block.
	\begin{itemize}
		\setlength\itemsep{1em}
		\item 
			Use a binary string of length $2s$ to identify a block and
			its Cartesian tree.
		\item
			The binary string is encoded in such a way that one can
			answer an in-block query by examining this binary string.
	\end{itemize}
\end{frame}

\begin{frame}
	There are two issues on examining this binary string:
	\begin{itemize}
		\setlength\itemsep{1em}
		\item 
			This requires counting the number of 1's {\em between} the
			last two 0's, which is {\em hard} to implement efficiently
			in a modern computer.
		\item
			Converting binary string to LCA table must be processed
			sequentially, and also is the overhead of RMQ.
	\end{itemize}
\end{frame}

\subsection{Rightmost-pops Encoding}
\begin{frame}
    \frametitle{Rightmost-pops Encoding}
    We propose a new encoding for blocks, called {\em rightmost-pops},
	instead of the binary string by Demaine, in order to improve the
	performance of range maximum query.
	\\~\\
	When we add the $i$-th data $a_i$ into the Cartesian tree, 
	\begin{itemize}
		\setlength\itemsep{1em}
		\item
			Keep popping data until the top of stack is no less than
			$a_i$.
		\item 
			$t_i$ be the number of nodes that need to be popped.
		\item
			$0 \le t_i < s$, where $s$ is the block size.
	\end{itemize}

\end{frame}

\begin{frame}
	\frametitle{An Example for Rightmost-pops Encoding}
	\begin{figure}[!thb]
	  \centering
	  \includegraphics[width=\linewidth]{\GraphicPath/fig-cartesian-encoding-stack.pdf}
	  \caption{Rightmost path in stack}
	  \label{fig:interval-cartesian}
	\end{figure}
\end{frame}

\begin{frame}
	\frametitle{Implementation of Rightmost-pops Encoding}
	We choose the block size $s = {{\frac{\log n}{4}}} = 16$.

	\begin{itemize}	
		\setlength\itemsep{1em}
		\item
			$t_i$'s are small than 16.
		\item
			Represent each of $t_i$ as a 4-bit integer.
		\item
			Concatenate sixteen of these 4-bit integers into a 64-bit
			integer to present a Cartesian tree for a block.
	\end{itemize}
	\vspace{1em}
	It can answer range maximum query for up to $n = 2^{64}$ data.  The
	time complexity of our parallel VGLCS algorithm becomes $O(n^2
	\log{n} / p + n \log n)$.
\end{frame}

\begin{frame}
	\frametitle{Pseudocode for Rightmost-pops Encoding}
	\begin{center}
		\scalebox{1} { \begin{minipage}{1\textwidth}
			\begin{algorithm}[H]
\SetAlgoNoLine
\LinesNumbered
\KwIn{$\textit{tmask}$: 64-bits Cartesian tree; $[l, r]$: query range}
\KwOut{$\textit{minIdx}$: the index of the minimum value in interval}
    
$\textit{minIdx} \gets l, \; x \gets 0$ \;
\For{$l \gets l+1$ to $r$} {
  $x \gets x+1 - ((\textit{tmask} \gg (l \ll 2)) \mathrel{\&} 15)$ \;
  \If{$x \le 0$} {
    $\textit{minIdx} \gets l$, $x \gets 0$ \;
  }
}
return $\textit{minIdx}$ \;

  \caption{Range Minimum Query in 64-bits Cartesian Tree}
  \label{alg:cartesian64bits-query}
\end{algorithm}
			\end{minipage}
		}
	\end{center}
\end{frame}