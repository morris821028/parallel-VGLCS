\documentclass{beamer}
\mode<presentation>
\usetheme{Warsaw}
\usefonttheme{professionalfonts}
\usefonttheme{serif}
%\usecolortheme{dove}

% Add frame number and total frame number in footline
%\setbeamertemplate{headline}{}
\defbeamertemplate*{footline}{shadow theme}{%
    \leavevmode%
    \hbox{%
        \begin{beamercolorbox}[wd=.87\paperwidth,ht=2.25ex,dp=1ex,center]{title in head/foot}%
            \usebeamerfont{title in head/foot}\insertshortinstitute
        \end{beamercolorbox}%
        \begin{beamercolorbox}[wd=.13\paperwidth,ht=2.25ex,dp=1ex,right]{date in head/foot}%
            \usebeamerfont{date in head/foot}{}\hspace*{2em}
            \insertframenumber{} / \inserttotalframenumber\hspace*{2ex}
    \end{beamercolorbox}}%
    \vskip0pt%
}

\makeatletter
\setbeamertemplate{headline}
{%
  \leavevmode%
  \@tempdimb=2.4375ex%
  \ifnum\beamer@subsectionmax<\beamer@sectionmax%
    \multiply\@tempdimb by\beamer@sectionmax%
  \else%
    \multiply\@tempdimb by\beamer@subsectionmax%
  \fi%
  \ifdim\@tempdimb>0pt%
    \advance\@tempdimb by 1.825ex%
    \begin{beamercolorbox}[wd=.4\paperwidth,ht=\@tempdimb]{section in head/foot}%
      \vbox to\@tempdimb{\vfil\insertsectionnavigation{.4\paperwidth}\vfil}%
    \end{beamercolorbox}%
    \begin{beamercolorbox}[wd=.6\paperwidth,ht=\@tempdimb]{subsection in head/foot}%
      \vbox to\@tempdimb{\vfil\insertsubsectionnavigation{.3\paperwidth}\vfil}%
    \end{beamercolorbox}%
  \fi%
}
% define the page without headline 
\makeatletter
    \newenvironment{withoutheadline}{
        \setbeamertemplate{headline}[default]
        \def\beamer@entrycode{\vspace*{-\headheight}}
    }{}
\makeatother

% Tikz related
\usepackage{tikz}
\usetikzlibrary{calc}
\usetikzlibrary{positioning}
\usetikzlibrary{shapes,arrows}
\usetikzlibrary{plotmarks}
\usepackage{subfigure}

% Use pfgplots to generate plots and labeled axes
\usepackage{pgfplots}

% Number the figures
\setbeamertemplate{caption}[numbered]

% Tables
\usepackage{array} 
\usepackage[ruled]{../patch/algorithm2e}[2015/10/18] % For algorithms
\renewcommand{\algorithmcfname}{ALGORITHM}
\SetAlFnt{\small}
\SetAlCapFnt{\small}
\SetAlCapNameFnt{\small}
\SetAlCapHSkip{0pt}
\IncMargin{-\parindent}
\newcommand\mycommentfont[1]{\footnotesize\itshape\textcolor{blue}{#1}}
\SetCommentSty{mycommentfont}


% Add outline page at begining of each section
\usepackage{multicol}
\AtBeginSection[]
{
    \begin{frame}<beamer>
        \frametitle{Outline}
        \begin{multicols}{2}
            \tableofcontents[currentsection,hideothersubsections]
        \end{multicols}
    \end{frame}
}

% Mathematics
\usepackage{amsmath}
\usepackage{mathtools}
\usepackage[nice]{nicefrac}

% Appendix
\usepackage{appendixnumberbeamer}

% Table
\usepackage{array}
\usepackage{diagbox}[2011/11/22]

% Code Block Setting
\usepackage{listings}
\lstset{language=C,
numberstyle=\footnotesize,
basicstyle=\ttfamily\footnotesize,
numbers=left,
stepnumber=1,
frame=shadowbox,
breaklines=true}

% ========= Variable ===================================================
\title{Parallel \\
Variable Gapped Longest Common Sequence \\ and \\ 
Incremental Range Maximum Query}
\author{Shiang-Yun Yang}
\institute{Department of Computer Science \& Information Engineering,\\
National Taiwan University}
%\date{Master's Thesis Defense\\\today}
\date{Master's Thesis Defense, June 5, 2017}

% ========= Variable ===================================================

\newcommand*{\PartialPath}{../partial}
\newcommand*{\GraphicPath}{../graphics}
\newcommand*{\AlgoPath}{./algorithms}
\newcommand*{\FormulaPath}{../formulas}
\newcommand*{\CodePath}{../codes}
\newcommand*{\TablePath}{../tables}

%=====================================================================


\begin{document}
\begin{frame}
    \titlepage
\end{frame}

\begin{frame}<beamer>
    \frametitle{Outline}
    \setcounter{tocdepth}{1}
    \tableofcontents
\end{frame}

\section{介紹} %Introduction
\label{sec:Introduction}

最長共同子序列 (\emph{longest common subsequence}, LCS)廣泛地使用在各個應用上。在多核心平台下,大多數的研究專注於如何高效率地在波前平行 (wavefront parallelism),而 Jiaoyun Yang ~\cite{jiaoyun} 提出的論文中改變一般的 LCS 遞迴定義以得到更好快取使用率。這裡,我們使用相關的理論來改善在 Iliopoulos ~\cite{iliopoulos} 提及的約束條件下的 LCS,如 \emph{fixed gap LCS } (FGLCS)要求任兩個挑選的距離在相對應的另一個字串中相等,同時距離最大為 $k+1$,可在時間複雜度在 $O(nm)$ 內解決,其中 $n$, $m$ 分別為兩個輸入的字串長度。

The \emph{longest common subsequence} (LCS) problem applied many products and fields widely.  In multi-core platform, most studies focus on the wavefront parallelism. Motivated by the definition of recursion in LCS, Jiaoyun Yang introduced a new formula to exploit more cache performance.  Here, we use the similar idea to improve LCS with variable contraints, which refer in Iliopoulos' study.  For example, \emph{fixed gap LCS} (FGLCS) require the distance between two consecutive matches limited at most $k+1$.  It can be solved in $O(nm)$, which $n, \; m$ is the length of input strings.

在眾多的約束條件類型中,我們將在這篇論文針對 \emph{variable gap LCS} (VGLCS) 進行探討。在 VGLCS 中,對各個不同的位置提供約束限制,如目前給定兩個字串 $A = \tt{RCLPCRR}$, $B = \tt{RPPLCPLRC}$,各自的約束限制為 $G_A = [2, 3, 0, 0, 3, 2, 2]$ 和 $G_B = [2, 0, 0, 0, 3, 0, 0, 2, 3]$,其中 $G_A(i)$ 表示當挑選第 $i$ 個位置時,與前一個挑選的位置最多差 $G_A(i)+1$,挑選的方式如圖 ~\ref{fig:VGLCSex}。這個問題已在 Yung-Hsing Peng ~\cite{yunghsing} 的論文針對 VGLCS 提出 $O(nm \alpha(n))$ 的解法。

這一篇論文,我們將在第二 \ref{sec:parallelSerial} 節部分將 Yung-Hsing Peng ~\cite{yunghsing} 提出的算法進行平行化。接著,在第三節 ~\ref{sec:parallelRMQ},在理論分析上提供易平行且時間複雜度 $O(nm)$ 的設計。次著,在第四節 ~\ref{sec:Implementation},我們將藉由快取忘卻 (cache-oblivious) 技術,在實作上提供更好的效能。最後,我們總結實驗結果與理論實務上的差異。

\begin{figure}[!thb]
  \centering
  \includegraphics[width=\linewidth]{graphics/fig-VGLCSex.pdf}
  \includegraphics[width=\linewidth]{graphics/fig-VGLCSex2.pdf}
  \caption{VGLCS 於兩個序列 $A = \tt{RCLPCRR}$, $B = \tt{RPPLCPLRC}$,各自的約束限制為 $G_A = [2, 3, 0, 0, 3, 2, 2]$ 和 $G_B = [2, 0, 0, 0, 3, 0, 0, 2, 3]$,的其中幾個可挑選的方案}
  \label{fig:VGLCSex}
\end{figure}
\section{平行化序列算法} %
\label{sec:parallelSerial}

在 $O(nm \alpha(n))$ 的序列算法中 (參照算法 ~\ref{alg:serial-VGLCS}),我們發現算法如大多數的變型 LCS 相同,依賴數個狀態以轉移當前狀態,大量的資料依賴性不易於細粒度平行。使用波前運行平行是一種常見的解決方案,由於這種平行對於運行時的快取不友善 (cache-unfriendly),所以在 Saeed Maleki ~\cite{saeed} 論文中提到如何使用 Rank Convergence 的特殊性質,拓展出更高平行度來解決動態規劃的相關問題。

\begin{algorithm*}[!thb]
  \caption{Algorithm for Finding VGLCS}
  \label{alg:serial-VGLCS}
  \begin{algorithmic}[1]
    \Require
      $A, B$: the input string;
      $G_A, G_B$: the array of variable gapped constraints;
    \Ensure Find the LCS with variable gapped constraints
    \State Create $m$ number of data structure $Q[m]$ to support ISMQ problem.
    \State Create an empty table $V[n][m]$.
    \For{$i \gets 1$ to $n$}
      \State Create a data structure $RQ$ to support ISMQ problem.
      \State $r \gets i - (GA[i]+1)$
      \For{$j \gets 1$ to $m$}
        \If{$A[i] = B[j]$}
            \State $t \gets $ query suffix maximum value from position $j - (GB[j]+1)$ to tail in $RQ$.
            \State $V[i][j] \gets t + 1$
            \State $t \gets $ get the suffix maximum value from position $r$ to $i$ in $Q[j]$
            \State Append value $t$ into $RQ$.
            \State Append value $V[i][j]$ into $Q[j]$.
        \Else
            \State $V[i][j] \gets 0$
            \State $t \gets $ get the suffix maximum value from position $r$ to $i$ in $Q[j]$
            \State Append value $t$ into $RQ$.
        \EndIf
      \EndFor
    \EndFor
    \State Retrieve the VGLCS by tracing $V[n][m]$
  \end{algorithmic}
\end{algorithm*}

序列算法的空間複雜度為 $O(nm)$。若使用波前平行,需要同時維護橫向的所有狀態,需要多付出一倍的空間量。若加入 Rank Convergence 的想法拓展出,勢必要記錄轉移的狀態,需要耗費更多的記憶體空間,用以在最後階段合併所用。

這裡我們傾向空間複雜度常數小且針對快取友善設計算法。平行算法主要分成兩個階段-縱向和橫向階段,縱向階段為數個列的後綴極值查找,橫向階段在行上運行 $n$ 個元素和 $n$ 組詢問。我們發現在橫向查找中使用 $O(\alpha(n))$ 操作的增長後綴最大值查找 (\emph{incremental suffix maximum query}, ISMQ),在過程中每插入一個元素便改動數據結構以支持下一個後綴詢問,這使得查詢難以平行化。為消除資料相依性,我們找到幾種區間詢問的替代方案。如:

\begin{itemize}
  \item 樹狀數組 (Binary Indexed Tree) -- $O(\log n)$: 對於任意前綴查找極值和更新元素,可以提供每次時間複雜度 $O(\log n)$,其運行常數比 Range Tree 低,只能支持前綴查找。若要運行區間查找,則必須在數學上符合加法原則。
  \item 線段樹 (Segment Tree) -- $O(\log n)$: 支持更高維度的正交區塊搜索,而我們用在區間極值查找需要 $O(\log n)$ 的時間完成所有區間查詢操作。
  \item 稀疏表 (Sparse Table) -- $O(n)$ -- $O(1)$:
    建立表格 $ST[j][i]$ 表示區間 $(i-2^j,i]$ 之間的極值。建表時間複雜度需 $O(n)$,對於任意區間詢問可以拆分 2 個 super-block 檢索和 2 個 in-block 檢索 (參照圖 ~\ref{fig:interval-decomposition} 的說明),轉換過程和存取時間皆需要 $O(1)$。
\end{itemize}

\begin{figure*}[!thb]
  \centering
  \includegraphics[width=\linewidth]{graphics/fig-interval-decomposition.pdf}
  \includegraphics[width=\linewidth]{graphics/fig-sparse-table.pdf}
  \caption{給定一陣列 $A$ 如上圖所述,並且拆成 5 個區塊,每個區塊皆有 4 個元素,若詢問區間 $[2, 18]$ 的最大值,將分成 $B1$ 的內部詢問 (in-block query) $Q_L$、$B5$ 的內部詢問 $Q_R$ 和兩個跨區間詢問 (super-block query) $SQ_L$、$SQ_R$}
  \label{fig:interval-decomposition}
\end{figure*}

稀疏表是我們認為最好的替代方案,其整合後為 VGLCS 平行算法 ~\ref{alg:parallel-VGLCS},算法的時間複雜度為 $O(n^2 \alpha(n) / p + n \log n)$,其中 $p$ 為處理器個數。在後續的章節,我們將提出新的數據結構取代并查集操作,使得序列算法的時間複雜度從 $O(n^2 \alpha(n))$ 降至 $O(n^2)$,且能在平行算法達到理想複雜度 $O(n^2 / p + n \log n)$。

\begin{algorithm*}[!thb]
  \caption{Parallel Algorithm for Finding VGLCS}
  \label{alg:parallel-VGLCS}
  \begin{algorithmic}[1]
    \Require
      $A, B$: the input string;
      $G_A, G_B$: the array of variable gapped constraints;
    \Ensure Find the LCS with variable gapped constraints
    \State Create $m$ number of data structure $Q[m]$ to support ISMQ problem.
    \State Create an empty table $V[n][m]$.
    \For{$i \gets 1$ to $n$}
      \State Create a sparse table data structure $\textit{sp}$, and initialize $\textit{sp}$ to zero.
      \ParFor{$j \gets 1$ to $m$}
        \State $\textit{sp}[j] \gets$ query suffix maximum value from position $r$ to tail in $Q[j]$.
      \EndParFor
      \State Build sparse table $\textit{sp}$ with $m$ elements in parallel $O(n/p \log n + \log n)$ time.
      \ParFor{$j \gets 1$ to $m$}
        \If{$A[i] = B[j]$}
            \State $t \gets $ query suffix maximum value from position $j - (GB[j] + 1)$ to $j-1$ in $\textit{sp}$
            \State $V[i][j] \gets t + 1$
            \State Append value $V[i][j]$ into $Q[j]$.
        \EndIf
      \EndParFor
    \EndFor
    \State Retrieve the VGLCS by tracing $V[n][m]$
  \end{algorithmic}
\end{algorithm*}
\section{Range Maximum Query}

\subsection{Range Maximum Query}
\begin{frame}
    \frametitle{Range Maximum Query}
    \begin{itemize}
    	\setlength\itemsep{1em}
    	\item 
    		The suffix maximum query is a special case of the range
			maximum query.
		\item
			In this section, we will describe our approach to address
			the challenges in the {\em second} stage of
			Algorithm~\ref{alg:parallel-VGLCS}.
    \end{itemize}
\end{frame}

\subsection{Blocked Range Maximum Query}
\begin{frame}
    \frametitle{Blocked Range Maximum Query}
	%We improve the efficiency of our parallel VGLCS algorithm with a
	%{\em blocked} sparse table proposed by Fischer.
	Fischer proposed {\em blocked} sparse table as follows:
	\begin{enumerate}
		\setlength\itemsep{1em}
		\item
			Partitions the data into blocks of size $s$, then
		\item
			Computes the maximum of each block, then
		\item
			Compute a sparse table $T_s$ for these maximums.
	\end{enumerate}
	\vspace{1em}
	It answers a range maximum query as two types of queries 
	\begin{itemize}
		\setlength\itemsep{1em}
		\item
			{\em super block} query
		\item
			{\em in-block} query.
	\end{itemize}
\end{frame}

\begin{withoutheadline}
\begin{frame}
    \frametitle{An Super Block Query Example}
	\begin{figure}[!thb]
		\centering \subfigure[Array] {
	    	\includegraphics[width=0.85\linewidth]{\GraphicPath/fig-interval-decomposition.pdf}
	  	} \subfigure[Blocked sparse table] {
	    	\includegraphics[width=0.4\linewidth]{\GraphicPath/fig-sparse-table.pdf}
	  	}
	  	\caption{A Sparse Table}
	  	\label{fig:block-interval-decomposition}
	\end{figure}
\end{frame}
\end{withoutheadline}

\begin{frame}
	\frametitle{Answer In-Block Query with Least Common Ancestor}
	\begin{itemize}
		\setlength\itemsep{1em}
		\item
			Fischer's algorithm scans through the data within a block
			and places them into a {\em Cartesian tree}.
		\item
			Cartesian tree as a {\em heap} where the data are in heap
			order, and the indexes of the data are in {\em sorted binary
			search tree order}.
		\item 
			As a result, the answer to an in-block range maximum query
			from the $i$-th to the $j$-th element of a block is located
			at their {\em least common ancestor} in the Cartesian tree.
	\end{itemize}
\end{frame}

\begin{withoutheadline}
\begin{frame}
	\frametitle{An In-Block Query Example}
	\begin{figure}[htbp]   
	  \centering
	  \includegraphics[width=\linewidth]{\GraphicPath/fig-interval-cartesian.pdf}
	  \caption{Least common ancestor tables}
	  \label{fig:ancesstor-cartesian}
	\end{figure}
\end{frame}
\end{withoutheadline}

\begin{frame}
	\frametitle{Fischer's Build Least Common Ancestor Table}
	Fischer's algorithm chooses block size $s=\frac{\log n}{4}$.
	\begin{itemize}
		\setlength\itemsep{1em}
		\item 
			Catalan number ${\cal C}_s$ is the number of different Cartesian
			trees, and also is the number of different binary trees of
			$s$ nodes.
		\item
			Space requirement is
			$O(s^2{\cal C}_s)=O(s^2\frac{1}{s+1}\binom{2s}{s})=O(n)$
		\item
			Construct LCA tables for blocks {\em on-demand}.
	\end{itemize}
\end{frame}

\begin{withoutheadline}
\begin{frame}
	\frametitle{Illustration for Constructing On-demand}
	\begin{figure}[htbp]   
	  \centering
	  \includegraphics[width=0.9\linewidth]{\GraphicPath/fig-fischer-ondemand.pdf}
	  \caption{Construct LCA table for blocks on-demand}
	\end{figure}
\end{frame}
\end{withoutheadline}

\begin{frame}
	\frametitle{The Least Common Ancestor Table in Parallel VGLCS}
	There are two performance issues:
	\begin{itemize}
		\setlength\itemsep{1em}
		\item 
			Building process repeats for as many times as the number of
			blocks, and may cause serious cache misses.
		\item
			The LCA table will be cached and this may {\em evict} other
			LCA tables from cache.
	\end{itemize}
\end{frame}

\begin{frame}
	\frametitle{The Static Least Common Ancestor Table}
	Demaine's algorithm does not check if an LCA table is in memory --
	instead it builds LCA table for {\em every possible} block.
	\begin{itemize}
		\setlength\itemsep{1em}
		\item 
			Use a binary string of length $2s$ to identify a block and
			its Cartesian tree.
		\item
			The binary string is encoded in such a way that one can
			answer an in-block query by examining this binary string.
	\end{itemize}
\end{frame}

\begin{frame}
	There are two issues on examining this binary string:
	\begin{itemize}
		\setlength\itemsep{1em}
		\item 
			This requires counting the number of 1's {\em between} the
			last two 0's, which is {\em hard} to implement efficiently
			in a modern computer.
		\item
			Converting binary string to LCA table must be processed
			sequentially, and also is the overhead of RMQ.
	\end{itemize}
\end{frame}

\subsection{Rightmost-pops Encoding}
\begin{frame}
    \frametitle{Rightmost-pops Encoding}
    We propose a new encoding for blocks, called {\em rightmost-pops},
	instead of the binary string by Demaine, in order to improve the
	performance of range maximum query.
	\\~\\
	When we add the $i$-th data $a_i$ into the Cartesian tree, 
	\begin{itemize}
		\setlength\itemsep{1em}
		\item
			Keep popping data until the top of stack is no less than
			$a_i$.
		\item 
			$t_i$ be the number of nodes that need to be popped.
		\item
			$0 \le t_i < s$, where $s$ is the block size.
	\end{itemize}

\end{frame}

\begin{withoutheadline}
\begin{frame}
	\frametitle{A Rightmost-pops Encoding Example}
	\begin{figure}[!thb]
	  \centering
	  \includegraphics[width=\linewidth]{\GraphicPath/fig-cartesian-encoding-stack.pdf}
	  \caption{Rightmost path in stack}
	  \label{fig:interval-cartesian}
	\end{figure}
\end{frame}
\end{withoutheadline}

\begin{frame}
	\frametitle{Implementation of Rightmost-pops Encoding}
	We choose the block size $s = {{\frac{\log n}{4}}} = 16$.

	\begin{itemize}	
		\setlength\itemsep{1em}
		\item
			All $t_i$'s are small than 16.
		\item
			Represent each of $t_i$ as a 4-bit integer.
		\item
			Concatenate sixteen of these 4-bit integers into a 64-bit
			integer to present a Cartesian tree for a block.
	\end{itemize}
	\vspace{1em}
	It can answer range maximum query for up to $n = 2^{64}$ data.  The
	time complexity of our parallel VGLCS algorithm becomes $O(n^2
	\log{n} / p + n \log n)$.
\end{frame}

\begin{frame}
	\frametitle{Pseudocode for Rightmost-pops Encoding}
	\begin{center}
		\scalebox{1} { \begin{minipage}{1\textwidth}
			\begin{algorithm}[H]
\SetAlgoNoLine
\LinesNumbered
\KwIn{$\textit{tmask}$: 64-bits Cartesian tree; $[l, r]$: query range}
\KwOut{$\textit{minIdx}$: the index of the minimum value in interval}
    
$\textit{minIdx} \gets l, \; x \gets 0$ \;
\For{$l \gets l+1$ to $r$} {
  $x \gets x+1 - ((\textit{tmask} \gg (l \ll 2)) \mathrel{\&} 15)$ \;
  \If{$x \le 0$} {
    $\textit{minIdx} \gets l$, $x \gets 0$ \;
  }
}
return $\textit{minIdx}$ \;

  \caption{Range Minimum Query in 64-bits Cartesian Tree}
  \label{alg:cartesian64bits-query}
\end{algorithm}
			\end{minipage}
		}
	\end{center}
\end{frame}
\section{RMQ on Incremental Data}

\subsection{Range Maximum Query on Incremental Data}
\begin{frame}
    \frametitle{Range Maximum Query on Incremental Data}
    \begin{itemize}
    	\setlength\itemsep{1em}
    	\item
    		Answer incremental {\em range} maximum queries on
			incrementally added data.
    	\item
    		This section describes our approach to address the
			challenges in the {\em first} stage of
			Algorithm~\ref{alg:parallel-VGLCS}.
    \end{itemize}
\end{frame}

\subsection{Parallel Building Least Common Ancestor Table}
\begin{frame}
    \frametitle{Parallel Building Least Common Ancestor Table}
    We need to address two issues in parallel environment:
    \begin{itemize}
    	\setlength\itemsep{1em}
    	\item 
    		How to label a binary tree into its {\em Catalan index}
		\item 
			How to find the least common ancestor of two nodes in a
			given Cartesian tree
	\end{itemize}
\end{frame}

\subsection{Catalan Tree Labeling}
\begin{frame}
    \frametitle{Catalan Tree Labeling}
    Our Cartesian tree labeling will enumerate all binary search trees
	in {\em lexicographical order} from $0$ to the ${\cal C}_n-1$.
	\\~\\
	A binary tree $x$ appears {\em before} another binary tree $y$ if
	any of the following condition is true.
	\begin{itemize}
		\setlength\itemsep{0.5em}
		\item 
			$x$ has more nodes than $y$ in the left subtree.
		\item 
			$x$ and $y$ have the same number of nodes in the left
  			subtree, and $x$'s left subtree appears before $y$'s left
  			subtree in lexicographical order.
		\item 
			$x$ and $y$ has the same left subtree, and $x$'s right
	  		subtree appears before $y$'s right subtree in
	  		lexicographical order.
	\end{itemize}
\end{frame}

\begin{frame}
	\frametitle{An Example for Catalan Tree Labeling}
	\begin{figure}[!thb]
	  \centering
	  \includegraphics[width=\linewidth]{\GraphicPath/fig-bst-encoding.pdf}
	  \caption{The labeling of binary search trees of sizes 1, 2, and 3.}
	  \label{fig:labelingBST}
	\end{figure}
\end{frame}

\subsection{Least Common Ancestor}
\begin{frame}
	\frametitle{Recursive Formula of Least Common Ancestor}
	Let ${\cal A}(s, t, p, q)$ denote the least common ancestor of the
	node $p$ and $q$ within a binary search tree of size $s$ that has a
	Catalan index $t$.
	\begin{center}
		\scalebox{.7} { \begin{minipage}{1.4\textwidth}
			\begin{equation*}
  \begin{split}
    &\mathit{LCA}(n, \mathit{tid}, p, q) \\
      &= \left\{\begin{matrix*}[l]
        \mathit{LCA}(\mathit{lsz}, \mathit{lid}, p, q) &&, p \le q < \mathit{lsz}\\ 
        \mathit{LCA}(\mathit{rsz}, \mathit{rid}, p-\mathit{lsz}-1, q-\mathit{lsz}-1)+\mathit{lsz}+1 &&, 
            \mathit{lsz} \le p \le q < n \\ 
        \mathit{lsz} && , 0 \le p \le \mathit{lsz}, \mathit{lsz} \le q \le i\\ 
        -1 && ,\mathit{otherwise}
      \end{matrix*}\right.
  \end{split}
\end{equation*}
			\end{minipage}
		}
	\end{center}
\end{frame}

\begin{withoutheadline}
\begin{frame}
	\frametitle{Illustration for Recursive Formula of LCA}
	\begin{center}
		\scalebox{0.8} { \begin{minipage}{1\textwidth}
			\begin{figure}[!thb]
			  \centering \subfigure[]{
			    \includegraphics[width=0.6\linewidth]{\GraphicPath/fig-bst-LCA-decomposition.pdf}
			  } \subfigure[]{
			    \includegraphics[width=0.6\linewidth]{\GraphicPath/fig-bst-LCA-decomposition-rec.pdf}
			  }
			  \caption{Recursive Formula of LCA}
			\end{figure}
			\end{minipage}
		}
	\end{center}
\end{frame}
\end{withoutheadline}

\begin{frame}
	\frametitle{Parallel Building LCA Algorithm}
	The time complexity of Algorithm~\ref{alg:parallel-LCA} is
	$O(\frac{s^3}{s+1} \binom{2s}{s} / p + s^2)$.
	\\~\\
	We choose the block size $s = {{\frac{\log n}{4}}}$, so time
	complexity is $O(\sqrt{n} \; (\log^{\nicefrac{3}{2}} n) / p + \log^2 n )$, where
	$p$ is the number of processors.
	\begin{center}
		\scalebox{.7} { \begin{minipage}{1.4\textwidth}
			\begin{algorithm}[!thb]
\SetAlgoNoLine
\KwIn{$s$: The maximum tree size}

\For{$n \gets 1$ to $s$} {
  \ForPar{$t \gets 0$ to $C_n - 1$} {
    \ForPar{$p \gets 0$ to $n-1$} {
      Compute $s_l$, $t_l$, $s_r$, and $t_r$ \;
      \For{$q \gets p$ to $n-1$} {
        Compute ${\cal A}[n][t][p][q]$ according to Equation~\ref{fun:LCA1} and \ref{fun:LCA2} \;
      }
    }
  }
}

\caption{A parallel algorithm that computes the least common ancesstor table}
\label{alg:parallel-LCA}
\end{algorithm}

			\end{minipage}
		}
	\end{center}
\end{frame}

\subsection{Catalan Index Computation}

\begin{frame}
	\frametitle{Catalan Index Computation}
	Determine Catalan index $t$ efficiently when given a block of data.
	There are two possible approaches

	\begin{itemize}
		\setlength\itemsep{1em}
		\item Build the tree
		\item Keep the rightmost path
	\end{itemize}
\end{frame}

\subsubsection{Build the Tree}
\begin{frame}
    \frametitle{Build the Tree}
    \begin{enumerate}
    	\setlength\itemsep{1em}
    	\item 
    		Build the tree from the data of block.
    	\item
    		Compute Catalan index from the the sizes and indices of the
			left and right subtrees.  
		\item 
			This requires a recursive traversal on the tree and has a
			$O(n)$ time complexity.\footnote{precomputing the {\em
			prefix sum} of the Catalan number products.}
	\end{enumerate}

    \begin{center}
		\scalebox{1} { \begin{minipage}{\textwidth}
			\begin{equation}  \label{fun:tid}
  {\cal T}(s_l, t_l, s_r, t_r) = t_l \; {\cal C}_{s_r} + t_r +
  \sum_{i = 0}^{s_l - 1} {\cal C}_i \; {\cal C}_{s_l + s_r - i}
\end{equation}

			\end{minipage}
		}
	\end{center}
\end{frame}

\begin{withoutheadline}
\begin{frame}
	\frametitle{Illustration for Build the Tree}
	\begin{figure}[htbp]   
	  \centering
	  \includegraphics[width=\linewidth]{\GraphicPath/fig-cartesian-encoding-buildtree.pdf}
	  \caption{Build the Tree}
	\end{figure}
\end{frame}
\end{withoutheadline}

\subsubsection{Keep the Rightmost Path}
\begin{frame}
	\frametitle{Keep the Rightmost Path}
	We propose a more efficient method than the previous computation of
	building the tree.

	\begin{itemize}
		\setlength\itemsep{1em}
		\item 
			Determines the Catalan index by keeps only the {\em
			rightmost path} in a {\em stack} $D$.
		\item
			After knowing the Catalan index $t$, we can compute LCA and
			answer queries with Algorithm~\ref{alg:parallel-LCA} and
			Equation~\ref{fun:tid}.
	\end{itemize}
\end{frame}

\begin{withoutheadline}
\begin{frame}
	\frametitle{Illustration for Keep the Rightmost Path}
	\begin{center}
		\scalebox{1} { \begin{minipage}{\textwidth}
			\begin{figure}[!thb]
			  \centering
			  \includegraphics[width=\linewidth]{\GraphicPath/fig-cartesian-encoding-static.pdf}
			  \caption{Compute Catalan index for a tree.  $A_l$ and $B_l$ denote
			    the left subtrees of $A$ and $B$ respectively.}
			  \label{fig:fig-cartesian-encoding-static}
			\end{figure}
			\end{minipage}
		}
	\end{center}
\end{frame}
\end{withoutheadline}

\begin{withoutheadline}
\begin{frame}
	\frametitle{Pseudocode of Catalan Index Computation}
	\begin{center}
		\scalebox{0.7} { \begin{minipage}{\textwidth}
			\begin{algorithm}
\SetAlgoNoLine
\KwIn{
  $A[1 \cdots s]$: storage array\;
  $s$: the number of elements\;
}
\KwOut{
  $\mathit{tid}$: this label
}
$\langle\mathit{lsz},\mathit{lid},\mathit{value}\rangle$ $D$[$s+1$] \;
$\textit{Dp} \gets 0$ \;
$D[0] \gets \langle 0,0,\infty \rangle$ \;
\For{$i \gets 1$ to $s$} {
  $v \gets A[i], \; \textit{lsz} \gets 0, \; \textit{lid} \gets 0$ \;
  \While{$D[\textit{Dp}].\textit{value} < v$} {
    $\textit{lid} \gets \textit{tid}(D[\textit{Dp}].\textit{lsz}, D[\textit{Dp}].\textit{lid}, \textit{lsz}, \textit{lid})$ \;
    $\textit{lsz} \gets \textit{lsz} + D[\textit{Dp}].\textit{lsz} + 1$ \;
    $\textit{Dp} \gets \textit{Dp} - 1$ \;
  }
  $\textit{Dp} \gets \textit{Dp} + 1$ \;
  $D[\textit{Dp}] \gets \langle\mathit{lsz},\mathit{lid},\mathit{v}\rangle$ \;
}

$\textit{lsz} \gets 0, \; \textit{lid} \gets 0$ \;
\While{$\textit{Dp} > 0$} {
  $\textit{lid} \gets \textit{tid}(D[\textit{Dp}].\textit{lsz}, D[\textit{Dp}].\textit{lid}, \textit{lsz}, \textit{lid})$ \;
  $\textit{lsz} \gets \textit{lsz} + D[\textit{Dp}].\textit{lsz} + 1$ \;
  $\textit{Dp} \gets \textit{Dp} - 1$ \;
}
return $\textit{lid}$ \;

\caption{Offline Type of Cartesian Tree}
\label{alg:cartesian-encode-offline}
\end{algorithm}
			\end{minipage}
		}
	\end{center}
\end{frame}
\end{withoutheadline}

\subsection{Dynamic Catalan Index Computation}
\begin{frame}
	\frametitle{Background -- Dynamic Catalan Index Computation}
	\begin{itemize}
		\setlength\itemsep{1em}
	 	\item 
	 		Fischer introduced the first encoding method.
		\item
			Masud presents a new encoding method that reduces the number
			of instructions.
		\item 
			Unfortunately all these algorithms are {\em off-line}, or
			could not use the encoding, which is presented by parallel
			building LCA algorithm.
	\end{itemize}
\end{frame}

\begin{frame}
	\frametitle{Dynamic Catalan Index Computation}
	\begin{itemize}
		\setlength\itemsep{1em}
	 	\item 
	 		Keeps the size of tree as a constant $n$.
	 	\item
	 		Append a path of $n-i$ {\em right-child-only} nodes to the
			rightmost path of the existing tree of size $i$.
		\item
			Compute the {\em difference} between the Catalan indices of
			these two different parts, we can compute the final $t^*$ by
			{\em patching} the Catalan index with this difference.
	\end{itemize}
\end{frame}

\begin{frame}
	\frametitle{Illustration for Dynamic Catalan Index Computation}
	\begin{center}
		\scalebox{0.8} { \begin{minipage}{1.3\textwidth}
			\begin{figure}[!thb]
			  \centering \subfigure[$t_{{\it root}_0}= {\cal C}_n - 1$]{
			    \includegraphics[width=0.26\linewidth]{\GraphicPath/fig-cartesian-encoding.pdf}
			    \label{fig:cartesianEncoding-init}
			  } \subfigure[$t_{{\it root}_i}$]{
			    \includegraphics[width=0.3\linewidth]{\GraphicPath/fig-cartesian-encoding-before.pdf}
			    \label{fig:cartesianEncoding-before}
			  } \subfigure[$t_{{\it root}_{i+1}} = t_{{\it root}_i} + t_x - t_A$]{
			    \includegraphics[width=0.3\linewidth]{\GraphicPath/fig-cartesian-encoding-after.pdf}
			    \label{fig:cartesianEncoding-after}
			  }
			  \caption{Normalization of Cartesian trees of increasing sizes by
			    adding a virtual path.}
			  \label{fig:cartesianEncoding}
			\end{figure}
			\end{minipage}
		}
	\end{center}
\end{frame}

\begin{withoutheadline}
\begin{frame}
	\frametitle{Pseudocode of Dynamic Catalan Index Computation}
	\begin{center}
		\scalebox{.8} { \begin{minipage}{\textwidth}
			\begin{algorithm}[!thb]
\SetAlgoNoLine
\KwIn{
  $S$: state of Cartesian Tree;  $v$: the added data;
  $D$: A stack where every element has $s$, $t$, and $v$ \;
}
\KwOut{
  $t$: The Catalan index of the input data block after adding $v$
}

$p \gets S.p$, $s \gets 0$, $t \gets 0$ \;
$s' \gets S.s - S.{i} + 1$ \;
$t' \gets C[s'] - 1$ \;

\While{$D[p].{value} < v$} {
  $t \gets t(D[p].s, D[p].t, s, t)$ \;
  $t' \gets t(D[p].s, D[p].t, s', t')$ \;
  $s \gets s + D[p].s+1$ \;
  $s' \gets s' + D[p].s+1$ \;
  $p \gets p - 1$ \;
}
$p \gets p + 1$ \;
$D[p] \gets \left \langle s, t, {v} \right \rangle$ \;
$S.p \gets p$ \;
$x.t \gets t(s, t, S.s-S.i, C[S.s-S.i]-1)$ \;
$S.t \gets S.t + t' - x.t$ \;
$S.i \gets S.i + 1$ \;
return $S.t$ \;

  \caption{Online Type of Cartesian Tree}
  \label{alg:cartesian-encode-online}
\end{algorithm}

			\end{minipage}
		}
	\end{center}
\end{frame}
\end{withoutheadline}

\subsection{Parallel VGLCS Algorithm with Blocked Sparse Table}
\begin{frame}
	\frametitle{Parallel VGLCS Algorithm with Blocked Sparse Table}
	The blocked sparse table solves range maximum query on incremental
	data.  It supports amortized $O(1)$ insertion and $O(1)$ query.
	\begin{itemize}
		\setlength\itemsep{1em}
		\item 
			The {\em first} stage of the VGLCS algorithm can be solved
			in $O(n)$ for the suffix maximum of {\em every column}.
		\item 
			The {\em second} stage of the VGLCS algorithm can be solved
			in $O(m/p + \log m)$ for the range maximum of the {\em row}.
		\item
			Consequently, the parallel VGLCS algorithm runs in $O(nm/p+n\log m)$
	\end{itemize}
\end{frame}
\section{Optimization}

\subsection{The Strategy of Disjoint Set}
\begin{frame}
    \frametitle{The Strategy of Disjoint Set}
\end{frame}

\subsection{Parallel Range Maximum Query}
\begin{frame}
    \frametitle{Parallel Range Maximum Query}
\end{frame}

\section{Experiment}

\subsection{Environment Settings}
\begin{frame}
    \frametitle{Environment}
\end{frame}

\subsection{Result}
\begin{frame}
    \frametitle{Speedup}
    \begin{table}
    	\caption{$n = 5000, \; \forall G[i] < 10$}
		\begin{tabular}{| l | l | l | l |}
			\hline
			Algorithm 		& n = 5000 & n = 2000 & n = 1000\\ \hline
			serial 			& 1.172 & 0.190 & 0.044\\ \hline
			parallel 	& 0.218 & 0.077 & 0.033\\ \hline
		\end{tabular}
	\end{table}
\end{frame}

\begin{frame}
    \frametitle{Scalability}
    \begin{table}
    	\caption{$n = 5000$}
		\begin{tabular}{| l | l | l | l | l |}
			\hline
			Algorithm 	& p = 1 & p = 2 & p = 4 & p = 8\\ \hline
			parallel 	& 0.928  & 0.508 & 0.303 & 0.240 \\ \hline
		\end{tabular}
	\end{table}
\end{frame}
\section{Related Work}

\subsection{Parallel Method}
\begin{frame}
	\frametitle{Parallel Method}
	\begin{itemize}
		\setlength\itemsep{1em}
		\item
			The wavefront computation is {\em not} cache-friendly.
			\begin{itemize}
				\item 
					To address this cache issue, Maleki et al. developed
					a technique to exploit more parallelism in the
					dynamic programming.
			\end{itemize}
		\item
			The alternative to wavefront method is the traditional
			row-by-row approach.
	\end{itemize}
\end{frame}

\subsection{Variable Gapped Longest Common Subsequence}
\begin{frame}
	\frametitle{Suffix Maximum Query in the VGLCS Algorithm}
	\begin{itemize}
		\setlength\itemsep{1em}
		\item
			Peng's sequential VGLCS algorithm uses disjoint sets for
			suffix maximum query. 
		\item
			We instead use {\em sparse table} to support incremental
			suffix/range maximum queries in our VGLCS algorithm.
	\end{itemize}
\end{frame}

\begin{frame}
	\frametitle{Blocked Sparse Table}
	\begin{itemize}
		\setlength\itemsep{1em}
		\item
			Fischer proposed blocked sparse table for better performance
			than the unblocked sparse table, and Fischer's algorithm
			builds least ancestor tables for answering range maximum
			query.
		\item
			We instead use a {\em rightmost-pops} encoding for Cartesian
			trees.
	\end{itemize}
\end{frame}

\begin{frame}
	\frametitle{Cache Issues on Blocked Sparse Table}
	\begin{itemize}
		\setlength\itemsep{1em}
		\item
			Demaine proposed {\em cache-aware} operations on Cartesian
			tree.
		\item
			Masud presents a new encoding method that reduces the number
			of instructions.
		\item 
			Our rightmost-pops encoding for Cartesian trees also reduces
			cache misses, and with a much simpler implementation than
			Demaine's encoding.
	\end{itemize}
\end{frame}
\section{Conclusion}

\subsection{Conclusion}
\begin{frame}
    \frametitle{Conclusion}
    \begin{itemize}
    	\setlength\itemsep{1em}
    	\item
    		VGLCS problem
    		\begin{itemize}
    			\setlength\itemsep{1em}
    			\item 
                    East-to-implements linear sparse table as
                    rightmost-pops encoding sparse table runs in $O(n^2
                    s/p+n\;\max(\log n, s))$.
				\item
					It can be solved in theorem $O(n^2/p+n\log n)$
		    		efficiently.
    		\end{itemize}
    	\item
    		Incremental ranged maximum query
    		\begin{itemize}
    			\setlength\itemsep{1em}
    			\item
    				It is more powerful than the incremental suffix
					maximum query.
				\item
					It can be solved in amortized $O(1)$ by our sparse
					table.
    		\end{itemize}
    \end{itemize}
\end{frame}
\section{Future Work}

\subsection{Future Work}
\begin{frame}
	\frametitle{Future Work}
	\begin{itemize}
		\setlength\itemsep{1em}
		\item
			Minimize the amount of computation in the encoding process
		\item
			Several register state instead of the memory access
	\end{itemize}
\end{frame}

\appendix
\section{Math Tricks}

\subsection{Catalan Number}
\begin{frame}
	\frametitle{Catalan Number}
	\begin{itemize}
		\setlength\itemsep{1em}
		\item 
			${\cal C}_0 = 1, \; {\cal C}_1 = 1$
		\item 
			${\cal C}_n = {\cal C}_0\;{\cal C}_{n-1} + {\cal
			C}_{1}\;{\cal C}_{n-2} + \cdots + {\cal C}_{n-1}\;{\cal
			C}_{0}$
		\item
			${\cal C}_n = \frac{1}{n \; + \;1} \binom{2n}{n} \sim \frac{4^n}{n^{\nicefrac{3}{2}} \sqrt{\pi}}$
		\item 
			${\cal C}_n = \frac{4n \;-\; 2}{\phantom{4} n \;+\; 1} {\cal C}_{n-1}$
	\end{itemize}
\end{frame}

\subsection{Ackermann Function}
\begin{frame}
	\frametitle{Ackermann Function}
	\begin{align*}
		A(m, n) = \left\{\begin{matrix*}[l]
			n+1 & \text{if } m = 0\\ 
			A(m-1, 1) & \text{if } m > 0 \text{ and } n = 0\\ 
			A(m-1, A(m, n-1)) & \text{if } m > 0 \text{ and } n > 0.
			\end{matrix*}\right.
	\end{align*}

	\begin{itemize}
		\setlength\itemsep{1em}
		\item 
			$f(n) = A(n, n)$
		\item 
			$\alpha(n) = f^{-1}(n)$
		\item
			$A(4, 4) = 2^{2^{2^{2^{16}}}}$, $\alpha(2^{2^{2^{2^{16}}}}) = 4$
	\end{itemize}
\end{frame}

\section{Code Tricks}

\subsection{Intel SIMD}
\begin{frame}
	\frametitle{Intel Intrinsics}
	\definecolor{links}{HTML}{2A1B81}
	\hypersetup{colorlinks,linkcolor=blue,urlcolor=links}
	\begin{itemize}
		\setlength\itemsep{1em}
		\item 
			\href{https://software.intel.com/sites/landingpage/IntrinsicsGuide/}{Intel Intrinsics Guide} \\
			\url{https://software.intel.com/sites/landingpage/IntrinsicsGuide/}
		\item
			C style functions that provide access to many Intel
			instructions
	\end{itemize}
\end{frame}

\subsection{Find the Horizontal Minimum/Maximum Value}
\begin{frame}
	\frametitle{Find the Horizontal Minimum/Maximum by SSE 2}
	{\tt SSE} 2
	\begin{itemize}
		\setlength\itemsep{1em}
		\item
			Find the minimum value with reduction
		\item
			\texttt{\_\_m128i \_mm\_shufflehi\_epi16 (\_\_m128i a, int imm8)}
		\item
			\texttt{\_\_m128d \_mm\_cmple\_pd (\_\_m128d a, \_\_m128d b)}
	\end{itemize}
\end{frame}

\begin{frame}
	\frametitle{Find the Horizontal Minimum/Maximum by SSE 3}
	{\tt SSE} 3
	\begin{itemize}
		\setlength\itemsep{1em}
		\item 
			The short width type has be supported by SSE 3.
		\item
			\texttt{\_\_m128i \_mm\_shuffle\_epi8 (\_\_m128i a, \_\_m128i b)}
	\end{itemize}
\end{frame}

\begin{frame}
	\frametitle{Find the Horizontal Minimum/Maximum by SSE 4.1}
	{\tt SSE} 4.1
	\begin{itemize}
		\setlength\itemsep{1em}
		\item
			\texttt{\_\_m128i \_mm\_minpos\_epu16 (\_\_m128i a)}
	\end{itemize}
	\lstinputlisting{./codes/minpos.c}
\end{frame}

\subsection{Round Up/Down with Power of Two}
\begin{frame}
	\frametitle{Round Up/Down with Power of Two}
	\lstinputlisting{./codes/roundwith2.c}
\end{frame}

\subsection{Exploit Register Allocation in RMQ Algorithm}
\begin{frame}
	\frametitle{Exploit Register Allocation in RMQ Algorithm}
	\lstinputlisting{./codes/reginrmq.c}
\end{frame}

\end{document}
