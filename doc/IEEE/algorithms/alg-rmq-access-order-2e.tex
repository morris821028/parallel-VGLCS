\begin{algorithm}
\SetAlgoNoLine
\KwIn{
  $A$: the input array\; 
  $P, \; S$: the prefix/suffix maximum arrays for each block \;
  ${\cal T}$: the Catalan index array for each block \;
  $T_s$: the blocked sparse table \; 
  $[l, r]$: the bound of the ranged query \;
}
\KwOut{
  $v$: the maximum value in $A[l .. r]$ \;
}

\If{$l$ and $r$ are in the same block $b$} {
  $i \gets$ Query the ranged maximum query $[l, r]$ on ${\cal T}_{k}$, where $k$ is the Catalan index of the $b$  \;
  return $A[i]$ \;
}
$v \gets - \infty$ \;
$l' \gets$ The starting index of the next block from the position $l$ \;
$r' \gets$ The ending index of the previous block from the position $r$ \;
$t \gets \lfloor \log_2 (r' - l' + 1) \rfloor$ \;
\If{$l' \le r'$} {
  $v \gets \max(T_s[t][l' + 2^t - 1], T_s[t][r'])$ \;
}

\If(\tcp*[f]{Access $P$ only when necessary}){$T_s[t][l' + 2^t - 2] > v$} {
  $v \gets \max(v, P[r])$ \;
}

\If(\tcp*[f]{Access $S$ only when necessary}){$T_s[t][r' + 1] > v$} {
  $v \gets \max(v, S[l])$ \;
}

return $v$ \;

\caption{The process of answering a range query. Note that it accesses
  $T_S$, then $P$, then $S$.}
\label{alg:rmq-access-order-2e}
\end{algorithm}
