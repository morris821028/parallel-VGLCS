\documentclass[format=acmsmall, review=false, screen=true]{acmart}

\usepackage{booktabs} % For formal tables

\usepackage[ruled]{algorithm2e} % For algorithms
\renewcommand{\algorithmcfname}{ALGORITHM}
\SetAlFnt{\small}
\SetAlCapFnt{\small}
\SetAlCapNameFnt{\small}
\SetAlCapHSkip{0pt}
\IncMargin{-\parindent}



%=====================================================================
\usepackage{subfigure}
\makeatletter
\DeclareRobustCommand*\cal{\@fontswitch\relax\mathcal}
\makeatother

\usepackage{listings}
\lstset{
  frame=single,
  language=C,
  basicstyle=\small,
}
\makeatletter
\def\lst@makecaption{%
  \def\@captype{table}%
  \@makecaption
}
\makeatother
%=====================================================================

% Metadata Information
\acmJournal{TWEB}
\acmVolume{9}
\acmNumber{4}
\acmArticle{39}
\acmYear{2010}
\acmMonth{3}
\copyrightyear{2009}
%\acmArticleSeq{9}

% Copyright
%\setcopyright{acmcopyright}
\setcopyright{acmlicensed}
%\setcopyright{rightsretained}
%\setcopyright{usgov}
%\setcopyright{usgovmixed}
%\setcopyright{cagov}
%\setcopyright{cagovmixed}

% DOI
\acmDOI{0000001.0000001}

% Paper history
\received{February 2007}
\received[revised]{March 2009}
\received[accepted]{June 2009}


% Document starts
\begin{document}
% Title portion. Note the short title for running heads 
\title[Parallel VGLCS and IRMQ]{Parallel Variable Gapped Longest Common Sequence and Incremental Range Maximum Query}  
\author{Shiang-Yun Yang}
\orcid{1234-5678-9012-3456}
\author{Pangfeng Liu}
\affiliation{%
  \institution{National Taiwan University}
  \department{Computer Science and Information Engineering}
  \streetaddress{XXXXXXXXXXXXXX}
  \city{Taipei}
  \state{XXXXXX}
  \postcode{XXXXXX}
  \country{Taiwan}}
\author{Jan-Jan Wu}
\affiliation{%
  \institution{University of Virginia}
  \department{School of Engineering}
  \city{Charlottesville}
  \state{VA}
  \postcode{22903}
  \country{USA}}

\begin{abstract}
Multifrequency media access control has been well understood in
general wireless ad hoc networks, while in wireless sensor networks,
researchers still focus on single frequency solutions. In wireless
sensor networks, each device is typically equipped with a single
radio transceiver and applications adopt much smaller packet sizes
compared to those in general wireless ad hoc networks. Hence, the
multifrequency MAC protocols proposed for general wireless ad hoc
networks are not suitable for wireless sensor network applications,
which we further demonstrate through our simulation experiments. In
this article, we propose MMSN, which takes advantage of
multifrequency availability while, at the same time, takes into
consideration the restrictions of wireless sensor networks. Through
extensive experiments, MMSN exhibits the prominent ability to utilize
parallel transmissions among neighboring nodes. 
\end{abstract}


%
% The code below should be generated by the tool at
% http://dl.acm.org/ccs.cfm
% Please copy and paste the code instead of the example below. 
%
\begin{CCSXML}
<ccs2012>
<concept>
<concept_id>10010147.10010169.10010170.10010171</concept_id>
<concept_desc>Computing methodologies~Shared memory algorithms</concept_desc>
<concept_significance>300</concept_significance>
</concept>
</ccs2012>
\end{CCSXML}

\ccsdesc[300]{Computing methodologies~Shared memory algorithms}

%
% End generated code
%

% We no longer use \terms command
%\terms{Design, Algorithms, Performance}

\keywords{range minimum query, incremental range maximum query,
incremental suffix maximum query, longest common sequence, parallel,
Cartesian tree}


\thanks{This work is supported by the National Science Foundation,
  under grant CNS-0435060, grant CCR-0325197 and grant EN-CS-0329609.

  Author's addresses: G. Zhou, Computer Science Department, College of
  William and Mary; Y. Wu {and} J. A. Stankovic, Computer Science
  Department, University of Virginia; T. Yan, Eaton Innovation Center;
  T. He, Computer Science Department, University of Minnesota; C.
  Huang, Google; T. F. Abdelzaher, (Current address) NASA Ames
  Research Center, Moffett Field, California 94035.}


\maketitle

% The default list of authors is too long for headers}
\renewcommand{\shortauthors}{G. Zhou et al.}

\newcommand*{\PartialPath}{../partial}
\newcommand*{\GraphicPath}{../graphics}
\newcommand*{\AlgoPath}{../algorithms}
\newcommand*{\FormulaPath}{../formulas}
\newcommand*{\CodePath}{../codes}
\newcommand*{\TablePath}{../tables}

\begin{abstract}

The longest common subsequence problem with variable gapped
constraints (VGLCS) is used in genes and molecular biology.  We can
find $O(nm)$ solution in the previous study, which use the efficient
incremental suffix maximum query (ISMQ).  ISMQ supports append a value
to array and get the suffix maximum value in amortized $O(1)$ time.
However, we can parallel origin algorithm by wavefront method, but not
have better performance.  In this paper, our algorithm and data
structure offer efficient operations and better theoretical time
complexity.  The VGLCS problem can be solved in $O(nm / p + n \log
n)$, which $p$ is the number of processors.  Simultaneously, we offer
that the incremental range minimum/maximum query problem is also be
solved in $O(n)$ -- $O(1)$ with sparse table.

\iffalse
可變的間隙限制最長同子序列應用於基因、分子生物學中。
在之前的研究中,已提出理論在 $O(nm)$ 的算法,
其使用特殊的 incremental tree set union 完成 $O(1)$ 的操作,
否則必須使用傳統并查集操作 $O(\alpha(n))$。
原先的序列算法無法以直觀的方式平行,我們修改了原本算法的資料結構,
且可解決遞增任意區間最大值詢問 (incremental RMQ),
可以在 $O(1)$ 時間內支援所有操作,其平行的運行時間為 $O(nm / p + n \log n)$。
在單一處理器上,理論時間複雜度仍保持 $O(nm)$,
其運行效能不輸 $O(nm\alpha(n))$ 實作版本。
\fi

\end{abstract}

\begin{IEEEkeywords}
range minimum query,
incremental range maximum query, incremental suffix maximum query,
longest common sequence, parallel, cartesian tree
\end{IEEEkeywords}


\section{Introduction} %Introduction
\label{sec:Introduction}

The \emph{longest common subsequence} (LCS) problem applied many products and fields widely.  In multi-core platform, most studies focus on the wavefront parallelism. Motivated by the definition of recursion in LCS, Jiaoyun Yang introduced a new formula to exploit more cache performance.  Here, we use the similar idea to improve LCS with variable contraints, which refer in Iliopoulos' study.  For example, \emph{fixed gap LCS} (FGLCS) require the distance between two consecutive matches limited at most $k+1$.  It can be solved in $O(nm)$, which $n, \; m$ is the length of input strings.

In many kinds of contraint, this paper foucus on \emph{variable gap LCS} (VGLS). The VGLCS require the distance between two consecutive matched limited at most $G_i$, which $G_i$ is the value of the position $i$ in input array. For example, two protein sequences $A = \tt{RCLPCRR}$ and $B = \tt{RPPLCPLRC}$, and two gap functions $G_A = [2, 3, 0, 0, 3, 2, 2]$ and $G_B = [2, 0, 0, 0, 3, 0, 0, 2, 3]$. Figure ~\ref{fig:VGLCSex} presents an example to show that tow motifs $\tt{R..C..R}$ and $\tt{R..C..C}$.

The remainder of the paper is organized as follows. In the section 2, we present the parallel algorithm to solve VGLCS problem. In the section 3, we present a new algorithm, which parallel easily and time complexity $O(nm)$ better than previous study. In the section 4 to 5, we provide the optimized implementation and the result of experiments. The last section has a brief conclusion in this paper.

\begin{figure}[!thb]
  \centering
  \includegraphics[width=\linewidth]{graphics/fig-VGLCSex.pdf}
  \includegraphics[width=\linewidth]{graphics/fig-VGLCSex2.pdf}
  \caption{An example for illustrating the VGLCS, which has two protein sequences $A = \tt{RCLPCRR}$ and $B = \tt{RPPLCPLRC}$, and two gap functions $G_A = [2, 3, 0, 0, 3, 2, 2]$ 和 $G_B = [2, 0, 0, 0, 3, 0, 0, 2, 3]$.}
  \label{fig:VGLCSex}
\end{figure}

\ifdefined\MasterThesis
\chapter{Related Work}
\else
\section{Related Work}
\fi
\label{sec:RelatedWork}

The {\em wavefront method} address this parallelization difficulty by
keeping the computation as wavefront.  However, the wavefront
computation is not cache-friendly, i.e., the wavefront algorithm cannot
effectively keep the required data in cache.  To address this cache
issue, Maleki et al.~\cite{Maleki2016EfficientPU} developed a technique
to exploit more parallelism in the dynamic programming, and we can use
this technique for the issue in our situation.

% still not clear about what you mean by rank convergence

\iffalse

先有 linear-tropical dynamic programming (LTDP) 的性質,
LTDP 可以拆成好幾個序列子問題,每個子問題依賴前一個子問題的最佳解,
我們可以將 LTDP 假想成好幾個子矩陣 (子問題) 的連乘 (合併),
平行方法:
(1) 子問題分開解決,並保留最佳解 
(2) 傳遞前一個子問題的最佳解 
(3) 修正錯誤的區塊,並更新最佳解

當 rank = 1 時,
子問題 u 的最佳解到子問題 v 的最佳解只會經過一個節點 (轉換)。

因此,原先的問題相當於 rank = n,最佳解會經過 n 個點。

The linear-tropical dynamic programming
problem~\cite{Maleki2016EfficientPU} which has the property of {\em
  rank convergence} can be divided into the sub-problems, such as a
sequence of stages, which a solution of the sub-problems depends on
only the solutions of the previous sub-problem.  About rank
convergence in VGLCS problem, the sub-problems has developed efficient
algorithm by Peng.  However, the parallel forward phase is difficult
to integrate the partial correct and incorrect solution in
parallelization algorithm which using rank convergence.
\fi

%need to say
%something abuut
%rank
%convergence


\section{Parallel Algorithm} %
\label{sec:parallelSerial}

In the serial algorithm which was designed by Yung-Hsing Peng, we observed that variants of LCS use many status to decide a new status. Its definition makes lots of data dependency, so we cannot parallel row by row intuitively. Even though it's hard to parallel by several modifications, we can parallel the serial algorithm by using wavefront method. Because the usage of cache  memory is hard to use efficiently in it, so Saeed Maleki~\cite{saeed} developed a technique that uses the property of rank convergence to exploit more parallism to solve dynamic programming problems.

\begin{algorithm}[!thb]
  \caption{Algorithm for Finding VGLCS~\cite{Peng2011TheLC}}
  \label{alg:serial-VGLCS}
  \begin{algorithmic}[1]
    \Require
      $A, B$: the input string;
      $G_A, G_B$: the array of variable gapped constraints;
    \Ensure Find the LCS with variable gapped constraints
    \State Create $m$ number of data structure $Q[m]$ to support ISMQ problem.
    \State Create an empty table $V[n][m]$.
    \For{$i \gets 1$ to $n$}
      \State Create a data structure $RQ$ to support ISMQ problem.
      \State $r \gets i - (GA[i]+1)$
      \For{$j \gets 1$ to $m$}
        \If{$A[i] = B[j]$}
            \State \multiline{Get the suffix maximum value $t$ from 
                    position $j - (GB[j]+1)$ to the end in $RQ$.}
            \State $V[i][j] \gets t + 1$
            \State \multiline{Get the suffix maximum value $t$ from 
                    position $r$ to $i$ in $Q[j]$.}
            \State Append value $t$ into $RQ$.
            \State Append value $V[i][j]$ into $Q[j]$.
        \Else
            \State $V[i][j] \gets 0$
            \State \multiline{Get the suffix maximum value $t$ from 
                    position $r$ to $i$ in $Q[j]$.}
            \State Append value $t$ into $RQ$.
        \EndIf
      \EndFor
    \EndFor
    \State Retrieve the VGLCS by tracing $V[n][m]$
  \end{algorithmic}
\end{algorithm}


In the algrithm ~\ref{alg:serial-VGLCS}, it run in $O(nm \alpha)$ time and $O(nm)$ space. If we use wavefront method to parallel, it must use extra storage space to record all row status compared to origin algorithm. If use the rank convergence technique, it also use extra storage space to reserve the informations of state translation, and spends more time to merge split parts.

In this paper, we tend to design the algorithm which has less space and be more cache-friendly. We define two stages in the serial algorithm, row and column stage. In the row stage, it use $O(\alpha(n))$ to maintain incremental suffix maximum query(ISMQ).



這裡我們著手設計算法空間複雜度常數小,並且針對快取友善。從序列算法中,發現在橫向查找中使用 $O(\alpha(n))$ 操作的增長後綴最大值查找 (\emph{incremental suffix maximum query}, ISMQ),這部分難以平行化。為消除資料相依性,我們找到幾種區間詢問的替代方案。如 

\begin{itemize}
  \item Binary Indexed Tree -- $O(\log n)$: 對於任意前綴查找極值和更新元素,可以提供每次時間複雜度 $O(\log n)$,其運行常數比 Range Tree 低,只能支持前綴查找。若要運行區間查找,則必須在數學上符合加法原則。
  \item Range Tree -- $O(\log n)$: 支持更高維度的正交區塊搜索,而我們用在區間極值查找需要 $O(\log n)$ 的時間完成所有操作。
  \item Sparse Table -- $O(n \log n)$ -- $O(1)$:
    建立表格 $T[i][j]$ 表示區間 $(i-2^j,i]$ 之間的極值。建表時間複雜度 $O(n \log n)$,對於任意區間詢問可以拆分兩個 super-block 的表格檢索,轉換過程和存取時間需要 $O(1)$。
\end{itemize}

根據 VGLCS 動態規劃時的單調性質,可以使用 Van Emde Boas Tree 作為輔助資料結構在 $O(\log \log n)$ 時間內完成操作。其中 Sparse Table 是我們認為最好的替代方案,其整合後的平行算法如下 ~\ref{alg:parallel-VGLCS},其時間複雜度為 $O(n^2 / p + n \log n)$,其中 $p$ 為處理器個數。

\begin{algorithm*}[!thb]
  \caption{Parallel Algorithm for Finding VGLCS}
  \label{alg:parallel-VGLCS}
  \begin{algorithmic}[1]
    \Require
      $A, B$: the input string;
      $G_A, G_B$: the array of variable gapped constraints;
    \Ensure Find the LCS with variable gapped constraints
    \State \tt{ISMQ} $Q[n]$
    \State \tt{int} $V[n][m]$
    \For{$i = 1$ to $n$}
      \State SparseTable $sp$
      \ParFor{$j = 1$ to $m$}
        \State $sp[j] = Q[j].get(r)$
      \EndParFor
      \State sp.build(m) \Comment{Parallel algorithm run in $O(n/p \log n + \log n)$}
      \ParFor{$j = 1$ to $m$}
        \If{$A[i] = B[j]$}
            \State $V[i][j] = sp.get(j - (GB[j]+1), j-1)+1$
            \State $Q[j].set(i, V[i][j])$
        \EndIf
      \EndParFor
    \EndFor
    \State Retrieve the VGLCS by tracing $V[n][m]$
  \end{algorithmic}
\end{algorithm*}

\section{Range Maximum Query} \label{sec:parallelRMQ}

In this section, we will describe our approach to address the
challenges in the second stage of Algorithm~\ref{alg:parallel-VGLCS},
i.e., an efficient {\em range maximum query}.  The range maximum query
problem is more complicated than the previous incremental suffix
maximum query problem because the suffix maximum query is a special
case of the range maximum query.

The operations to support range maximum query is similar to those for
suffix maximum query.  A {\sc make} operation creates an empty array
$A$, an {\sc Append(V)} operation appends a value $V$ to the end of an
array $A$.  Finally, a {\sc Query(L, R)} operation finds the {\em
  maximum} value among the $L$-th value to the $R$-th value of an
array $A$.

%% We can use a parallel sparse table implementation to answer range
%% maximum queries with $p$ processors, so that it requires $O(n \log n /
%% p + \log n)$ in preprocessing, and takes only $O(1)$ time to answer a
%% query.  On the other hand, it is difficult to efficiently parallelize
%% the querying on tree-like data structures, e.g., disjoint sets.

\subsection{Blocked Sparse Table} \label{sec:blocked-sparse-table}

\begin{figure}[!thb]
  \centering \subfigure[Array] {
    \includegraphics[width=0.65\linewidth]{\GraphicPath/fig-interval-decomposition.pdf}
  } \subfigure[Blocked sparse table] {
    \includegraphics[width=0.25\linewidth]{\GraphicPath/fig-sparse-table.pdf}
  }
  \caption{A Sparse Table}
  \label{fig:block-interval-decomposition}
\end{figure}

We improve the efficiency of our parallel VGLCS algorithm with a {\em
  blocked} sparse table proposed by
Fischer~\cite{Fischer2006TheoreticalAP}.  The blocked approach first
partitions the data into blocks of size $s$, then it computes the
maximum of each block, and compute a sparse table $T_s$ for these
maximums.  Recall that the {\em unblocked} sparse table in
Section~\ref{sec:sparse-table} does {\em not} partition data into
blocks, but builds a table of $\log n$ rows of maximum for different
lengths of intervals.

The blocked approach answers a range maximum query as follows.  We
consider two types of queries -- {\em super block} query and {\em
  in-block} query.  A super block query queries the answer for {\em
  consecutive} blocks, and an in-block query queries a segment {\em
  within} a block.  It is easy to see that we can answer a super block
query by querying $T_s$ at most {\em twice}, as described in
Section~\ref{sec:sparse-table}.  We can also answer an in-block
queries answered by a {\em single} lookup into an {\em least common
  ancestor table}.  We will provide more details on this table later.
Since we can split {\em any} range query into at most {\em two}
queries into $T_s$ and {\em two} in-block queries, we need at most
{\em four} memory access to answer any range maximum query.  Also
since the super block query is easy to answer with $T_s$, we will now
focus on in-block query.

Fischer's algorithm~\cite{Fischer2006TheoreticalAP} scans through the
data within a block and places them into a {\em Cartesian tree}.  Each
node of this Cartesian tree stores a data and the index of this data
within the block.  One can think of this Cartesian tree as a {\em heap}
where the data are in heap order, and the indexes of the data are in
{\em sorted binary search tree order}.  Please refer to
Figure~\ref{fig:ancesstor-cartesian} for an illustration.  Note that in
Figure~\ref{fig:ancesstor-cartesian} the horizontal position of tree
nodes reflects their position in the data block.  As a result, the
answer to an in-block range maximum query from the $i$-th to the $j$-th
element of a block is located at their {\em least common ancestor} in
the Cartesian tree.  For example, the maximum from the fourth (with data
3) to the six elements (with data 2) is located at the fifth element
(with data 7).

\begin{figure}[!thb]   
  \centering
  \includegraphics[width=0.9\linewidth]{\GraphicPath/fig-interval-cartesian.pdf}
  \caption{Least common ancestor tables}
  \label{fig:ancesstor-cartesian}
\end{figure}

To answer in-block queries, Fischer's algorithm computes a {\em least
common ancestor table} for every block.  After scanning the data in a
block, Fischer's algorithm builds a Cartesian tree and its least common
ancestor (LCA) table.  An LCA table stores all $((i, j), k)$'s where the
$k$-th data is the common ancestor for the $i$-th and $j$-th data in a
block.  Please refer to Figure~\ref{fig:ancesstor-cartesian} for an
illustration.  Now we can answer an in-block range maximum query from
the $i$-th to the $j$-th element simply by look into the LCA table and
return $k$,  least common ancestor table of this block.  One can think
of the LCA table as a mapping table from an in-block query $(i, j)$ to
its answer $k$.

Note that the algorithm does {\em not} maintain the value of the $k$-th
element.  Instead, it stores the {\em index}, i.e., $k$, of the least
common ancestor so that two blocks with the {\em same relative key
order} can {\em share} a least common ancestor table.  For example, the
first three blocks in Figure~\ref{fig:ancesstor-cartesian} can share the
same LCA table because they have the same Cartesian tree.  Consequently,
an in-block range maximum query $(1, 3)$ to {\em any} of these three
blocks will return the {\em same} answer $2$.

The main idea of Fischer's algorithm is to compute an LCA table for
every block, and answer an in-block query directly by looking into its
LCA table.  It is easy to see that there are ${\cal C}_s$ different Cartesian
trees, where ${\cal C}_s$ is number of different binary trees of $s$ nodes. It
is also easy to see that each block can be identified by the shape of
its Cartesian tree, so it can be represented by an index.  For ease of
notation we will refer to this index as its {\em Catalan index}. By
knowing the Catalan index of a block, we can answer an in-block range
maximum query by looking into its corresponding LCA table. Please refer
to Figure~\ref{fig:ancesstor-cartesian} for an illustration.

Fischer's algorithm~\cite{Fischer2006TheoreticalAP} builds least
ancestor tables by choosing $s = \frac{\log n}{4}$ as the block size
for performance reason.  Recall that $C_s = \frac{1}{s+1}\binom{2s}{s}
= O(\frac{4^s}{s^{1.5}})$.  As a result the time to scan data and to
build Cartesian tree and LCA tables is $O(n)$, and the space
requirement is $O(s^2 \frac{4^s}{s^{1.5}}) = O(n)$.  That is, a
sequential Fischer's algorithm requires $O(n)$ time in preprocessing,
and $O(1)$ time to answer a query.  It is easy to see that both
preprocessing time and query answering are optimal.

One drawback of Fischer's algorithm is that it causes {\em serious
  cache miss} when the number of data is large.  Fischer's algorithm
will construct LCA tables for blocks {\em on-demand}.  When the
algorithm finds that the corresponding LCA table is {\em not} in
memory, it will build the LCA table, which will be cached and this may
{\em evict} other LCA tables from cache.  This LCA table building
process repeats for as many times as the number of blocks, and may
cause serious cache misses.

In order to reduce cache miss, Demaine~\cite{Demaine2009OnCT} proposed
{\em cache-aware} operations on Cartesian
tree~\cite{Vuillemin1980AUL}.  Demaine's algorithm does not check if
an LCA table is in memory -- instead it builds LCA table for {\em
  every possible} block.  To do so, Demaine's algorithm uses a binary
string of length $2s$ to identify a block and its Cartesian tree.  The
binary string is encoded in such a way that one can answer an in-block
range maximum query by examining this binary string only.  However,
this examination requires counting the number of 1's {\em between} the
last two 0's, which is {\em hard} to implement efficiently in a modern
computer.

% We replace lookup operation to naive operation.  The naive operation
% is find the maximum value by comparing each element on the compressed
% data. On the other hand, the lookup operation find the index of
% maximum value from the ancestor table. When loading a element from index
% table, it also bring some useless data to caches.   In order to use
% cache efficiently, the naive operation is better than the lookup
% operation because the access pattern is almost one by one in our VGLCS
% algorithm.

\subsection{Rightmost-pops Encoding} \label{sec:cct}

We propose a new encoding for blocks, called {\em rightmost-pops},
instead of the binary string by Demaine~\cite{Demaine2009OnCT}, in order
to improve the performance of range maximum query.  This rightmost-pops
encoding is inspired by Demaine's algorithm and Cartesian tree.

The rightmost-pops encoding encodes a Cartesian tree by keeping only
the {\em rightmost} path of the Cartesian tree in a {\em stack}, and
keeping track of the {\em number of pops} from the stack when we add
data into the Cartesian tree.  Please refer to
Figure~\ref{fig:interval-cartesian} for an illustration.  To maintain
the heap property of the Cartesian tree, when we add the $i$-th data
$a_i$ into the Cartesian tree, we need to {\em pop} the data in the
stack, which stores the rightmost path of the Cartesian tree, that are
{\em smaller} than $a_i$.  We keep popping data until the top of stack
is no less than $a_i$, then we push $a_i$ into the stack.  Let $t_i$
be the number of nodes that need to be popped, and it is easy to see
that $0 \le t_i < s$, where $s$ is the block size.  We use these
$t_i$, the number of pops on the right most path, to encode a
Cartesian tree.

Consider the example in Figure~\ref{fig:interval-cartesian}.  When we
insert $a_1 = 0$, we just insert it into the stack since $a_0 = 1$, no
data is popped, and $t_1$ is $0$.  When we insert $a_2 = 4$ we need to
pop both $a_0$ and $a_1$ out of the stack, since they are smaller than
$a_2$, so $t_2$ is $2$.  As defined earlier, the contents of the stack
is exactly the rightmost path of the Cartesian tree, and we keep these
$t_i$'s to represent a Cartesian tree.

\begin{figure}[!thb]
  \centering
  \includegraphics[width=\linewidth]{\GraphicPath/fig-cartesian-encoding-stack.pdf}
  \caption{Rightmost path in stack}
  \label{fig:interval-cartesian}
\end{figure}


The key idea of rightmost-pops encoding is that we can use $t_i$'s to
{\em implicitly identify} the Cartesian tree of this block of data, and
that we can answer in-block range queries simply by examining these
$t_i$'s.  Suppose we want to answer an in-block maximum query that
ranges from $l$ to $r$ with these $t_i$'s.  We maintain the number of
times data are {\em popped} from the stack in a variable $x$, and
initialize $x$ to 0.  We then loop through $t_l$ to $t_r$ and let index
$j$ run from $l$ to $r$.  Every iteration adds 1 to $x$ then subtracts
$t_j$ from $x$.  We need to remember the index $j$ when $x$ becomes
smaller or equal to 0.  Finally, we report the {\em last} index $j$ when
$x$ becomes smaller or equal to 0, as the answer.  The pseudo code of
the query answering algorithm is in
Algorithm~\ref{alg:cartesian64bits-query}.  All the operations of
Algorithm~\ref{alg:cartesian64bits-query} map directly to machine
instructions so that unlike Demaine's algorithm,
Algorithm~\ref{alg:cartesian64bits-query} is extremely intuitive to
implement.

\begin{algorithm}[H]
\SetAlgoNoLine
\KwIn{$\textit{tmask}$: 64-bits Cartesian tree; $[l, r]$: query range}
\KwOut{$\textit{minIdx}$: the index of the minimum value in interval}
    
$\textit{minIdx} \gets l, \; x \gets 0$ \;
\For{$l \gets l+1$ to $r$} {
  $x \gets x+1 - ((\textit{tmask} \gg (l \ll 2)) \mathrel{\&} 15)$ \;
  \If{$x \le 0$} {
    $\textit{minIdx} \gets l$, $x \gets 0$ \;
  }
}
return $\textit{minIdx}$ \;

  \caption{Range Minimum Query in 64-bits Cartesian Tree}
  \label{alg:cartesian64bits-query}
\end{algorithm}

The correctness proof of Algorithm~\ref{alg:cartesian64bits-query} is
in Theorem~\ref{thm:correctness}.  The intuition is that we only need
to know the root of the tree when the last element in range was
inserted.

\begin{theorem} \label{thm:correctness}
  Algorithm~\ref{alg:cartesian64bits-query} correctly answers an
  in-block range maximum query.
\end{theorem}
\begin{proof}
When $x$, the number of poppings, is {\em smaller than or equal} to 0,
it means the added data has become the {\em root} of the tree of the
queried range.  As a result, when we add a data and it became the root
of the tree for the {\em last} time, the added data is indeed the
maximum among this interval, because according to the heap property, the
root is the maximum among all nodes within this tree.
\end{proof}

We also note that since all $t_i$'s are small than 16, we can represent
each of them as a 4-bit integer.  We then concatenate sixteen of these
4-bit integers into a 64-bit integer to present a Cartesian tree for a
block of sixteen data.  The pseudo code on how to build a 64-bit integer
to represent a block of 16 data is in
Algorithm~\ref{alg:cartesian-to-64bits}, which runs in time $O(s)$.
where $s$ is the block size.  Note that all the operations, e.g., shift,
addition, subtraction, in Algorithm~\ref{alg:cartesian-to-64bits} map
directly to machine instructions and are straightforward to implement.

\begin{algorithm}
\SetAlgoNoLine
\KwIn{$A[1 .. 16]$: input data block}
\KwOut{$t$: a 64 bit rightmost-pops encoding of $A$}

$\tt{LOGS} \gets 4$, $\tt{POWS} \gets 2^{\tt{LOGS}}$ \;
Create an array $D$ of size $\tt{POWS}+1$ \;
$p \gets 0$, $D[0] \gets \infty$ \;

$t \gets 0$ \;
\For{$i \gets 1$ to $\tt{POWS}$} {
  $v \gets A[i], \; c \gets 0$\;
  \While{$D[p] < v$} {
    $p \gets p - 1$, $c \gets c + 1$ \;
  }
  $p \gets p + 1$ \;
  $D[p] \gets v$ \;
  $t \gets t \mathrel{|} (c \ll ((i-1) \ll 2))$ \;
}
return $t$ \;

  \caption{Encode a data block of sixteen data with rightmost-pops
    encoding into a 64-bits integer.}
  \label{alg:cartesian-to-64bits}
\end{algorithm}


We choose the block size as $s = \frac{\log n}{4} = 16$ for performance
reason.  Modern CPUs support 64-bit register and fast operation on them.
When we pack 16 $t_i$ in to a 64-bit integer, we can leverage fast
64-bit instructions and improve performance.  In addition, we do {\em
not} build least common ancestor tables {\em explicitly} since we
implicitly maintain the Cartesian tree information within these 64-bit
$t_i$.  This approach reduces memory usage and improves cache
performance, it can also efficiently answer one-dimension range maximum
query for up to $n = 2^{64}$ data.

Note that our rightmost-pops encoding does improve cache performance,
but will increase the time complexity in answering queries.
Algorithm~\ref{alg:cartesian64bits-query} accesses data in a very
regular manner and has a better data locality than Fischer's algorithm.
The preprocessing time is $O(n)$, as same as in Fischer's algorithm.
However, a single query now needs $O(s)$ time, where $s$ is the block
size $\frac{\log n}{4}$.  This is acceptable in practice since we set
the block size $s$ to $16$ for $n$ up to $2^{64}$, so $s$ is a small
constant.  The space complexity is $O(n)$ as in Fischer's algorithm. The
overall time complexity of the parallel version of our VGLCS algorithm
becomes $O(n^2 \log{n} / p + n \log n)$.  Note that the $\log n$ comes
from the block size $s = O(\log n)$


\section{Query on Incrementally Added Data} \label{sec:QIUD}

This section describes our approach to address the challenges in the
first stage of Algorithm~\ref{alg:parallel-VGLCS}, where we compute
the suffix maximum on every {\em column} of $V$ while new data is
added incrementally.  Here we generalize our technique so that we can
also answer incremental {\em ranged} maximum on incrementally added
data, so that our technique can be applied to other cases that require
ranged maximum query.

% not here, somewhere else
%\begin{table}
  %\tiny
  \centering
  \caption{   Our study shows in the bold front. We use the fixed size
$s=16$ on Cartesian tree. The small amortized constant will not
encounter serious load imbalance problem.   }

  \label{tlb:cmp-complexity}
  \begin{tabular}{ccc}
    \toprule
     & serial & parallel \\
    \midrule
    horizontal & \begin{tabular}{@{}c@{}}
              $\left \langle n \alpha(n) \right \rangle$ \cite{yunghsing} \\ 
              amortized $\left \langle n \right \rangle$\end{tabular}
              & \begin{tabular}{@{}c@{}}
                $\left \langle n \alpha(n)/p + \alpha(n) \right \rangle$ \cite{yunghsing} \\
                amortized $\left \langle n /p + o(1) \right \rangle$
                \end{tabular} \\
    vertical & \begin{tabular}{@{}c@{}}
                $\left \langle n \alpha(n) \right \rangle$ \\
                amortized $\left \langle n \right \rangle$
                \end{tabular}
            & \begin{tabular}{@{}c@{}}
                impossible \\
                amortized $\left \langle n /p + o(1) \right \rangle$
              \end{tabular}
              \\
    total & \begin{tabular}{@{}c@{}}
              $\left \langle n^2 \alpha(n) \right \rangle$ \cite{yunghsing} \\ 
              amortized $\left \langle n^2 \right \rangle$\end{tabular}
          & \begin{tabular}{@{}c@{}}
              impossible \\ 
              amortized $\left \langle n^2 /p + n \log n \right \rangle$\end{tabular} \\
    \bottomrule
  \end{tabular}
\end{table}

\subsection{Build Ancestor Table}

Recall from the discussion of Cartesian tree in
Section~\ref{sec:parallelRMQ} that finding the {\em lowest common
  ancestor} is important for answering ranged maximum queries.  Here
we need to address two issues -- how to map a binary tree into its
{\em Catalan index} and how to find the lowest common ancestor of two
nodes in a given tree.

%% In VGLCS problem, we could not use sorting to improve cache miss
%% because the number of elements and queries are similar.  

\subsubsection{Cartesian Tree Mapping}

Our Cartesian tree mapping lists {\em all} binary search trees in {\em
  lexicographical order} and label them from $0$ to the $n$-th Catalan
number minus 1.  The lexicographical order among binary search trees
of the {\em same} number of nodes is defined {\em recursively} as
follows.  A binary tree $a$ appears {\em before} another binary if $a$
has more nodes than $b$ in the left subtree, or $a$ and $b$ has the
same number of nodes in the left subtree, and $a$'s left subtree
appears before $b$'s left subtree in lexicographical order, or $a$ and
$b$ has the same left subtree, and $a$'s right subtree appears before
$b$'s right subtree in lexicographical order.
Figure~\ref{fig:labelingBST} shows our Catalan indexing of binary
search tree for 1, 2 and 3 nodes.

\begin{figure}[!thb]
  \centering
  \includegraphics[width=\linewidth]{\GraphicPath/fig-bst-encoding.pdf}
  \caption{The labeling of binary search trees}
  \label{fig:labelingBST}
\end{figure}

\subsubsection{Lowest Common Ancestor}

We also need to determine the lowest common ancestor {\em efficiently}
for answering ranged maximum queries on incrementally added data.  Let
$t$ be the Catalan index of the search tree, so $t$ is between 0 and
$s - 1$, where $s$ is the number of search trees.  Let ${\cal A}(s, t,
p, q)$ denote the {\em lowest common ancestor} of the node $p$ and $q$
within a binary search tree with $s$ nodes and Catalan index $t$.  For
example, ${\cal A}(3, 2, 0, 2) = 1$ from Figure~\ref{fig:labelingBST}.
Also we consider the tree of $s$ nodes with label $t$, and let $l_s$
denote the size of the left subtree, $r_s$ denote the size of the
right subtree, $l_t$ be the Catalan index of its left subtree, and
$r_t$ be the Catalan index of its right subtree.  With these notations
we can define the lowest common ancestor {\em recursively} as in
Equation~\ref{fun:LCA1} when $l_s \le p \le q < n$.  Other cases are
defined in Equation~\ref{fun:LCA2}.  A Pseudo code is given in
Algorithm~\ref{alg:parallel-LCA}.

\begin{equation*}
  \begin{split}
    &\mathit{LCA}(n, \mathit{tid}, p, q) \\
      &= \left\{\begin{matrix*}[l]
        \mathit{LCA}(\mathit{lsz}, \mathit{lid}, p, q) &&, p \le q < \mathit{lsz}\\ 
        \mathit{LCA}(\mathit{rsz}, \mathit{rid}, p-\mathit{lsz}-1, q-\mathit{lsz}-1)+\mathit{lsz}+1 &&, 
            \mathit{lsz} \le p \le q < n \\ 
        \mathit{lsz} && , 0 \le p \le \mathit{lsz}, \mathit{lsz} \le q \le i\\ 
        -1 && ,\mathit{otherwise}
      \end{matrix*}\right.
  \end{split}
\end{equation*}

\begin{algorithm}[!thb]
\SetAlgoNoLine
\KwIn{$s$: the maximum tree size}

\For{$n \gets 1$ to $s$} {
  \ForPar{$t \gets 0$ to $C_n - 1$} {
    \ForPar{$p \gets 0$ to $n-1$} {
      Compute $s_l$, $t_l$, $s_r$, and $t_r$ \;
      \For{$q \gets p$ to $n-1$} {
        Compute ${\cal A}[n][t][p][q]$ according to Equation~\ref{fun:LCA} \;
      }
    }
  }
}

\caption{A parallel algorithm that computes the least common ancestor
  table $\cal A$}
\label{alg:parallel-LCA}
\end{algorithm}


We first analyze the space complexity of
Algorithm~\ref{alg:parallel-LCA}.  The lookup table records the all
the binary tree of sizes from 1 to $s$.  When tree size is $m$, the
number of different binary trees is the $n$-th Catalan number $C_m$,
which is $\frac{1}{m+1}\binom{2m}{m} = O(\frac{4^m}{m^{1.5}})$.  For
each binary tree of size $m$, we store the lowest common ancestor of
{\em every} pair of nodes into the table, so the size of the table is
$O(m^2)$.  Therefore, the space complexity is $O(s \times
\frac{1}{s+1}\binom{2s}{s} \times s^2)$, where $s$ is the number of
elements in a block.  When we set $s$ to $\frac{\log n}{4}$, the space
complexity is $O(\sqrt{n} \log ^{1.5} n)$.  Note that the query will
be on the tree size of $s$ {\em only}.  However, we do need the space
for tables of {\em smaller} tree sizes as intermiedate data to compute
the table of tree size $s$.  Also since the number of operations in
Equation~\ref{fun:LCA1} and \ref{fun:LCA2} is a constant, the time
complexity is also $O(\sqrt{n} \log ^{1.5} n)$ when we set $s$ to
$\frac{\log n}{4}$.

We now analyze the time complexity of the parallel version of
Algorithm~\ref{alg:parallel-LCA}.  As deescribed earlier, the
sequential time complexity of Algorithm~\ref{alg:parallel-LCA} is
$O(\frac{s^3}{s+1} \binom{2s}{s})$.  We observe that the computation
of the $C_m$ trees of size $m$ are independent, hence can be done in
parallel.  However, the time to find the sizes and ids of subtrees (in
line 4) is $O(m)$ for a tree of size $m$.  Since both line 4 and 5 are
in the same loop body, it is not necessary to parallel line 4 since
line 5 will dominate the time of the loop body.  As a result we can
only parallelize the double loops in line 2 and 3 in
Algortihm~\ref{alg:parallel-LCA}, and the time complexity of our
parallel algorithm is $O(\frac{s^3}{s+1} \binom{2s}{s} / p + s^2) =
O(\sqrt{n} (\log ^{1.5} n) / p + \log^2 n )$, where $p$ is the number
of processors.


%% % how to compute subtree information from t

%% Note that in line 4 of Algorithm~\ref{alg:parallel-LCA}, when given
%% the tree id $t$, we need to compute the sizes and ids of the left and
%% right subtrees in our encoding.  We can do this in $O(n)$ time, where
%% $n$ is the number of tree nodes.

\subsection{Catalan Index Computation}

Note that Algorithm~\ref{alg:parallel-LCA} requires Catalan index $t$,
so we need to determine $t$ efficiently when given a block of data.

% how to compute t from tree data structure

\subsubsection{Build the Tree}

In order to find the Catalan index of the block, we build a Cartesian
tree corresponding to the elements of the block, and then find the
index of the Cartesian tree.  That is, we build the tree and compute
it from the the sizes and ids of the left and right subtrees.  This
require a recursive traversal on the tree and has a $O(n)$ tgime
complexity, where $n$ is the number of tree nodes.  The conversion is
as in Equation~\ref{fun:tid}.  Recall that $l_s$ denotes the size of
the left subtree, $r_s$ denotes the size of the right subtree, $l_t$
is the Catalan index of the left subtree, and $r_t$ is the Catalan
index of the right subtree.

% \begin{algorithm}[!thb]
  \caption{Get $tid$ from $\langle\mathit{lsz},\mathit{lid},\mathit{rsz},\mathit{rid}\rangle$ in $\theta(1)$ time}
  \label{alg:encode-tid}
  \begin{algorithmic}[1]
    \Require
      $\langle\mathit{lsz},\mathit{lid},\mathit{rsz},\mathit{rid}\rangle$: size and label in left/right subtree
    \Ensure
      $\mathit{tid}$: this label
    \If{$\mathit{rsz} = 0$}
      \State return $\mathit{lid}$
    \EndIf
    \State $n \gets \mathit{lsz}+\mathit{rsz}+1$
    \State $\mathit{base} \gets 0$
    \For{$i \gets 0$ to $\mathit{lsz}-1$}
      \State $\mathit{base} \gets \mathit{base} + C_i \cdot C_{n-i-1}$
    \EndFor
    \State $\mathit{offset} \gets \mathit{lid} \cdot C_{\mathit{rsz}}$ + $\mathit{rid}$
    \State return $\mathit{base}$ + $\mathit{offset}$
  \end{algorithmic}
\end{algorithm}

\begin{eqnarray}  \label{fun:tid}
  {\cal T}({\mathit l}_t, {\mathit l}_s, {\mathit r}_t, {\mathit r}_s)
    = {\mathit l}_t \cdot C_{{\mathit r}_s} + {\mathit r}_t + 
          \sum_{i = 0}^{{\mathit l}_s - 1} C_i C_{{\mathit l}_s + {\mathit r}_s - i}
\end{eqnarray}


We can further optimize Equation~\ref{fun:tid} by pre-computing the
{\em prefix sum} of Catalan numbers.  Then we store these sums in
memory, so that we can use them directly, instead of recomputing them
as in Equation~\ref{fun:tid}.  That is, we can pre-compute these
summation, and replace the summation in Equation~\ref{fun:tid} with a
table lookup.

% how to compute t with rightmost path of the tree

\subsubsection{Keep the Rightmost Path}

The previous computation of Catalan index requires building the tree
to obtain subtree information, and may not be efficient.  We propose a
method that detremines the Catalan index by keeps only the {\em
  rightmost path} in a {\em stack} without building the entire tree.
This technique is similar to the {\em compressed cartesian tree} in
Section~\ref{sec:cct}.  After knowing the Catalan index $t$ we can
compute LCA and answer queries with Algorithm~\ref{alg:parallel-LCA}.

We compute the Catalan index $t$ efficiently by the matintaining its
{\em rightmost path}.  The Cartesian tree for a sequence of data can
be constructed in linear time using a {\em stack} as follows.  The
stack maintains the Catalan indexes and sizes of every left subtree
along the right most path.  That is, we will {\em not} build these
left subtrees, but only keep their Catalan indexes and sizes.

\begin{figure}[!thb]
  \centering
  \includegraphics[width=0.6\linewidth]{\GraphicPath/fig-cartesian-encoding-static.pdf}
  \caption{Compute Catalan index for a tree.}
  \label{fig:fig-cartesian-encoding-static}
\end{figure}


The pseudo code of this Catalan index computation is in
Algorithm~\ref{alg:cartesian-encode-offline}.  Note that each node of
the stack $D$ has three members -- $v$ as the data, $s$ as the size of
its subtree, and $t$ as the index of its left subtree.  We also use a
pointer $p$ to point to the top of the stack.  In the first double
loop the outer loop goes through every input and the inner loop
inserts a data at the {\em end} of the right most path, which is at
the top of the stack $D[p]$, and traverse towards the root by popping
any {\em smaller} data out of the stack $D$.  When we rotate nodes
along the rightmost path to update the Cartesian tree, we compute the
new {\em index} $t$ and size $s$ of the new left subtree whenever the
newly inserted data replaces it.  As a result the new Catalan index
$t$ can be recomputed with Equation~\ref{fun:tid} by the indexes and
sizes of the left and right subtrees in the stack.  Please refer to
the first while loop of Algorithm~\ref{alg:cartesian-encode-offline}
and Figure~\ref{fig:fig-cartesian-encoding-static} for an
illustration.  After popping all smaller data in the stack the while
loop stops and the size, index, and input are pushed into the new top
of stack.  Finally we pop all data out of the stack and compute the
Catalan index for the entire block.

Algorithm~\ref{alg:cartesian-encode-offline} can compute any Catalan
index for Cartesian trees with the entire block of data and the block
size.  The algorithm runs in $O(s)$ time since an elelment is
pushed/popped at most {\em once}.

\begin{algorithm}
\SetAlgoNoLine
\KwIn{
  $A[1 .. n]$: input data block; $n$: the number of elements\;
}
\KwOut{
  $t$: The Catalan index of the input data block
}
Create a stack $D$ of $n+1$ elements.  Every element has $s$, $t$, and $v$ \;
$p \gets 0$, $D[0] \gets \langle 0,0,\infty \rangle$ \;
\For{$i \gets 1$ to $n$} {
  $v \gets A[i]$ \; $s \gets 0$, $t \gets 0$ \;
  \While{$D[p].{v} < v$} {
    $t \gets {\cal T}(D[p].s, D[p].t, s, t)$ \;
    $s \gets s + D[p].s + 1$ \;
    $p \gets p - 1$ \;
  }
  $p \gets p + 1$ \;
  $D[p] \gets \langle s,t,{v}\rangle$ \;
}

$s \gets 0$, $t \gets 0$ \;
\While{$p > 0$} {
  $t \gets {\cal T}(D[p].s, D[p].t, s, t)$ \;
  $s \gets s + D[p].s + 1$ \;
  $p \gets p - 1$ \;
}
return $t$ \;

\caption{Catalan index computation for a data block}
\label{alg:cartesian-encode-offline}
\end{algorithm}


\subsection{Dynamic Catalan Index Computation}


Several encoding methods were proposed for indexing Cartesian search
trees.  Fischer~\cite{Fischer2006TheoreticalAP} introduced the first
encoding method and Masud~\cite{Hasan2010CacheOA} presents a new
encoding method to reduce the number of instructions.  Unfortunately
all these algorithm work in an off-line model, i.e., they assume all
data are given in advance, and cannot cope with incrementally added
data.  In addition, they require a preprocessing of time $O(n)$, where
$n$ is the number of data.  The preprocessing need more memory
transfer to find the information of the block of an input array, or
read external files from disk.

% Morris: Cartesian tree can be used to solve above problem

We can generalize our Catalan index computation technique for
incrementally added data.

It is easy to see that our stack-based
Algorithm~\ref{alg:cartesian-encode-offline} can easily support the
{\tt append} operation on Cartesian tree, as long as we provide a
dynamic encoding method.

That is, we can update all the Catalan index whenever we insert a new
data.  That is, we can {\em dynamically} maintain a lookup table to
obtain the maximum value in a range, so as to answer a ranged query
{\tt query(L, R)}, by our Cartesian tree encoding.


%% Now, we provide the dynamic encoding method so that each operation is
%% amortized $O(1)$ time.

We use five variables to record the state of a Cartesian tree, so as
to support dynamic encoding,

This dynamic encoding method is based on
Algorithm~\ref{alg:parallel-LCA} and Equation~\ref{fun:tid}.

Consider the step to insert the $i$-th element.  Let the index of the
current tree be $t$ and the rightmost path of the Cartesian tree is
presented by two variables -- stack pointer $Dp$ and a stack $D$.  We
first initialize a state set $i$ to be empty, $s$ to $\frac{\log
  n}{4}$, ${t}$ to $C_s - 1$, $Dp$ to 0, and the top of stack $D$ to
infinity.

The structure of state is as follows:

\iffalse
我們定義轉移狀態由 5 個變數來決定動態笛卡爾樹的編碼,當前插入第 $i$ 個
元素,最終填充 $s$ 個元素,當前的樹編號 $\mathit{tid}$,以及笛卡爾樹的
右鏈狀態指針 $Dp$ 與其堆疊 $D$,其結構如下:
\fi

% State(i = 0, s = n, tid = C[n]-1, Dp = 0, D[0].val = INF)

%\begin{minipage}{\linewidth}

\lstinputlisting[frame=single,basicstyle=\tt,caption=State of Cartesian Tree]{\CodePath/cartesian-state.h}

%\end{minipage}

% Morris: Idea for encode Cartesian tree dynamically: virtual node &
% propagation

In order to encode Cartesian tree dynamically, we initialize the $s$
number of virtual node on the rightmost path, and that is why the
default Catalan index ${\mathit t}$ is $C_s - 1$, which $C_s$ is the
$s$-th Catalan number.

Following the elements insertion, we assume the sequence of elements
which is not yet inserted are increasing.  In lexicographical order,
the rightmost path of Cartesian tree is belonged to the lower
dimension in the row-major like.

Simultaneously, building a Cartesian tree only modify the rightmost
path.  We use the propagation to get the Catalan index in the next
insertion.  Finally, we propose the difference
algorithm~\ref{alg:cartesian-encode-online} to satisfy above
requirement.

\iffalse 為了解決在線詢問操作,取 $s = \frac{\log n}{4}$。根據字典順序
的編碼性質,一開始建立虛設點 $s$ 個在右鏈上,其樹編號 $\mathit{tid} =
C_s - 1$ 。隨著插入元素的增加,尚未加入的元素都預設嚴格遞減,加上根據
編碼順序,我們藉由差值來維護在線編碼 (如圖
~\ref{fig:cartesianEncoding})。根據上述的編碼想法,我們得到算法
~\ref{alg:cartesian-encode-online}。\fi

We give an example of difference algorithm in the
figure~\ref{fig:cartesianEncoding}.  Each block has $s$ number of
elements.  We will build a Cartesian tree with $s$ number of nodes to
solve in-block query.  In initialization, it assume $s$ number of
nodes on the rightmost path and the default Catalan index
$\mathit{tid} = C_s - 1$.  When inserting $i$-th element, the Catalan
index is $\mathit{tid}_i$, and the Catalan index of the subtree $A$ is
$A.\mathit{tid}$.  If the value of $(i+1)$-th element is $x$, it will
rotate onto the node $A$.  After rotation, $A$ is a left subtree of
$A$, and we can compute the index of subtree $A$ during rotation.
Then, $x.\mathit{tid}$ can be computed by the $s-(i+1)$ number of
virtual nodes on the rightmost path and $A.\mathit{tid}$.  According
to the lexicographical order, we get $\mathit{tid}_{i+1} =
\mathit{tid}_i + (x.\mathit{tid} - A.\mathit{tid})$.

\begin{algorithm}[!thb]
\SetAlgoNoLine
\KwIn{
      $\mathit{state}$: state of Cartesian Tree\;
      $v$: the value which append to array\;
}
\KwOut{
      $\mathit{tid}$: this label
}

$\textit{Dp} \gets \textit{state}.\textit{Dp}$, $\textit{lsz} \gets 0$, $\textit{lid} \gets 0$ \;
$\textit{bsz} \gets \textit{state}.\textit{s} - \textit{state}.\textit{i} + 1$ \;
$\textit{bid} \gets C[\textit{bsz}] - 1$ \;

\While{$\textit{state}.D[\textit{Dp}].\textit{value} < v$} {
  $\textit{lid} \gets \textit{tid}(\textit{state}.D[\textit{Dp}].\textit{lsz}, \textit{state}.D[\textit{Dp}].\textit{lid}, \textit{lsz}, \textit{lid})$ \;
  $\textit{bid} \gets \textit{tid}(\textit{state}.D[\textit{Dp}].\textit{lsz}, \textit{state}.D[\textit{Dp}].\textit{lid}, \textit{bsz}, \textit{bid})$ \;
  $\textit{lsz} \gets \textit{lsz} + \textit{state}.D[\textit{Dp}].\textit{lsz}+1$ \;
  $\textit{bsz} \gets \textit{bsz} + \textit{state}.D[\textit{Dp}].\textit{lsz}+1$ \;
  $\textit{Dp} \gets \textit{Dp} - 1$ \;
}
$\textit{Dp} \gets \textit{Dp} + 1$ \;
$\textit{state}.D[\textit{Dp}] \gets \left \langle \textit{lsz}, \textit{lid}, \textit{v} \right \rangle$ \;
$\textit{state}.\textit{Dp} \gets \textit{Dp}$ \;
$x.\textit{tid} \gets \textit{tid}(\textit{lsz}, \textit{lid}, \textit{state}.s-\textit{state}.i, C[\textit{state}.s-\textit{state}.i]-1)$ \;
$\textit{state}.\textit{tid} \gets \textit{state}.\textit{tid} + \textit{bid} - x.\textit{tid}$ \;
$\textit{state}.i \gets \textit{state}.i + 1$ \;
return $\textit{state}.\textit{tid}$ \;

  \caption{Online Type of Cartesian Tree}
  \label{alg:cartesian-encode-online}
\end{algorithm}

\begin{figure*}[!thb]
  \centering
  \includegraphics[width=\linewidth]{\GraphicPath/fig-cartesian-encoding.pdf}

  \caption{An example for difference algorithm to encode Cartesian tree.}

  \label{fig:cartesianEncoding}
\end{figure*}

Finally, we do not increase the time complexity of the building
Cartesian tree algorithm because each operation is $O(1)$.  For the in-
block query, we get the index of the Cartesian tree in amortized $O(1)$.

\iffalse
最後,我們不改變原本的建立笛卡爾樹算法,便能在過程中擭得樹的編號,
每一次的 in-block 詢問只需要一次記憶體存取,得到任一操作攤銷複雜度 $\theta(1)$。
\fi


%\section{Implementation Optimization} \label{sec:Implementation}

This section describes the optimization in our implementations of VGLCS
algorithms, both in sequential and parallel environments.  Note that
some of these techniques address hardware characteristics, e.g., cache
behavior, and are not confined to asymptotically analysis only.

\subsection{The Strategy of Disjoint Set Implementation}

We now describe the optimization in the implementations of disjoint
sets, whose applications include VGLCS.

%% There two techniques to efficiently merge two disjoint sets -- {\em
%%   path compression}, and {\em rank strategy}.

\subsubsection{Cache Performance}

The cache is essential to efficient implementation of disjoint sets.
Patwary, Blair, and Manne~\cite{Patwary2010ExperimentsOU} conducted
experiments on disjoint-set and showed that different implementations
have different impact on different levels of caches misses.  In
practice the cache miss is strongly related to how we go from a child
to its ancestors through pointer chasing during path compression.
Usually, an algorithm with a lower time complexity, e.g.,  will have more
``long jumps'' than an algorithm with a higher time complexity.  Here
the long jump means the pointer chasing will go from a memory address
to a far away memory address.  Please refer to
Figure~\ref{fig:long-short-jump-disjoint} for an illustration.

\begin{figure}[!thb]
  \centering \subfigure[An algorithm with a lower time complexity] {
    \includegraphics[width=0.42\linewidth]{\GraphicPath/fig-rem-long-jump.pdf}
  } \subfigure[An algorithm with a higher time complexity] {
    \includegraphics[width=0.42\linewidth]{\GraphicPath/fig-rem-short-jump.pdf}
  }
  \caption{The parent jump in disjoint set}
  \label{fig:long-short-jump-disjoint}
\end{figure}

% what is the full name for RemSP? What does it mean???

The Rem's algorithm ({\sc Rem})~\cite{dijkstra1976a} achieves better
cache performance by a {\em merge-by-index} technique.  Traditional
disjoint set merging techniques are {\em merge-by-rank}\cite{XXX} and
{\em merge-by-size}\cite{XXX}, which determine the root of new tree by
the ranks and the sizes of the two trees respectively.  Despite that
they do have asymptotically better time complexity, put their better
time complexity here.  However, their performance in practice is not
so impressive due to the previously mentioned cache issue.  In
contrast Rem's algorithm assigns an {\em index} to each node, and
merge two disjoint sets according the index of the roots of the two
trees being merged.  That is, the root of the new merged tree will be
the root with the ``larger'' index.

In our experiments we still use merge-by-rank to merge disjoint set
trees.  However, when we have two {\em equally ranked} disjoint set
trees to merge, we will apply the {\em merge-by-index} technique to
break the tie, so as to improve cache performance.  Experiments show
that the merge-by-index technique improves by up to 5 \%.

\subsubsection{Application on VGLCS}

% VGLCS

It is possible to further optimize the implementation of Peng's
sequential algorithm (Algorithm~\ref{alg:serial-VGLCS}) by a {\em lazy
  insertion} technique.  Recall that in
Algorithm~\ref{alg:serial-VGLCS} when the $i$-th characters from one
input string does not match the $j$-th character of the other input
string, we need to place a zero into the dynamic programming table at
position $V[i][j]$.  In practice this this mismatch will happen very
frequently, and causing frequent insertions of zeros into the disjoint
sets.  Each insertion then links the newly inserted singleton node of
zero the previous node of zero that was inserted one iteration ago. 

We implemented an optimization that resolves this repeated insertion,
and linking, of zeros.  Our implementation will scan through a series
of zeros, and locate the next non-zero $V$ element (denoted as $v$) in
the same column, and insert them as a batch.  Since all values in $V$
are non-negative, and hence larger than zero, we can link all zeros to
$v$.  From the experiments we observe that this technique causes less
pointer chasing and updates and does improve performance.

Note that we do {\em not} apply the lazy insertion optimization in our
parallel implementation.  For a multi-core platform, the efficiency of
thread synchronization is essential to the performance.  Since the
threads of a parallel VGLCS algorithm needs to synchronize at every
column of the dynamic programming table in order to process a row, the
lazy insertion, which can ignore this dependency in a sequential
environment, is not beneficial, and is not adopted in our parallel
implementation.

\subsection{Parallel Range Query in VGLCS}

%% %???

%% In the VGLCS problem, the information of range query can be reused a
%% lots of times.  We can reduce the amount of computation for our
%% dynamic programming problem by remove duplicate computation and
%% imposing a boundary limitation.  For example, the logarithm function
%% is often used in the query of a sparse table, and we can preprocessing
%% all the result of the requirements.  Therefore, the reduce-boundary
%% Algorithm~\ref{alg:reduce-boundary} runs $O(n \log n)$ time, which $n$
%% is the length of the input sequence.  It would not increase the time
%% complexity because the VGLCS problem is solved in $O(n^2 / p + n \log
%% n)$, which $p$ is the number of the processors.

%% \iffalse
%% 在 VGLCS 問題中,平行區間詢問的每個區間範圍都是已知的,每一個會重複使用好幾次,
%% 由於已知詢問的區間資訊,我們可以藉由建表範圍縮小,
%% 從算法 \ref{alg:reduce-boundary} 推導邊界來減少計算量。
%% \fi

%% \begin{algorithm}[!thb]
\SetAlgoNoLine
\KwIn{
      $G[1 \cdots n]$: the variable gap constraints\;
}
\KwOut{
      $\textit{limD}$: boundary for doubling algorithm\;
}
Create an array $\textit{limD}[n]$ with all zero elements.\;
  \For{$i \gets n$ to $1$} {
    $\textit{limD}[i] \gets \max(\textit{limD}[i], \lfloor \log_2(\min(G[i]+1, \; i)) \rfloor)$\;
    \For{$k \gets 1$ to $\textit{limD}[i]$} {
      $\textit{limD}[i-2^{k-1}] \gets \max(\textit{limD}[i-2^{k-1}], \; k-1)$\;
    }
  }
return \textit{limD}\;

\caption{Reduce Boundary Dynamic Programming}
\label{alg:reduce-boundary}
\end{algorithm}

We further improve the performance of range query by maintaining
several tables in the blocked sparse table approach described in
Section~\ref{sec:blocked-sparse-table}.  There are three tables --
block maximum $T_S$, prefix maximum $P$, and suffix maximum $S$.  Each
entry in the block maximum table is the maximum within that block.  As
described in Section~\ref{sec:blocked-sparse-table}, we also maintain
a sparse table on $T_S$ so that we can easily answer ranged query on
$B$, which is equal to the maximum within several {\em consecutive}
blocks from the input data.  The prefix maximum table $P$ maintains
the maximum of the prefix within a block, and the suffix maximum table
$S$ maintains the maximum for prefix.  It is easy to see that any
range query can be answered by two queries into the sparse table on
$B$, and one query into prefix $P$, and one into $S$.  For example, in
Figure~\ref{fig:compressed-sp-opt}, the query from index 2 to 18 can
be broken into two queries into the sparse table for $T_S$ -- one from
block 1 to block 2, and one from 2 to 3, and one query for the suffix
of length 2 into the first block , and one query for the prefix of
length 3 into the last block.  Please refer to
Figure~\ref{fig:compressed-sp-opt} for an illustration.

%% The in-block query is a very small probability event due to small $s =
%% \frac{\log n}{4}$.

%% We can use prefix and suffix maximum array to reduce the number of
%% lookup operation on LCA table.

\begin{figure}[!thb]
  \centering \subfigure[The prefix/suffix maximum tables for blocks] {
    \includegraphics[width=0.50\linewidth]{\GraphicPath/fig-compressed-sp-prefix-suffix.pdf}
  } \subfigure[A Sparse Table for $T_S$] {
    \includegraphics[width=0.4\linewidth]{\GraphicPath/fig-sparse-table.pdf}
  } \caption{Block maximum $T_S$, prefix maximum $P$, and suffix
    maximum $S$.}
  \label{fig:compressed-sp-opt}
\end{figure}

We argue that the {\em order} to access these maximum tables is
important.  Our implementation accesses the block maximum {\em first},
then the prefix maximum {\em second}, then the suffix maximum {\em
  last}.  The reasoning for this order is as follows.  Since we need
to access two elements in the sparse table for $T_S$ in {\em the same
  level}, it is very likely they will be in the same cache line, so
access the first will bring in the other automatically by hardware
cache.  For example, in Figure~\ref{fig:compressed-sp-opt}, we will
access both $T_{S}[1][2]$ and $T_{S}[1][3]$ in order to find the
maximum from block 1 to 3.  In addition, our implementation will also
``peek'' into the two neighboring elements $T_{S}[1][1]$ and
$T_{S}[1][4]$, which are very likely also present in cache because
they are in the same level of the sparse table.  If the maximum from
the block maximum, i.e., the maximum of $T_{S}[1][2]$ and
$T_{S}[1][3]$, is already larger than the overall block maximum where
the prefix belongs, i.e., in $T_{S}[1][4]$, then we do not need to
check the $P$ table.  Similarly if If the maximum from the block
maximum and the prefix maximum is larger than the overall block maximum
where the suffix belongs, i.e., in $T_{S}[1][1]$, then we do not need
to check the $S$ table either.  Note that we check for the prefix
before the suffix since the value of the dynamic programming table is
{\em increasing}, so it is more likely that a prefix maximum can save a
check into the suffix table.




  
In our application, we even predict whether the Cartesian tree is
necessary to use for the in-block query. If not, we can reduce time to
compute it. These arrays need extra $O(n)$ space, but improve the
performance on lookup operation.

Algorithm~\ref{alg:rmq-access-order-2e} shows the access order for the
ranged maximum query to reduce the cache miss.  The first level of the
compressed sparse table $T_s[0]$ consider as a small cache which can
provide a probabilistic test to determine whether the load the value
from prefix/suffix maximum array is necessary.

\begin{algorithm}
\SetAlgoNoLine
\KwIn{
  $A$: the input array\; 
  $P, \; S$: the prefix/suffix maximum array for each block of the compressed sparse table \;
  ${\cal T}$: the Catalan index array for each block of the compressed sparse table \;
  $T_s$: the compressed sparse table \; 
  $[l, r]$: ranged query \;
}
\KwOut{
  $v$: the maximum value in $A[l .. r]$ \;
}

\If{$l$ and $r$ in the same block} {
  $i \gets$ Query the ranged maximum query $[l, r]$ on ${\cal T}_\text{block(l)}$ \;
  return $A[i]$ \;
}
$v \gets - \infty$ \;
$l' \gets$ The lower bound of the next block by the position $l$ \;
$r' \gets$ The upper bound of the previous block by the position $r$ \;
\If{$l' \le r'$} {
  $t \gets \lfloor \log_2 (r'-l'+1) \rfloor$ \;
  $SQ_L \gets T_s[t][l' + 2^t - 1]$, $SQ_R \gets T_s[t][r']$ \;
  $v \gets \max(SQ_L, SQ_R)$ \;
}

\If(\tcc*[f]{$T_s[0]$ is a cache to avoid impossible event}){$T_s[0][\text{block}(r)] > v$} {
  $v \gets \max(v, P[r])$ \;
}

\If(\tcc*[f]{Place very low possibility event to the end}){$T_s[0][\text{block}(l)] > v$} {
  $v \gets \max(v, S[l])$ \;
}

return $v$ \;

\caption{Access order of ranged maximum query}
\label{alg:rmq-access-order-2e}
\end{algorithm}




%\section{Experiment} \label{sec:Experiment}

We conduct three sets of experiments.  The first set compares the
performance of blocked and unblocked spare tables in supporting parallel
range maximum query.  The second set evaluates the performance of
incremental suffix maximum query using various data structures,
especially the blocked sparse tables.  Finally, the third experiment
evaluates the effects of different data structures on the overall
performance of VGLCS computation.

We conduct experiments on an Intel Xeon E5-2620 2.4 Ghz processor with
384K bytes of level 1 cache, 1536K bytes of level 2 cache, and 15M
bytes of shared level 3 cache.  The Intel CPU supports
hyper-threading, and each processor has six cores.  The operating
system is Ubuntu 14.04.  We implemented all algorithms in C++ and
OpenMP and compiled them using gcc with {\tt -O2} and {\tt -fopenmp}
flag.

\subsection{Blocked and Unblocked Sparse Table}

We first compare the performance of {\em unblocked} sparse table
(Section~\ref{sec:sparse-table}) and {\em blocked} sparse tables
(Section~\ref{sec:blocked-sparse-table}) for supporting parallel range
maximum query.  For the blocked sparse table, we use the
rightmost-pops encoding instead of the LCA tables, since we will show
that the rightmost-pops encoding is more efficient than LCA table in
the next set of experiments.  We test all possible query range sizes
and the number of range queries is set to be the number of elements
$N$.

Table~\ref{tlb:CORMQ} compares the parallel range query time of
unblocked sparse table and blocked sparse table, and we observe that
the blocked sparse table using rightmost-pops sparse table is more
efficient than the unblocked sparse table when $N$ increases.  For
example, the blocked sparse table is $1.4$ times faster than the
unblocked sparse table when $N$ reaches $100000$.

We believe that the maximum length of interval query ($L$ in
Table~\ref{tlb:CORMQ}) affects the cache performance of accessing a
sparse table.  In addition, even when we fix the number of data $N$,
the speedup of blocked sparse table over unblocked one does increase
as $L$ increases.  We believe that the reason of this speedup is as
follows.  Even with a very large query length, a blocked sparse table
will only need to access a limited range of memory.  For example, the
rightmost-pops encoding only needs to access two elements that are
$\log {N/s}$ apart in memory (the sparse table for the block maximums)
for super block query, and two in-block queries $O(s)$ in consecutive
memory.  In contrast, the unblocked sparse table can have up to $\log
N$ levels, and different queries of different sizes will need to
access different levels of the sparse table.  This does not help cache
locality because consecutive queries will unlikely to access
consecutive memory.  When $N$ is small this will not be problem for
unblocked sparse table since its size will be small.  When $N$
increases the cache locality of blocked sparse table becomes more
significant.

% ($t_i$ the number of pops) 

\begin{table}[!thb]
  %\tiny
  \caption{Total running time (ms) for finding RMQ of different sizes $N$ and maximum interval sizes $L$.}
  \label{tlb:CORMQ}
  \centering
  \begin{tabular}{l c c c c}
    \firsthline
      & \multicolumn{4}{c}{$N$} \\
      \cline{2-5}
        & \multicolumn{2}{c}{$30000$} & $50000$ & $100000$ \\
      $L$ & $2^{10}$ & $2^{15}$ & $2^{15}$ & $2^{15}$ \\
      \hline
      parallel-\tt{RMQ}     & $903$ & $1516$ & $1874$ & $4116$ \\
      parallel-\tt{CORMQ}   & $995$ & $1475$ & $1689$ & $2594$ \\
      parallel-\tt{CORMQ-opt} & $843$ & $1373$ & $1136$ & $1745$ \\
      \hline
      Speedup & $1.07\times$ & $1.10\times$ & $1.64\times$ & $2.35\times$\\
    \lasthline
  \end{tabular}
\end{table}

\subsection{Rightmost-pops and LCA Tables}

We now compare the performance of {\em four} data structures for
supporting {\em incremental suffix} query.  The four data structures
are {\em disjoint set}, {\em unblocked} sparse table, blocked table
with {\em rightmost-pops encoding}, and blocked sparse table with
LCA. Note that since we are testing suffix queries so we can use
disjoint set.  Please refer to Section~\ref{sec:parallelRMQ} for
details. For disjoint set, we implemented both {\em path compression}
and {\em merge-by-rank} strategies so that the amortized time
$O(\alpha(n))$. In our implementation of blocked sparse tables, we set
the block size $s$ to $8$.  All sparse tables are allocated in memory
in a row-major manner to reduce cache miss.  Please refer to
Algorithm~\ref{alg:parallel-VGLCS} and
Figure~\ref{fig:interval-decomposition} for details.

% Here we need a table of all complexity.

We first tried a simple scenario in which we alternate between {\em
  appending} a data and {\em querying} a range.  Both the length and
the position of the range query are uniformly distributed among all
possibilities.  Figure~\ref{fig:fig-ISMQcmp} shows the performance of
the four data structures for supporting incremental suffix maximum
queries under this simple scenario.  We note that our rightmost-pops
encoding implementation runs faster than all other
implementations. For example, it runs faster than the disjoint set
implementation when the number of data $n$ reaches $10^6$.  In
particular, when $n$ is greater than $10^7$, the rightmost-pops
encoding runs $1.8$ times faster than the disjoint set.


\begin{figure}[!thb]
  \centering
  \includegraphics[width=0.75\linewidth]{\GraphicPath/fig-ISMQ.pdf}
  \caption{The performance of the different data structures for
    supporting incremental suffix maximum query on an E5-2620 server}
  \label{fig:fig-ISMQcmp}
\end{figure}

Now we conduct another experiment in a more complex scenario.  In a
dynamic programming various factors affect performance, these factors
include the distribution of values being inserted, the maximum
interval being queried, and the ratio between the numbers of insertion
and queries.  For ease of notation, we use $p$ to denote the
probability of insert a larger next element, $q$ for the probability
of inserting a zero, and $L$ for the maximum interval length being
queried.  In our experiments, we set the number of data $N$ to be
$10^7$, and vary the maximum interval sizes $L$ from 4 to 16, and vary
$p$ and $q$ from 0 to 100\%.  We also set the number of the queries to
be {\em ten} times of the number of insertions.

From Figure~\ref{fig:fig-ISMQcmp}, we observe that blocked sparse
table implementations outperform disjoint set and unblocked sparse
table, so we will focus on comparing the two implementations for
blocked sparse table, i.e., rightmost-pops and LCA tables.  Note that
it is easy to extend rightmost-pops encoding for incremental data
since we only need to maintain the number of pops incrementally for
the last block.

Table~\ref{tlb:ISMQcmp} compares the time for answering incremental
suffix maximum query using rightmost-pops and LCA tables.  Despite
that the LCA table method has a theoretically better amortized $O(1)$
query time, we observe that the rightmost-pops runs up to $1.5$ times
faster than the LCA table.  We believe that there are two reasons for
this.  First, the LCA table implementation requires more instructions
to compute the Catalan index.  In contrast, the rightmost-pops
encoding does not require the computation for tree index.  Instead it
uses the stored number of pops to answer the query directly.  Second,
despite that in Theory~\cite{Fischer2006TheoreticalAP} the optimal
block size is $\frac{\log n}{4}$, this block size is usually too large
for LCA table implementations.  That is, we will need to build a very
large LCA lookup table.  Even worse, we will {\em not} access all of
it, which means it is very unlikely that we will access contiguous
memory, which causes serious cache misses.


\begin{table}[htbp]
  \caption{The timing (in seconds) of answering incremental suffix
    maximum query using rightmost-pops sparse table and the theoretically
    better LCA table sparse table (in bold
    font).} \label{tlb:ISMQcmp} \tiny
  \begin{tabular}{|r|rrrrr|rrrrr|rrrrr|r|} 
    \hline
      & \multicolumn{5}{c|}{$L = 4$} & \multicolumn{5}{c|}{$L=8$} & \multicolumn{5}{c|}{$L=16$} &  \\ 
      \hline 
      \diagbox{$q$}{$p$} & 0\% & 25\% & 50\% & 75\% & 100\% & 0\% & 25\% & 50\% & 75\% & 100\% & 0\% & 25\% & 50\% & 75\% & 100\% & speedup\\
      \hline
      $0\%$ &
            {\bf 1.15} & {\bf 0.89} & {\bf 0.86} & {\bf 0.88} & {\bf 0.91}
          & {\bf 0.88} & {\bf 0.87} & {\bf 0.87} & {\bf 0.85} & {\bf 0.87}   
          & {\bf 1.02} & {\bf 1.00} & {\bf 0.99} & {\bf 1.00} & {\bf 1.02} & 1.56 \\
        & 1.30 & 1.05 & 1.05 & 1.05 & 1.05   & 1.32 & 1.32 & 1.32 & 1.32 & 1.32   & 1.35 & 1.34 & 1.34 & 1.34 & 1.34 & \\ \hline
      $20\%$ & 
            {\bf 0.98} & {\bf 0.95} & {\bf 0.98} & {\bf 0.99} & {\bf 0.96}   
          & {\bf 1.16} & {\bf 1.16} & {\bf 1.19} & {\bf 1.19} & {\bf 1.18}   
          & {\bf 1.24} & {\bf 1.28} & {\bf 1.31} & {\bf 1.25} & {\bf 1.21} & 1.26 \\
        & 1.09 & 1.09 & 1.09 & 1.09 & 1.09   & 1.40 & 1.40 & 1.40 & 1.40 & 1.40   & 1.53 & 1.53 & 1.53 & 1.53 & 1.53 & \\ \hline
      $40\%$ & 
            {\bf 1.01} & {\bf 1.01} & {\bf 1.02} & {\bf 1.02} & {\bf 0.99}   
          & {\bf 1.23} & {\bf 1.24} & {\bf 1.25} & {\bf 1.24} & {\bf 1.21}   
          & {\bf 1.39} & {\bf 1.43} & {\bf 1.45} & {\bf 1.31} & {\bf 1.26} & 1.28 \\
        & 1.13 & 1.13 & 1.13 & 1.13 & 1.12   & 1.46 & 1.46 & 1.47 & 1.47 & 1.45   & 1.62 & 1.62 & 1.62 & 1.62 & 1.61 & \\ \hline
      $60\%$ & 
            {\bf 1.01} & {\bf 1.02} & {\bf 1.04} & {\bf 1.02} & {\bf 0.99}   
          & {\bf 1.23} & {\bf 1.25} & {\bf 1.27} & {\bf 1.26} & {\bf 1.20}   
          & {\bf 1.44} & {\bf 1.48} & {\bf 1.51} & {\bf 1.34} & {\bf 1.26} & 1.28 \\
        & 1.13 & 1.14 & 1.15 & 1.14 & 1.12   & 1.47 & 1.48 & 1.50 & 1.49 & 1.45   & 1.63 & 1.64 & 1.66 & 1.65 & 1.61 & \\ \hline
      $80\%$ & 
            {\bf 0.99} & {\bf 1.01} & {\bf 1.03} & {\bf 1.01} & {\bf 0.97}   
          & {\bf 1.20} & {\bf 1.22} & {\bf 1.25} & {\bf 1.23} & {\bf 1.15}   
          & {\bf 1.38} & {\bf 1.43} & {\bf 1.49} & {\bf 1.31} & {\bf 1.19} & 1.31 \\
        & 1.11 & 1.12 & 1.15 & 1.12 & 1.10   & 1.44 & 1.46 & 1.49 & 1.47 & 1.41   & 1.59 & 1.61 & 1.65 & 1.63 & 1.56 & \\ \hline
      $100\%$ &
            {\bf 0.94} & {\bf 0.96} & {\bf 1.00} & {\bf 0.97} & {\bf 0.91}   
          & {\bf 0.96} & {\bf 1.01} & {\bf 1.01} & {\bf 0.99} & {\bf 1.03}   
          & {\bf 1.09} & {\bf 1.12} & {\bf 1.16} & {\bf 1.34} & {\bf 1.20} & 1.39 \\
        & 1.04 & 1.06 & 1.10 & 1.07 & 1.03   & 1.34 & 1.36 & 1.39 & 1.36 & 1.33   & 1.39 & 1.41 & 1.44 & 1.41 & 1.39 & \\ \hline
  \end{tabular}
\end{table}


We also observe that the probability $p$ affects the performance gain
of our ``peeking'' technique.  When the probability $p$ is close to 0
or 1, the peeking achieves excellent performance gain.  The
performance gain of the peeking operation also depends the block size.
In addition, the best block size for LCA sparse table is different
from the best block size for the rightmost-pops sparse table.

We observe that the performance of the rightmost-pops sparse table is
better than that of LCA sparse table, but the performance gain
decreases when $q$ is close to 1.  LCA sparse table requires the
Cartesian index computation, which is not required in the
rightmost-pops sparse table implementation.  When the probability $q$
is close to 1, LCA sparse table will require less instructions on
computing Cartesian index, so its performance will be close to that of
the the rightmost-pops sparse table.

The maximum length of interval query ($L$ in Table~\ref{tlb:ISMQcmp})
affects the performance of interval query in a rightmost-pops sparse
table, but not the performance of the LCA sparse table.  Note that in
Algorithm~\ref{alg:cartesian64bits-query} the block size $s$ is set to
16, which is the maximum number of times the loop in
Algorithm~\ref{alg:cartesian64bits-query} will iterate for an query.
Therefore, the query time will increase when $L$ increases.  On the
other hand, the LCA sparse table answers a query by a single lookup
into the LCA table, so its performance is not affected by $L$.

\subsection{Variable Gapped Longest Common Subsequence}

We now compare all four data structures in the final experiments and
evaluate their overall performance on VGLCS.  We implemented {\em
  four} combinations of data structures in solving the VGLCS problem
and evaluated their performance.  The first combination is a {\em
  sequential} implementation of Peng's algorithm using disjoint set on
{\em both} the first stage and second stage.  Note that since the
disjoint set only supports suffix query, so the implementation must be
sequential if the second stage uses the disjoint set.  On the other
hand, the other three combinations are all implemented in
parallel. The {\em DS-ST} combination uses disjoint set in the first
stage and the standard {\em unblocked} sparse table in the second
stage.  The {\em DS-BST} combination uses disjoint set in the first
stage and {\em blocked} sparse table with rightmost-pops encoding in
the second stage.  The reason we use rightmost-pops encoding, instead
of LCA tables, was describe earlier.  Finally, the {\em BST2}
combination uses blocked sparse table with rightmost-pops encoding in
both stages.

Figure~\ref{fig:fig-parallel} compares the execution time of all four
data structure combinations under different lengths of inputs.  The
input strings are generated {\em randomly} from the alphabet $\{A, T,
C, G\}$, as this alphabet is most popular in bioinformatics.  We note
that, {\em BST2} outperforms all other parallel implementations. % why

Figure~\ref{fig:fig-parallel-scala} shows the scalability of the best
parallel combination {\em BST2}.  It is {\em eight} times faster than
a serial implementation on our server with 6 cores and
hyper-threading.

\begin{figure}[!thb]
  \centering
  \subfigure[Execution time of four data structure combinations]{
    \includegraphics[width=0.45\linewidth]{\GraphicPath/fig-parallel-n.pdf}
    \label{fig:fig-parallel}
  }
  \subfigure[Scalability of the sparse table with rightmost-pops encoding]{
    \includegraphics[width=0.45\linewidth]{\GraphicPath/fig-parallel-p.pdf}
    \label{fig:fig-parallel-scala}
  }
  \caption{The execution time and scalability results of our parallel
    implementations on an E5-2620 server with 6 cores and
    hyper-threading}
\end{figure}


%\section{Conclusion}
\label{sec:Conclusion}

Our parallel VGLCS algorithm run in $\theta(n \log n)$ time, and we
use sparse table to solve incremental suffix maximum query(ISMQ)
problem.  We provide the amortized sparse table which supports
incremental range maximum query(IRMQ) problem.  Finally, we can use
them to solve VGLCS problem in $\theta(nm)$ time, and also use
$\theta(nm)$ time to solve variable range gapped LCS problem which is
hard than variable gapped LCS problem.

In the practice, we presented easy-to-implements data structure for
const-time IRMQ-retrieval and provide the extra dynamic programming to
reduce the computing boundary, compressed the space to reduce cache
miss, and the auxiliary prefix/suffix array to avoid building
cartesian trees. Finally, the CORMQ-opt run $2.35 \times$ faster than
the origin the algorithm in the parallel environment.

In the incremental range maximum/minimum query, we provide the
theoretical $\theta(n)$ - amortized $\theta(1)$ algorithm by
lexicographical order encoding.

%\input{\PartialPath/feature-en.tex}

\nocite{Rahman2006AlgorithmsFC}
\nocite{Peng2011TheLC}
\nocite{Hasan2010CacheOA}
\nocite{Gabow1983ALA}
\nocite{Fischer2007ANS}

% Bibliography
\bibliographystyle{ACM-Reference-Format}
\bibliography{bibliography}


\end{document}
