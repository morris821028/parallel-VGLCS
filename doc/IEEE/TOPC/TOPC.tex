\documentclass[format=acmsmall, review=false, screen=true]{acmart}

\usepackage{booktabs} % For formal tables

\usepackage[ruled]{../patch/algorithm2e}[2015/10/18] % For algorithms
\renewcommand{\algorithmcfname}{ALGORITHM}
\SetAlFnt{\small}
\SetAlCapFnt{\small}
\SetAlCapNameFnt{\small}
\SetAlCapHSkip{0pt}
\IncMargin{-\parindent}
\newcommand\mycommentfont[1]{\footnotesize\itshape\textcolor{blue}{#1}}
\SetCommentSty{mycommentfont}



%=====================================================================
\usepackage{subfigure}
\makeatletter
\DeclareRobustCommand*\cal{\@fontswitch\relax\mathcal}
\makeatother

\usepackage{listings}
\lstset{
  frame=single,
  language=C,
  basicstyle=\small,
}
\makeatletter
\def\lst@makecaption{%
  \def\@captype{table}%
  \@makecaption
}
\makeatother

\usepackage{diagbox}[2011/11/22]
\usepackage[nice]{nicefrac}
%=====================================================================

\newcommand*{\PartialPath}{../partial}
\newcommand*{\GraphicPath}{../graphics}
\newcommand*{\AlgoPath}{../algorithms}
\newcommand*{\FormulaPath}{../formulas}
\newcommand*{\CodePath}{../codes}
\newcommand*{\TablePath}{../tables}

%=====================================================================

% Metadata Information
\acmJournal{TOPC}
\acmVolume{9}
\acmNumber{4}
\acmArticle{39}
\acmYear{2017}
\acmMonth{10}
\copyrightyear{2017}
%\acmArticleSeq{9}

% Copyright
%\setcopyright{acmcopyright}
%\setcopyright{acmlicensed}
%\setcopyright{rightsretained}
%\setcopyright{usgov}
%\setcopyright{usgovmixed}
%\setcopyright{cagov}
%\setcopyright{cagovmixed}

% DOI
\acmDOI{0000001.0000001}

% Paper history
\received{February 2007}
\received[revised]{March 2009}
\received[accepted]{June 2009}


% Document starts
\begin{document}
% Title portion. Note the short title for running heads 
\title[Parallel VGLCS and IRMQ]{Parallel Variable Gapped Longest Common Sequence and Incremental Range Maximum Query}  
\author{Shiang-Yun Yang}
\orcid{0000-0001-6495-3386}
\author{Pangfeng Liu}
\affiliation{%
  \institution{National Taiwan University}
  \department{Computer Science and Information Engineering}
  \streetaddress{No.1, Sec. 4, Roosevelt Rd., Zhongzheng Dist.}
  \city{Taipei City}
  \postcode{100}
  \country{Taiwan}}
\author{Jan-Jan Wu}
\affiliation{%
  \institution{Institute of Information Science, Academia Sinica}
  \department{School of Engineering}
  \city{Taipei City}
  \country{Taiwan}}



%
% The code below should be generated by the tool at
% http://dl.acm.org/ccs.cfm
% Please copy and paste the code instead of the example below. 
%
\begin{CCSXML}
<ccs2012>
<concept>
<concept_id>10010147.10010169.10010170.10010171</concept_id>
<concept_desc>Computing methodologies~Shared memory algorithms</concept_desc>
<concept_significance>300</concept_significance>
</concept>
</ccs2012>
\end{CCSXML}

\ccsdesc[300]{Computing methodologies~Shared memory algorithms}

%
% End generated code
%

\begin{abstract}

The longest common subsequence problem with variable gapped
constraints (VGLCS) is widely used in genes and molecular biology.  An
$O(nm)$ solution has been proposed in the previous study, by reduction
to the efficient incremental suffix maximum query (ISMQ) problem.
Algorithm for solving ISMQ supports appending a value to array and
querying the suffix maximum value in amortized $O(1)$ time. However,
we try to parallelize origin algorithm by wavefront method, but failed
to achieve better performance.  In this paper, our algorithm and data
structure can achieve a better theoretical time complexity on both
querying and appending.  The VGLCS problem can be solved in $O(nm / p
+ n \log n)$, where $p$ is the number of parallel running processors.
And also, the incremental range maximum query problem can be
solved by sparse table within $O(n)$ -- $O(1)$.

\end{abstract}

%\begin{IEEEkeywords}
%range minimum query,
%incremental range maximum query, incremental suffix maximum query,
%longest common sequence, parallel, cartesian tree
%\end{IEEEkeywords}


% We no longer use \terms command
%\terms{Design, Algorithms, Performance}

\keywords{range minimum query, incremental range maximum query,
incremental suffix maximum query, longest common sequence, parallel,
Cartesian tree}


\thanks{}


\maketitle

% The default list of authors is too long for headers}
\renewcommand{\shortauthors}{S.Y. Yang et al.}

\section{Introduction} %Introduction
\label{sec:Introduction}

% This is a topic sentence by itself.
% say something about sequence alignment

The {\em longest common subsequence} (LCS)~\cite{Hirschberg1975ALS} is
a famous problem in string processing.  For example, the {\tt diff}
utility show the difference between texts by LCS, and the revison
control systems such as SVN/Git use LCS to reconciling multiple
changes.  In bioinformatics, the best-known application of the LCS
problem is the sequence alignment~\cite{mount2001bioinformatics,
Ann2010EfficientAF}, which identify the region of similarity between
the sequences of DNA, RND, or protein.


\iffalse 最長共同子序列 (\emph{longest common subsequence}, LCS) 廣泛
地使用在各個應用上。在多核心平台下,大多數的研究專注於如何高效率地在波
前平行 (wavefront parallelism),而 Jiaoyun Yang ~\cite{jiaoyun} 提出的
論文中改變一般的 LCS 遞迴定義以得到更好快取使用率。在這篇論文中,針對
在 Iliopoulos 和 Rahman ~\cite{iliopoulos} 提及的約束條件下的 LCS 問題
使用相關的想法來改善效能。\fi

% give citation

Iliopoulos and Rahman~\cite{Rahman2006AlgorithmsFC} introduced many
contrainted versions of LCS.  For example, a {\em fixed gap LCS}
(FGLCS) requires that the distance between consecutive characers in
the LCS is at most $k + 1$ characters away.  Fixed gap LCS can be
solved in $O(nm)$, where $n$ and $m$ are the lengths of the two input
strings~\cite{citation}.  On the other hand, a {\em variable   gap
LCS} (VGLCS) requires that each charachter has a {\em gap} value and
two consecutive characters in LCS must be with distance of the gap of
the latter character plus 1.  One can think of the FGLCS as a special
case of VGLCS in which the gap values of all characters are $k$.

\begin{figure}[!thb]
  \centering
  \includegraphics[width=0.8\linewidth]{graphics/fig-VGLCSex.pdf}
  \includegraphics[width=0.8\linewidth]{graphics/fig-VGLCSex2.pdf}
  \caption{An example VGLCS example}    \label{fig:VGLCSex}
\end{figure}

We use an exmaple to illustrte the gap function and VGLCS.  Let string
$A$ be {\tt GCGCAATG} with gap values $(3, 1, 1, 2, 0, 0, 2, 1)$, and
let string $B$ be {\tt GCCCTAGCG} with gap values $(2, 0, 3, 2, 0, 1,
2, 0, 1)$.  Please refer to Figure~\ref{fig:VGLCSex} for an
illustration.  Now the LCS $GCCT$ is a VGLCS because every character
in the LCS can find its predecessor in the LCS with distance at most
its gap value plus 1.

This paper focuses on finding efficient parallel algorithm to solve
VGLS.  Peng~\cite{Peng2011TheLC} gives a $O(nm \alpha(n))$ algorithm
that is easy to implement and a asymptotically better $O(nm)$
algorithm.  Then, we propose our $O(nm)$ algorithm which is easy to
implement and run efficiently in parallel environment.

% then we???

The parallelization of LCS on most multi-core platforms focuses on {\em
wavefront} parallelism.  The wavefront parallelism is motivated by the
recursive solution of LCS.  For example, Jiaoyun
Yang~\cite{Yang2010AnEP} introduced a new formula to exploit more
cache performance.  Our algorithm use more powerful sparse table
instead of the disjoint set in the Peng's serial algorithm and get
better cache performance.

% Then we will ???

\iffalse 在約束條件下的 LCS 中,如 \emph{fixed gap LCS } (FGLCS)要求任
兩個挑選的距離在相對應的另一個字串中相等,同時距離最大為 $k+1$,可在時
間複雜度在 $O(nm)$ 內解決,其中 $n$, $m$ 分別為兩個輸入的字串長度。我
們將在這篇論文針對 \emph{variable gap LCS} (VGLCS) 進行探討。在 VGLCS
中,對各個不同的位置提供約束限制,如目前給定兩個字串 $A =
\tt{GCGCAATG}$, $B = \tt{GCCCTAGCG}$,各自的約束限制為 $G_A = [3, 1,
  1, 2, 0, 0, 2, 1]$ 和 $G_B = [2, 0, 3, 2, 0, 1, 2, 0, 1]$,其中
$G_A(i)$ 表示當挑選第 $i$ 個位置時,與前一個挑選的位置最多差
$G_A(i)+1$,同理 $G_B(i)$;我們可以得到兩組 VGLCS 的解
$\tt{G..C..C..A}$ 和 $\tt{G..C..C..T}$,挑選的方式如圖
~\ref{fig:VGLCSex}。在 Yung-Hsing Peng ~\cite{yunghsing} 的論文已對
VGLCS 提出易於實作的 $O(nm \alpha(n))$ 和理論 $O(nm)$ 的解法。\fi

% Use reference in all section numbering

The remainder of the paper is organized as follows.  In Section
\ref{sec:parallelVGLCS}, we present the parallel algorithm to solve
VGLCS problem.  In Section \ref{sec:parallelRMQ} and
\ref{sec:parallelIRMQ}, we present a new algorithm that be easily
parallelized with a time complexity $O(nm)$, which is better than
previous works.  In Section \ref{sec:Implementation} and
\ref{sec:Experiment}, we describe our optimized implementation and the
result of our experiments. Section xx conclude this paper with lessons
learned and posiible future works.

\iffalse 這一篇論文,我們將在第二 \ref{sec:parallelVGLCS} 節部分將
Yung-Hsing Peng ~\cite{yunghsing} 提出的算法進行平行化。在第三節
~\ref{sec:parallelRMQ},在理論分析上提供易平行且時間複雜度 $O(nm)$ 的
設計。在第四節 ~\ref{sec:Implementation},我們將藉由快取忘卻
(cache-oblivious) 技術,在實作上提供更好的效能。最後,我們總結實驗結果
與理論實務上的差異。\fi



\section{Parallel VGLCS Algorithm} %
\label{sec:parallelVGLCS}

\begin{algorithm*}[!thb]
  \caption{Algorithm for Finding VGLCS}
  \label{alg:serial-VGLCS}
  \begin{algorithmic}[1]
    \Require
      $A, B$: the input string;
      $G_A, G_B$: the array of variable gapped constraints;
    \Ensure Find the LCS with variable gapped constraints
    \State Create $m$ number of data structure $Q[m]$ to support ISMQ problem.
    \State Create an empty table $V[n][m]$.
    \For{$i \gets 1$ to $n$}
      \State Create a data structure $RQ$ to support ISMQ problem.
      \State $r \gets i - (GA[i]+1)$
      \For{$j \gets 1$ to $m$}
        \If{$A[i] = B[j]$}
            \State $t \gets $ query suffix maximum value from position $j - (GB[j]+1)$ to tail in $RQ$.
            \State $V[i][j] \gets t + 1$
            \State $t \gets $ get the suffix maximum value from position $r$ to $i$ in $Q[j]$
            \State Append value $t$ into $RQ$.
            \State Append value $V[i][j]$ into $Q[j]$.
        \Else
            \State $V[i][j] \gets 0$
            \State $t \gets $ get the suffix maximum value from position $r$ to $i$ in $Q[j]$
            \State Append value $t$ into $RQ$.
        \EndIf
      \EndFor
    \EndFor
    \State Retrieve the VGLCS by tracing $V[n][m]$
  \end{algorithmic}
\end{algorithm*}

The serial VGLCS algorithm~\ref{alg:serial-VGLCS} by
Peng~\cite{Peng2011TheLC} and other variants of LCS are difficult to
parallelize.  These algorithms use several states to determine a new
state during dynamic programming.  This construction requires {\em
  heavy data dependency}, and makes it difficult to parallelize the
computation in a naive row-by-row manner.  If we parallel Peng's
algorithm with wavefront method intuitively, it will require extra
space to record all row status, which is not required in the origin
algorithm.  On the other hand, if we use the
Maleki's~\cite{Maleki2016EfficientPU} technique, it also uses extra
space to maintain state translation, and spends more time to merge
split parts.  Therefore, it is crucial that our parallelization
conserves both memory and time.

In order to find VGLCS efficiently, we need to address the {\em
  incremental suffix maximum query} (ISMQ) problem, which was
proposed by Peng~\cite{Peng2011TheLC}.  A data structure that supports
incremental suffix maximum queries should support the following three
operations.  First, a {\tt make} operation creates an empty array $A$.
Second, an {\tt append(V)} operation appends a value $V$ to array $A$.
Finally an ISMQ {\tt query(x)} finds the {\em maximum} value among the
$x$-th value to the end of an array $A$.

The sequential version of Peng's algorithm has two stages.  In the
first stage, the algorithm uses a {\em disjoint set} to efficiently
answer ISMQ to compute the answer on every column, which form an array
$S$.  In the second stage, the algorithm again uses ISMQ on array $S$
obtained from the first stage to get the final answers.

Peng uses a {\em disjoint-set} data structure to answer incremental
suffix maximum queries, and finds a VGLCS with answers from ISMQ. The
disjoint-set data structure by Gabow~\cite{Gabow1983ALA} and
Tarjan~\cite{Tarjan1975EfficiencyOA} solves the {\em union-and-find
  problem}.  The set of data are stored in a sequence of disjoint
sets, and maintain the property that the {\em maximum} of disjoint
sets are at the root and in {\em decreasing} order.  When we add a
value $x$ into the data structure, we put it at the end as a set of
itself.  Then we start joining (with union operation) from the last
set to its previous set until the maximum of the previous set is {\em
  larger}.  It is easy to see that the {\tt query(x)} operation is
simply a {\em find} operation that finds the root, which has the {\em
  maximum}, of the tree that $x$ belongs to.  The amortized time per
union/find operation is $O(\alpha(n))$.

However, a disjoint set implementation of Peng's algorithm is difficult
to parallelize for two reasons.  First, the query in the second stage
will change the data structure because the lookup operation will
compress the path to the root, so it is difficult to maintain a
consistent view of the data structure when multiple processing units are
compressing the path simultaneously.  Second, in the first stage when
multiple processing units are compressing different paths, the load
among them could very different, and incurs load imbalance.  Third, in
the first stage there will be a large number of threads that work on
different part of the disjoint-set forests, therefore it will be
difficult to synchronize them efficiently.

Since the disjoint set cannot support ISMQ efficiently in parallel, we
consider the following data structures to support ISMQ efficiently in
parallel.

\begin{itemize}
  \item Segment tree~\cite{berg2000computational} supports ranged
    maximum query and update in multi-dimensions.  The time complexity
    of both update and query is $O(\log n)$ in one-dimension.
  \item Sparse table~\cite{Berkman1993RecursiveSP} requires a $O(n
    \log n)$ preprocessing, and can support ranged maximum query in
    $O(1)$ time on one dimensional data.  A sparse table is a two
    dimensional array.  The element of a sparse table in the $j$-th
    row and $i$-th column is the maximum among the $i$-th elements
    and its $2^j - 1$ predecessors in the input array.
\end{itemize}

\begin{figure}[!thb]
  \centering \subfigure[Array]{
    \includegraphics[width=\linewidth]{graphics/fig-interval-decomposition-origin.pdf}
    \label{fig:fig-interval-decomposition}
  } \subfigure[Sparse Table]{
    \includegraphics[width=\linewidth]{graphics/fig-sparse-table-origin.pdf}
    \label{fig:fig-sparse-table}
  }
  \caption{A sparse table example}
  \label{fig:interval-decomposition}
\end{figure}

We give an example of the sparse table.  The input is in array $A$. We
split array $A$ into five blocks so that each block has four elements.
The we build a sparse table $ST$ on $A$ as described earlier.  Now a
ranged maximum query on $A$ can be answered by at most {\em two}
queries into the sparse table.  For example, if the query is of the
range from 2 to 13, then the answer is the maximum of the two answers
-- one from 2 to 9, and one from 6 to 13.

We now present a simple version of our parallel VGLCS algorithm.  The
algorithm uses a sparse table and its time complexity is $O(n^2 \log n
/ p + n \log n)$, where $p$ is the number of processors.  In
Section~\ref{sec:parallelIRMQ}, we present a more complicated version
that uses a variant of the sparse table, and runs in $O(n^2 / p + n
\log n)$.

The operations on a sparse table are much easily to parallelize than
those on a disjoint, which is used in the second stage of Peng's
sequential VGLCS algorithm.  The second stage of Peng's algorithm
alternates between append and query operations.  This alternation
between append and query incurs heavy data dependency.  In addition,
the parallelism of operations on a disjoint tree is limited by the
length of path under compression.  The length is usually very short
and provide very limited parallelism.  In contrast our parallel sparse
table implementation in Algorithm~\ref{alg:parallel-sparse-table} runs
in only $O(n \log n / p + \log n)$, where $n$ is the number of
elements and $p$ is the number of processors, and is very easy to
parallelize and to implement.

\begin{algorithm}[!thb]
  \caption{Parallel Sparse Table Algorithm}
  \label{alg:parallel-sparse-table}
  \begin{algorithmic}[1]
    \Require
      $A[0 \cdots N]$: The input array with $N$ elements.
    \Ensure
      $ST$: The sparse table of the input array $A$.
    \State Create two-dimensional $ST[\log N][N]$.
    \State Copy $A$ to $ST[0]$.
    \For{$j \gets 0$ to $\log N$}
      \ParFor{$i \gets 2^j$ to $N$}
        \State $ST[j][i] \gets \max(ST[j-1][i-2^{j}], ST[j-1][i])$
      \EndParFor
    \EndFor
  \end{algorithmic}
\end{algorithm}

The pseudo code of our simple version parallel VGLCS algorithm is in
Algorithm~\ref{alg:parallel-VGLCS}

\begin{algorithm*}[!thb]
  \caption{Parallel Algorithm for Finding VGLCS}
  \label{alg:parallel-VGLCS}
  \begin{algorithmic}[1]
    \Require
      $A, B$: the input string;
      $G_A, G_B$: the array of variable gapped constraints;
    \Ensure Find the LCS with variable gapped constraints
    \State Create $m$ number of data structure $Q[m]$ to support ISMQ problem.
    \State Create an empty table $V[n][m]$.
    \For{$i \gets 1$ to $n$}
      \State Create a sparse table data structure $\textit{sp}$, and initialize $\textit{sp}$ to zero.
      \ParFor{$j \gets 1$ to $m$}
        \State $\textit{sp}[j] \gets$ query suffix maximum value from position $r$ to tail in $Q[j]$.
      \EndParFor
      \State Build sparse table $\textit{sp}$ with $m$ elements in parallel $O(n/p \log n + \log n)$ time.
      \ParFor{$j \gets 1$ to $m$}
        \If{$A[i] = B[j]$}
            \State $t \gets $ query suffix maximum value from position $j - (GB[j] + 1)$ to $j-1$ in $\textit{sp}$
            \State $V[i][j] \gets t + 1$
            \State Append value $V[i][j]$ into $Q[j]$.
        \EndIf
      \EndParFor
    \EndFor
    \State Retrieve the VGLCS by tracing $V[n][m]$
  \end{algorithmic}
\end{algorithm*}


\section{Ranged Maximum Query} \label{sec:parallelRMQ}

In this section we will describe our approach to address the
challenges in the second stage of Algorithm~\ref{alg:parallel-VGLCS},
i.e., an efficient {\em ranged maximum query} on the sparse table $T$.
The ranged maximum query problem is more complicated than the previous
incremental suffix maximum query problem.  Again a {\tt make}
operation creates an empty array $A$, an {\tt append(V)} operation
appends a value $V$ to the end of an array $A$.  Finally, a {\tt
  query(L, R)} operation finds the {\em maximum} value among the
$L$-th value to the $R$-th value of an array $A$.  One can think of
the suffix maximum query as a special case of the ranged maximum
query.

%% We can use a parallel sparse table implementation to answer ranged
%% maximum queries with $p$ processors, so that it requires $O(n \log n /
%% p + \log n)$ in preprocessing, and takes only $O(1)$ time to answer a
%% query.  On the other hand, it is difficult to efficiently parallelize
%% the querying on tree-like data structures, e.g., disjoint sets.

\subsection{Blocked Sparse Table}

\begin{figure}[!thb]   \label{fig:interval-decomposition}
  \centering 
    \includegraphics[width=0.9\linewidth]{\GraphicPath/fig-interval-decomposition.pdf}

    \includegraphics[width=0.9\linewidth]{\GraphicPath/fig-sparse-table.pdf}
  \caption{A Sparse Table}
\end{figure}

We improve the efficiency of our parallel VGLCS algorithm by a {\em
  blocked} sparse table proposed by
Fischer~\cite{Fischer2006TheoreticalAP}.  The blocked approach first
partition the data into $s$ blocks, then it computes the maximum of
each block, and compute a sparse table $T_s$ for these maximums.

The blocked approach answers a ranged maximum query as follows.  We
consider two types of queries -- {\em super block} query and {\em
  in-block} query.  A super block query queries the answer for
consecutive blocks, and a in-block query queries a segment {\em
  within} a block.  It is easy to see that we can answer a super block
query by querying $T_s$ at most {\em twice}.  We can also answer an
in-block queries answered by a {\em single} lookup into an {\em least
  common ancestor table}.  We will provide more details on this later.
Since we can split {\em any} ranged query into at most {\em two}
queries into $T_s$ and {\em two} in-block queries, we need at most
{\em four} memory access to answer any ranged maximum query.

Fischer's algorithm~\cite{Fischer2006TheoreticalAP} scans through the
data within a block, and places the data into a {\em Cartesian tree}.
Each node of the Cartesian tree has a data and the index of this data
in the block.  One can think of this Cartesian tree as a heap where
the data are in {\em heap order}, and the indexes of the data are in
{\em sorted binary search tree order}.  Please refer to
Figure~\ref{fig:ancesstor-cartesian} for an illustration.  As a result
the answer to the in-block ranged maximum query from the $i$-th to the
$j$-th element of a block is located at their {\em least common
  ancestor} in the tree.

\begin{figure}[htbp]   
  \centering
  \includegraphics[width=\linewidth]{\GraphicPath/fig-interval-cartesian.pdf}
  \caption{Least common ancestor tables}
  \label{fig:ancesstor-cartesian}
\end{figure}

Fischer's algorithm computes a {\em least common ancestor table} for
every block.  After scanning the data in a block, the algorithm will
build a Cartesian tree, and is able to answer the in-block ranged
maximum query from the $i$-th to the $j$-th element.  Let the index of
this least common ancestor be $k$. then the algorithm add the key
value pairs $((i, j), k)$ into the least common ancestor table of this
block.  Please refer to Figure~\ref{fig:ancesstor-cartesian} for an
illustration.  That is, one can think of the least common ancestor
table as a mapping table from the in-block query $(i, j)$ to its
answer $k$.  Note that the algorithm does {\em not} maintain the value
the $k$-th element.  Instead it keeps the {\em index}, i.e., $k$, of
the least common ancestor so that two blocks with the {\em same
  relative key order} can {\em share} a least common ancestor table.
For example, the first three blocks in
Figure~\ref{fig:ancesstor-cartesian} can share the same least common
ancestor table because they have their same Cartesian tree.
Consequently an in-block ranged maximum query $(1, 3)$ to {\em any} of
these three blocks will return the {\em same} answer $2$.

The main idea of Fischer's algorithm is to compute an ancestor table
for each block, and answer an in-block query directly by looking into
its ancestor table.  It is easy to see that there are $C_s$ different
Cartesian trees, where $C_s$ is number of different binary trees of
$s$ nodes.  It is also easy to see that each block can be identified
by the shape of its Cartesian tree, so it can be represented by an
index.  For ease of notation we will refer to this index as its {\em
  Catalan index}.  By knowing the Catalan index of a block, we can
answer a in-block ranged maximum query by looking into its
corresponding ancestor table.  Please refer to
Figure~\ref{fig:ancesstor-cartesian} for an illustration.

Fischer's algorithm~\cite{Fischer2006TheoreticalAP} builds ancestor
tables by choosing $s = \frac{1}{4} \log n$ as the block size for
performance reason.  Recall that $C_s = \frac{1}{s+1}\binom{2s}{s} =
O(\frac{4^s}{s^{1.5}})$.  As a result the preprocessing time is
$O(n)$, and the space complexity is $O(s^2 \frac{4^s}{s^{1.5}}) =
O(n)$.  That is, a sequential version requires $O(n)$ time in
preprocessing, and $O(1)$ time in answering a query, and both
preprocessing time and query answering are the best.

Fischer's algorithm causes {\em serious cache miss} when the number of
data is large.  Fischer's algorithm will construct least common
ancestor tables for blocks in an on-demand manner.  When the algorithm
finds that the corresponding least common ancestor table is {\em not}
present in memory, it will build it in memory, which will be cached.
This process will repeat the number of blocks times, and causes
serious cache misses.

In order to reduce cache miss, Demaine~\cite{Demaine2009OnCT} proposed
{\em cache-aware} operations on Cartesian
tree~\cite{Vuillemin1980AUL}.  Demaine's algorithm skips the checking
for existing least common ancestor table, and builds an least common
ancestor table for {\em every} block.  In addition, Demaine's
algorithm uses a binary string of length $2s$ as an identifier of a
block to its Cartesian tree.  The binary string is encoded in such a
way that one can answer an in-block ranged maximum query by examining
this binary string.  However, this operation requires counting the
number of 1's {\em between} the last two 0's, which is {\em hard} to
implement efficiently in a modern computer.

% We replace lookup operation to naive operation.  The naive operation
% is find the maximum value by comparing each element on the compressed
% data. On the other hand, the lookup operation find the index of
% maximum value from the ancestor table. When loading a element from index
% table, it also bring some useless data to caches.   In order to use
% cache efficiently, the naive operation is better than the lookup
% operation because the access pattern is almost one by one in our VGLCS
% algorithm.

\subsection{Compressed Cartesian Tree} \label{sec:cct}

We propose a new encoding for blocks called {\em compressed Cartesian
  tree}, instead of the binary string by
Demaine~\cite{Demaine2009OnCT}, in order to improve performance.  The
compressed sparse table is inspired by Fischer's sparse table and
Cartesian tree.

The compressed sparse table works by keeping only the {\rm rightmost}
path of the Cartesian tree in a {\em stack}.  Please refer to
Figure~\ref{fig:interval-cartesian} for an illustration.  It is easy
to see that when we add the $i$-th data $a_i$ into the Cartesian tree,
we need to {\em pop} the data in the stack, which stores the rightmost
path of the Cartesian tree, that are {\em smaller} than $a_i$.  We
keep popping data until the top of stack is no less than $a_i$, then
we push $a_i$ into the stack.  Let $t_i$ be the number of nodes that
need to be popped, and it is easy to see that $0 \le t_i < s$, where
$s$ is the block size.  

Consider the example in Figure~\ref{fig:interval-cartesian}.  When we
insert $a_1 = 0$, we just insert it into the stack since $a_0 = 1$,
and no data is popped, so $t_1$ is $0$.  When we insert $a_2 = 4$ we
need to pop both $a_0$ and $a_1$ out of the stack, since they are
smaller than $a_2$, so $t_2$ is $2$.  It is easy to see that the
contents of the stack is exactly the rightmost path of the Cartesian
tree.

\begin{figure}[!thb]
  \centering
  \includegraphics[width=\linewidth]{\GraphicPath/fig-cartesian-encoding-stack.pdf}
  \caption{Right most path in stack}
  \label{fig:interval-cartesian}
\end{figure}


The key observation of compressed Cartesian tree is that we can use
$t_i$'s to {\em implicitly identify} the Cartesian tree of this block
of data, so that we can answer in-block ranged queries simply by
examining these $t_i$'s.  More details on how to answer queries with
these $t_i$'s will be described later.  Now since all $t_i$'s are
small than 16, we can represent each of them as a 4-bit integer.  We
then merge 16 of these 4-bit integers into a 64-bit integer to present
a Cartesian tree for a block.  The pseudo code on how to build a 64
bit integer to represent a block of 16 data is in
Algorithm~\ref{alg:cartesian-to-64bits}, which runs in time
$O(s)$. where $s$ is the block size.  Note that all the operations,
e.g., shift, addition, subtraction, in
Algorithm~\ref{alg:cartesian-to-64bits} map directly to machine
instructions and are straightforward to implement.

\begin{algorithm}
\SetAlgoNoLine
\KwIn{$A[1 .. 16]$: input data block}
\KwOut{$t$: a 64 bit right-most-pops encoding of $A$}

$\tt{LOGS} \gets 4$, $\tt{POWS} \gets 2^{\tt{LOGS}}$ \;
Create an array $D$ of size $\tt{POWS}+1$ \;
$p \gets 0$, $D[0] \gets \infty$ \;

$t \gets 0$ \;
\For{$i \gets 1$ to $\tt{POWS}$} {
  $v \gets A[i], \; c \gets 0$\;
  \While{$D[p] < v$} {
    $p \gets p - 1$, $c \gets c + 1$ \;
  }
  $p \gets p + 1$ \;
  $D[p] \gets v$ \;
  $t \gets t \mathrel{|} (c \ll ((i-1) \ll 2))$ \;
}
return $t$ \;

  \caption{Encode a data block of sixteen data with right-most-pops
    encoding into a 64-bits integer.}
  \label{alg:cartesian-to-64bits}
\end{algorithm}


We answer an in-block maximum query that ranges from $l$ to $r$ with
these $t_i$'s as follows.  We maintain the number of times data are
{\em popped} from the stack in a variable $x$, and initialize $x$ to
0.  We then loop through $t_l$ to $t_r$ and let the index run from $l$
to $r$.  Every iteration adds 1 to $x$ then subtracts $t_j$ from $x$.
We need to remember the index when $x$ becomes smaller or equal to 0.
We report the index $j$, the {\em last} index when $x$ becomes smaller
or equal to 0, as the answer.  The pseudo code of the query answering
algorithm is in Algorithm~\ref{alg:cartesian64bits-query}.  The
correctness proof of Algorithm~\ref{alg:cartesian64bits-query} is in
Theorem~\ref{thm:correctness}.  Again all the operations of
Algorithm~\ref{alg:cartesian64bits-query} map directly to machine
instructions so that unlike Demaine's algorithm,
Algorithm~\ref{alg:cartesian64bits-query} is extremely intuitive to
implement.

%\begin{eqnarray}  \label{fun:rmq-stack-pos}
	\text{stack position: } \textit{pos}_i = i - \sum_{j=0}^{i} t_j && t_i \text{ is \#pop operation in $i$-th insertion}
\end{eqnarray}

\begin{eqnarray}  \label{fun:rmq-reduction}
	\max(A[l, r]) = \max(A[k, r])
		&& \text{if } k \in [l, r]\text{ and } \forall \alpha \in [l, r): a_\alpha \le a_k
\end{eqnarray}

\begin{eqnarray}  \label{fun:rmq-reduction-compute}
	\textit{pos}_k \le \textit{pos}_l 
		&\Rightarrow& k - \sum_{j=0}^{k} t_j \le l - \sum_{j=0}^{l} t_j \\
		&\Rightarrow& (k-l) - \sum_{j=l+1}^{k} t_j \le 0
\end{eqnarray}



\begin{algorithm}[H]
\SetAlgoNoLine
\LinesNumbered
\KwIn{$\textit{tmask}$: 64-bits Cartesian tree; $[l, r]$: query range}
\KwOut{$\textit{minIdx}$: the index of the minimum value in interval}
    
$\textit{minIdx} \gets l, \; x \gets 0$ \;
\For{$l \gets l+1$ to $r$} {
  $x \gets x+1 - ((\textit{tmask} \gg (l \ll 2)) \mathrel{\&} 15)$ \;
  \If{$x \le 0$} {
    $\textit{minIdx} \gets l$, $x \gets 0$ \;
  }
}
return $\textit{minIdx}$ \;

  \caption{Range Minimum Query in 64-bits Cartesian Tree}
  \label{alg:cartesian64bits-query}
\end{algorithm}

\begin{theorem} \label{thm:correctness}
  Algorithm~\ref{alg:cartesian64bits-query} correctly answers a
  in-block ranged maximum query.
\end{theorem}
\begin{proof}
The algorithm will correctly answer a in-block ranged maximum query.
When the $x$, the number of poppings, is {\em smaller than or equal}
to 0, it means the added data has become the {\em root} of the subtree
of the queried interval.  As a result when we report that the added
data became the root of the subtree for the {\em last} time, the
reported root is indeed the maximum among this interval, because
according to the heap property, the root is the maximum among the
nodes within this subtree.
\end{proof}

We choose the block size as $s = {{\frac{\log n}{4}}} = 16$ for
performance reason.  Modern CPUs support 64 bit register and fast
operation on them.  When we pack 16 $t_i$ in to a 64 integer, we can
leverage fast 64 bit instructions and improve performance.  In
addition, we do {\em not} build least common ancestor tables {\em
  explicitly} since we implicitly maintain the Cartesian tree
information within these 64 bit $t_i$.  This approach reduces memory
usage and improve cache performance, it also can efficiently answer
one-dimension ranged maximum query for up to $n = 2^{64}$ data.

Note that our compressed Cartesian table approach does improve cache
performance, but will increase the time complexity.
Algorithm~\ref{alg:cartesian64bits-query} access data in a very
regular manner and has a better data locality than Fischer's
algorithm.  The preprocessing time is $O(n)$, as same as in Fischer's
algorithm.  However, a single query now needs $O(s)$ time, where $s$
is the block size.  This is acceptable in practice since we use $s =
16$, which is a small constant.  The space complexity is $O(n)$ as in
Fischer's algorithm.  The overall time complexity of the parallel
version of our VGLCS algorithm becomes $O(n^2 s / p + n \times
\max(\log n, s))$.


\section{Query on Incrementally Updated Data} \label{sec:parallelIRMQ}

Recall that our VGLCS algorithm has two stages.  In the first stage,
the algorithm uses incremental suffix maximum query to compute
intermediate data on every {\em column} of a two dimensional matrix.
In the second stage, the algorithm uses {\em ranged maximum query} on
the rows of the matrix from the first stage to get the final answers.
These two stages work together to find the maximum in a given
rectangle area.

The implementation for two stages have different challenges.  The
first stage is easier to parallelize because the operations on
individual columns are independent.  However, it will insert new data
into the data structure, and it still needs to answer ranged query
efficiently. The second stage does not requires insertion so it is
more static, and easier.  However, it requires working on several
columns simultaneously and synchronously.  Fortunately we have address
this data synchronous issue by computing the index table in a off-line
manner, as described in previous Section~\ref{sec:parallelIRMQ}.  As a
result in this section we will focus on how to answer ranged maximum
queries efficiently on {\em incrementally} updated data.

% not here, somewhere else
%\begin{table}
  %\tiny
  \centering
  \caption{   Our study shows in the bold front. We use the fixed size
$s=16$ on Cartesian tree. The small amortized constant will not
encounter serious load imbalance problem.   }

  \label{tlb:cmp-complexity}
  \begin{tabular}{ccc}
    \toprule
     & serial & parallel \\
    \midrule
    horizontal & \begin{tabular}{@{}c@{}}
              $\left \langle n \alpha(n) \right \rangle$ \cite{yunghsing} \\ 
              amortized $\left \langle n \right \rangle$\end{tabular}
              & \begin{tabular}{@{}c@{}}
                $\left \langle n \alpha(n)/p + \alpha(n) \right \rangle$ \cite{yunghsing} \\
                amortized $\left \langle n /p + o(1) \right \rangle$
                \end{tabular} \\
    vertical & \begin{tabular}{@{}c@{}}
                $\left \langle n \alpha(n) \right \rangle$ \\
                amortized $\left \langle n \right \rangle$
                \end{tabular}
            & \begin{tabular}{@{}c@{}}
                impossible \\
                amortized $\left \langle n /p + o(1) \right \rangle$
              \end{tabular}
              \\
    total & \begin{tabular}{@{}c@{}}
              $\left \langle n^2 \alpha(n) \right \rangle$ \cite{yunghsing} \\ 
              amortized $\left \langle n^2 \right \rangle$\end{tabular}
          & \begin{tabular}{@{}c@{}}
              impossible \\ 
              amortized $\left \langle n^2 /p + n \log n \right \rangle$\end{tabular} \\
    \bottomrule
  \end{tabular}
\end{table}

\subsection{Build a Lookup Table for LCA}

% why LCA is importnat

To find the lowest common ancestor is important for answering ranged
maximum queries, which is important to find VGLCS.  We need to address
two issues -- how to encode a binary tree and how to find the lowest
common ancestor under the given tree encoding.

%% In VGLCS problem, we could not use sorting to improve cache miss
%% because the number of elements and queries are similar.  

\subsubsection{Cartesian Tree Encoding}

Our Cartesian tree encoding method lists {\em all} binary search trees
in {\em lexicographical order} and label them from $0$ to the $n$-th
Catalan number minus 1.  The lexicographical order among binary search
trees of the {\em same} number of nodes is defined {\em recursively}
as follows.  A binary tree $a$ appears {\em before} another binary if
$a$ has more nodes than $b$ in the left subtree, or $a$ and $b$ has
the same number of nodes in the left subtree, and $a$'s left subtree
appears before $b$'s left subtree in lexicographical order, or $a$ and
$b$ has the same left subtree, and $a$'s right subtree appears before
$b$'s right subtree in lexicographical order.
Figure~\ref{fig:labelingBST} shows our labeling of binary search tree
for 1, 2 and 3 nodes.

\begin{figure}[!thb]
  \centering
  \includegraphics[width=\linewidth]{graphics/fig-bst-encoding.pdf}
  \caption{The labeling of binary search trees}
  \label{fig:labelingBST}
\end{figure}

\subsubsection{Lowest Common Ancestor}

We also need to determine the lowest common ancestor efficiently for
answering ranged maximum queries.  For ease of discussion we define
several notations.  Let $t$ be the index of the search tree in our
encoding, so $t$ is between 0 and $s - 1$, where $s$ is the number of
search trees.  Let ${\cal A}(s, t, p, q)$ denote the {\em lowest
  common ancestor} of the node $p$ and $q$ within a binary search tree
with $s$ nodes and label $t$.  For example, ${\cal A}(3, 2, 0, 2) = 1$
from Figure~\ref{fig:labelingBST}.  Also we consider the tree of $s$
nodes with label $t$, and let $l_s$ denote the size of the left
subtree, $r_s$ denote the size of the right subtree, $l_t$ be the
label of its left subtree, and $r_t$ be the label of its right
subtree.  With these notations we can define the lowest common
ancestor {\em recursively} as in Equation~\ref{fun:LCA1} when $l_s
\le p \le q < n$.  Other cases are defined in Equation~\ref{fun:LCA2}.

\begin{equation*}
  \begin{split}
    &\mathit{LCA}(n, \mathit{tid}, p, q) \\
      &= \left\{\begin{matrix*}[l]
        \mathit{LCA}(\mathit{lsz}, \mathit{lid}, p, q) &&, p \le q < \mathit{lsz}\\ 
        \mathit{LCA}(\mathit{rsz}, \mathit{rid}, p-\mathit{lsz}-1, q-\mathit{lsz}-1)+\mathit{lsz}+1 &&, 
            \mathit{lsz} \le p \le q < n \\ 
        \mathit{lsz} && , 0 \le p \le \mathit{lsz}, \mathit{lsz} \le q \le i\\ 
        -1 && ,\mathit{otherwise}
      \end{matrix*}\right.
  \end{split}
\end{equation*}

In order to store all binary search trees, the space complexity of
Algorithm~\ref{alg:parallel-LCA} is in Equation~\ref{eq:space}, and the
time complexity of parallel Algorithm~\ref{alg:parallel-LCA} is in
Equation~\ref{eq:parallel-time}.

% need a formal proof here for the space and time complexity.

The lookup table records the all the binary tree with the number of tree
nodes from 1 to $s$.  When the number of tree nodes is $n$, the number
of rooted binary trees is the $n$-th Catalan number $C_n$, which $C_n =
\frac{1}{n+1}\binom{2n}{n} = O(\frac{4^n}{n^{1.5}})$.  For each binary
tree, we store the lowest common ancestor of every pair of nodes into
table, and the number of pairs is $O(n^2)$.  Therefore, the space
complexity is $O(s \times \frac{1}{s+1}\binom{2s}{s} \times s^2)$
corresponding the number of the answers in different binary trees.  When
pick $s = \frac{\log n}{4}$, the space complexity is $O(n)$.

The Algorithm~\ref{alg:parallel-LCA} compute all the answers.  Because
the number of answers is $O(\frac{s^3}{s+1} \binom{2s}{s})$ and the cost
of translation in formula is $O(1)$.  The sequence version run in
$O(\frac{s^3}{s+1} \binom{2s}{s})$ time.  With $p$ processor, we find
that all the computation in the same number of tree nodes are
independent, but the time complexity of finding
$\langle\mathit{lsz},\mathit{lid},\mathit{rsz},\mathit{rid}\rangle$ is
$O(s)$.  Therefore, the parallel algorithm can be found in $O(s^2)$ time
easily.

\begin{equation} \label{eq:space}
O\left(\frac{s^3}{s+1} \binom{2s}{s}\right) =
O\left(n\right)
\end{equation}

\begin{equation} \label{eq:parallel-time}
O\left(\frac{s^3}{s+1} \binom{2s}{s} \bigg/ p + s^2 \right)
\end{equation}.

% how to compute subtree information from t

Note that in line 4 of Algorithm~\ref{alg:parallel-LCA}, when given
the tree id $t$, we need to compute the sizes and ids of the left and
right subtrees in our encoding.  We can do this in $O(n)$ time, where
$n$ is the number of tree nodes.

\begin{algorithm}[!thb]
  \caption{Parallel Algorithm for building LCA}
  \label{alg:parallel-LCA}
  \begin{algorithmic}[1]
    \Require
      $s$: Maximum size for required the number of BST
    \For{$n \gets 1$ to $s$}
      \ParFor{$\mathit{tid} \gets 0$ to $C_n - 1$}
        \ParFor{$p \gets 0$ to $n-1$}
          \State compute $\langle\mathit{lsz},\mathit{lid},\mathit{rsz},\mathit{rid}\rangle$
          \For{$q \gets p$ to $n-1$}
            \State $\textit{LCA}[n][\mathit{tid}][p][q] \gets$ Equation~\ref{fun:LCA1} and \ref{fun:LCA2}
          \EndFor
        \EndParFor
      \EndParFor
    \EndFor
  \end{algorithmic}
\end{algorithm}


\subsection{Tree Index Computation}

Note that Algorithm~\ref{alg:parallel-LCA} requires tree index $t$
under our encoding scheme, so we need to determine $t$ efficiently,
given a block of data.

% how to compute t from tree data structure

One way to determine the tree index $t$ is to build the tree and
compute it from the the sizes and ids of the left and right subtrees.
This require a recursive traversal on the tree and has a $O(n)$ time
complexity, where $n$ is the number of tree nodes.  The pseudo code of
the conversion is given as Algorithm~\ref{alg:encode-tid}.

\begin{algorithm}
  \caption{Get $tid$ from $\langle\mathit{lsz},\mathit{lid},\mathit{rsz},\mathit{rid}\rangle$ in $\theta(1)$ time}
  \label{alg:encode-tid}
  \begin{algorithmic}[1]
    \Require
      $\langle\mathit{lsz},\mathit{lid},\mathit{rsz},\mathit{rid}\rangle$: size and label in left/right subtree
    \Ensure
      $\mathit{tid}$: this label
    \If{$\mathit{rsz} = 0$}
      \State return $\mathit{lid}$
    \EndIf
    \State $n = \mathit{lsz}+\mathit{rsz}+1$
    \State $\mathit{base} = 0$
    \For{$i=0$ to $\mathit{lsz}-1$}
      \State $\mathit{base}$ = $\mathit{base} + C_i \cdot C_{n-i-1}$
    \EndFor
    \State $\mathit{offset}$ = $\mathit{lid} \cdot C_{\mathit{rsz}}$ + $\mathit{rid}$
    \State return $\mathit{base}$ + $\mathit{offset}$
  \end{algorithmic}
\end{algorithm}

We can further optimize Algorithm~\ref{alg:encode-tid} by
pre-computing the prefix sum of Catalan numbers, and store them in
memory, so that we can use them directly, instead of recomputing them
in Algorithm~\ref{alg:encode-tid}.  That is, we can pre-compute these
offsets, and replace the loop at line 6 of
Algorithm~\ref{alg:encode-tid} with a table lookup.

% how to compute t with rightmost path of the tree

In order to find the index of the block, we build a Cartesian tree
corresponding to the elements of the block, and then find the id of the
Cartesian tree. The previous computation of tree index requires build
the tree to obtain subtree information, and may not be efficient.  We
propose a method that keeps only the {\em rightmost path} in a stack
without building the entire tree.  After knowing tree index $t$ we can
compute LCA and answer queries with Algorithm~\ref{alg:parallel-LCA}.

We compute the tree index $t$ on the rightmost path efficiently. Because
the Cartesian tree for a sequence is constructed in linear time using a
stack-based algorithm, we reserve the $\mathit{lid}, \mathit{lsz}$
instead of making a edge of nodes for each nodes, which these are on the
rightmost path which is stored in the stack.  In the rotate operation of
building Cartesian tree algorithm, we maintain the $\mathit{rid},
\mathit{rsz}$ in the pop operations, and the subtree index $t$ can be
recomputed with Algorithm~\ref{alg:encode-tid} by the $\mathit{rid},
\mathit{rsz}$ and the information $\mathit{lid}, \mathit{lsz}$ on the
stack.

The Algorithm~\ref{alg:cartesian-encode-offline} is shown for offline
encoding any Cartesian tree.

\begin{algorithm*}
  \caption{Offline Type of Cartesian Tree}
  \label{alg:cartesian-encode-of}
  \begin{algorithmic}[1]
    \Require
      $A[1 \cdots s]$: storage array;
      $s$: the number of elements;
    \Ensure
      $\mathit{tid}$: this label
    \State $\langle\mathit{lsz},\mathit{lid},\mathit{value}\rangle$ $D$[$s+1$]
    \State $\textit{Dp} \gets 0$
    \State $D[0] \gets \langle 0,0,\infty \rangle$
    \For{$i \gets 1$ to $s$}
      \State $v \gets A[i], \; \textit{lsz} \gets 0, \; \textit{lid} \gets 0$
      \While{$D[\textit{Dp}].\textit{value} < v$}
        \State $\textit{lid} \gets \textit{tid}(D[\textit{Dp}].\textit{lsz}, D[\textit{Dp}].\textit{lid}, \textit{lsz}, \textit{lid})$
        \State $\textit{lsz} \gets \textit{lsz} + D[\textit{Dp}].\textit{lsz} + 1$
        \State $\textit{Dp} \gets \textit{Dp} - 1$
      \EndWhile
      \State $\textit{Dp} \gets \textit{Dp} + 1$
      \State $D[\textit{Dp}] \gets \langle\mathit{lsz},\mathit{lid},\mathit{v}\rangle$
    \EndFor
    \State $\textit{lsz} \gets 0, \; \textit{lid} \gets 0$
    \While{$\textit{Dp} > 0$} \Comment{pop all elements}
      \State $\textit{lid} \gets \textit{tid}(D[\textit{Dp}].\textit{lsz}, D[\textit{Dp}].\textit{lid}, \textit{lsz}, \textit{lid})$
      \State $\textit{lsz} \gets \textit{lsz} + D[\textit{Dp}].\textit{lsz} + 1$
      \State $\textit{Dp} \gets \textit{Dp} - 1$
    \EndWhile
    \State return $\textit{lid}$
  \end{algorithmic}
\end{algorithm*}

% Morris: remove the cache issue

There is a main issues in answering incremental ranged maximum query --
Cartesian tree encoding.  Fischer introduced the first encoding method
and the Masud~\cite{Hasan2010CacheOA} presents a new encoding method to
reduce the number of instructions.  Unfortunately all these algorithm
work in a off-line model, i.e., they assume all $n$ data are given in
advance, therefore they cannot cope with incrementally updated data in
our problem.  In addition, they require a preprocessing of time $O(n)$.
The preprocessing need more memory transfer to find the information of
the block of an input array, or read external files from disk.

However, the sparse table support range maximum query problem is more
powerful than suffix maximum value problem.

In this section, we provide a solution to make the sparse table to
support the append operation. The problem which is more powerful than
ISMQ problem is named incremental range maximum query (IRMQ).  IRMQ
support three operations.  First, a {\tt make} operation creates an
empty array $A$.  Second, a {\tt append(V) } operation appends a value
$V$ to the end of an array $A$.  Finally, a {\tt query(L, R)}
operation finds the {\em maximum} value among the $L$-th value to the
$R$-th value of an array $A$.

\iffalse

ISMQ 已知解法有二,其一使用並查集在 $O(\alpha(n))$ 解決單一操作,其二
使用樸素的稀疏表在 $O(\log n)$完成插入操作、$O(1)$ 完成詢問操作。其二,
Fischer \cite{fischer} 提出的 $\theta(n)$ -- $\theta(1)$ 無法應用在此,
其原因在於插入元素時,無法動態決定 in-block 的最大值,必須等到整個
in-block 塞滿至預設值才可解決。

在我們的應用中維護後綴最大值,
拓展其操作成為增長區間最大值查找 (\emph{incremental range maximum query}, IRMQ),
其支援兩項操作:
\fi

Now, we provide the dynamic encoding method so that each operation is
amortized $\theta(1)$ time.  First, we need to fully recognize the
formulas encoder and decoder, so that each step in the algorithm is
$\theta(1)$.


\iffalse

接下來的幾段中,我們提供動態的編碼方式使得每一操作皆均攤 $\theta(1)$
完成。首先,我們需要充分認知編碼相互轉換的公式,藉以在算法中完成每一步
皆為 $\theta(1)$ 的要求。

在上一節提出對於任意編號 $\mathit{tid}$ 可以在 $O(n)$ 時間內得到
$\langle\mathit{lsz},\mathit{lid},\mathit{rsz},\mathit{rid}\rangle$;
相反地,可以在 $\theta(1)$ 時間內逆推得到 $\mathit{tid}$,如算法
~\ref{alg:encode-tid}。透過預處理,事先將所有前綴和保存下來,在算法中
的迴圈可視為一次內存存取,使得時間複雜度 $\theta(1)$。\fi




\iffalse
根據先前的字典順序編碼,只需要維護笛卡爾樹的右鏈,實作上與堆疊結構相同。
基於 row-major 順序和遞迴定義 ~\ref{fun:LCA},修改之前論文對於的離線編碼,
其對應方案如算法 \ref{alg:cartesian-encode-offline}。
\fi

In order to support online encoding, we use five variables to present
the state of the Cartesian tree.  The next step will insert $i$-th
elements and final stage fill $s$ number of elements.  The identity of
the current tree is $\mathit{tid}$ and the rightmost path of the
Cartesian tree is presented by two variable, stack pointer
$\mathit{Dp}$ and the stack $\mathit{D}$. The structure of state is as
follows:

\iffalse
我們定義轉移狀態由 5 個變數來決定動態笛卡爾樹的編碼,當前插入第 $i$ 個
元素,最終填充 $s$ 個元素,當前的樹編號 $\mathit{tid}$,以及笛卡爾樹的
右鏈狀態指針 $Dp$ 與其堆疊 $D$,其結構如下:
\fi

\begin{minipage}{0.9\linewidth}
\begin{lstlisting}[frame=single,caption=State of Cartesian Tree]
struct Node {
  int lsz, lid, val;
};
struct State {
  int i, s, tid, Dp;
  struct Node D[s+1];
  State(i = 0, s = n, 
          tid = C[n]-1, Dp = 0,
           D[0].val = INF)
};
\end{lstlisting}
\end{minipage}

For the online query, we choose $s=\frac{\log n}{4}$ by the Fischer's
RMQ.  In our encoding method, we initialize the $s$ number of virtual
node on the rightmost path, so the default tree label $\mathit{tid}$
is $C_s - 1$, which $C_s$ is the $s$-th Catalan number.  Following the
elements insertion, we assume the sequence of elements which is not
yet inserted are increasing.

Because of lexicographical order, the rightmost path of Cartesian tree
is belonged to the lower dimension in the row-major like.
Simultaneously, building a Cartesian tree only modify the rightmost
path. We can use the propagation characteristics to get the identity of
the tree.  Finally, we propose the difference algorithm~\ref{alg
:cartesian-encode-online} to satisfy above requirement.

\iffalse 為了解決在線詢問操作,取 $s = \frac{\log n}{4}$。根據字典順序
的編碼性質,一開始建立虛設點 $s$ 個在右鏈上,其樹編號 $\mathit{tid} =
C_s - 1$ 。隨著插入元素的增加,尚未加入的元素都預設嚴格遞減,加上根據
編碼順序,我們藉由差值來維護在線編碼 (如圖
~\ref{fig:cartesianEncoding})。根據上述的編碼想法,我們得到算法
~\ref{alg:cartesian-encode-online}。\fi

We give an example of difference algorithm in the
figure~\ref{fig:cartesianEncoding}.  Each block has $s$ number of
elements.  We will build a Cartesian tree with $s$ number of nodes to
solve in-block query.  In initialization, it assume $s$ number of
nodes on the rightmost path and the default tree identity
$\mathit{tid} = C_s - 1$.  When inserting $i$-th element, the tree
identity is $\mathit{tid}_i$, and the tree identity of the subtree $A$
is $A.\mathit{tid}$.  If the value of $(i+1)$-th element is $x$, it
will rotate onto the node $A$.  After rotation, $A$ is a left subtree
of $A$, and we can compute the identify of subtree $A$ during
rotation.  Then, $x.\mathit{tid}$ can be computed by the $s-(i+1)$
number of virtual nodes on the rightmost path and
$A.\mathit{tid}$. According to the lexicographical order, we get
$\mathit{tid}_{i+1} = \mathit{tid}_i + (x.\mathit{tid} -
A.\mathit{tid})$.

\begin{algorithm}[!thb]
  \caption{Online Type of Cartesian Tree}
  \label{alg:cartesian-encode-online}
  \begin{algorithmic}[1]
  \Require
      $\mathit{state}$: state of Cartesian Tree;
      $v$: the value which append to array
  \Ensure
      $\mathit{tid}$: this label
  \State $\textit{Dp} \gets \textit{state}.\textit{Dp}$, $\textit{lsz} \gets 0$, $\textit{lid} \gets 0$
  \State $\textit{bsz} \gets \textit{state}.\textit{s} - \textit{state}.\textit{i} + 1$
  \State $\textit{bid} \gets C[\textit{bsz}] - 1$
  \While{$\textit{state}.D[\textit{Dp}].\textit{value} < v$}
    \State $\textit{lid} \gets \textit{tid}(\textit{state}.D[\textit{Dp}].\textit{lsz}, \textit{state}.D[\textit{Dp}].\textit{lid}, \textit{lsz}, \textit{lid})$
    \State $\textit{bid} \gets \textit{tid}(\textit{state}.D[\textit{Dp}].\textit{lsz}, \textit{state}.D[\textit{Dp}].\textit{lid}, \textit{bsz}, \textit{bid})$
    \State $\textit{lsz} \gets \textit{lsz} + \textit{state}.D[\textit{Dp}].\textit{lsz}+1$
    \State $\textit{bsz} \gets \textit{bsz} + \textit{state}.D[\textit{Dp}].\textit{lsz}+1$
    \State $\textit{Dp} \gets \textit{Dp} - 1$
  \EndWhile
  \State $\textit{Dp} \gets \textit{Dp} + 1$
  \State $\textit{state}.D[\textit{Dp}] \gets \left \langle \textit{lsz}, \textit{lid}, \textit{v} \right \rangle$
  \State $\textit{state}.\textit{Dp} \gets \textit{Dp}$
  \State $x.\textit{tid} \gets \textit{tid}(\textit{lsz}, \textit{lid}, \textit{state}.s-\textit{state}.i, C[\textit{state}.s-\textit{state}.i]-1)$
  \State $\textit{state}.\textit{tid} \gets \textit{state}.\textit{tid} + \textit{bid} - x.\textit{tid}$
  \State $\textit{state}.i \gets \textit{state}.i + 1$
  \State return $\textit{state}.\textit{tid}$
  \end{algorithmic}
\end{algorithm}

\begin{figure*}[!thb]
  \centering
  \includegraphics[width=\linewidth]{graphics/fig-cartesian-encoding.pdf}

  \caption{An example for difference algorithm to encode Cartesian tree.}

  \label{fig:cartesianEncoding}
\end{figure*}

Finally, we do not increase the time complexity of the building
Cartesian tree algorithm because each operation is $O(1)$.  For the
in-block query, we get the identity of the Cartesian tree in $O(1)$,
and look up table to find the result.

\iffalse
最後,我們不改變原本的建立笛卡爾樹算法,便能在過程中擭得樹的編號,
每一次的 in-block 詢問只需要一次記憶體存取,得到任一操作攤銷複雜度 $\theta(1)$。
\fi


\section{Implementation}
\label{sec:Implementation}

\subsection{The Strategy of Disjoint Set}

The merging of disjoint set has two main strategies, path compression,
and rank strategy.  Patwary, Blair, and
Manne~\cite{Patwary2010ExperimentsOU} make the experiments for the
disjoint-set data structure, which impact of implementation will
encounter different level of cache miss. The {\textrm RemSP} strategy
tends to fewer cache misses and fewer parent jumps.  Therefore, we
consider above situations to implement our application carefully.

\iffalse
運行 VGLCS 時,將耗費 $\theta(n^2)$ 的內存空間。使用遞增後綴最大值 (ISMQ) 時,
採用並查集實作將會遭遇到很多不平衡的工作負載,其原因在於合併的策略,
常見的有路徑壓縮和啟發式合併兩種策略,這間接影響到不同次數的分枝判斷。
實務上須考慮到快取未中
\fi

In the serial algorithm, we tend to use lazy propagation because of the
cache miss.  Due to the tendency of the dynamic programming, there are
two cases in which the trend of the inserted value.  The first case is
the continuous zero value insert because it violates the definition.
The second case is the insertion of incremental elements.  Finally, we
can use the lazy propagation to improve the performance in
implementation.

For the multi-core platform, the efficiency of thread synchronization is
the important part of the performance, so we tends to make the worst
case as more smaller as possible.  Therefore, we always merge the groups
as possible.

\iffalse
每個執行緒負責數個完整的并查集,操作時應偏向延遲標記操作,
儘早合併的策略易造成快取未中。由於動態規劃的傾向中,插入值的趨勢有兩種情況,
其一為連續不合定義的零元素插入,其二為遞增元素的插入,在這兩者穿插的趨勢中,
我們發現延遲操作將會帶來較能改善快取未中問題。
\fi

\subsection{Parallel Range Query}

We can reduce the amount of computation for our dynamic programming
problem by remove duplicate computation and imposing a boundary
limitation.  For example, the logarithm function is often used in the
query of a sparse table, and we can preprocessing all the result of the
requirements.  In the VGLCS problem, the information of range query is
known for using.  Therefore, the reduce-boundary 
algorithm~\ref{alg:recude-boundary} runs $O(n \log n)$ time, which $n$ 
is the length of the input sequence.  It would not increase the time 
complexity because the VGLCS problem is solved in 
$O(n^2 / p + n \log n)$, which $p$ is the number of the processors.


\iffalse
運行區間查找時,一般依賴內建函數在 $O(1)$ 時間完成對數取整,
然而,在 VGLCS 這類型的動態規劃中,區間查找的對數結果是可以被預測的,預先將每一組詢問的區段對數結果儲存在陣列中,便可降低指令次數。
\fi

\iffalse
由於已知所有詢問區間,建立稀疏表時,可藉由動態規劃在 $O(n \log n)$ 排除掉不可能的計算 (參照算法 ~\ref{alg:reduce-boundary}),
降低過程中的計算量。由於 VGLCS 在平行操作需要 $O(n \log n)$,故使用動態規劃不影響我們的最終結果。
\fi

\begin{algorithm*}[!thb]
  \caption{Reduce Boundary Dynamic Programming}
  \label{alg:reduce-boundary}
  \begin{algorithmic}[1]
  \Require
      $G[1 \cdots n]$: the variabled gap contraints;
  \Ensure
      $\textit{limD}$: boundary for doubling algorithm;
  \State Create an array $\textit{limD}[n]$, and initialize with all zero elements.
  \For{$i \gets n$ to $1$}
    \State $\textit{limD}[i] \gets \max(\textit{limD}[i], \lfloor \log_2(\min(G[i]+1, \; i)) \rfloor)$
    \For{$k \gets 1$ to $\textit{limD}[i]$}
      \State $\textit{limD}[i-2^{k-1}] \gets \max(\textit{limD}[i-2^{k-1}], \; k-1)$
    \EndFor
  \EndFor
  \State return \textit{limD}
  \end{algorithmic}
\end{algorithm*}

Due to small $s = \frac{\log n}{4}$, the in-block query is a very
small probability event.  We can use prefix and suffix maximum array
to instead of the lookup table.  In our application, we even predict
whether the Cartesian tree is necessary to use for the in-block query.
If not, we can reduce time to compute it.  These two arrays brought
$O(n)$ space, but improve the performance by strength reduction.

\iffalse
從機率分佈的角度來看,因 $s = \frac{1}{4} \log n$ 過小,區間詢問完全落於 block 的機率低,
故額外維護區段前綴和後綴最大值 (prefix/suffix maximum value in block) 取代笛卡爾樹的建立。
藉這兩個額外儲存空間,將會增加空間複雜度的常數,卻能有效地降低整體的指令次數。
\fi


\section{Experiment}
\label{sec:Experiment}

We conduct experiments on a Intel Xeon E5-2620 2.4 Ghz processor with
384K bytes of level 1 cache, 1536K bytes of level 2 cache, and 15M
bytes of shared level 3 cache.  The Intel CPU supports
hyper-threading, and each processor has six cores.  The opearating
system is Ubuntu 14.04.  We implemented all algorithms in C++ and
OpenMP and compiled them using gcc with {\tt -O2} and {\tt -fopenmp}
flag.

We implement {\em disjoint set} and {\em sparse table} in our
experiments and evaluate their performance.  Recall that there are three
data structures we can use to answer incremental suffix maximum query.
The first one is {\em disjoint set}, and it has an amortized time
$O(\alpha(n))$ for each query.  Its variant, {\em incremental   tree set
union}~\cite{Gabow1983ALA}, has an amortized time $O(1)$ for each query.
The second data structure is {\em sparse table}.  It takes $O(\log n)$
time to append a new value and $O(1)$ time to answer a query.  Although
the amortized complexity of answering a query with a disjoint set is
$O(\alpha(n))$ or $O(1)$, it causes inefficient synchronizations in a
parallel environment, therefore, we chose to implement the last data
structures and conduct experiments to evaluate their performance.

\subsection{VGLCS with Compressed Sparse Table}

In the random test data, we get the better performance with compressed
sparse table compared to theoretical $O(1)$ sparse table.  The
figure~\ref{fig:fig-parallel} shows the different runtime in different
parallelism, and the figure~\ref{fig:fig-parallel-scala} shows the
scalability of our parallel algorithm.  On our server, the parallel
algorithm can get $8 \times$ faster than the serial algorithm.

\iffalse
我們運行優化策略中的空間壓縮版本,而非理論分析的 $\theta(1)$ 操作,
單次詢問落在 $O(s)$ 中,在實作上由於可以完全壓在暫存器上操作,效能表現較佳。
\fi

\begin{figure}
  \centering
  \subfigure[Runtime]{
    \includegraphics[width=0.45\linewidth]{\GraphicPath/fig-parallel-n.pdf}
    \label{fig:fig-parallel}
  }
  \subfigure[Scalability]{
    \includegraphics[width=0.45\linewidth]{\GraphicPath/fig-parallel-p.pdf}
    \label{fig:fig-parallel-scala}
  }
  \caption{Serial and Parallel Algorithm run on E5-2620}
\end{figure}

%\subsection{理論常數 VGLCS}

%尚未完成

\subsection{Incremental Suffix Maximum Query}

We use the random test cases without any limitation, e.g. the length of
range query has a uniform distribution.  After each append operation,
there is only one query operation in the experiment.  Then, we compare
the performance between the four data structure without parallel
environment as follows:

\iffalse
針對插入和詢問次數相同的 ISMQ 問題,運行以下四種數據結構:
\fi

\begin{itemize}
  \item 

Disjoint set: Average run time $o(\alpha(n))$ which is implemented by
path compression and rank.

  \item 

Sparse table: Insertion in $O(\log n)$ time, and query in $O(1)$ time.
We allocate $\tt{table}[\log N][N]$ which arranged in row-major to
reduce cache miss.  The Algorithm~\ref{alg:parallel-VGLCS} and
Figure~\ref{fig:interval-decomposition} illustrate the method of this
sparse table.

%   \item 

% Binary indexed tree: Both insert and query operation consume $O(\log
% n)$ time.

  \item 

Compressed sparse table: The insert operation consume amortized
$O(1)$ time.  The query operation consumes $O(s)$ time, which $s$ is the
size of the block.  Here, the program implement by the fixed size $s =
16$.  We can maintain extra the prefix and suffix maximum in block to
reduce the time complexity $O(s)$ to $O(1)$ in most queries.  It is introduced in the 

  \item

Amortized $O(1)$ sparse table:  The insert operation consume amortized
$O(1)$ time.  The query operation consume $O(1)$ time.  In the
implementation, we choose the fixed size $s = 8$.  The other
optimization is same as compressed sparse table.

\end{itemize}

\iffalse
\begin{itemize}
  \item 并查集 (Disjoint Set): 平均運行時間 $o(\alpha(n))$。只使用路徑壓縮技巧。
  \item 稀疏表 (Sparse Table): 插入 $O(\log n)$、詢問 $O(1)$。實作陣列宣告採用 $\tt{table}[\log N][N]$ 以減少快取未中。
  \item 樹狀數組 (Binary Indexed Tree): 插入、詢問均為 $O(\log n)$。
  \item 壓縮稀疏表 (Compressed Sparse Tree): 插入均攤 $O(1)$、詢問操作 $O(s)$,
  其中 $s$ 為拆分到區塊大小。實作時,維護區塊前綴和後綴最大值降低詢問複雜度至 $O(1)$,當發生 in-block 詢問再運行 $O(s)$ 算法。
\end{itemize}
\fi

The Figure~\ref{fig:fig-ISMQcmp} illustrate our compressed sparse table
run faster than the disjoint set when $n$ is greater than  $10^6$.  When
$n$ is greater than $10^7$, the compressed sparse table run $1.8 \times$
faster than the disjoint set in random test data.

\begin{figure}[!thb]
  \centering
  \includegraphics[width=\linewidth]{\GraphicPath/fig-ISMQ.pdf}
  \caption{
  The performance of the ISMQ problem with different data structures on 
  E5-2620 server.
  }
  \label{fig:fig-ISMQcmp}
\end{figure}

However, we observe that our amortized $O(1)$ sparse table is more
slow than compressed sparse table.  In the dynamic programming, the
tendency of insertion value will inflict the performance of data
structure, so we experiment the different probability for each data
structure.  The number of the query is ten multiple of the number of
insertion.  The result is shown on table~\ref{tlb:ISMQcmp}.  The
table~\ref{tlb:ISMQcmp} show the performance of the ISMQ with sizes $N
= 10^7$, different maximum interval sizes $L$, the probability $p$ of
incremental elements and the probability $q$ of zero elements. There
are two strategies for each $(p, q, L)$, implemented by compressed
sparse table and amortized $O(1)$ sparse table.  The amortized $O(1)$
sparse table run $1.5 \times$ faster than the compressed table in
above situation.

\iffalse 當運行 $n > 10^6$ 時,我們提出的壓縮稀疏表的效能已經勝過并查
集的版本,其運行結果如圖表 ~\ref{fig:fig-ISMQcmp}。在 $n = 10^7$ 時,
加速 $1.25 \times$。然而,我們提供的 amortized $\theta(1)$ 的稀疏表慢
於并查集,我們做了深入的機率探討 (參照表 ~\ref{tlb:ISMQcmp}),由於大部
分的操作都被區塊後綴和前綴解決,沒有實際運用到內部詢問,約束區間詢問的
大小為 $L$,在 $N = 10^7$ 時,最多能加速 $1.26 \times$,其中插入和詢問
比例為 1:10,當詢問比重更大時,將有更明顯的加速。\fi

\begin{table*}
  \normalsize
  \caption{Total running time (second) for ISMQ with sizes $N = 10^7$, different maximum interval sizes $L$, the probability $p$ of incremental elements and the probility $q$ of zero elements. There are three strategies for each $(p, q, L)$, implemented by disjoint set, compressed sparse table, and O(1) sparse table.}
  \label{tlb:ISMQcmp}
  \centering
  \setlength\tabcolsep{0pt}
  \begin{tabular}{@{\extracolsep{4pt}}r c c c c c c c c c c c c c c c c}
    \firsthline
      & \multicolumn{5}{c}{$L=4$} & \multicolumn{5}{c}{$L=8$} & \multicolumn{5}{c}{$L=16$}\\
      \cline{2-6} \cline{7-11} \cline{12-16}
      $q$ & $0\%$ & $25\%$ & $50\%$ & $75\%$ & $100\%$ 
        & $0\%$ & $25\%$ & $50\%$ & $75\%$ & $100\%$ 
        & $0\%$ & $25\%$ & $50\%$ & $75\%$ & $100\%$ 
        & Speedup\\
      $p$ \\
      \hline
$0\%$ & \begin{tabular}{@{}r@{}} \textbf{1.72}\\1.95\\1.91 \end{tabular}& \begin{tabular}{@{}r@{}} \textbf{1.48}\\1.74\\1.78 \end{tabular}& \begin{tabular}{@{}r@{}} \textbf{1.48}\\1.74\\1.70 \end{tabular}& \begin{tabular}{@{}r@{}} \textbf{1.48}\\1.74\\1.75 \end{tabular}& \begin{tabular}{@{}r@{}} \textbf{1.48}\\1.74\\1.79 \end{tabular}& \begin{tabular}{@{}r@{}} \textbf{1.62}\\2.20\\1.78 \end{tabular}& \begin{tabular}{@{}r@{}} \textbf{1.62}\\2.20\\1.69 \end{tabular}& \begin{tabular}{@{}r@{}} \textbf{1.62}\\2.19\\1.72 \end{tabular}& \begin{tabular}{@{}r@{}} \textbf{1.62}\\2.20\\1.75 \end{tabular}& \begin{tabular}{@{}r@{}} \textbf{1.62}\\2.19\\1.71 \end{tabular}& \begin{tabular}{@{}r@{}} \textbf{1.83}\\2.23\\1.87 \end{tabular}& \begin{tabular}{@{}r@{}} \textbf{1.81}\\2.23\\1.88 \end{tabular}& \begin{tabular}{@{}r@{}} \textbf{1.81}\\2.23\\1.88 \end{tabular}& \begin{tabular}{@{}r@{}} \textbf{1.81}\\2.23\\1.87 \end{tabular}& \begin{tabular}{@{}r@{}} \textbf{1.81}\\2.23\\1.84 \end{tabular}& 0.98\\ \hline
$20\%$ & \begin{tabular}{@{}r@{}} \textbf{1.56}\\2.13\\2.04 \end{tabular}& \begin{tabular}{@{}r@{}} \textbf{1.57}\\2.14\\2.05 \end{tabular}& \begin{tabular}{@{}r@{}} \textbf{1.59}\\2.15\\2.06 \end{tabular}& \begin{tabular}{@{}r@{}} \textbf{1.57}\\2.16\\2.05 \end{tabular}& \begin{tabular}{@{}r@{}} \textbf{1.54}\\2.10\\1.99 \end{tabular}& \begin{tabular}{@{}r@{}} \textbf{1.93}\\3.31\\2.14 \end{tabular}& \begin{tabular}{@{}r@{}} \textbf{2.05}\\3.28\\2.15 \end{tabular}& \begin{tabular}{@{}r@{}}2.26\\3.21\\ \textbf{2.16} \end{tabular}& \begin{tabular}{@{}r@{}} \textbf{2.05}\\3.29\\2.16 \end{tabular}& \begin{tabular}{@{}r@{}} \textbf{1.78}\\3.32\\2.11 \end{tabular}& \begin{tabular}{@{}r@{}}2.66\\3.60\\ \textbf{2.31} \end{tabular}& \begin{tabular}{@{}r@{}}2.77\\3.55\\ \textbf{2.31} \end{tabular}& \begin{tabular}{@{}r@{}}2.92\\3.44\\ \textbf{2.32} \end{tabular}& \begin{tabular}{@{}r@{}}2.67\\3.57\\ \textbf{2.31} \end{tabular}& \begin{tabular}{@{}r@{}}2.45\\3.69\\ \textbf{2.28} \end{tabular}& \textbf{1.26}\\ \hline
$40\%$ & \begin{tabular}{@{}r@{}} \textbf{1.62}\\2.40\\2.12 \end{tabular}& \begin{tabular}{@{}r@{}} \textbf{1.65}\\2.46\\2.14 \end{tabular}& \begin{tabular}{@{}r@{}} \textbf{1.72}\\2.47\\2.16 \end{tabular}& \begin{tabular}{@{}r@{}} \textbf{1.65}\\2.46\\2.14 \end{tabular}& \begin{tabular}{@{}r@{}} \textbf{1.59}\\2.33\\2.00 \end{tabular}& \begin{tabular}{@{}r@{}} \textbf{2.21}\\4.05\\2.33 \end{tabular}& \begin{tabular}{@{}r@{}}2.46\\4.00\\ \textbf{2.34} \end{tabular}& \begin{tabular}{@{}r@{}}2.81\\3.79\\ \textbf{2.36} \end{tabular}& \begin{tabular}{@{}r@{}} \textbf{2.29}\\4.04\\2.35 \end{tabular}& \begin{tabular}{@{}r@{}} \textbf{1.81}\\4.04\\2.25 \end{tabular}& \begin{tabular}{@{}r@{}}2.81\\4.40\\ \textbf{2.50} \end{tabular}& \begin{tabular}{@{}r@{}}2.94\\4.22\\ \textbf{2.54} \end{tabular}& \begin{tabular}{@{}r@{}}3.04\\3.95\\ \textbf{2.53} \end{tabular}& \begin{tabular}{@{}r@{}}2.67\\4.37\\ \textbf{2.52} \end{tabular}& \begin{tabular}{@{}r@{}} \textbf{2.38}\\4.63\\2.46 \end{tabular}& \textbf{1.20}\\ \hline
$60\%$ & \begin{tabular}{@{}r@{}} \textbf{1.65}\\2.45\\2.15 \end{tabular}& \begin{tabular}{@{}r@{}} \textbf{1.71}\\2.58\\2.17 \end{tabular}& \begin{tabular}{@{}r@{}} \textbf{1.82}\\2.67\\2.19 \end{tabular}& \begin{tabular}{@{}r@{}} \textbf{1.69}\\2.52\\2.17 \end{tabular}& \begin{tabular}{@{}r@{}} \textbf{1.60}\\2.24\\2.03 \end{tabular}& \begin{tabular}{@{}r@{}} \textbf{2.29}\\4.30\\2.38 \end{tabular}& \begin{tabular}{@{}r@{}}2.60\\4.31\\ \textbf{2.41} \end{tabular}& \begin{tabular}{@{}r@{}}2.99\\4.08\\ \textbf{2.45} \end{tabular}& \begin{tabular}{@{}r@{}} \textbf{2.20}\\4.36\\2.44 \end{tabular}& \begin{tabular}{@{}r@{}} \textbf{1.75}\\3.98\\2.28 \end{tabular}& \begin{tabular}{@{}r@{}}2.97\\4.67\\ \textbf{2.55} \end{tabular}& \begin{tabular}{@{}r@{}}3.05\\4.51\\ \textbf{2.59} \end{tabular}& \begin{tabular}{@{}r@{}}3.12\\4.22\\ \textbf{2.63} \end{tabular}& \begin{tabular}{@{}r@{}}2.67\\4.81\\ \textbf{2.61} \end{tabular}& \begin{tabular}{@{}r@{}} \textbf{2.33}\\4.56\\2.47 \end{tabular}& \textbf{1.22}\\ \hline
$80\%$ & \begin{tabular}{@{}r@{}} \textbf{1.62}\\2.31\\2.12 \end{tabular}& \begin{tabular}{@{}r@{}} \textbf{1.68}\\2.53\\2.14 \end{tabular}& \begin{tabular}{@{}r@{}} \textbf{1.82}\\2.74\\2.19 \end{tabular}& \begin{tabular}{@{}r@{}} \textbf{1.64}\\2.38\\2.14 \end{tabular}& \begin{tabular}{@{}r@{}} \textbf{1.55}\\1.97\\2.01 \end{tabular}& \begin{tabular}{@{}r@{}} \textbf{2.19}\\4.00\\2.29 \end{tabular}& \begin{tabular}{@{}r@{}}2.57\\4.22\\ \textbf{2.35} \end{tabular}& \begin{tabular}{@{}r@{}}3.00\\4.15\\ \textbf{2.42} \end{tabular}& \begin{tabular}{@{}r@{}} \textbf{2.01}\\4.15\\2.38 \end{tabular}& \begin{tabular}{@{}r@{}} \textbf{1.69}\\3.13\\2.16 \end{tabular}& \begin{tabular}{@{}r@{}}2.99\\4.40\\ \textbf{2.46} \end{tabular}& \begin{tabular}{@{}r@{}}3.04\\4.42\\ \textbf{2.52} \end{tabular}& \begin{tabular}{@{}r@{}}3.15\\4.29\\ \textbf{2.62} \end{tabular}& \begin{tabular}{@{}r@{}}2.59\\4.65\\ \textbf{2.57} \end{tabular}& \begin{tabular}{@{}r@{}} \textbf{2.11}\\3.64\\2.34 \end{tabular}& \textbf{1.24}\\ \hline
$100\%$ & \begin{tabular}{@{}r@{}} \textbf{1.58}\\2.09\\2.03 \end{tabular}& \begin{tabular}{@{}r@{}} \textbf{1.61}\\2.33\\2.09 \end{tabular}& \begin{tabular}{@{}r@{}} \textbf{1.68}\\2.62\\2.16 \end{tabular}& \begin{tabular}{@{}r@{}} \textbf{1.57}\\2.07\\2.08 \end{tabular}& \begin{tabular}{@{}r@{}} \textbf{1.52}\\1.79\\1.79 \end{tabular}& \begin{tabular}{@{}r@{}} \textbf{1.68}\\3.29\\1.99 \end{tabular}& \begin{tabular}{@{}r@{}} \textbf{1.93}\\3.66\\2.04 \end{tabular}& \begin{tabular}{@{}r@{}}2.42\\3.90\\ \textbf{2.10} \end{tabular}& \begin{tabular}{@{}r@{}} \textbf{1.72}\\3.13\\2.04 \end{tabular}& \begin{tabular}{@{}r@{}} \textbf{1.68}\\2.19\\1.81 \end{tabular}& \begin{tabular}{@{}r@{}}2.22\\3.71\\ \textbf{2.19} \end{tabular}& \begin{tabular}{@{}r@{}}2.65\\3.80\\ \textbf{2.25} \end{tabular}& \begin{tabular}{@{}r@{}}2.87\\3.96\\ \textbf{2.33} \end{tabular}& \begin{tabular}{@{}r@{}} \textbf{2.06}\\3.50\\2.25 \end{tabular}& \begin{tabular}{@{}r@{}} \textbf{1.82}\\2.25\\2.02 \end{tabular}& \textbf{1.23}\\ \hline
Speedup & 0.90& 0.83& 0.87& 0.84& 0.85& 0.96& \textbf{1.09}& \textbf{1.24}& 0.98& 0.95& \textbf{1.21}& \textbf{1.21}& \textbf{1.26}& \textbf{1.16}& \textbf{1.07}\\

    \lasthline
  \end{tabular}
\end{table*}

\subsection{Parallel Range Query}

Each stage will have $n$ number of elements and the $n$ number of the
query.  In this special cases, we run the {\tt CORMQ} (compressed RMQ)
to improve the performance compared to the origin.  If we can predict
all range size before building, we get the {\tt CORMQ-opt} which run
$2.35$ faster than origin RMQ.  The table~\ref{tlb:CORMQ} is shown the
result of the strategy of the parallel range query.

\iffalse 每一次有 $n$ 個元素和 $n$ 組詢問,針對這種特殊性質的問題,我
們運行樸素的 \texttt{CORMQ} (compressed RMQ) 得到效能改善,搭配可預測
的分析降低運算量 (參照 \texttt{CORMQ-opt}),得到更好的改善。在
\texttt{CORMQ-opt} 策略中,得到 $2.35 \times$ 倍的加速,結果如表
~\ref{tlb:CORMQ}。\fi

\begin{table*}[!thb]
  %\tiny
  \caption{Total running time (ms) for finding RMQ of different sizes $N$ and maximum interval sizes $L$.}
  \label{tlb:CORMQ}
  \centering
  \begin{tabular}{l c c c c}
    \firsthline
      & \multicolumn{4}{c}{$N$} \\
      \cline{2-5}
        & \multicolumn{2}{c}{$30000$} & $50000$ & $100000$ \\
      $L$ & $2^{10}$ & $2^{15}$ & $2^{15}$ & $2^{15}$ \\
      \hline
      parallel-\tt{RMQ}     & $903$ & $1516$ & $1874$ & $4116$ \\
      parallel-\tt{CORMQ}   & $995$ & $1475$ & $1689$ & $2594$ \\
      parallel-\tt{CORMQ-opt} & $843$ & $1373$ & $1136$ & $1745$ \\
      \hline
      Speedup & $1.07\times$ & $1.10\times$ & $1.64\times$ & $2.35\times$\\
    \lasthline
  \end{tabular}
\end{table*}


\section{Related Work} \label{sec:RelatedWork}

Most of the parallelization of LCS on most multi-core platforms
focuses on {\em wavefront} parallelism.  The wavefront method keeps
the computation as a wavefront that sweep through the entire dynamic
programming tables.  The wavefront computation is {\em not}
cache-friendly, i.e., the wavefront algorithm cannot effectively keep
the required data in cache.  To address this cache issue, Maleki et
al.~\cite{Maleki2016EfficientPU} developed a technique to exploit more
parallelism in the dynamic programming.

The alternative to wavefront method is the traditional row-by-row
approach, in which the dynamic programming tables is built in a
row-by-row manner.  For example, Peng~\cite{Peng2011TheLC} gives a
$O(nm \alpha(n))$ VGLCS algorithm that is easy to implement and an
asymptotically better $O(nm)$ algorithm, where $\alpha$ is the inverse
of Ackermann's function~\cite{Banachowski1980ACT}.

It is difficult to parallelize traditional row-by-row approach for
VGLCS due to the difficulty in efficient suffix and range query in a
parallel environment.  Peng's sequential VGLCS algorithm uses disjoint
sets by Gabow~\cite{Gabow1983ALA} and
Tarjan~\cite{Tarjan1975EfficiencyOA} for suffix maximum query.  We
instead use {\em sparse table}~\cite{Berkman1993RecursiveSP} to
support incremental suffix/range maximum queries in our VGLCS
algorithm.  The sparse is simple to implement and provides sufficient
parallelism for good performance in a parallel environment.

Fischer~\cite{Fischer2006TheoreticalAP} proposed blocked sparse table
for better performance than the unblocked sparse
table~\cite{Berkman1993RecursiveSP}.  We also adopted blocked sparse
table and tested its implementation in our experiments.  Fischer's
algorithm~\cite{Fischer2006TheoreticalAP} builds least ancestor tables
for answering range maximum query.  We instead use a {\em
  right-most-pops} encoding for Cartesian trees.

Demaine~\cite{Demaine2009OnCT} also proposed {\em cache-aware}
operations on Cartesian tree~\cite{Vuillemin1980AUL}, to address the
cache miss issues in Fischer's least common ancestor table
building~\cite{Fischer2006TheoreticalAP}.
Masud~\cite{Hasan2010CacheOA} presents a new encoding method that
reduces the number of instructions.  Our right-most-pops encoding for
Cartesian trees also reduces cache misses, and with a much simpler
implementation than Demaine's encoding.

Finally the authors would like to point out that to the best of our
knowledge, we are not aware of any {\em dynamic} encoding for
Cartesian trees.  All previous works are {\em off-line}, i.e., they
assume all data are given in advance, as a result, they cannot cope
with incrementally added data.  In contrast our dynamic Catalan Index
computation technique in Section~\ref{sec:dynamic} does support
efficient range query on an incremental data sets.


\section{Conclusion}
\label{sec:Conclusion}

Our parallel VGLCS algorithm run in $\theta(n \log n)$ time, and we
use sparse table to solve incremental suffix maximum query(ISMQ)
problem.  We provide the amortized sparse table which supports
incremental range maximum query(IRMQ) problem.  Finally, we can use
them to solve VGLCS problem in $\theta(nm)$ time, and also use
$\theta(nm)$ time to solve variable range gapped LCS problem which is
hard than variable gapped LCS problem.

In practice, we presented easy-to-implements data structure for
constant-time IRMQ-retrieval and provide the extra dynamic programming to
reduce the computing boundary, compressed the space to reduce cache
miss, and the auxiliary prefix/suffix array to avoid building
Cartesian trees. Finally, the CORMQ-opt run $2.35 \times$ faster than
the origin the algorithm in the parallel environment.

In the incremental range maximum query, we provide the theoretical
$\theta(n)$ - amortized $O(1)$ algorithm by lexicographical order
encoding.

\iffalse
我們修改 VGLCS 的序列算法,將其平行化於 $\theta(n \log n)$ 時間內,
並以稀疏表實作 ISMQ 問題。提出的稀疏表能解決比 VGLCS 更困難的 Variable Range Gapped LCS,
致使 VIGLCS 可在時間複雜度 $\theta(nm)$ 被解決。

在實務上,我們提供以動態規劃減少計算量,以及使用空間壓縮降低快取未中的策略,
最終平行 RMQ 獲取 $2.35 \times$ 倍的加速;在增長區間最大值詢問 (IRMQ) 問題中,
以字典順序的編碼策略,提出理論 $\theta(n)$ -- amortized $\theta(1)$  的算法。
\fi

\ifdefined\MasterThesis
\chapter{Future Work}
\else
\section{Future Work}
\fi
\label{sec:Future}

Many studies focused on how to minimize the amount of computation in
the encoding process, such as changing the recursion definition
formula.  We can use the transformation of several register state
instead of the memory access.  These methods are used offline coding.
In the case of the maximum value of the suffix,  our approach still
does not apply all the transformation in registers.  If we can change
the mathematical definition to reduce memory access, the performance
will be more significantly improved.


\nocite{Rahman2006AlgorithmsFC}
\nocite{Peng2011TheLC}
\nocite{Hasan2010CacheOA}
\nocite{Gabow1983ALA}
\nocite{Fischer2007ANS}

% Bibliography
\bibliographystyle{ACM-Reference-Format}
\bibliography{bibliography}


\end{document}
