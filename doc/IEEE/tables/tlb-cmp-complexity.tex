\begin{table*}
  %\tiny
  \centering
  \caption{   Our study shows in the bold front. We use the fixed size
$s=16$ on Cartesian tree. The small amortized constant will not
encounter serious load imbalance problem.   }

  \label{tlb:cmp-complexity}
  \begin{tabular}{rlcc}
    \toprule
    & data structure & serial & parallel \\
    \midrule
    First stage &\begin{tabular}{@{}l@{}}
              disjoint set~\cite{Peng2011TheLC} \\ 
              tree set union~\cite{Gabow1983ALA} \\
              sparse table\end{tabular}
              & \begin{tabular}{@{}c@{}}
              amortized $O(n \alpha(n))$ \\ 
              amortized $O(n)$ \\ 
              amortized $O(n)$\end{tabular}
              & \begin{tabular}{@{}c@{}}
                $O(n \alpha(n)/p + \alpha(n))$ \\
                amortized $O(n /p + 1)$ \\
                amortized $O(n /p + 1)$
                \end{tabular} \\ \hline
    Second stage &\begin{tabular}{@{}l@{}}
              disjoint set~\cite{Peng2011TheLC} \\ 
              tree set union~\cite{Gabow1983ALA} \\
              sparse table\end{tabular}
            & \begin{tabular}{@{}c@{}}
                $O(n \alpha(n))$ \\
                amortized $O(n)$ \\
                amortized $O(n)$
                \end{tabular}
            & \begin{tabular}{@{}c@{}}
                difficult \\
                difficult \\
                amortized $O(n/p + \log n)$
              \end{tabular}
              \\ \hline
    Overall &\begin{tabular}{@{}l@{}}
              disjoint set~\cite{Peng2011TheLC} \\ 
              tree set union~\cite{Gabow1983ALA} \\
              sparse table\end{tabular}
            & \begin{tabular}{@{}c@{}}
              $O(n^2 \alpha(n))$\\ 
              $O(n^2)$\\ 
              amortized $O(n^2)$\end{tabular}
            & \begin{tabular}{@{}c@{}}
              difficult \\ 
              difficult \\ 
              amortized $O(n^2 /p + n \log n)$\end{tabular} \\
    \bottomrule
  \end{tabular}
\end{table*}