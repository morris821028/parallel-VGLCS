\ifdefined\MasterThesis
\chapter{Related Work}
\else
\section{Related Work}
\fi
\label{sec:RelatedWork}

The {\em wavefront method} address this parallelization difficulty by
keeping the computation as wavefront.  However, the wavefront
computation is not cache-friendly, i.e., the wavefront algorithm cannot
effectively keep the required data in cache.  To address this cache
issue, Maleki et al.~\cite{Maleki2016EfficientPU} developed a technique
to exploit more parallelism in the dynamic programming, and we can use
this technique for the issue in our situation.

% still not clear about what you mean by rank convergence

\iffalse

先有 linear-tropical dynamic programming (LTDP) 的性質,
LTDP 可以拆成好幾個序列子問題,每個子問題依賴前一個子問題的最佳解,
我們可以將 LTDP 假想成好幾個子矩陣 (子問題) 的連乘 (合併),
平行方法:
(1) 子問題分開解決,並保留最佳解 
(2) 傳遞前一個子問題的最佳解 
(3) 修正錯誤的區塊,並更新最佳解

當 rank = 1 時,
子問題 u 的最佳解到子問題 v 的最佳解只會經過一個節點 (轉換)。

因此,原先的問題相當於 rank = n,最佳解會經過 n 個點。

The linear-tropical dynamic programming
problem~\cite{Maleki2016EfficientPU} which has the property of {\em
  rank convergence} can be divided into the sub-problems, such as a
sequence of stages, which a solution of the sub-problems depends on
only the solutions of the previous sub-problem.  About rank
convergence in VGLCS problem, the sub-problems has developed efficient
algorithm by Peng.  However, the parallel forward phase is difficult
to integrate the partial correct and incorrect solution in
parallelization algorithm which using rank convergence.
\fi

%need to say
%something abuut
%rank
%convergence
