\section{增長區間最大值詢問} \label{sec:QIUD}

在這一章節中,我們將著重於算法~\ref{alg:parallel-VGLCS} 第一階段的挑戰,
這裡我們將計算 $V$ 每一列的後綴最大值,同時提供泛用性更高的「增長區間最大值詢問」,
增長區間最大值詢問將逐漸地加入新的元素到陣列尾端,並且隨問任意區間的最大值,

\subsection{建立最小共同祖先表}

在章節~\ref{sec:parallelRMQ} 中,我們曾討論如何使用最小共同祖先來回答區間最大值詢問,
這裡我們著重兩個議題如下:第一個問題-「如何給予一個二元搜尋樹相應的『卡塔蘭索引值』」,
以及第二個問題-「如何找到任意兩個樹上節點的共同祖先」

\subsubsection{標記笛卡爾樹}

我們標記笛卡爾樹時,採用「字典順序」({\em lexicographical order}),
從 $0$ 編號到卡塔蘭數的第 $n$ 個值減一。二元索引樹的字典順序參照乙下的遞迴定義:
當一個樹 $x$ 出現在樹 $y$ 前需滿足下述的任意條件,
請參照圖~\ref{fig:labelingBST} 說明節點大小為 1、2、3 的標記情況。

\begin{itemize}
\item $x$ 的左子樹節點個數比 $y$ 的左子數節點個數多
\item $x$ 和 $y$ 的左子樹節點個數相同,且 $x$ 的左子樹字典順序小於 $y$ 的左子樹字典順序。
\item $x$ 和 $y$ 的左子樹節點個數相同且 $x$ 和 $y$ 的左子樹字典順序相同,
  且 $x$ 的右子樹字典順序小於 $y$ 的右子樹字典順序。
\end{itemize}

\begin{figure}[!thb]
  \centering
  \includegraphics[width=0.8\linewidth]{\GraphicPath/fig-bst-encoding.pdf}
  \caption{二元搜尋樹的標記情況,由左而右分別是節點個數為 1、2 和 3}
  \label{fig:labelingBST}
\end{figure}

\subsubsection{最小共同祖先}

隨著加入的資料,我們需要有效率地決定「最小共同祖先」以回答區間最大值。
令 $t$ 為二元搜尋樹的卡塔蘭索引值,則 $t$ 介於 $0$ 到 ${\cal C}_s-1$ 之間,
其中 $s$ 為二元搜尋樹的節點個數。定義 ${\cal A}(s, t, p, q)$ 為在節點個數為 $s$ 時,
卡塔蘭索引值為 $t$ 的二元索引樹上的節點 $p$ 和 $q$ 的最小共同祖先。
例如,從圖~\ref{fig:labelingBST} 中,得知 ${\cal A}(3, 2, 0, 2) = 1$。
令 $s_l$ 為左子樹節點個數、$s_r$ 為右子樹節點個數,
而 $t_l, \; t_r$ 分別為左、右子樹的的卡塔蘭索引值。
藉由上述的定義,得到最小共同祖先 ${\cal A}$ 遞迴如公式~\ref{fun:LCA},
其相應的平行算法~\ref{alg:parallel-LCA}。

\begin{equation*}
  \begin{split}
    &\mathit{LCA}(n, \mathit{tid}, p, q) \\
      &= \left\{\begin{matrix*}[l]
        \mathit{LCA}(\mathit{lsz}, \mathit{lid}, p, q) &&, p \le q < \mathit{lsz}\\ 
        \mathit{LCA}(\mathit{rsz}, \mathit{rid}, p-\mathit{lsz}-1, q-\mathit{lsz}-1)+\mathit{lsz}+1 &&, 
            \mathit{lsz} \le p \le q < n \\ 
        \mathit{lsz} && , 0 \le p \le \mathit{lsz}, \mathit{lsz} \le q \le i\\ 
        -1 && ,\mathit{otherwise}
      \end{matrix*}\right.
  \end{split}
\end{equation*}

\begin{algorithm}[!thb]
\SetAlgoNoLine
\KwIn{$s$: The maximum tree size}

\For{$n \gets 1$ to $s$} {
  \ForPar{$t \gets 0$ to $C_n - 1$} {
    \ForPar{$p \gets 0$ to $n-1$} {
      Compute $s_l$, $t_l$, $s_r$, and $t_r$ \;
      \For{$q \gets p$ to $n-1$} {
        Compute ${\cal A}[n][t][p][q]$ according to Equation~\ref{fun:LCA1} and \ref{fun:LCA2} \;
      }
    }
  }
}

\caption{A parallel algorithm that computes the least common ancesstor table}
\label{alg:parallel-LCA}
\end{algorithm}


分析算法~\ref{alg:parallel-LCA} 的空間複雜度,從表 ${\cal A}$ 維護所有節點大小 $1$ 到 $s$,
當子樹節點個數為 $m$ 時,不同的二元搜尋樹個數為第 $m$ 個卡塔蘭數 ${\cal C}_m$,
其 ${\cal C}_m$ 可表示成 $\frac{1}{m+1}\binom{2m}{m}=O(\frac{4^m}{m^{\nicefrac{3}{2}}})$,
對於每一個節點個數 $m$ 的二元樹,我們儲存所有節點之間的最小共同祖先,故需要 $O(m^2)$。
整體而言,空間複雜度為 $O(s \; \frac{1}{s+1}\binom{2s}{s} \; s^2)$,
其中 $s$ 為塊大小,倘若挑選 $s$ 為 $\frac{\log n}{4}$,
其空間複雜度為 $O(\sqrt{n} \; \log^{\nicefrac{3}{2}} n)$。
然而,對於任意區間詢問只會在節點個數為 $s$ 上運行,我們仍需要個數小於 $s$ 的表來建立表 ${\cal A}$。

分析算法~\ref{alg:parallel-LCA} 的時間複雜度,在算法~\ref{alg:parallel-LCA} 的循序版本,
其時間複雜度為 $O(\frac{s^3}{s+1} \binom{2s}{s})$,因公式~\ref{fun:LCA} 相應的結果皆為 $O(1)$。
當選用 $s$ 為 $\frac{\log n}{4}$ 時,時間複雜度為 $O(\sqrt{n} \; \log^{\nicefrac{3}{2}} n)$。

對於算法~\ref{alg:parallel-LCA} 的平行版本,我們觀察計算不同個 ${\cal C}_m$ 的樹是「彼此獨立」的,
故可以平行化處理之。然而,當節點個數為 $m$ 時,找到左右子樹的卡塔蘭索引值的複雜度為 $O(m)$ 
(算法中的行~\ref{alg:parallel-LCA-compute} 所示),
同時行~\ref{alg:parallel-LCA-compute} 和行~\ref{alg:parallel-LCA-loop} 為同一層,故沒有平行的必要性。
最後平行化最外層的兩個迴圈,得到平行算法~\ref{alg:parallel-LCA} 的時間複雜度為
$O(\frac{s^3}{s+1} \binom{2s}{s} / p + s^2) = O(\sqrt{n} \; (\log^{\nicefrac{3}{2}} n) / p + \log^2 n)$
,其 $p$ 為處理器個數。

\subsection{計算卡塔蘭索引值}

算法~\ref{alg:parallel-LCA} 中需要卡塔蘭索引值 $t$,所以我們需要從塊中的資料中高效地決定 $t$ 為何,
這裡有兩種方法來計算-「建立樹」或者「維護右側路徑」

% how to compute t from tree data structure

\subsubsection{建立樹}

根據塊中數個元素決定卡塔蘭索引值,我們可以建立笛卡爾樹後,再決定其索引值為何,
意即我們確切地把樹建造後,利用左右子樹的索引值和節點個數組合出卡塔蘭索引值,
這將會要求遞迴遍歷整棵樹於 $O(n)$ 時間,其 $n$ 為節點總數。
令 ${\cal T}$ 為卡塔蘭索引值,根據公式~\ref{fun:tid} 的定義,
從左右子樹的節點個數 $s_l, \; s_r$、左右子樹的卡塔蘭索引值 $t_l, \; t_r$,
以及第 $i$ 個卡塔蘭數 ${\cal C}_i$ 得到。

\begin{equation}  \label{fun:tid}
  {\cal T}(s_l, t_l, s_r, t_r) = t_l \; {\cal C}_{s_r} + t_r +
  \sum_{i = 0}^{s_l - 1} {\cal C}_i \; {\cal C}_{s_l + s_r - i}
\end{equation}


更近一步優化等式~\ref{fun:tid} 的計算,我們預處理卡塔蘭數乘積的「前綴和」,
並儲存這些結果於記憶體中,將可以直接地使用它們,而不用再次地在等式~\ref{fun:tid} 計算之。

% how to compute t with rightmost path of the tree

\subsubsection{維護右側路徑}

先前計算卡塔蘭索引值的方法依賴完整的子樹資訊,這是較沒效率的做法。
我們提出使用「棧」維護笛卡爾的「右側路徑」,不需要維護整棵樹的資訊,
這一種方法類似章節~\ref{sec:cct} 提及的「右側棧出編碼」。
在我們知道卡塔蘭索引值 $t$ 後,
我們可解決算法~\ref{alg:parallel-LCA} 和公式~\ref{fun:tid} 中使用的 LCA 和索引值的需求。

以棧維護右側路徑,將可以高效地計算卡塔蘭索引值 $t$,我們不再需要建立完整的樹才能得到結果,
相應的代價是維護左子樹的節點個數與其卡塔蘭數索引值於棧 $D$ 中,
棧 $D$ 的每一個元素皆包含其左子樹的卡塔蘭索引值和左子樹的節點個數,
請參照算法~\ref{alg:cartesian-encode-offline} 的說明。
在算法~\ref{alg:cartesian-encode-offline} 中,棧 $D$ 每一個元素帶有元素值 $v$、
左子樹的節點個數 $s$、左子樹的卡塔蘭索引值 $t$,並且我們使用 $p$ 表示指向棧頂的指針。

\begin{algorithm}
\SetAlgoNoLine
\KwIn{
  $A[1 \cdots s]$: storage array\;
  $s$: the number of elements\;
}
\KwOut{
  $\mathit{tid}$: this label
}
$\langle\mathit{lsz},\mathit{lid},\mathit{value}\rangle$ $D$[$s+1$] \;
$\textit{Dp} \gets 0$ \;
$D[0] \gets \langle 0,0,\infty \rangle$ \;
\For{$i \gets 1$ to $s$} {
  $v \gets A[i], \; \textit{lsz} \gets 0, \; \textit{lid} \gets 0$ \;
  \While{$D[\textit{Dp}].\textit{value} < v$} {
    $\textit{lid} \gets \textit{tid}(D[\textit{Dp}].\textit{lsz}, D[\textit{Dp}].\textit{lid}, \textit{lsz}, \textit{lid})$ \;
    $\textit{lsz} \gets \textit{lsz} + D[\textit{Dp}].\textit{lsz} + 1$ \;
    $\textit{Dp} \gets \textit{Dp} - 1$ \;
  }
  $\textit{Dp} \gets \textit{Dp} + 1$ \;
  $D[\textit{Dp}] \gets \langle\mathit{lsz},\mathit{lid},\mathit{v}\rangle$ \;
}

$\textit{lsz} \gets 0, \; \textit{lid} \gets 0$ \;
\While{$\textit{Dp} > 0$} {
  $\textit{lid} \gets \textit{tid}(D[\textit{Dp}].\textit{lsz}, D[\textit{Dp}].\textit{lid}, \textit{lsz}, \textit{lid})$ \;
  $\textit{lsz} \gets \textit{lsz} + D[\textit{Dp}].\textit{lsz} + 1$ \;
  $\textit{Dp} \gets \textit{Dp} - 1$ \;
}
return $\textit{lid}$ \;

\caption{Offline Type of Cartesian Tree}
\label{alg:cartesian-encode-offline}
\end{algorithm}

\begin{figure}[!thb]
  \centering
  \includegraphics[width=0.4\linewidth]{\GraphicPath/fig-cartesian-encoding-static.pdf}
  \caption{計算樹的卡塔蘭索引值,令 $A_l$、$B_l$ 分別為 $A$、$B$ 的左子樹}
  \label{fig:fig-cartesian-encoding-static}
\end{figure}

算法~\ref{alg:cartesian-encode-offline} 根據塊內元素和棧 $D$ 計算出卡塔蘭索引值。
在第一個雙層迴圈中,外層迴圈依序加入元素到右側路徑的尾端,其中棧頂為 $D[p]$,
每次將從 $D$ 中推出「較小」的元素,當我們推出時相當於翻轉笛卡爾樹的子樹,
需要計算新的索引值 $t$ 以及節點個數 $s$,並且抽換掉右側路徑的狀態。
最終,整個樹的卡塔蘭索引值 $t$ 將根據公式~\ref{fun:tid},
統合棧 $D$ 上的左右子樹的索引值和節點個數計算得到,
請參照算法~\ref{alg:cartesian-encode-offline} 和圖~\ref{fig:fig-cartesian-encoding-static} 的說明。
圖中 $A_l$、$B_l$ 分別為 $A$、$B$ 的左子樹,在運行完所有插入元素的推出操作後,
我們需要從棧 $D$ 中推出所有元素得到整棵樹的卡塔蘭索引值,
也就是算法~\ref{alg:cartesian-encode-offline} 最後一個迴圈所運行的內容。


算法~\ref{alg:cartesian-encode-offline} 可以計算任何笛卡爾樹的卡塔蘭索引值,
由於每一個元素「最多」進出棧一次,故只需 $O(s)$ 時間完成。

\subsection{動態計算卡塔蘭索引值} \label{sec:dynamic}

先前的研究中,已經有數種編碼笛卡爾樹的方法。
如 Fischer~\cite{Fischer2006TheoreticalAP} 介紹了第一種方法,
以及 Masud~\cite{Hasan2010CacheOA} 在其論文中也描述如何運用編碼方法降低運行指令次數。
不幸地,這些算法的建造方案都是離線操作,意即它們需要給予完整塊才能計算出索引值,同時,
這造成它們無法適應持續地附加新的資料。此外,它們要求前處理時間為 $O(n)$,其 $n$ 為元素個數。
這些前處理需要額外的空間,或者需要從硬碟中提取資料。

概括說明我們提出的笛卡爾樹標記方法,我們加入新的資料時,使用「正規化」的方式,
將樹的節點個數統一為 $n$,意即對於任何的笛卡爾樹將會填充節點個數直到 $n$。
當節點個數為 $i$ 時,我們將附加 $n-i$ 個「右傾節點」({\em right-child-only}) 於樹上。
因此,當樹為空,我們添加 $n$ 個右傾節點,請參照圖~\ref{fig:cartesianEncoding-init} 的說明。
當插入第 $i$ 節點時,添加 $n-i$ 個右傾節點於右側路徑上,如圖~\ref{fig:cartesianEncoding-before} 所示。
為了說明這一加入的路徑,我們將其命名為「虛擬路徑」({\em virtual path})。

\begin{figure*}[!thb]
  \centering \subfigure[初始化 $t_{{\it root}_0}= {\cal C}_n - 1$]{
    \includegraphics[width=0.26\linewidth]{\GraphicPath/fig-cartesian-encoding.pdf}
    \label{fig:cartesianEncoding-init}
  } \subfigure[$t_{{\it root}_i}$]{
    \includegraphics[width=0.3\linewidth]{\GraphicPath/fig-cartesian-encoding-before.pdf}
    \label{fig:cartesianEncoding-before}
  } \subfigure[$t_{{\it root}_{i+1}} = t_{{\it root}_i} + t_x - t_A$]{
    \includegraphics[width=0.3\linewidth]{\GraphicPath/fig-cartesian-encoding-after.pdf}
    \label{fig:cartesianEncoding-after}
  }
  \caption{正規化笛卡爾樹的過程,使用增加虛擬路徑來增加節點個數}
  \label{fig:cartesianEncoding}
\end{figure*}

樹的正規化有兩項優點。其一,將可以減化卡塔蘭索引值的計算,
意即插入新的元素時,更新卡塔蘭索引值較為容易。其二,正規化操作補齊最大元素到塊的尾端,
對於現存的元素不受任何影響,可以運行相應的區間詢問,
同時使得編碼算法可以「動態」維護相應的 LCA 表,便可支持增長區間詢問。

動態卡塔蘭索引值的計算將使用樹的正規化操作,其使用方法如下:
令當前笛卡爾樹的卡塔蘭索引值為 $t^*$,笛卡爾樹的右側路徑在棧 $D$ 上,
隨著插入新的元素 $x$ 到樹上時,結合算法~\ref{alg:parallel-LCA} 和公式~\ref{fun:tid},
並且加入正規化操作得到在線版本的算法~\ref{alg:cartesian-encode-online},
在任意時刻 $i$,插入的第 $i$ 個散速,將可以得到當前的卡塔蘭索引值 $t^*$,
接著拿 $t^*$ 解決區間最大值詢問。

為方便表示,使用 $t$ 和 $s$ 表示卡塔蘭索引值和笛卡爾樹節點個數,
這裡的笛卡爾樹並「不包含」虛擬路徑。這些變數命名類似算法~\ref{alg:cartesian-encode-offline} 使用的,
額外使用 $t'$ 和 $s'$ 表示「包含」虛擬路徑的卡塔蘭索引值和笛卡爾樹節點個數。
請參照圖~\ref{fig:cartesianEncoding},紅色虛線為包含虛擬路徑的部分。

在算法~\ref{alg:cartesian-encode-online} 初使化 $s$ 和 $t$ 的方法,
接著如算法~\ref{alg:cartesian-encode-offline} 運行模式相當。然而,
在算法~\ref{alg:cartesian-encode-online} 中,額外初使化 $s'$ 為 $n-i$、$t'$ 為 ${\cal C}_{s'}-1$,
以應對「包含」虛擬路徑長 $n-i$ 的運行需求。

算法~\ref{alg:cartesian-encode-online} 在插入第 $i$ 個元素 $x$ 時,分成兩個階段。
在算法中的行~\ref{alg:cartesian-encode-online-pop-c1} 和 \ref{alg:cartesian-encode-online-pop-c2} 
皆推出小於 $x$ 的元素出棧,並更新 $t,\; s$ 為最後一個子樹「不包含」虛擬路徑的値,
同理,在算法中的行~\ref{alg:cartesian-encode-online-pop-virtual-c1} 和 \ref{alg:cartesian-encode-online-pop-virtual-c2} 
更新 $t', \; s'$ 為「包含」虛擬路徑長 $n-s'$ 的結果。
請參照算法~\ref{alg:cartesian-encode-online} 和圖~\ref{fig:cartesianEncoding} 的說明。

\begin{algorithm}[!thb]
\SetAlgoNoLine
\KwIn{
  $S$: state of Cartesian Tree;  $v$: the added data;
  $D$: A stack where every element has $s$, $t$, and $v$ \;
}
\KwOut{
  $t$: The Catalan index of the input data block after adding $v$
}

$p \gets S.p$, $s \gets 0$, $t \gets 0$ \;
$s' \gets S.s - S.{i} + 1$ \;
$t' \gets C[s'] - 1$ \;

\While{$D[p].{value} < v$} {
  $t \gets t(D[p].s, D[p].t, s, t)$ \;
  $t' \gets t(D[p].s, D[p].t, s', t')$ \;
  $s \gets s + D[p].s+1$ \;
  $s' \gets s' + D[p].s+1$ \;
  $p \gets p - 1$ \;
}
$p \gets p + 1$ \;
$D[p] \gets \left \langle s, t, {v} \right \rangle$ \;
$S.p \gets p$ \;
$x.t \gets t(s, t, S.s-S.i, C[S.s-S.i]-1)$ \;
$S.t \gets S.t + t' - x.t$ \;
$S.i \gets S.i + 1$ \;
return $S.t$ \;

  \caption{Online Type of Cartesian Tree}
  \label{alg:cartesian-encode-online}
\end{algorithm}


接著,更新棧 $D$ 且調整「整體的」卡塔蘭索引值 $t^*$,此時棧 $D$ 頂部必為元素 $x$ 的資訊,
其包含左子樹的節點個數 $s$ 和左子樹的卡塔蘭索引值 $t$,
請參照圖~\ref{fig:cartesianEncoding-after} 的說明。
整體的卡塔蘭索引值 $t^*$ 將會在更新棧後,請參照圖~\ref{fig:cartesianEncoding-before} 
和圖~\ref{fig:cartesianEncoding-after} 的說明。

觀察圖~\ref{fig:cartesianEncoding-before} 
和圖~\ref{fig:cartesianEncoding-after} 的紅色虛線框出的差異,
我們只需要找到紅色虛線部分的卡塔蘭索引值的「差值」,使用「補丁」的方式,
加上差值即可修正 $t^*$。所幸地,我們提供的字典順序採用左子樹先於右子樹,故圖~\ref{fig:cartesianEncoding-before} 
和圖~\ref{fig:cartesianEncoding-after} 的紅色虛線區域的差值為 $t' - {\cal T}(s, t, s - i,
{\cal C}_{s-i} - 1)$。最後,新的 $t^*$ 將會增加 $t' - {\cal T}(s, t, s - i, {\cal C}_{s-i} - 1)$。

我們提出的算法~\ref{alg:cartesian-encode-online} 並不會增加原先的時間複雜度,
儘管提供動態加入新的元素到樹上,複雜度仍同於先前提到的靜態索引值計算的算法~\ref{alg:cartesian-encode-offline},
其時間複雜度為 $O(n)$,因每個元素恰好被推出棧一次。相似地,
運行 $n$ 次算法~\ref{alg:cartesian-encode-online},
又因在迴圈內的任何計算都屬於 $O(1)$ 操作,攤銷複雜度仍為 $O(n)$。
最後,我們得知在塊內詢問與卡塔蘭索引值計算皆為攤銷 $O(1)$ 的操作。
