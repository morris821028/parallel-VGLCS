\section{區間最大值} \label{sec:parallelRMQ}

在這一章節中,我們將著重於算法~\ref{alg:parallel-VGLCS} 第二階段的挑戰,
意即高效的「區間最大值詢問」,增長區間最大值詢問遠比增長後綴最大值詢問來得困難,
原因在於後綴為區間的一種特例。

增長區間最大值詢問類似於後綴最大值詢問,分成三個操作:第一種 {\sc Make},建立空陣列 $A$;
第二種 {\sc Append}$(V)$ 操作,附加元素 $V$ 於陣列 $A$ 的尾端;
第三種 {\sc Query}$(L, R)$,找到陣列 $A$ 位置 $L$ 到 $R$ 的元素最大值。

\subsection{塊狀稀疏表} \label{sec:blocked-sparse-table}

為了改善平行 VGLCS 算法,我們採用「塊狀稀疏表」(blocked sparse table),
其最早被 Fischer~\cite{Fischer2006TheoreticalAP} 提及。
塊狀方法先將原輸入陣列拆成好幾個塊,每一塊有 $s$ 個元素,將每一塊的最大值建立成稀疏表 $T_s$。
相較於章節~\ref{sec:sparse-table} 提到的稀疏表,未分塊的稀疏表單一元素為一塊,
並且需建造 $\log n$ 層的區間最大值。


\begin{figure}[!thb]
  \centering \subfigure[原輸入陣列] {
    \includegraphics[width=0.65\linewidth]{\GraphicPath/fig-interval-decomposition.pdf}
  } \subfigure[塊狀稀疏表] {
    \includegraphics[width=0.25\linewidth]{\GraphicPath/fig-sparse-table.pdf}
  }
  \caption{塊狀稀疏表例子}
  \label{fig:block-interval-decomposition}
\end{figure}

塊狀方法解決區間最大值的方法如下所述:首先,
我們分成兩種詢問「超塊詢問」({\em super block} query) 以及「塊內詢問」({\em
  in-block} query),其中超塊詢問解決「連續完整」的塊狀最大值,
而塊內詢問解決單一塊內的區間詢問。如先前章節~\ref{sec:sparse-table} 提及的稀疏表算法,
超塊詢問最多拆成 2 個稀疏表上的查找。塊內詢問則需要「一次」查找,
這個查找需在「最小共同祖先表」({\em least common ancestor} table) 上運作,
關於這個表將在後續的章節中闡述細節。因此,任何區間詢問將會被拆成至多 2 次 $T_s$ 上的查找以及 2 次的塊內詢問,
整體而言最多有 4 次記憶體存取。由於超塊詢問與先前的稀疏表相當,接著將探討塊內詢問如何被解決。

Fischer 算法~\cite{Fischer2006TheoreticalAP} 掃描整個輸入陣列所產生的塊,
接著將其轉換成「笛卡爾樹」(Cartersian tree)。笛卡爾樹的每個節點儲存兩個鍵值,分別為資料的索引值與值。
我們可以將笛卡爾樹視為「堆」(heap),按照資料值呈現最大/最小堆的特性;
若只看資料的索引值,笛卡爾樹具有「二元搜尋樹」(binary search tree)的特性,
請參照圖~\ref{fig:ancesstor-cartesian} 的說明。
圖~\ref{fig:ancesstor-cartesian} 的每一個樹節點的水平位置反應出其原始位置的相對關係,
而垂直位置反應數值大小關係。換而言之,若需要得知塊內詢問的第 $i$ 位置到 $j$ 位置上的最大值,
找到相應的笛卡爾樹的「最小共同祖先」({\em least common ancestor})。
例如詢問區間位置 4 到位置 6 之間的最大值,從表格中得知「最大值位置」為 5,最後得到最大值為 7。

\begin{figure}[!thb]   
  \centering
  \includegraphics[width=0.9\linewidth]{\GraphicPath/fig-interval-cartesian.pdf}
  \caption{最小共同祖先表}
  \label{fig:ancesstor-cartesian}
\end{figure}

為了解決塊內詢問,Fischer 算法需要計算每一塊相應的最小共同祖先表 (LCA table,簡稱「LCA 表」),
在掃描每一塊的資訊後,建造相應的 LCA 表,此一表儲存資訊為 $((i, j), k)$,
對於任意位置 $i$ 到 $j$ 的最大值位置 $k$,請參照圖~\ref{fig:ancesstor-cartesian} 的說明。
我們可以從 LCA 表中提取位置 $i$ 到位置 $j$ 的最大值位置 $k$,
隨後映射回原輸入陣列得到區間 $(i, j)$ 的最大值。

特別注意到算法中,沒有儲存第 $k$ 個元素的值,而是儲存其「索引值」,
意即保留元素值的相對關係,因此兩個塊之間可以共用同一個 LCA 表。
如圖~\ref{fig:ancesstor-cartesian} 所示,前三個區塊共用同一個 LCA 表,
因此,無論在哪塊中詢問區間 $(1, 3)$ 最大值,任何塊內詢問皆回答 $2$。

Fischer 算法的核心在於如何計算每一塊相應的 LCA 表,間接地應答所有塊內詢問。
我們也可以發現將會有 ${\cal C}_s$ 種不同形的笛卡爾樹,
其中 ${\cal C}_s$ 為節點個數為 $s$ 的二元搜尋樹的不同形狀個數。
每一塊將對應一個笛卡爾樹,接著我們將使用「卡塔蘭索引值」({\em Catalan index}) 表示塊與笛卡爾樹的相應關係,
這相應關係如圖~\ref{fig:ancesstor-cartesian} 所描述。

Fischer 算法~\cite{Fischer2006TheoreticalAP} 建立 LCA 表前,
基於效能考量而選用塊大小 $s = \frac{\log n}{4}$,因卡塔蘭數 ${\cal C}_s =
\frac{1}{s+1}\binom{2s}{s} = O(\frac{4^s}{s^{\nicefrac{3}{2}}})$,
故掃描和建立笛卡爾樹的時間皆為 $O(n)$。意即 Fischer 算法在前處理階段需要 $O(n)$ 時間,
對於任意區間詢問需 $O(1)$ 時間,容易地明白前處理和詢問操作皆已經理論最佳的結果。

然而,Fischer 算法在大量資料下易造成「快取未中」(cache miss),
而算法中採用「請求」 ({\em on-demand}) 式的建立所需要的 LCA 表,
唯有在掃瞄過程中找到尚未建立的 LCA 表時,才進行新的 LCA 表建造。
一旦使用 LCA 表,這意味著使用上隨著帶入 LCA 表至快取時,易將另一個 LCA 表逐出快取,
反覆使用的次數一多,就容易造成嚴重的快取未中。

為減少快取未中的次數,Demaine 算法~\cite{Demaine2009OnCT} 
提出在笛卡爾數上的「快取適性」(cache-aware) 操作,
Demaine 算法並不需要使用記憶體檢查是否需要建立 LCA 表,而是建立所有的可能笛卡爾樹的表。
因此,Demaine 算法使用以長度為 $2s$ 的二元字串表示笛卡爾樹,
這個二元字串上可回答塊內區間最大值。然而,檢測過程中將會需要計算兩個相鄰 1 的位元之間有多少個 0,
這一操作難以在當代計算機架構上高效地運行。

\subsection{右側棧出編碼} \label{sec:cct}

我們提出另一種塊的編碼方法-名為「右側棧出」({\em rightmost-pops}),
有別於 Demaine~\cite{Demaine2009OnCT} 提出的二元字串,其為了改善查詢區間極值查找的效能。
然而,右側棧出主要仍來自 Demaine 和笛卡爾樹的想法。


右側棧出編碼主要維護笛卡爾樹的「右側路徑」({\em rightmost path}) 於棧 (stack),
同時,當加入新的元素到笛卡爾樹時,保留在棧上的推出 (pop) 次數,
請參照圖~\ref{fig:interval-cartesian} 的說明。
為了維護加入第 $i$ 的元素 $a_i$ 滿足笛卡爾樹的堆性質,
將會把樹上右側較小的元素翻轉到 $a_i$ 的左子樹,也就是將棧上小於等於 $a_i$ 的元素推出,
其推出的次數令為 $t_i$,在過程中滿足 $0 \le t_i < s$,其中 $s$ 為塊大小。
接著,這些 $t_i$ 將會是編碼笛卡爾樹的主要素材。

請參照圖~\ref{fig:interval-cartesian},當我們插入元素 $a_1 = 0$ 時,將不會有任何元素被推出棧,
接著插入元素 $a_2 = 4$ 時,將會把 $a_0$ 和 $a_1$ 推出棧,此時得到 $t_2$ 為 $2$,
從先前的定義中,我們可以明確地將棧表示為笛卡爾樹的最右側結構,接著將這些 $t_i$ 組合來表示笛卡爾樹的狀態。

\begin{figure}[!thb]
  \centering
  \includegraphics[width=\linewidth]{\GraphicPath/fig-cartesian-encoding-stack.pdf}
  \caption{表示右側路徑於棧}
  \label{fig:interval-cartesian}
\end{figure}

「右側棧出編碼」主要藉由數個 $t_i$ 「隱式辨識」(implicitly identify) 出個別的笛卡爾樹。
對於任意的塊內詢問,只需要在數個 $t_i$ 上測試即可。
假設回答塊狀詢問區間 $l$ 到 $r$ 於這些 $t_i$ 上的最大值,我們維護棧上的推出次數 $x$,
初始值令 $x$ 為 0,接著依序檢測 $t_l$ 到 $t_r$,令迴圈索引 $j$ 依序從 $l$ 到 $r$,
每經過一次迴圈,將 $x$ 扣除 $t_j - 1$,我們只需要紀錄最後一次發生 $x$ 小於等於 0 的索引值 $j$ 為何,
則「最後的」索引值 $j$ 即是需要的「最大值位置」,請參照算法~\ref{alg:cartesian64bits-query} 的說明。
算法~\ref{alg:cartesian64bits-query} 中提供的每一步操作都可以明確地在當代電腦中找到相應的指令,
別於當初 Demaine 提出的算法,這大幅度地容易以簡單指令表示之。

\begin{algorithm}[H]
\SetAlgoNoLine
\LinesNumbered
\KwIn{$\textit{tmask}$: 64-bits Cartesian tree; $[l, r]$: query range}
\KwOut{$\textit{minIdx}$: the index of the minimum value in interval}
    
$\textit{minIdx} \gets l, \; x \gets 0$ \;
\For{$l \gets l+1$ to $r$} {
  $x \gets x+1 - ((\textit{tmask} \gg (l \ll 2)) \mathrel{\&} 15)$ \;
  \If{$x \le 0$} {
    $\textit{minIdx} \gets l$, $x \gets 0$ \;
  }
}
return $\textit{minIdx}$ \;

  \caption{Range Minimum Query in 64-bits Cartesian Tree}
  \label{alg:cartesian64bits-query}
\end{algorithm}

算法~\ref{alg:cartesian64bits-query} 的正確性請參照理論~\ref{thm:correctness} 的證明。
直觀的方法來自於找到樹根,其表示區間內最後插入到根的元素。

\begin{theorem} \label{thm:correctness}
  算法~\ref{alg:cartesian64bits-query} 正確回答塊內區間最大值詢問
\end{theorem}
\begin{proof}
當推出次數 $x$ 小於等於 0 時,其意涵著該位置為區間內的樹根,
則最後一次成為樹根的位置則為區間內的最大值。因為笛卡爾樹具有堆性質,根為樹內最大值。
\end{proof}

選用塊大小 $s=16$,則所有 $t_i$ 小於 16 且每一個 $t_i$ 可以表示成 4-bit 整數,
串接這十六個 4-bit 整數可以壓縮到 64-bit 整數來表示一個完整的笛卡爾樹,
請參照算法~\ref{alg:cartesian-to-64bits},其算法只需要 $O(s)$ 時間。
算法~\ref{alg:cartesian-to-64bits} 中的左移、右移和加減法都可以很明確地對應到機器指令。

\begin{algorithm}
\SetAlgoNoLine
\KwIn{$A[1 .. 16]$: input data block}
\KwOut{$t$: a 64 bit right-most-pops encoding of $A$}

$\tt{LOGS} \gets 4$, $\tt{POWS} \gets 2^{\tt{LOGS}}$ \;
Create an array $D$ of size $\tt{POWS}+1$ \;
$p \gets 0$, $D[0] \gets \infty$ \;

$t \gets 0$ \;
\For{$i \gets 1$ to $\tt{POWS}$} {
  $v \gets A[i], \; c \gets 0$\;
  \While{$D[p] < v$} {
    $p \gets p - 1$, $c \gets c + 1$ \;
  }
  $p \gets p + 1$ \;
  $D[p] \gets v$ \;
  $t \gets t \mathrel{|} (c \ll ((i-1) \ll 2))$ \;
}
return $t$ \;

  \caption{Encode a data block of sixteen data with right-most-pops
    encoding into a 64-bits integer.}
  \label{alg:cartesian-to-64bits}
\end{algorithm}


考量到效能問題,選用塊大小 $s = \frac{\log n}{4} = 16$ 的原因來自於大部分的機器皆為 64-bit 暫存器,
且有許多非常高效率的暫存器操作。此外,這一類的方法並沒有使用到 LCA 表,
這意涵著我們只需要維護 64-bit 的 $t_i$ 表示笛卡爾樹的狀態,充分地改善記憶體使用和快取效能,
從理論上已經可以支持到 $n = 2^{64}$ 大小的問題。

我們提出的「右側棧出編碼」改善了快取問題,卻增加了詢問操作的複雜度。
相較於 Fischer 算法,算法~\ref{alg:cartesian64bits-query} 提供較高的資料局部性和較有規律性的管理,
前處理同樣需要 $O(n)$ 時間,單一詢問操作需要 $O(s)$ 的時間,其中 $s$ 為塊大小 $\frac{\log n}{4}$。
在實作中,我們選用 $s$ 為 16 來支持 $n = 2^{64}$ 的問題,所以 $s$ 可以視為一個很小的常數,
空間複雜度仍為 $O(n)$。帶入平行 VGLCS 算法中使用,整體時間複雜度為 $O(n^2 \log{n} / p + n \log n)$,
其中的 $\log n$ 來自於塊大小的選用 $s = O(\log n)$ 影響。
