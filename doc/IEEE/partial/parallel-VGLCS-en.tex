\section{Parallel VGLCS Algorithm} \label{sec:parallelVGLCS}

\subsection{Basic Dynamic Programming}

We first describe a basic dynamic programming for
VGLCS~\cite{Peng2011TheLC}. Let $A$ and $B$ denote two input strings of
length $n$ and $m$ respectively, and $G_A$ and $G_B$ be the arrays of
the variable gap constraints.  We define $V[i, j]$ to be the {\em
maximum} length of the variable gapped longest common subsequence
between substring $A[1, i]$ and $B[1, j]$.  It is easy to see that $V[i,
j]$ is the {\em maximum} among $V[k, l]$, where $k$ is between $i -1$
and $i-G_A(i)-1$, and $l$ is between $j -1 $ and $j-G_B(j)-1$, i.e., a
rectangle within $V$ on the left and upper of $V[i,j]$.  Please refer to
Figure~\ref{fig:fig-VGLCS-dp-naive} for an illustration.

\begin{figure}[!thb]
  \centering \subfigure[How to compute $V$.] {
    \includegraphics[width=0.45\linewidth]{\GraphicPath/fig-VGLCS-dp-naive.pdf}
    % \caption{How to compute $V$.}
    \label{fig:fig-VGLCS-dp-naive}
  } \subfigure[Compute $V$ with incremental suffix maximum queries.] {
    \includegraphics[width=0.45\linewidth]{\GraphicPath/fig-VGLCS-dp.pdf}
    \label{fig:fig-VGLCS-dp}
  }
  \caption{The basic dynamic programming for VGLCS}
  \label{fig:basic-dp-VGLCS}
\end{figure}

The computation of $V$ can be optimized as follows.  Note that the
computation of all $V[i, j]$'s with the same $i$ has the {\em same}
gap constraint $G_A(i)$, so the maximum within the rectangle can be
computed in two steps.  First we compute the maximum of {\em every
  column} of this rectangle, and place them into another array $R$.
Then we compute the maximum of the {\em suffix} of length $G_B$ on
$R$, which is exactly $V[i, j]$.  Please refer to
Figure~\ref{fig:fig-VGLCS-dp} for an illustration.

We note that this optimization requires maximum queries on the suffix
{\em incrementally} in the following sense.  Recall that in the first
step of the optimization we need to compute the maximum of every
column within the rectangle.  This is just like finding the maximum of
the {\em suffix} of every column in that rectangle.  After we compute
the $i$-th row of $V$ and go to the next row to compute $V[i + 1, *]$,
we will then need the maximum of the suffix of every column of length
$G_A(i + 1)$.  It will be beneficial if we put the $V$'s in each
column into a data structure that supports suffix maximum query.
Similarly, the computation within the same row, that is, from $V[i,j]$
to $V[i, j + 1]$, also requires the maximum of the suffix on $R$.  We
will refer to this type of queries as {\em incremental suffix maximum
  query}, i.e., we would like to maintain a data structure form which
we can find the maximum of its suffix efficiently while we add data at
its end.

% \begin{figure}[!thb]
%   \includegraphics[width=0.40\linewidth]{\GraphicPath/fig-VGLCS-dp.pdf}
%   \caption{Compute $V$ with incremental suffix maximum queries.}
%   \label{fig:fig-VGLCS-dp}
% \end{figure}

\subsection{Peng's Algorithm}

The sequential VGLCS algorithm~\ref{alg:serial-VGLCS} by
Peng~\cite{Peng2011TheLC} applies the optimization and is shown as
Algorithm~\ref{alg:serial-VGLCS}.  The outer loop goes through every
row, and the inner loop goes through every element of a row from left
to right.  We use an array of $C$ to answer incremental maximum
queries on all columns.  That is, we can think of $C[j]$ as a data
structure that supports incremental suffix maximum queries on the
$j$-th column of $V$.  From the previous observation that all
computation of elements in the $i$-th row of $V$ share the same gap
$G_A$, we will query each $C$ for the maximum in the suffix that ends
at row $i-1$ with length $G_A + 1$, and place these maximums in
another data structure $R$ that also supports incremental suffix
maximum queries.  It is easy to see that the value of $V[i][j]$ can be
obtained by querying $R$ with a suffix maximum query of length $G_B +
1$ as shown in Figure~\ref{fig:fig-VGLCS-dp}.

We update $V$, $C$, and $R$ as follows.  If the $i$-th character of
$A$ matches the $j$-th character of $B$ then $V[i][j]$ is the maximum
among the rectangle plus 1, as shown in Figure~\ref{fig:fig-VGLCS-dp}.
This maximum can be found by querying $R$ for the maximum among the
last $G_B[j]$ elements in it.  Note that $R$ contains the information
of the previous row, up to the element of the $j-1$ element.  After
that we add the maximum of last $G_B + 1$ in the $j$-the column into
$R$, and the newly computed $V[i][j]$ into $C[j]$, which supports ISMQ
on the $j$-the column, before going to column $j + 1$.  If the $i$-th
character of $A$ does {\em not} match the $j$-th character of $B$ then
we simply set $V$ to 0 since it does not affect the answer, then again
update $R$ accordingly.
 
\begin{algorithm}[!thb]
\SetAlgoNoLine
\KwIn{$A, B$: the input string; $G_A, G_B$: the array of variable gapped constraints;}
\KwOut{Find the LCS with variable gap constraints.}

Create an array of $m$ data structures $C[m]$ that support ISMQ\;
Create an empty table $V[n][m]$\;

\For{$i \gets 1$ to $n$} {
  Create a data structure $R$ that supports ISMQ\;
  \For{$j \gets 1$ to $m$} {
    \uIf{$A[i] = B[j]$} {
        $t \gets$ Query $R$ for the maximum among the last $G_B(j)+1$ elements \;
        $V[i][j] \gets t + 1$\; 
        $t \gets$ Query $C[j]$ for the maximum among the last $G_A(i)+1$ elements \;
        Append $t$ to $R$ \;
    }
    \Else {
        $V[i][j] \gets 0$ \;
        $t \gets$ Query $C[j]$ for the maximum among the last $G_A(i)+1$ elements \;
        Append $t$ to $R$ \;
    }
    Append $V[i][j]$ into $C[j]$ \;
  }
}
Retrieve the VGLCS by tracing $V[n][m]$\;

  \caption{Peng's algorithm for finding VGLCS~\cite{Peng2011TheLC}}
  \label{alg:serial-VGLCS}
\end{algorithm}


\subsection{Incremental Suffix Maximum Query}

From the previous discussion of Peng's algorithm, we note that in
order to find VGLCS efficiently, we need to address the {\em
  incremental suffix maximum query} (ISMQ) problem.  A data structure
that supports incremental suffix maximum queries should support the
three operations.  First, a {\sc Make} operation creates an empty
array $A$. Second, an {\sc Append(V)} operation appends a value $V$ to
array $A$. Finally an ISMQ {\sc Query(x)} finds the {\em maximum}
value among those from $x$ to the end of an array $A$.

Peng uses a {\em disjoint-set} data structure to answer incremental
suffix maximum queries in his VGLCS algorithm.  The disjoint-set data
structure was proposed by Gabow~\cite{Gabow1983ALA} and
Tarjan~\cite{Tarjan1975EfficiencyOA} to solves the {\em union-and-find
  problem}.  The set of data are stored in a sequence of disjoint
sets, and the {\em maximum} of each disjoint set is at the root of the
tree, and these maximum are in {\em decreasing} order.  When we add a
value $x$, we make it as a disjoint set with a single element by
itself, and as the {\em last} disjoint set in the sequence.  Then we
start joining (with union operation) from the last set to its previous
set until the maximum of the previous set is {\em larger} than $x$.
It is easy to see that the {\sc Query(x)} operation is simply a {\em
  find} operation that finds the root, which has the {\em maximum}, of
the tree that $x$ belongs to.  The amortized time per union/find
operation is $O(\alpha(n))$.

\subsection{A Parallel VGLCS Algorithm with Sparse Table}

The sequential VGLCS algorithm~\ref{alg:serial-VGLCS} by
Peng~\cite{Peng2011TheLC} and other variants of LCS are difficult to
parallelize in a row-by-row manner.  These algorithms use several
states to determine a new state with a dynamic programming.  This
construction requires {\em heavy data dependency}, and is difficult to
parallelize in a naive row-by-row manner because an element of the
dynamic table needs the values of elements in the {\em same} row to
compute its value.

It is also difficult to parallelize Peng's algorithm with the
wavefront method because it requires {\em extra space} to keep track
of row status, which is not required in a sequential algorithm.
Recall that VGLCS requires a rectangle of data in $V$ to compute a new
$V$ element.  If those $V$'s want to compute are on a diagonal wave
front, those rectangles of data will require extra bookkeeping since
the gap constraints of those elements on the wavefront could be very
different.  Please refer to Figure~\ref{fig:fig-VGLCS-dp-wavefront}
for an illustration of those data that must be present (in solid
line).  Also the new elements computed must be appended to the data
strictures according to the wavefront, which incurs more book keeping
and overheads.

\begin{figure}[!thb]
  \centering \subfigure[The first stage]{
    \includegraphics[width=0.45\linewidth]{\GraphicPath/fig-VGLCS-dp-wavefront-second.pdf}
  } \subfigure[The second stage]{
    \includegraphics[width=0.45\linewidth]{\GraphicPath/fig-VGLCS-dp-wavefront-first.pdf}
  }
  \caption{The book keeping data of the wavefront method}
  \label{fig:fig-VGLCS-dp-wavefront}
\end{figure}

We propose a parallel VGLCS algorithm with an optimized row-by-row
approach, which maintains only {\em one} row data structure that
collects suffix maximum values from all columns.  That is, our
optimized row-by-row approach removes the data dependency among
elements {\em within the same row}.

Our optimized row-by-row approach uses less space, has a more balanced
workload, and a smaller thread synchronization overhead, than the
wavefront method.  We observe that the length of the critical path of
a wavefront method is greater than that of a row-by-row method.  In
addition, if we can removes the data dependency among elements within
the same row, then the computation on the elements of the same row can
be {\em fully} and {\em evenly} parallelized.  The result is a much
more balanced workload distribution and a much easier synchronization
among threads.


% On the other hand, if we use the
% Maleki's~\cite{Maleki2016EfficientPU} technique, % need to explain
% this technique it also uses extra space to maintain state
% translation, and spends more time to merge data.  Therefore, it is
% crucial that our parallelization conserves {\em both} memory and
% time.

% It seems that you should use the new figures to explain these???
% Morris: how to make it to row-by-row manner


A sketch of our algorithm is as follows.  Our algorithm computes $V$ one
row at a time.  The computation of each row has two stages.  In the
first stage, the algorithm queries each data structure $C$ for every
column within the rectangle {\em in parallel}, so as to obtain the
maximums of suffix of length $G_A + 1$ of every column, and place them
into an array $R$. Recall from Algorithm~\ref{alg:serial-VGLCS} that
every column of $V$ has a data structure $C$ that supports incremental
suffix maximum on $V$.  Please refer to
Figure~\ref{fig:fig-VGLCS-dp-rmq} for an illustration.

In the second stage, our algorithm issues $m$ {\em range maximum
queries}, one for each column, on $R$ to compute all $m$ elements of the
$i$-th row of $V$ {\em in parallel}.  Note that unlike the sequential
algorithm, we compute all elements in the $i$-th row of $V$ in parallel,
so we cannot query the {\em suffix} of $R$.  Instead we need to query a
{\em range} of $R$ for the maximum, where the range is the gap
constraint on that column.  Please refer to
Figure~\ref{fig:fig-VGLCS-dp-rmq} for an illustration.  Note that we
need to add the newly computed $V[i, j]$ into the $C$ of the $j$-th
column {\em incrementally}, so that they will contain the correct
information for the computation of the $(i+1)$-th row.  Also since the
algorithm iterates in rows, these $C$'s only need to support suffix
maximum query.  No range query on them is required.  In contrast we do
need to support range maximum query on $R$, and these queries will be in
parallel.

\begin{figure}
  \centering \subfigure[The first stage]{
    \includegraphics[width=0.45\linewidth]{\GraphicPath/fig-VGLCS-dp-rmq-first.pdf}
  } \subfigure[The second stage]{
    \includegraphics[width=0.45\linewidth]{\GraphicPath/fig-VGLCS-dp-rmq-second.pdf}
  }
  \caption{Two stages of the computation of one row of $V$.}
  \label{fig:fig-VGLCS-dp-rmq}
\end{figure}

To resolve the data dependency we need to consider a good data
structure that can handle incremental suffix/range maximum query {\em
  in parallel}.  We note that it is {\em not} feasible to parallelize
the disjoint set implementation for three reasons.  First, a query for
disjoint set will change the data structure because a lookup will {\em
  compress} the path to the root.  It is difficult to maintain a
consistent view of the data structure when multiple threads are
compressing the path {\em simultaneously}.  Second, when multiple
threads are compressing different paths, the load among them could be
very different, and this will incur load imbalance.  Third, there will
be a large number of threads working on different parts of the
disjoint set, therefore it will be difficult to synchronize them
efficiently.

\subsubsection{Sparse Table} \label{sec:sparse-table}

Since the disjoint set cannot be implemented efficiently in parallel,
we use {\em sparse table}~\cite{Berkman1993RecursiveSP} to support
incremental suffix/range maximum queries in our VGLCS algorithm.
Sparse table~\cite{Berkman1993RecursiveSP} requires a $O(n \log n)$
preprocessing, and can support range maximum query in $O(1)$ time on
one dimensional data.  A sparse table is a two dimensional array.  The
element of a sparse table in the $j$-th row and $i$-th column is the
maximum among the $i$-th elements and its $2^j - 1$ predecessors in
the input array.

We give an example of the sparse table
(Figure~\ref{fig:interval-decomposition}).  The input is in array
$A$. We split array $A$ into five blocks so that each block has four
elements.  The we build a sparse table $T$ on $A$ as described
earlier.  Now a ranged maximum query on $A$ can be answered by at most
{\em two} queries into the sparse table.  For example, if the query is
from 2 to 13, then the answer is the maximum of from 2 to 9
($T[3][9]$), and from 6 to 13 ($T[3]13]$).  Both are from the third
  level of the table since each has the maximum of $2^3 = 8$ elements
  in the input.

\begin{figure}[!thb]
  \centering \subfigure[Array]{
    \includegraphics[width=0.45\linewidth]{\GraphicPath/fig-interval-decomposition-origin.pdf}
    \label{fig:fig-interval-decomposition}
  } \subfigure[Sparse table]{
    \includegraphics[width=0.45\linewidth]{\GraphicPath/fig-sparse-table-origin.pdf}
    \label{fig:fig-sparse-table}
  }
  \caption{A sparse table example}
  \label{fig:interval-decomposition}
\end{figure}

It is easy to see that one can build a sparse table in parallel
efficiently.  Please refer to
Algorithm~\ref{alg:parallel-sparse-table} for details.
Algorithm~\ref{alg:parallel-sparse-table} builds sparse table in
parallel and in $O(n \log n / p + \log n)$ time, where $n$ is the
number of elements and $p$ is the number of processors.  This
algorithm is very easy to parallelize and implement.

\begin{algorithm}[H]
\SetAlgoNoLine
\LinesNumbered
\KwIn{$A[0 .. N-1]$: the input array}
\KwOut{$T$: a sparse table for the input array $A$}

Create a two-dimensional array $T[\log N][N]$ \;
Copy $A$ to $T[0]$ \;
\For{$j \gets 0$ to $\log N$} {
  \ForPar{$i \gets 2^j$ to $N$} {
    $T[j][i] \gets \max(T[j-1][i-2^{j}], T[j-1][i])$ \;
  }
}
  \caption{A Parallel Sparse Table Building Algorithm}
  \label{alg:parallel-sparse-table}
\end{algorithm}


\subsubsection{A Parallel VGLCS with Sparse Table}

The operations on a sparse table are much easily to parallelize than
those on a disjoint set, which is used within the inner loop of Peng's
sequential VGLCS algorithm.  The inner loop of Peng's algorithm
alternates between append and query operations on $R$.  Please refer
to Algorithm~\ref{alg:serial-VGLCS} for details.  This alternation
between appending and querying incurs heavy data dependency.  In
addition, the parallelism of operations on a disjoint tree is limited
by the length of path under compression.  The length is usually very
short and provides very limited parallelism.

The pseudo code of our parallel VGLCS algorithm with sparse table is
given in Algorithm~\ref{alg:parallel-VGLCS}.  The algorithm computes
$V$ one row at a time.  The computation of each row has two stages.
In the first stage, the algorithm queries the $C$'s {\em in parallel}
to obtain the maximums of suffix of length $G_A + 1$ and place them
into an array $R$.  Then we build a sparse table $T$ with the data of
$R$.  Then in the second stage, the algorithm queries $T$ to find the
range maximum in $R$ to compute all elements in the $i$-th row of $V$
{\em in parallel}.

\begin{algorithm}[!thb]
\SetAlgoNoLine
\KwIn{$A, B$: the input string; $G_A, G_B$: the array of variable gapped constraints;}
\KwOut{Find the LCS with variable gapped constraints}
    
Create an array of $m$ data structure $C[m]$ to support ISMQ \;
Create an empty table $V[n][m]$ \;
\For{$i \gets 1$ to $n$} {
  \ForPar{$j \gets 1$ to $m$} {
    $M[j] \gets$ suffix maximum of length $G_A(i) + 1$ from $C[j]$ \;
  }
  Build a sparse table $T$ with data of $M$ in parallel with Algorithm~\ref{alg:parallel-sparse-table}\; 
  \ForPar{$j \gets 1$ to $m$} {
    \If{$A[i] = B[j]$} {
        $t \gets $ the range maximum among the $G_B(j) + 1$ elements before $M[j]$ by querying $T$ \;
        $V[i][j] \gets t + 1$ \;
        Append $V[i][j]$ to $C[j]$ \;
    }
  }
}
Retrieve the VGLCS by tracing $V[n][m]$ \;
  \caption{Parallel Algorithm for Finding VGLCS}
  \label{alg:parallel-VGLCS}
\end{algorithm}


The implementation of the two stages have different challenges.  The
first stage is easier to parallelize because the operations on
individual columns are {\em independent}.  However, it will insert new
data into $C$, and still needs to answer suffix queries efficiently in
order to build the $R$ array.  The second stage does {\em not}
requires insertion so it is more static.  However, since we compute
all $V$'s in the same row in parallel, it requires {\em ranged
  queries}, instead of suffix queries, on the sparse table $T$.  The
next two sections will describe our approaches to address these
challenges of two stages.  For ease of presentation we will describe
our approach for the second stage in Section~\ref{sec:parallelRMQ}
first.  Then we address the first stage in Section~\ref{sec:QIUD}.
