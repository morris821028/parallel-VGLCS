\section{Parallel VGLCS Algorithm with Sparse Table} %
\label{sec:parallelVGLCS}

% use line number in the following to explain the idea.

The sequential VGLCS algorithm~\ref{alg:serial-VGLCS} by
Peng~\cite{Peng2011TheLC} (shown as Algorithm~\ref{alg:serial-VGLCS}
below) works as follows.  

% these figures were never referenced????

\begin{figure}[!thb]
  \centering
  \subfigure[Na\"{i}ve]{
    \includegraphics[width=0.40\linewidth]{\GraphicPath/fig-VGLCS-dp-naive.pdf}
    \label{fig:fig-VGLCS-dp-naive}
  }
  \subfigure[ISMQ]{
    \includegraphics[width=0.40\linewidth]{\GraphicPath/fig-VGLCS-dp.pdf}
    \label{fig:fig-VGLCS-dp}
  }
  \caption{Workflow}
\end{figure}

\begin{enumerate}
  \item 

The VGLCS problem has the two input string $A$ and $B$, and the variable
gapped constraints $G_A$ and $G_B$.  The state $V[i, j]$ is defined as
the maximum length of the variable gapped longest common subsequence
between substring $A[1, i]$ and $B[1, j]$.  The na\"{i}ve algorithm can
be constructed by solve $V[i, j]$ by finding the maximum value in the
area $V[i-G_A(i)-1 \cdots i-1, j-G_B(j)-1 \cdots j-1]$ as the area
maximum value queries.  The Figure~\ref{fig:fig-VGLCS-dp-naive}
illustrates the workflow of the na\"{i}ve algorithm to solve the VGLCS
problem.

  \item 

When computing the $V[i, j]$ with the same $i$, the gapped constraint of
the $G_A(i)$ is same.  Therefore, the area maximum queries can be
combined by the multiple results of incremental suffix maximum queries
because of the same upper and lower $x$-coordinate.  Please refer to
Figure~\ref{fig:fig-VGLCS-dp} for an illustration.

  \item 

Then, the concept of the sweep line algorithm is the main point to solve
each area maximum queries.  We transform the sequential area maximum
queries into the incremental suffix maximum queries when moving sweep
line.


\end{enumerate}

% The algorithm has two stages.  In the first
% stage, the algorithm uses a {\em disjoint set} to efficiently answer
% ISMQ to compute an answer for every column.  The answers on all
% columns form an array $S$.  In the second stage, the algorithm again
% uses ISMQ on array $S$ obtained from the first stage to get the final
% answers.


\begin{algorithm}[!thb]
\SetAlgoNoLine
\KwIn{$A, B$: the input string; $G_A, G_B$: the array of variable gapped constraints;}
\KwOut{Find the LCS with variable gap constraints.}

Create an array of $m$ data structures $C[m]$ that support ISMQ\;
Create an empty table $V[n][m]$\;

\For{$i \gets 1$ to $n$} {
  Create a data structure $R$ that supports ISMQ\;
  \For{$j \gets 1$ to $m$} {
    \uIf{$A[i] = B[j]$} {
        $t \gets$ Query $R$ for the maximum among the last $G_B(j)+1$ elements \;
        $V[i][j] \gets t + 1$\; 
        $t \gets$ Query $C[j]$ for the maximum among the last $G_A(i)+1$ elements \;
        Append $t$ to $R$ \;
    }
    \Else {
        $V[i][j] \gets 0$ \;
        $t \gets$ Query $C[j]$ for the maximum among the last $G_A(i)+1$ elements \;
        Append $t$ to $R$ \;
    }
    Append $V[i][j]$ into $C[j]$ \;
  }
}
Retrieve the VGLCS by tracing $V[n][m]$\;

  \caption{Peng's algorithm for finding VGLCS~\cite{Peng2011TheLC}}
  \label{alg:serial-VGLCS}
\end{algorithm}


\subsection{Incremental Suffix Maximum Query}

In order to find VGLCS efficiently, we need to address the {\em
incremental suffix maximum query} (ISMQ) problem, which was proposed by
Peng~\cite{Peng2011TheLC}.  A data structure that supports incremental
suffix maximum queries should support the following three operations.
First, a {\tt make} operation creates an empty array $A$. Second, an
{\tt append(V)} operation appends a value $V$ to array $A$. Finally an
ISMQ {\tt query(x)} finds the {\em maximum} value among the $x$-th value
to the end of an array $A$.


Peng uses a {\em disjoint-set} data structure to answer incremental
suffix maximum queries, and finds a VGLCS with answers from ISMQ. The
disjoint-set data structure by Gabow~\cite{Gabow1983ALA} and
Tarjan~\cite{Tarjan1975EfficiencyOA} solves the {\em union-and-find
  problem}.  The set of data are stored in a sequence of disjoint
sets, and maintain the property that the {\em maximum} of disjoint
sets are at the root and in {\em decreasing} order.  When we add a
value $x$ into the data structure, we put it at the end as a set of
itself.  Then we start joining (with union operation) from the last
set to its previous set until the maximum of the previous set is {\em
  larger}.  It is easy to see that the {\tt query(x)} operation is
simply a {\em find} operation that finds the root, which has the {\em
  maximum}, of the tree that $x$ belongs to.  The amortized time per
union/find operation is $O(\alpha(n))$.

% Both the following TWO paragraphs explain why Peng's algorithm are hard to parallelize.
% Morris: why intuitive parallel algorithm is not good

The sequential VGLCS algorithm~\ref{alg:serial-VGLCS} by
Peng~\cite{Peng2011TheLC} and other variants of LCS are difficult to
parallelize.  These algorithms use several states to determine a new
state during dynamic programming.  This construction requires {\em heavy
data dependency}, and makes it difficult to parallelize the computation
in a naive row-by-row manner.  If we parallelize Peng's algorithm with
wavefront method intuitively, it will require extra space to record all
row status, which is not required in the origin sequential algorithm.
Simultaneously, the length of the critical path of wavefront method is
greater than the row-by-row manner.


% On the other hand, if we use the
% Maleki's~\cite{Maleki2016EfficientPU} technique, % need to explain this technique
% it also uses extra
% space to maintain state translation, and spends more time to merge
% data.  Therefore, it is crucial that our parallelization conserves
% {\em both} memory and time.

% It seems that you should use the new figures to explain these???
% Morris: how to make it to row-by-row manner

Therefore, we focus on parallel VGLCS algorithm by row-by-row manner.
The parallel method is decided by the data structure, and the data
structure is often an bottleneck for the performance of the parallel
program.  We observe that the disjoint set implementation of Peng's
algorithm is difficult to parallelize for three reasons.  First, the
query in the second stage will change the data structure because the
lookup operation will compress the path to the root, so it is difficult
to maintain a consistent view of the data structure when multiple
processing units are compressing the path {\em simultaneously}.  Second,
in the first stage when multiple processing units are compressing
different paths, the load among them could be very different, and this
will incur load imbalance. Third, in the first stage there will be a
large number of threads that work on different part of the disjoint-set
forests, therefore it will be difficult to synchronize them efficiently.

\subsection{Sparse Table}

Since the disjoint set cannot support ISMQ efficiently in parallel, we
consider the following data structures to support ISMQ efficiently in
parallel.  Therefore, we find two major data structures which can
parallel build table and answer queries, and the workflow of the VGLCS
problem is shown in Figure~\ref{fig:fig-VGLCS-dp-rmq}.

% these figures were never referenced????

\begin{figure}
  \includegraphics[width=0.82\linewidth]{\GraphicPath/fig-VGLCS-dp-rmq.pdf}
  \caption{Two-step workflow}
  \label{fig:fig-VGLCS-dp-rmq}
\end{figure}


\begin{itemize}
  \item Segment tree~\cite{berg2000computational} supports ranged
    maximum query and update in multi-dimensions.  The time complexity
    of both update and query is $O(\log n)$ in one-dimension.
  \item Sparse table~\cite{Berkman1993RecursiveSP} requires a $O(n
    \log n)$ preprocessing, and can support ranged maximum query in
    $O(1)$ time on one dimensional data.  A sparse table is a two
    dimensional array.  The element of a sparse table in the $j$-th
    row and $i$-th column is the maximum among the $i$-th elements
    and its $2^j - 1$ predecessors in the input array.
\end{itemize}

\begin{figure}[!thb]
  \centering \subfigure[Array]{
    \includegraphics[width=0.45\linewidth]{\GraphicPath/fig-interval-decomposition-origin.pdf}
    \label{fig:fig-interval-decomposition}
  } \subfigure[Sparse Table]{
    \includegraphics[width=0.45\linewidth]{\GraphicPath/fig-sparse-table-origin.pdf}
    \label{fig:fig-sparse-table}
  }
  \caption{A sparse table example}
  \label{fig:interval-decomposition}
\end{figure}

We give an example of the sparse table.  The input is in array $A$. We
split array $A$ into five blocks so that each block has four elements.
The we build a sparse table $ST$ on $A$ as described earlier.  Now a
ranged maximum query on $A$ can be answered by at most {\em two}
queries into the sparse table.  For example, if the query is of the
range from 2 to 13, then the answer is the maximum of the two answers
-- one from 2 to 9, and one from 6 to 13.

\subsection{Parallel VGLCS with Sparse Table}

We now present a simple version of our parallel VGLCS algorithm.  The
algorithm uses a sparse table and its time complexity is $O(n^2 \log n
/ p + n \log n)$, where $p$ is the number of processors.  In
Section~\ref{sec:parallelIRMQ}, we present a more complicated version
that uses a variant of the sparse table, and runs in $O(n^2 / p + n
\log n)$.

\begin{algorithm}[H]
\SetAlgoNoLine
\LinesNumbered
\KwIn{$A[0 .. N-1]$: the input array}
\KwOut{$T$: a sparse table for the input array $A$}

Create a two-dimensional array $T[\log N][N]$ \;
Copy $A$ to $T[0]$ \;
\For{$j \gets 0$ to $\log N$} {
  \ForPar{$i \gets 2^j$ to $N$} {
    $T[j][i] \gets \max(T[j-1][i-2^{j}], T[j-1][i])$ \;
  }
}
  \caption{A Parallel Sparse Table Building Algorithm}
  \label{alg:parallel-sparse-table}
\end{algorithm}


The operations on a sparse table are much easily to parallelize than
those on a disjoint, which is used in the second stage of Peng's
sequential VGLCS algorithm.  The second stage of Peng's algorithm
alternates between append and query operations.  This alternation
between append and query incurs heavy data dependency.  In addition,
the parallelism of operations on a disjoint tree is limited by the
length of path under compression.  The length is usually very short
and provide very limited parallelism.  In contrast our parallel sparse
table implementation in Algorithm~\ref{alg:parallel-sparse-table} runs
in only $O(n \log n / p + \log n)$, where $n$ is the number of
elements and $p$ is the number of processors, and is very easy to
parallelize and to implement.  The pseudo code of our simple version
parallel VGLCS algorithm using sparse table is given in
Algorithm~\ref{alg:parallel-VGLCS}.

\begin{algorithm}[!thb]
\SetAlgoNoLine
\KwIn{$A, B$: the input string; $G_A, G_B$: the array of variable gapped constraints;}
\KwOut{Find the LCS with variable gapped constraints}
    
Create an array of $m$ data structure $C[m]$ to support ISMQ \;
Create an empty table $V[n][m]$ \;
\For{$i \gets 1$ to $n$} {
  \ForPar{$j \gets 1$ to $m$} {
    $M[j] \gets$ suffix maximum of length $G_A(i) + 1$ from $C[j]$ \;
  }
  Build a sparse table $T$ with data of $M$ in parallel with Algorithm~\ref{alg:parallel-sparse-table}\; 
  \ForPar{$j \gets 1$ to $m$} {
    \If{$A[i] = B[j]$} {
        $t \gets $ the range maximum among the $G_B(j) + 1$ elements before $M[j]$ by querying $T$ \;
        $V[i][j] \gets t + 1$ \;
        Append $V[i][j]$ to $C[j]$ \;
    }
  }
}
Retrieve the VGLCS by tracing $V[n][m]$ \;
  \caption{Parallel Algorithm for Finding VGLCS}
  \label{alg:parallel-VGLCS}
\end{algorithm}

