\section{Parallel VGLCS Algorithm} %
\label{sec:parallelSerial}

In the serial algorithm which was designed by Yung-Hsing Peng, we
observed that variants of LCS use many statuses to decide a new
status. Its definition makes lots of data dependency, so we cannot
parallel row by row intuitively.  Even though it's hard to parallel by
several modifications, we can parallel the serial algorithm by using
wavefront method.  Because the usage of the cache memory is hard to
use efficiently in it, so Saeed Maleki~\cite{Maleki2016EfficientPU}
developed a technique that uses the property of rank convergence to
exploit more parallelism to solve dynamic programming problems.

\iffalse
在 $O(nm \alpha(n))$ 的序列算法 \ref{alg:serial-VGLCS} 中,
我們發現算法如大多數的變型 LCS 相同,依賴數個狀態以轉移當前狀態,
大量的資料依賴性不易於細粒度平行。使用波前運行平行是一種常見的解決方案,
由於這種平行對於運行時的快取不友善 (cache-unfriendly),
在 Saeed Maleki ~\cite{saeed} 論文中提到如何使用 Rank Convergence 的特殊性質,
拓展出更高平行度來解決動態規劃的相關問題。
\fi

\begin{algorithm*}[!thb]
  \caption{Algorithm for Finding VGLCS}
  \label{alg:serial-VGLCS}
  \begin{algorithmic}[1]
    \Require
      $A, B$: the input string;
      $G_A, G_B$: the array of variable gapped constraints;
    \Ensure Find the LCS with variable gapped constraints
    \State Create $m$ number of data structure $Q[m]$ to support ISMQ problem.
    \State Create an empty table $V[n][m]$.
    \For{$i \gets 1$ to $n$}
      \State Create a data structure $RQ$ to support ISMQ problem.
      \State $r \gets i - (GA[i]+1)$
      \For{$j \gets 1$ to $m$}
        \If{$A[i] = B[j]$}
            \State $t \gets $ query suffix maximum value from position $j - (GB[j]+1)$ to tail in $RQ$.
            \State $V[i][j] \gets t + 1$
            \State $t \gets $ get the suffix maximum value from position $r$ to $i$ in $Q[j]$
            \State Append value $t$ into $RQ$.
            \State Append value $V[i][j]$ into $Q[j]$.
        \Else
            \State $V[i][j] \gets 0$
            \State $t \gets $ get the suffix maximum value from position $r$ to $i$ in $Q[j]$
            \State Append value $t$ into $RQ$.
        \EndIf
      \EndFor
    \EndFor
    \State Retrieve the VGLCS by tracing $V[n][m]$
  \end{algorithmic}
\end{algorithm*}

In the algorithm ~\ref{alg:serial-VGLCS}, it run in $O(nm)$
time and $O(nm)$ space.  If we use wavefront method to parallel, it
must use extra storage space to record all row status compared to
origin algorithm.  If use the rank convergence technique, it also uses
extra storage space to reserve the information of state translation
and spends more time to merge split parts.

\iffalse
序列算法的空間複雜度為 $O(nm)$。若使用波前平行,需要同時維護橫向的所有狀態,
需要多付出一倍的空間量。若加入 Rank Convergence 的想法拓展出,
勢必要記錄轉移的狀態,需要耗費更多的記憶體空間,用以在最後階段合併所用。
\fi

In this paper, we tend to design the algorithm which has less space
and be more cache-friendly.  We define two stages in the serial
algorithm, row and column stage.  In the row stage, it uses disjoint
set to maintain \emph{incremental suffix maximum query} (ISMQ).
However, ISMQ change itself state of the data structure when it
appends a new value each time, so it makes harder to parallel.  We
study some alternative plans as follows:

\iffalse
這裡我們傾向空間複雜度常數小且針對快取友善設計算法。
平行算法主要分成兩個階段-縱向和橫向階段,縱向階段為數個列的後綴極值查找,
橫向階段在行上運行 $n$ 個元素和 $n$ 組詢問。
在橫向階段,我們需要解決增長後綴最大值查找 (\emph{incremental suffix maximum query}, ISMQ)
易於實作的并查集支持單一操作 $O(\alpha(n))$。
然而,在過程中每插入一個元素便改動數據結構以支持下一個後綴詢問,
這部分使得查詢難以平行化。為消除資料相依性,我們找到幾種區間詢問的替代方案。如:
\fi

\begin{itemize}
  \item 

Binary Indexed Tree\cite{Fenwick1994AND} -- $O(\log n)$: It supports
prefix sum query and update a single value in $O(\log n)$.  In our
application, it also supports updated maximum value and query suffix
maximum value.  It runs faster than segment tree.

  \item 

Segment Tree\cite{berg2000computational} -- $O(\log n)$: It supports
query maximum value and update values in multi-dimension.  It can use
$O(\log n)$ time to solve range maximum query in one-dimension.

  \item 

Sparse Table\cite{Berkman1993RecursiveSP} -- $O(n \log n)$ -- $O(1)$:
We use $ST[j][i]$ to present the maximum value in array $(i-2^j,i]$.
It spend $O(n \log n)$ time to build table and require $O(1)$ time to
get the range maximum query. In order to support append value to tail,
it cannot run $O(n)$ -- $O(1)$ solution which Fischer
~\cite{Fischer2006TheoreticalAP} introduced.

\end{itemize}

\iffalse
\begin{itemize}
  \item 樹狀數組 (Binary Indexed Tree) -- $O(\log n)$: 對於任意前綴查找極值和更新元素,單一操作的時間複雜度為 $O(\log n)$,其運行常數比線段樹低。
  \item 線段樹 (Segment Tree) -- $O(\log n)$: 支持更高維度的正交區塊搜索,而我們用在區間極值查找需要 $O(\log n)$ 的時間完成所有區間查詢操作。
  \item 稀疏表 (Sparse Table) -- $O(n)$ -- $O(1)$:
    建立表格 $ST[j][i]$ 表示區間 $(i-2^j,i]$ 之間的極值。建表時間複雜度需 $O(n)$
    ,對於任意區間詢問可以拆分 2 個 super-block 檢索和 2 個 in-block 檢索,
    如圖 \ref{fig:interval-decomposition} 的說明,轉換過程和存取時間皆需要 $O(1)$。
\end{itemize}
\fi

\begin{figure*}[!thb]
  \centering
  \subfigure[Array]{
    \includegraphics[width=0.45\linewidth]{graphics/fig-interval-decomposition.pdf}
    \label{fig:fig-interval-decomposition}
  }
  \subfigure[Sparse Table]{
    \includegraphics[width=0.45\linewidth]{graphics/fig-sparse-table.pdf}
    \label{fig:fig-sparse-table}
  }
  \caption{
An example for illustrating the sparse table, which has an array $A$.
$A$ is split into five blocks, each block has four elements. A range
minimum/maximum query can be decomposed into at most four different
sub-queries.  If the query range maximum value in $[2, 18]$, it will
merge four maximum results $B1$, $Q_L$, $B5$, and $Q_R$.
}

  \label{fig:interval-decomposition}
\end{figure*}

Among of them, we consider that the sparse table is the best
alternative plan.  We present the parallel algorithm ~\ref{alg
:parallel-VGLCS} for the VGLCS problem and its time complexity is
$O(n^2 / p + n \log n)$, which $p$ is the number of processors. In the
next chapter, we provide a new data structure to instead of disjoint
set. In the next section, we reduce unstable algorithm to more stable
in amortized $O(n^2)$ theoretically.

\iffalse
稀疏表是我們認為最好的替代方案,其整合後為 VGLCS 平行算法 \ref{alg:parallel-VGLCS},
算法的時間複雜度為 $O(n^2 / p + n \log n)$,其中 $p$ 為處理器個數。
在後續的章節,我們將提出新的數據結構取代并查集操作,
且能在平行算法達到理想複雜度 $O(n^2 / p + n \log n)$。
\fi

\begin{algorithm*}[!thb]
  \caption{Parallel Algorithm for Finding VGLCS}
  \label{alg:parallel-VGLCS}
  \begin{algorithmic}[1]
    \Require
      $A, B$: the input string;
      $G_A, G_B$: the array of variable gapped constraints;
    \Ensure Find the LCS with variable gapped constraints
    \State Create $m$ number of data structure $Q[m]$ to support ISMQ problem.
    \State Create an empty table $V[n][m]$.
    \For{$i \gets 1$ to $n$}
      \State Create a sparse table data structure $\textit{sp}$, and initialize $\textit{sp}$ to zero.
      \ParFor{$j \gets 1$ to $m$}
        \State $\textit{sp}[j] \gets$ query suffix maximum value from position $r$ to tail in $Q[j]$.
      \EndParFor
      \State Build sparse table $\textit{sp}$ with $m$ elements in parallel $O(n/p \log n + \log n)$ time.
      \ParFor{$j \gets 1$ to $m$}
        \If{$A[i] = B[j]$}
            \State $t \gets $ query suffix maximum value from position $j - (GB[j] + 1)$ to $j-1$ in $\textit{sp}$
            \State $V[i][j] \gets t + 1$
            \State Append value $V[i][j]$ into $Q[j]$.
        \EndIf
      \EndParFor
    \EndFor
    \State Retrieve the VGLCS by tracing $V[n][m]$
  \end{algorithmic}
\end{algorithm*}