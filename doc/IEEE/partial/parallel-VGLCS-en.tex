\section{Parallel VGLCS Algorithm} %
\label{sec:parallelVGLCS}

The serial VGLCS algorithm by Peng~\ref{alg:serial-VGLCS} and other
variants of LCS are difficult to parallelize.  These algorithms
maintain several states to determine a new state during dynamic
programming.  This construction requires heavy data dependency, and
make it difficult to parallelize the computation in a naive row-by-row
manner.

The {\em wavefront method} address this parallelization difficulty by
keeping the computation as wavefront.  However, the wavefront
computation is not cache-friendly, i.e., the wavefront algorithm cannot
effectively keep the required data in cache effectively.  To address
this cache issue Maleki~\cite{Maleki2016EfficientPU} developed a
technique that uses the property of {\em rank convergence} to exploit
more parallelism in the dynamic programming.  

The linear-tropical dynamic programming
problem~\cite{Maleki2016EfficientPU} which has the property of {\em rank
convergence} can be divided into the sub-problems , such as a sequence
of stages, which a solution of the sub-problems depends on only the
solutions of the previous sub-problem.  About rank convergence in VGLCS
problem, the sub-problems has developed efficient algorithm by Peng.
However, the parallel forward phase is difficult to integrate the partial
correct and incorrect solution in parallelization algorithm which using
rank convergence.

%need to say
%something abuut
%rank
%convergence

\iffalse
在 $O(nm \alpha(n))$ 的序列算法 \ref{alg:serial-VGLCS} 中,
我們發現算法如大多數的變型 LCS 相同,依賴數個狀態以轉移當前狀態,
大量的資料依賴性不易於細粒度平行。使用波前運行平行是一種常見的解決方案,
由於這種平行對於運行時的快取不友善 (cache-unfriendly),
在 Saeed Maleki ~\cite{saeed} 論文中提到如何使用 Rank Convergence 的特殊性質,
拓展出更高平行度來解決動態規劃的相關問題。
\fi

\begin{algorithm*}[!thb]
  \caption{Algorithm for Finding VGLCS}
  \label{alg:serial-VGLCS}
  \begin{algorithmic}[1]
    \Require
      $A, B$: the input string;
      $G_A, G_B$: the array of variable gapped constraints;
    \Ensure Find the LCS with variable gapped constraints
    \State Create $m$ number of data structure $Q[m]$ to support ISMQ problem.
    \State Create an empty table $V[n][m]$.
    \For{$i \gets 1$ to $n$}
      \State Create a data structure $RQ$ to support ISMQ problem.
      \State $r \gets i - (GA[i]+1)$
      \For{$j \gets 1$ to $m$}
        \If{$A[i] = B[j]$}
            \State $t \gets $ query suffix maximum value from position $j - (GB[j]+1)$ to tail in $RQ$.
            \State $V[i][j] \gets t + 1$
            \State $t \gets $ get the suffix maximum value from position $r$ to $i$ in $Q[j]$
            \State Append value $t$ into $RQ$.
            \State Append value $V[i][j]$ into $Q[j]$.
        \Else
            \State $V[i][j] \gets 0$
            \State $t \gets $ get the suffix maximum value from position $r$ to $i$ in $Q[j]$
            \State Append value $t$ into $RQ$.
        \EndIf
      \EndFor
    \EndFor
    \State Retrieve the VGLCS by tracing $V[n][m]$
  \end{algorithmic}
\end{algorithm*}

% XXX -> last name
% XXX = Peng

Efficient parallelization of VGLCS is difficult in both memory and time.
Peng proposed a serial algorithm for the VGLCS problem~\ref{alg:serial-
VGLCS}, which runs in $O(nm)$ time and $O(nm)$ space.  Simultaneously,
Peng transfer the VGLCS problem into {\em incremental suffix maximium
query} problem

If we parallel this algorithm with wavefront method
intuitively, it will require extra space to record all row status,
which is not required in the origin algorithm.  On the other hand, if
we use the rank convergence technique, it also uses extra space to
reserve the information of state translation, and spends more time to
merge split parts.  It is crucial that our parallelization conserves
both memory and time.

\iffalse 序列算法的空間複雜度為 $O(nm)$。若使用波前平行,需要同時維護橫
向的所有狀態,需要多付出一倍的空間量。若加入 Rank Convergence 的想法拓
展出,勢必要記錄轉移的狀態,需要耗費更多的記憶體空間,用以在最後階段合
併所用。\fi

Peng found the VGLCS with {\em incremental suffix maximium query} (ISMQ)
and handling ISMQ by union and find.  The disjoint-set data structure
for solving the union-find problem can be found in
Gabow's~\cite{Gabow1983ALA} and Tarjan's~\cite{Tarjan1975EfficiencyOA}
article, and its amortized time is only $O(\alpha(n))$.  In the parallel
environment, the unstable operation make load imbalance to run
efficiently.  In this paper, we propose a parallel VGLCS algorithm that
is space-efficient and cache-friendly.  The sequential version has two
stages -- row and column stage.  In the row stage, the algorithm uses a
{\em disjoint set} to efficiently answer {\em incremental suffix
maximum query} (ISMQ).  However, a ISMQ will change the state of the
data structure when it appends a new value, so it is difficult to
parallelize.

% query ususally does not change anything. Are you sure it is qurey?
% Morris: Yes, disjoint set use path compression during query.
% ISMQ is out of the blue.  You need to give an exmaple and  motivation 

\iffalse 這裡我們傾向空間複雜度常數小且針對快取友善設計算法。平行算法主
要分成兩個階段-縱向和橫向階段,縱向階段為數個列的後綴極值查找,橫向階
段在行上運行 $n$ 個元素和 $n$ 組詢問。在橫向階段,我們需要解決增長後綴
最大值查找 (\emph{incremental suffix maximum query}, ISMQ)易於實作的并
查集支持單一操作 $O(\alpha(n))$。然而,在過程中每插入一個元素便改動數據
結構以支持下一個後綴詢問,這部分使得查詢難以平行化。為消除資料相依性,
我們找到幾種區間詢問的替代方案。如:\fi

We consider several data structures to support ISMQ efficiently in
parallel.

\begin{itemize}
  \item Binary Indexed Tree\cite{Fenwick1994AND} -- $O(\log n)$: It
    supports prefix sum query and update a single value in $O(\log
    n)$.  In our application, it also supports updated maximum value
    and query suffix maximum value.  It runs faster than segment tree.
  \item Segment Tree\cite{berg2000computational} -- $O(\log n)$: It
    supports query maximum value and update values in multi-dimension.
    It can use $O(\log n)$ time to solve range maximum query in
    one-dimension.
  \item Sparse Table\cite{Berkman1993RecursiveSP} -- $O(n \log n)$ --
    $O(1)$: We use $ST[j][i]$ to present the maximum value in array
    $(i-2^j,i]$.  It spend $O(n \log n)$ time to build table and
    require $O(1)$ time to get the range maximum query. In order to
    support append value to tail, it cannot run $O(n)$ -- $O(1)$
    solution which Fischer ~\cite{Fischer2006TheoreticalAP}
    introduced.
\end{itemize}

\iffalse
\begin{itemize}
\item 樹狀數組 (Binary Indexed Tree) -- $O(\log n)$: 對於任意前綴查找極
  值和更新元素,單一操作的時間複雜度為 $O(\log n)$,其運行常數比線段樹
  低。
  \item 線段樹 (Segment Tree) -- $O(\log n)$: 支持更高維度的正交區塊搜
    索,而我們用在區間極值查找需要 $O(\log n)$ 的時間完成所有區間查詢操
    作。
  \item 稀疏表 (Sparse Table) -- $O(n)$ -- $O(1)$:建立表格 $ST[j][i]$
    表示區間 $(i-2^j,i]$ 之間的極值。建表時間複雜度需 $O(n)$,對於任意
      區間詢問可以拆分 2 個 super-block 檢索和 2 個 in-block 檢索,如圖
      \ref{fig:interval-decomposition} 的說明,轉換過程和存取時間皆需要
      $O(1)$。
\end{itemize}
\fi

\begin{figure*}[!thb]
  \centering \subfigure[Array]{
    \includegraphics[width=0.45\linewidth]{graphics/fig-interval-decomposition.pdf}
    \label{fig:fig-interval-decomposition}
  } \subfigure[Sparse Table]{
    \includegraphics[width=0.45\linewidth]{graphics/fig-sparse-table.pdf}
    \label{fig:fig-sparse-table}
  }
  \caption{An example for illustrating the sparse table, which has an
    array $A$.  $A$ is split into five blocks, each block has four
    elements. A range minimum/maximum query can be decomposed into at
    most four different sub-queries.  If the query range maximum value
    in $[2, 18]$, it will merge four maximum results $B1$, $Q_L$,
    $B5$, and $Q_R$.}
  \label{fig:interval-decomposition}
\end{figure*}

We present our parallel algorithm for the VGLCS problem.  The algorithm
use {\em sparse table} and and its time complexity is $O(n^2 / p + n
\log n)$, where $p$ is the number of processors.  In
Chapter~\ref{sec:parallelIRMQ}, we present a new data structure --
variant sparse table instead of disjoint set.  Also, we use them to
develop more stable algorithm in amortized $O(n^2)$ theoretically.

% what data structure
% Morris: our compressed sparse table and O(1) sparse table to support ISMQ
% reduce?? stable??
% Morris: The sparse table is more stable version compared to disjoint set version.

The pseudo code of our algorithm is in Algorithm~\ref{alg:parallel-VGLCS}

\iffalse 稀疏表是我們認為最好的替代方案,其整合後為 VGLCS 平行算法
\ref{alg:parallel-VGLCS},算法的時間複雜度為 $O(n^2 / p + n \log n)$,
其中 $p$ 為處理器個數。在後續的章節,我們將提出新的數據結構取代并查集操
作,且能在平行算法達到理想複雜度 $O(n^2 / p + n \log n)$。\fi

\begin{algorithm*}[!thb]
  \caption{Parallel Algorithm for Finding VGLCS}
  \label{alg:parallel-VGLCS}
  \begin{algorithmic}[1]
    \Require
      $A, B$: the input string;
      $G_A, G_B$: the array of variable gapped constraints;
    \Ensure Find the LCS with variable gapped constraints
    \State Create $m$ number of data structure $Q[m]$ to support ISMQ problem.
    \State Create an empty table $V[n][m]$.
    \For{$i \gets 1$ to $n$}
      \State Create a sparse table data structure $\textit{sp}$, and initialize $\textit{sp}$ to zero.
      \ParFor{$j \gets 1$ to $m$}
        \State $\textit{sp}[j] \gets$ query suffix maximum value from position $r$ to tail in $Q[j]$.
      \EndParFor
      \State Build sparse table $\textit{sp}$ with $m$ elements in parallel $O(n/p \log n + \log n)$ time.
      \ParFor{$j \gets 1$ to $m$}
        \If{$A[i] = B[j]$}
            \State $t \gets $ query suffix maximum value from position $j - (GB[j] + 1)$ to $j-1$ in $\textit{sp}$
            \State $V[i][j] \gets t + 1$
            \State Append value $V[i][j]$ into $Q[j]$.
        \EndIf
      \EndParFor
    \EndFor
    \State Retrieve the VGLCS by tracing $V[n][m]$
  \end{algorithmic}
\end{algorithm*}
