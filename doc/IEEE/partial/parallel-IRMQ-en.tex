\section{Query on Incrementally Updated Data} \label{sec:parallelIRMQ}

Recall that our VGLCS algorithm has two stages.  In the first stage,
the algorithm uses incremental suffix maximum query to compute
intermediate data on every {\em column} of a two dimensional matrix.
In the second stage, the algorithm uses {\em ranged maximum query} on
the rows of the matrix from the first stage to get the final answers.
These two stages work together to find the maximum in a given
rectangle area.

The implementation for two stages have different challenges.  The
first stage is easier to parallelize because the operations on
individual columns are independent.  However, it will insert new data
into the data structure, and it still needs to answer ranged query
efficiently. The second stage does not requires insertion so it is
more static, and easier.  However, it requires working on several
columns simultaneously and synchronously.  Fortunately we have address
this data synchronous issue by computing the index table in a off-line
manner, as described in previous Section~\ref{sec:parallelIRMQ}.  As a
result in this section we will focus on how to answer ranged maximum
queries efficiently on {\em incrementally} updated data.

% not here, somewhere else
%\begin{table}
  %\tiny
  \centering
  \caption{   Our study shows in the bold front. We use the fixed size
$s=16$ on Cartesian tree. The small amortized constant will not
encounter serious load imbalance problem.   }

  \label{tlb:cmp-complexity}
  \begin{tabular}{ccc}
    \toprule
     & serial & parallel \\
    \midrule
    horizontal & \begin{tabular}{@{}c@{}}
              $\left \langle n \alpha(n) \right \rangle$ \cite{yunghsing} \\ 
              amortized $\left \langle n \right \rangle$\end{tabular}
              & \begin{tabular}{@{}c@{}}
                $\left \langle n \alpha(n)/p + \alpha(n) \right \rangle$ \cite{yunghsing} \\
                amortized $\left \langle n /p + o(1) \right \rangle$
                \end{tabular} \\
    vertical & \begin{tabular}{@{}c@{}}
                $\left \langle n \alpha(n) \right \rangle$ \\
                amortized $\left \langle n \right \rangle$
                \end{tabular}
            & \begin{tabular}{@{}c@{}}
                impossible \\
                amortized $\left \langle n /p + o(1) \right \rangle$
              \end{tabular}
              \\
    total & \begin{tabular}{@{}c@{}}
              $\left \langle n^2 \alpha(n) \right \rangle$ \cite{yunghsing} \\ 
              amortized $\left \langle n^2 \right \rangle$\end{tabular}
          & \begin{tabular}{@{}c@{}}
              impossible \\ 
              amortized $\left \langle n^2 /p + n \log n \right \rangle$\end{tabular} \\
    \bottomrule
  \end{tabular}
\end{table}

\subsection{Build a Lookup Table for LCA}

% why LCA is important

To find the lowest common ancestor is important for answering ranged
maximum queries, which is important to find VGLCS.  We need to address
two issues -- how to encode a binary tree and how to find the lowest
common ancestor under the given tree encoding.

%% In VGLCS problem, we could not use sorting to improve cache miss
%% because the number of elements and queries are similar.  

\subsubsection{Cartesian Tree Encoding}

Our Cartesian tree encoding method lists {\em all} binary search trees
in {\em lexicographical order} and label them from $0$ to the $n$-th
Catalan number minus 1.  The lexicographical order among binary search
trees of the {\em same} number of nodes is defined {\em recursively}
as follows.  A binary tree $a$ appears {\em before} another binary if
$a$ has more nodes than $b$ in the left subtree, or $a$ and $b$ has
the same number of nodes in the left subtree, and $a$'s left subtree
appears before $b$'s left subtree in lexicographical order, or $a$ and
$b$ has the same left subtree, and $a$'s right subtree appears before
$b$'s right subtree in lexicographical order.
Figure~\ref{fig:labelingBST} shows our labeling of binary search tree
for 1, 2 and 3 nodes.

\begin{figure}[!thb]
  \centering
  \includegraphics[width=\linewidth]{graphics/fig-bst-encoding.pdf}
  \caption{The labeling of binary search trees}
  \label{fig:labelingBST}
\end{figure}

\subsubsection{Lowest Common Ancestor}

We also need to determine the lowest common ancestor {\em efficiently}
for answering ranged maximum queries.  For ease of discussion we
define several notations.  Let $t$ be the index of the search tree in
our encoding, so $t$ is between 0 and $s - 1$, where $s$ is the number
of search trees.  Let ${\cal A}(s, t, p, q)$ denote the {\em lowest
  common ancestor} of the node $p$ and $q$ within a binary search tree
with $s$ nodes and label $t$.  For example, ${\cal A}(3, 2, 0, 2) = 1$
from Figure~\ref{fig:labelingBST}.  Also we consider the tree of $s$
nodes with label $t$, and let $l_s$ denote the size of the left
subtree, $r_s$ denote the size of the right subtree, $l_t$ be the
label of its left subtree, and $r_t$ be the label of its right
subtree.  With these notations we can define the lowest common
ancestor {\em recursively} as in Equation~\ref{fun:LCA1} when $l_s \le
p \le q < n$.  Other cases are defined in Equation~\ref{fun:LCA2}.

\begin{equation*}
  \begin{split}
    &\mathit{LCA}(n, \mathit{tid}, p, q) \\
      &= \left\{\begin{matrix*}[l]
        \mathit{LCA}(\mathit{lsz}, \mathit{lid}, p, q) &&, p \le q < \mathit{lsz}\\ 
        \mathit{LCA}(\mathit{rsz}, \mathit{rid}, p-\mathit{lsz}-1, q-\mathit{lsz}-1)+\mathit{lsz}+1 &&, 
            \mathit{lsz} \le p \le q < n \\ 
        \mathit{lsz} && , 0 \le p \le \mathit{lsz}, \mathit{lsz} \le q \le i\\ 
        -1 && ,\mathit{otherwise}
      \end{matrix*}\right.
  \end{split}
\end{equation*}

We analyze the space and time complexity of
Algorithm~\ref{alg:parallel-LCA}.  The lookup table records the all
the binary tree of sizes from 1 to $s$.  When tree size is $m$, the
number of different binary trees is the $n$-th Catalan number $C_m$,
which is $\frac{1}{m+1}\binom{2m}{m} = O(\frac{4^m}{m^{1.5}})$.  For
each binary tree of size $m$, we store the lowest common ancestor of
{\em every} pair of nodes into the table, so the size of the table is
$O(m^2)$.  Therefore, the space complexity is $O(s \times
\frac{1}{s+1}\binom{2s}{s} \times s^2)$, where $s$ is the number of
elements in a block.  When we set $s$ to $\frac{\log n}{4}$, the space
complexity is $O(\sqrt{n} \log ^{1.5} n)$.  Note that the query will
be on the tree size of $s$ {\em only}.  However, we do need the space
for tables of {\em smaller} tree sizes as intermiedate data to compute
the table of tree size $s$.  Also since the number of operations in
Equation~\ref{fun:LCA1} and \ref{fun:LCA2} is a constant, the time
complexity is also $O(\sqrt{n} \log ^{1.5} n)$ when we set $s$ to
$\frac{\log n}{4}$.

We now analyze the time complexity of the parallel version of
Algorithm~\ref{alg:parallel-LCA}.  As deescribed earlier, the
sequential time complexity of Algorithm~\ref{alg:parallel-LCA} is
$O(\frac{s^3}{s+1} \binom{2s}{s})$.Now we have $p$ processor, and we
find that the computation of the $C_m$ tree of size $m$ are
independent, and can be done in parallel.  However, the time
complexity of finding the sizes and idsof subtrees wil take $O(m)$ for
a tree of size $m$.  Since both line 4 and 5 are in the same loop
body, it is not necessary to parallel line 6 since line 5 will
dominate the time of the loop body.  As a result we can only
parallelize the double loops in line 2 and 3 in
Algortihm~\ref{alg:parallel-LCA}, and the time complexity of our
parallel algorithm is $O(\frac{s^3}{s+1} \binom{2s}{s} / p + s^2) =
O(\sqrt{n} (\log ^{1.5} n) / p + \log^2 n )$.


%% % how to compute subtree information from t

%% Note that in line 4 of Algorithm~\ref{alg:parallel-LCA}, when given
%% the tree id $t$, we need to compute the sizes and ids of the left and
%% right subtrees in our encoding.  We can do this in $O(n)$ time, where
%% $n$ is the number of tree nodes.

\begin{algorithm}[!thb]
  \caption{Parallel Algorithm for building LCA}
  \label{alg:parallel-LCA}
  \begin{algorithmic}[1]
    \Require
      $s$: Maximum size for required the number of BST
    \For{$n \gets 1$ to $s$}
      \ParFor{$\mathit{tid} \gets 0$ to $C_n - 1$}
        \ParFor{$p \gets 0$ to $n-1$}
          \State compute $\langle\mathit{lsz},\mathit{lid},\mathit{rsz},\mathit{rid}\rangle$
          \For{$q \gets p$ to $n-1$}
            \State $\textit{LCA}[n][\mathit{tid}][p][q] \gets$ Equation~\ref{fun:LCA1} and \ref{fun:LCA2}
          \EndFor
        \EndParFor
      \EndParFor
    \EndFor
  \end{algorithmic}
\end{algorithm}


\subsection{Tree Index Computation}

Note that Algorithm~\ref{alg:parallel-LCA} requires tree index $t$
under our encoding scheme, so we need to determine $t$ efficiently,
given a block of data.

% how to compute t from tree data structure

One way to determine the tree index $t$ is to build the tree and compute
it from the the sizes and ids of the left and right subtrees. This
require a recursive traversal on the tree and has a $O(n)$ time
complexity, where $n$ is the number of tree nodes.  The
Equation~\ref{fun:tid} is given as the conversion.

% \begin{algorithm}
  \caption{Get $tid$ from $\langle\mathit{lsz},\mathit{lid},\mathit{rsz},\mathit{rid}\rangle$ in $\theta(1)$ time}
  \label{alg:encode-tid}
  \begin{algorithmic}[1]
    \Require
      $\langle\mathit{lsz},\mathit{lid},\mathit{rsz},\mathit{rid}\rangle$: size and label in left/right subtree
    \Ensure
      $\mathit{tid}$: this label
    \If{$\mathit{rsz} = 0$}
      \State return $\mathit{lid}$
    \EndIf
    \State $n = \mathit{lsz}+\mathit{rsz}+1$
    \State $\mathit{base} = 0$
    \For{$i=0$ to $\mathit{lsz}-1$}
      \State $\mathit{base}$ = $\mathit{base} + C_i \cdot C_{n-i-1}$
    \EndFor
    \State $\mathit{offset}$ = $\mathit{lid} \cdot C_{\mathit{rsz}}$ + $\mathit{rid}$
    \State return $\mathit{base}$ + $\mathit{offset}$
  \end{algorithmic}
\end{algorithm}

\begin{equation}  \label{fun:tid}
  {\cal T}(s_l, t_l, s_r, t_r) = t_l \; {\cal C}_{s_r} + t_r +
  \sum_{i = 0}^{s_l - 1} {\cal C}_i \; {\cal C}_{s_l + s_r - i}
\end{equation}


We can further optimize Equation~\ref{fun:tid} by pre-computing the
prefix sum of Catalan numbers, and store them in memory, so that we can
use them directly, instead of recomputing them in
Equation~\ref{fun:tid}.  That is, we can pre-compute these summation,
and replace the summation in Equation~\ref{fun:tid} with a table lookup.

% how to compute t with rightmost path of the tree

In order to find the index of the block, we build a Cartesian tree
corresponding to the elements of the block, and then find the id of the
Cartesian tree. The previous computation of tree index requires build
the tree to obtain subtree information, and may not be efficient.  We
propose a method that keeps only the {\em rightmost path} in a stack
without building the entire tree.  After knowing tree index $t$ we can
compute LCA and answer queries with Algorithm~\ref{alg:parallel-LCA}.

We compute the tree index $t$ on the rightmost path efficiently. Because
the Cartesian tree for a sequence is constructed in linear time using a
stack-based algorithm, we reserve the $\mathit{lid}, \mathit{lsz}$
instead of making a edge of nodes for each nodes, which these are on the
rightmost path which is stored in the stack.  In the rotate operation of
building Cartesian tree algorithm, we maintain the $\mathit{rid},
\mathit{rsz}$ in the pop operations, and the subtree index $t$ can be
recomputed with Equation~\ref{fun:tid} by the $\mathit{rid},
\mathit{rsz}$ and the information $\mathit{lid}, \mathit{lsz}$ on the
stack.

The Algorithm~\ref{alg:cartesian-encode-offline} is shown for offline
encoding any Cartesian tree.  Each element can take part in at most two
operations, so the algorithm~\ref{alg:cartesian-encode-offline} run in
$O(s)$ time.

\begin{algorithm*}
  \caption{Offline Type of Cartesian Tree}
  \label{alg:cartesian-encode-of}
  \begin{algorithmic}[1]
    \Require
      $A[1 \cdots s]$: storage array;
      $s$: the number of elements;
    \Ensure
      $\mathit{tid}$: this label
    \State $\langle\mathit{lsz},\mathit{lid},\mathit{value}\rangle$ $D$[$s+1$]
    \State $\textit{Dp} \gets 0$
    \State $D[0] \gets \langle 0,0,\infty \rangle$
    \For{$i \gets 1$ to $s$}
      \State $v \gets A[i], \; \textit{lsz} \gets 0, \; \textit{lid} \gets 0$
      \While{$D[\textit{Dp}].\textit{value} < v$}
        \State $\textit{lid} \gets \textit{tid}(D[\textit{Dp}].\textit{lsz}, D[\textit{Dp}].\textit{lid}, \textit{lsz}, \textit{lid})$
        \State $\textit{lsz} \gets \textit{lsz} + D[\textit{Dp}].\textit{lsz} + 1$
        \State $\textit{Dp} \gets \textit{Dp} - 1$
      \EndWhile
      \State $\textit{Dp} \gets \textit{Dp} + 1$
      \State $D[\textit{Dp}] \gets \langle\mathit{lsz},\mathit{lid},\mathit{v}\rangle$
    \EndFor
    \State $\textit{lsz} \gets 0, \; \textit{lid} \gets 0$
    \While{$\textit{Dp} > 0$} \Comment{pop all elements}
      \State $\textit{lid} \gets \textit{tid}(D[\textit{Dp}].\textit{lsz}, D[\textit{Dp}].\textit{lid}, \textit{lsz}, \textit{lid})$
      \State $\textit{lsz} \gets \textit{lsz} + D[\textit{Dp}].\textit{lsz} + 1$
      \State $\textit{Dp} \gets \textit{Dp} - 1$
    \EndWhile
    \State return $\textit{lid}$
  \end{algorithmic}
\end{algorithm*}

% Morris: what is the incremental ranged maximum query?

The incremental ranged maximum query problem which is more powerful than
incremental suffix maximum query problem, which supports three
operations.  First, a {\tt make} operation creates an empty array $A$.
Second, a {\tt append(V) } operation appends a value $V$ to the end of
an array $A$.  Finally, a {\tt query(L, R)} operation finds the {\em
maximum} value among the $L$-th value to the $R$-th value of an array
$A$.  

% Morris: Cartesian tree can be used to solve above problem

By the interactive encoding Cartesian tree, we can use the index of tree
to support the {\tt query(L, R)} operation.  The stack-based algorithm
of a Cartesian tree can support the {\tt append(V)} operation, but we
could not cope with a {\tt query(L, R)} operation on a Cartesian tree
intuitively.

% Morris: How to use tree index to lookup? Then, what's encode method we
% know in previous study.

In order to solve {\tt query(L, R)} operation on a Cartesian tree, we
can build lookup table to answer the maximum value by the tree index.
There is a main issue in answering incremental ranged maximum query --
Cartesian tree encoding.  Fischer introduced the first encoding method
and the Masud~\cite{Hasan2010CacheOA} presents a new encoding method to
reduce the number of instructions.  Unfortunately all these algorithm
work in a off-line model, i.e., they assume all $n$ data are given in
advance, therefore they cannot cope with incrementally updated data in
our problem.  In addition, they require a preprocessing of time $O(n)$.
The preprocessing need more memory transfer to find the information of
the block of an input array, or read external files from disk.

% Morris: Our encoding with parallel algorithm without find the block of
% input array or external files.

Now, we provide the dynamic encoding method so that each operation is
amortized $O(1)$ time.  This encoding method is designed corresponding
to the Algorithm~\ref{alg:parallel-LCA} and Equation~\ref{fun:tid}.

In order to support online encoding, we use five variables to present
the state of the Cartesian tree.  The next step will insert $i$-th
elements and final stage fill $s$ number of elements.  The index of the
current tree is ${\mathit t}$ and the rightmost path of the Cartesian
tree is presented by two variable, stack pointer ${\mathit Dp}$ and the
stack ${\mathit D}$.  Before any insertion, the initialization state set
$i$ to zero, $s$ to $\frac{\log n}{4}$, ${\mathit t}$ to $C_s$ subtract
one, ${\mathit Dp}$ to zero, and the top element in stack ${\mathit D}$
to infinity.

The structure of state is as follows:

\iffalse
我們定義轉移狀態由 5 個變數來決定動態笛卡爾樹的編碼,當前插入第 $i$ 個
元素,最終填充 $s$ 個元素,當前的樹編號 $\mathit{tid}$,以及笛卡爾樹的
右鏈狀態指針 $Dp$ 與其堆疊 $D$,其結構如下:
\fi

% State(i = 0, s = n, tid = C[n]-1, Dp = 0, D[0].val = INF)

\begin{minipage}{0.9\linewidth}

\lstinputlisting[frame=single,basicstyle=\tt,caption=State of Cartesian Tree]{codes/cartesian-state.h}

\end{minipage}

% Morris: Idea for encode Cartesian tree dynamically: virtual node &
% propagation

In order to encode Cartesian tree dynamically, we initialize the $s$
number of virtual node on the rightmost path, and that is why the
default tree index ${\mathit t}$ is $C_s - 1$, which $C_s$ is the $s$-th
Catalan number.  Following the elements insertion, we assume the
sequence of elements which is not yet inserted are increasing.  In
lexicographical order, the rightmost path of Cartesian tree is belonged
to the lower dimension in the row-major like.  Simultaneously, building
a Cartesian tree only modify the rightmost path.  We use the propagation
to get the tree index in the next insertion.  Finally, we propose the
difference algorithm~\ref{alg:cartesian-encode-online} to satisfy above
requirement.

\iffalse 為了解決在線詢問操作,取 $s = \frac{\log n}{4}$。根據字典順序
的編碼性質,一開始建立虛設點 $s$ 個在右鏈上,其樹編號 $\mathit{tid} =
C_s - 1$ 。隨著插入元素的增加,尚未加入的元素都預設嚴格遞減,加上根據
編碼順序,我們藉由差值來維護在線編碼 (如圖
~\ref{fig:cartesianEncoding})。根據上述的編碼想法,我們得到算法
~\ref{alg:cartesian-encode-online}。\fi

We give an example of difference algorithm in the
figure~\ref{fig:cartesianEncoding}.  Each block has $s$ number of
elements.  We will build a Cartesian tree with $s$ number of nodes to
solve in-block query.  In initialization, it assume $s$ number of nodes
on the rightmost path and the default tree index $\mathit{tid} = C_s -
1$.  When inserting $i$-th element, the tree index is $\mathit{tid}_i$,
and the tree index of the subtree $A$ is $A.\mathit{tid}$.  If the value
of $(i+1)$-th element is $x$, it will rotate onto the node $A$.  After
rotation, $A$ is a left subtree of $A$, and we can compute the index of
subtree $A$ during rotation.  Then, $x.\mathit{tid}$ can be computed by
the $s-(i+1)$ number of virtual nodes on the rightmost path and
$A.\mathit{tid}$.  According to the lexicographical order, we get
$\mathit{tid}_{i+1} = \mathit{tid}_i + (x.\mathit{tid} -
A.\mathit{tid})$.

\begin{algorithm}[!thb]
  \caption{Online Type of Cartesian Tree}
  \label{alg:cartesian-encode-online}
  \begin{algorithmic}[1]
  \Require
      $\mathit{state}$: state of Cartesian Tree;
      $v$: the value which append to array
  \Ensure
      $\mathit{tid}$: this label
  \State $\textit{Dp} \gets \textit{state}.\textit{Dp}$, $\textit{lsz} \gets 0$, $\textit{lid} \gets 0$
  \State $\textit{bsz} \gets \textit{state}.\textit{s} - \textit{state}.\textit{i} + 1$
  \State $\textit{bid} \gets C[\textit{bsz}] - 1$
  \While{$\textit{state}.D[\textit{Dp}].\textit{value} < v$}
    \State $\textit{lid} \gets \textit{tid}(\textit{state}.D[\textit{Dp}].\textit{lsz}, \textit{state}.D[\textit{Dp}].\textit{lid}, \textit{lsz}, \textit{lid})$
    \State $\textit{bid} \gets \textit{tid}(\textit{state}.D[\textit{Dp}].\textit{lsz}, \textit{state}.D[\textit{Dp}].\textit{lid}, \textit{bsz}, \textit{bid})$
    \State $\textit{lsz} \gets \textit{lsz} + \textit{state}.D[\textit{Dp}].\textit{lsz}+1$
    \State $\textit{bsz} \gets \textit{bsz} + \textit{state}.D[\textit{Dp}].\textit{lsz}+1$
    \State $\textit{Dp} \gets \textit{Dp} - 1$
  \EndWhile
  \State $\textit{Dp} \gets \textit{Dp} + 1$
  \State $\textit{state}.D[\textit{Dp}] \gets \left \langle \textit{lsz}, \textit{lid}, \textit{v} \right \rangle$
  \State $\textit{state}.\textit{Dp} \gets \textit{Dp}$
  \State $x.\textit{tid} \gets \textit{tid}(\textit{lsz}, \textit{lid}, \textit{state}.s-\textit{state}.i, C[\textit{state}.s-\textit{state}.i]-1)$
  \State $\textit{state}.\textit{tid} \gets \textit{state}.\textit{tid} + \textit{bid} - x.\textit{tid}$
  \State $\textit{state}.i \gets \textit{state}.i + 1$
  \State return $\textit{state}.\textit{tid}$
  \end{algorithmic}
\end{algorithm}

\begin{figure*}[!thb]
  \centering
  \includegraphics[width=\linewidth]{graphics/fig-cartesian-encoding.pdf}

  \caption{An example for difference algorithm to encode Cartesian tree.}

  \label{fig:cartesianEncoding}
\end{figure*}

Finally, we do not increase the time complexity of the building
Cartesian tree algorithm because each operation is $O(1)$.  For the
in-block query, we get the identity of the Cartesian tree in $O(1)$,
and look up table to find the result.

\iffalse
最後,我們不改變原本的建立笛卡爾樹算法,便能在過程中擭得樹的編號,
每一次的 in-block 詢問只需要一次記憶體存取,得到任一操作攤銷複雜度 $\theta(1)$。
\fi
