\section{Maximum Interval Query on Incremental Data} \label{sec:QIUD}

This section describes our approach to address the challenges in the
first stage of Algorithm~\ref{alg:parallel-VGLCS}, where we compute
the suffix maximum on every {\em column} of $V$ while new data are
added.  Here we generalize our technique so that we can also answer
incremental {\em range} maximum queries on incrementally added data,
so that our technique can be applied to other cases that require
queries on an interval.

% not here, somewhere else
%\begin{table}
  %\tiny
  \centering
  \caption{   Our study shows in the bold front. We use the fixed size
$s=16$ on Cartesian tree. The small amortized constant will not
encounter serious load imbalance problem.   }

  \label{tlb:cmp-complexity}
  \begin{tabular}{ccc}
    \toprule
     & serial & parallel \\
    \midrule
    horizontal & \begin{tabular}{@{}c@{}}
              $\left \langle n \alpha(n) \right \rangle$ \cite{yunghsing} \\ 
              amortized $\left \langle n \right \rangle$\end{tabular}
              & \begin{tabular}{@{}c@{}}
                $\left \langle n \alpha(n)/p + \alpha(n) \right \rangle$ \cite{yunghsing} \\
                amortized $\left \langle n /p + o(1) \right \rangle$
                \end{tabular} \\
    vertical & \begin{tabular}{@{}c@{}}
                $\left \langle n \alpha(n) \right \rangle$ \\
                amortized $\left \langle n \right \rangle$
                \end{tabular}
            & \begin{tabular}{@{}c@{}}
                impossible \\
                amortized $\left \langle n /p + o(1) \right \rangle$
              \end{tabular}
              \\
    total & \begin{tabular}{@{}c@{}}
              $\left \langle n^2 \alpha(n) \right \rangle$ \cite{yunghsing} \\ 
              amortized $\left \langle n^2 \right \rangle$\end{tabular}
          & \begin{tabular}{@{}c@{}}
              impossible \\ 
              amortized $\left \langle n^2 /p + n \log n \right \rangle$\end{tabular} \\
    \bottomrule
  \end{tabular}
\end{table}

\subsection{Build Least Common Ancestor Table}

Recall from the discussion of Cartesian tree in
Section~\ref{sec:parallelRMQ} that finding the {\em least common
  ancestor} is important for answering range maximum queries.  Here we
need to address two issues -- how to label a binary tree with a {\em
  Catalan index} and how to find the least common ancestor of two
nodes in a given Cartesian tree.

\subsubsection{Cartesian Tree Labeling}

Our Cartesian tree labeling will enumerate all binary search trees in
{\em lexicographical order} from $0$ to the $n$-th Catalan number
minus 1.  This lexicographical order among binary search trees of the
{\em same} number of nodes is defined {\em recursively} as follows.  A
binary tree $x$ appears {\em before} another binary tree $y$ if any of
the the following condition is true.  Figure~\ref{fig:labelingBST}
shows our lexicographical labeling of binary search trees of sizes 1,
2 and 3.

\begin{itemize}
\item $x$ has more nodes than $y$ in the left subtree.
\item $x$ and $y$ have the same number of nodes in the left subtree,
  and $x$'s left subtree appears before $y$'s left subtree in
  lexicographical order.
\item $x$ and $y$ has the same left subtree, and $x$'s right subtree
  appears before $y$'s right subtree in lexicographical order.
\end{itemize}

\begin{figure}[!thb]
  \centering
  \includegraphics[width=0.8\linewidth]{\GraphicPath/fig-bst-encoding.pdf}
  \caption{The labeling of binary search trees of sizes 1, 2, and 3.}
  \label{fig:labelingBST}
\end{figure}

\subsubsection{Least Common Ancestor}

We also need to determine the {\em least common ancestor} efficiently
for answering range maximum queries when data are incrementally added.
Let $t$ be the Catalan index of the binary search tree from our
lexicographical labeling, so $t$ is between 0 and ${\cal C}_s-1$, where
$s$ is the number of search trees.  Let ${\cal A}(s, t, p, q)$ denote
the least common ancestor of the node $p$ and $q$ within a binary search
tree of size $s$ that has a Catalan index $t$.  For example, ${\cal
A}(3, 2, 0, 2) = 1$ from Figure~\ref{fig:labelingBST}.  Now let $s_l$
denote the size of the left subtree, $s_r$ denote the size of the right
subtree, $t_l$ be the Catalan index of its left subtree, and $t_r$ be
the Catalan index of its right subtree.  With these notations we can
define the least common ancestor $\cal A$ {\em recursively} as in
Equation~\ref{fun:LCA}.  A Pseudo code for computing the LCA table $\cal
A$ is given in Algorithm~\ref{alg:parallel-LCA}.

\begin{equation*}
  \begin{split}
    &\mathit{LCA}(n, \mathit{tid}, p, q) \\
      &= \left\{\begin{matrix*}[l]
        \mathit{LCA}(\mathit{lsz}, \mathit{lid}, p, q) &&, p \le q < \mathit{lsz}\\ 
        \mathit{LCA}(\mathit{rsz}, \mathit{rid}, p-\mathit{lsz}-1, q-\mathit{lsz}-1)+\mathit{lsz}+1 &&, 
            \mathit{lsz} \le p \le q < n \\ 
        \mathit{lsz} && , 0 \le p \le \mathit{lsz}, \mathit{lsz} \le q \le i\\ 
        -1 && ,\mathit{otherwise}
      \end{matrix*}\right.
  \end{split}
\end{equation*}

\begin{algorithm}[!thb]
\SetAlgoNoLine
\KwIn{$s$: The maximum tree size}

\For{$n \gets 1$ to $s$} {
  \ForPar{$t \gets 0$ to $C_n - 1$} {
    \ForPar{$p \gets 0$ to $n-1$} {
      Compute $s_l$, $t_l$, $s_r$, and $t_r$ \;
      \For{$q \gets p$ to $n-1$} {
        Compute ${\cal A}[n][t][p][q]$ according to Equation~\ref{fun:LCA1} and \ref{fun:LCA2} \;
      }
    }
  }
}

\caption{A parallel algorithm that computes the least common ancesstor table}
\label{alg:parallel-LCA}
\end{algorithm}


We first analyze the space complexity of
Algorithm~\ref{alg:parallel-LCA}.  The table $\cal A$ keeps all LCA
information for binary trees of sizes from 1 to $s$.  When tree size is
$m$, the number of different binary trees is the $m$-th Catalan number
${\cal C}_m$, which is
$\frac{1}{m+1}\binom{2m}{m}=O(\frac{4^m}{m^{1.5}})$. For each binary
tree of size $m$, we store the least common ancestor of {\em every} pair
of nodes into the table, so the size of the table is $O(m^2)$.
Therefore, the space complexity is $O(s \times
\frac{1}{s+1}\binom{2s}{s} \times s^2)$, where $s$ is the number of
elements in a block.  When we set $s$ to $\frac{\log n}{4}$, the space
complexity is $O(\sqrt{n} \; \log ^{1.5} n)$.  Note that the size of
range query is up to tree size $s$.  However, we do need the space for
tables of {\em smaller} tree sizes as intermediate data to compute the
table of tree size $s$.

We now analyze the time complexity of the parallel version of
Algorithm~\ref{alg:parallel-LCA}.  The time complexity of a sequential
version of Algorithm~\ref{alg:parallel-LCA} is $O(\frac{s^3}{s+1}
\binom{2s}{s})$ because the number of operations in
Equation~\ref{fun:LCA} is $O(1)$.  The time complexity hence becomes
$O(\sqrt{n} \; \log ^{1.5} n)$ when we set $s$ to $\frac{\log n}{4}$.

Now for the parallel version, we observe that the computation of the
${\cal C}_m$ trees of size $m$ are {\em independent}, hence can be done
in parallel.  However, the time to find the sizes and Catalan indices of
subtrees (in line \ref{alg:parallel-LCA-compute}) is $O(m)$ for a tree
of size $m$.  Since both line \ref{alg:parallel-LCA-compute} and
\ref{alg:parallel-LCA-loop} are in the same loop body, it is not
necessary to parallelize line~\ref{alg:parallel-LCA-compute} because the
loop starting at line~\ref{alg:parallel-LCA-loop} will dominate the
time.  As a result, we only parallelize the double loops in
line~\ref{alg:parallel-LCA-for-2} and \ref{alg:parallel-LCA-for-3} in
Algorithm~\ref{alg:parallel-LCA}, and the time complexity of our
parallel algorithm is $O(\frac{s^3}{s+1} \binom{2s}{s} / p + s^2) =
O(\sqrt{n} \; (\log^{1.5} n) / p + \log^2 n )$, where $p$ is the number
of processors.

%% % how to compute subtree information from t

%% Note that in line 4 of Algorithm~\ref{alg:parallel-LCA}, when given
%% the tree id $t$, we need to compute the sizes and ids of the left and
%% right subtrees in our encoding.  We can do this in $O(n)$ time, where
%% $n$ is the number of tree nodes.

\subsection{Catalan Index Computation}

Note that Algorithm~\ref{alg:parallel-LCA} requires Catalan index $t$,
so we need to determine $t$ efficiently when given a block of data.
There are two possible approaches -- build the tree or keep only the
rightmost path.

% how to compute t from tree data structure

\subsubsection{Build the Tree}

In order to find the Catalan index of a block of data, we can build a
Cartesian tree corresponding to the data of the block, and then find the
index of the Cartesian tree.  That is, we build the tree and compute it
from the the sizes and indices of the left and right subtrees.  This
requires a recursive traversal on the tree and has a $O(n)$ time
complexity, where $n$ is the number of tree nodes.  Let ${\cal T}$
denote the Catalan index.  After knowing these information, we can
compute the Catalan index ${\cal T}$ by Equation~\ref{fun:tid}.  Recall
that $s_l$ denotes the size of the left subtree, $s_r$ denotes the size
of the right subtree, $t_l$ is the Catalan index of the left subtree,
and $t_r$ is the Catalan index of the right subtree.  Recall that ${\cal
C}_i$ denotes the Catalan number of tree of size $i$.

% \begin{algorithm}
  \caption{Get $tid$ from $\langle\mathit{lsz},\mathit{lid},\mathit{rsz},\mathit{rid}\rangle$ in $\theta(1)$ time}
  \label{alg:encode-tid}
  \begin{algorithmic}[1]
    \Require
      $\langle\mathit{lsz},\mathit{lid},\mathit{rsz},\mathit{rid}\rangle$: size and label in left/right subtree
    \Ensure
      $\mathit{tid}$: this label
    \If{$\mathit{rsz} = 0$}
      \State return $\mathit{lid}$
    \EndIf
    \State $n = \mathit{lsz}+\mathit{rsz}+1$
    \State $\mathit{base} = 0$
    \For{$i=0$ to $\mathit{lsz}-1$}
      \State $\mathit{base}$ = $\mathit{base} + C_i \cdot C_{n-i-1}$
    \EndFor
    \State $\mathit{offset}$ = $\mathit{lid} \cdot C_{\mathit{rsz}}$ + $\mathit{rid}$
    \State return $\mathit{base}$ + $\mathit{offset}$
  \end{algorithmic}
\end{algorithm}

\begin{equation}  \label{fun:tid}
  {\cal T}(s_l, t_l, s_r, t_r) = t_l \; {\cal C}_{s_r} + t_r +
  \sum_{i = 0}^{s_l - 1} {\cal C}_i \; {\cal C}_{s_l + s_r - i}
\end{equation}


We can further optimize Equation~\ref{fun:tid} by precomputing the
{\em prefix sum} of the Catalan number products.  Then we store these
sums in memory, so that we can use them directly, instead of
recomputing them as in Equation~\ref{fun:tid}.  

% how to compute t with rightmost path of the tree

\subsubsection{Keep the Rightmost Path}

The previous computation of Catalan index requires building the tree
to obtain subtree information, and may not be efficient.  We propose a
more efficient method that determines the Catalan index by keeps only
the {\em rightmost path} in a {\em stack}, without building the entire
tree.  This technique is similar to the {\em rightmost-pops encoding}
in Section~\ref{sec:cct}.  After knowing the Catalan index $t$ we can
compute LCA and answer queries with Algorithm~\ref{alg:parallel-LCA}
and Equation~\ref{fun:tid}.

We compute the Catalan index $t$ efficiently by the maintaining its {\em
rightmost path} in a {\em stack}.  That is, we will {\em not} build
these left subtrees along the rightmost path.  Instead, we keep their
Catalan indices and sizes in the stack (denoted by $D$).  That is, the
stack $D$ has the Catalan {\em indices} and {\em sizes} of every {\em
left subtree} along the rightmost path.  Please refer to
Algorithm~\ref{alg:cartesian-encode-offline} for the details on the
stack.  In Algorithm~\ref{alg:cartesian-encode-offline} each node of the
stack $D$ has three members -- $v$ as the data, $s$ as the size of its
subtree, and $t$ as the index of its left subtree.  We also use a
pointer $p$ to point to the top of the stack.

\begin{algorithm}
\SetAlgoNoLine
\KwIn{
  $A[1 \cdots s]$: storage array\;
  $s$: the number of elements\;
}
\KwOut{
  $\mathit{tid}$: this label
}
$\langle\mathit{lsz},\mathit{lid},\mathit{value}\rangle$ $D$[$s+1$] \;
$\textit{Dp} \gets 0$ \;
$D[0] \gets \langle 0,0,\infty \rangle$ \;
\For{$i \gets 1$ to $s$} {
  $v \gets A[i], \; \textit{lsz} \gets 0, \; \textit{lid} \gets 0$ \;
  \While{$D[\textit{Dp}].\textit{value} < v$} {
    $\textit{lid} \gets \textit{tid}(D[\textit{Dp}].\textit{lsz}, D[\textit{Dp}].\textit{lid}, \textit{lsz}, \textit{lid})$ \;
    $\textit{lsz} \gets \textit{lsz} + D[\textit{Dp}].\textit{lsz} + 1$ \;
    $\textit{Dp} \gets \textit{Dp} - 1$ \;
  }
  $\textit{Dp} \gets \textit{Dp} + 1$ \;
  $D[\textit{Dp}] \gets \langle\mathit{lsz},\mathit{lid},\mathit{v}\rangle$ \;
}

$\textit{lsz} \gets 0, \; \textit{lid} \gets 0$ \;
\While{$\textit{Dp} > 0$} {
  $\textit{lid} \gets \textit{tid}(D[\textit{Dp}].\textit{lsz}, D[\textit{Dp}].\textit{lid}, \textit{lsz}, \textit{lid})$ \;
  $\textit{lsz} \gets \textit{lsz} + D[\textit{Dp}].\textit{lsz} + 1$ \;
  $\textit{Dp} \gets \textit{Dp} - 1$ \;
}
return $\textit{lid}$ \;

\caption{Offline Type of Cartesian Tree}
\label{alg:cartesian-encode-offline}
\end{algorithm}

\begin{figure}[!thb]
  \centering
  \includegraphics[width=0.4\linewidth]{\GraphicPath/fig-cartesian-encoding-static.pdf}
  \caption{Compute Catalan index for a tree.  $A_l$ and $B_l$ denote
    the left subtrees of $A$ and $B$ respectively.}
  \label{fig:fig-cartesian-encoding-static}
\end{figure}

Algorithm~\ref{alg:cartesian-encode-offline} computes the Catalan index
for a given block of data with a stack $D$.  In the first double loop,
the outer loop goes through every input and the inner loop inserts a
data at the {\em end} of the rightmost path, which is at the top of the
stack $D[p]$, and traverse towards the root by popping any {\em smaller}
data out of the stack $D$.  When we rotate nodes along the rightmost
path to update the Cartesian tree, we compute the new {\em index} $t$
and size $s$ of the new left subtree whenever the newly inserted data
replaces a node on the rightmost path.  As a result, the new Catalan
index $t$ can be recomputed with Equation~\ref{fun:tid} by the indices
and sizes of the left and right subtrees in the stack.  Please refer to
the first while loop of Algorithm~\ref{alg:cartesian-encode-offline} and
Figure~\ref{fig:fig-cartesian-encoding-static} for an illustration. Note
that $A_l$ and $B_l$ denote the left subtrees of $A$ and $B$
respectively.  After popping all smaller data in the stack the while
loop stops and the size, index, and input are pushed into the new top of
stack.  Finally, we pop all data out of the stack and compute the
Catalan index for the entire block in the last loop of
Algorithm~\ref{alg:cartesian-encode-offline}.

Algorithm~\ref{alg:cartesian-encode-offline} can compute any Catalan
index for Cartesian trees with the entire block of data and the block
size.  The algorithm runs in $O(s)$ time since an element is
pushed/popped at most {\em once}.

\subsection{Dynamic Catalan Index Computation} \label{sec:dynamic}

Several encoding methods were proposed for indexing Cartesian search
trees.  Fischer~\cite{Fischer2006TheoreticalAP} introduced the first
encoding method and Masud~\cite{Hasan2010CacheOA} presents a new
encoding method that reduces the number of instructions.
Unfortunately all these algorithms are {\em off-line}, i.e., they
assume all data are given in advance, as a result, they cannot cope
with incrementally added data.  In addition, they require a
preprocessing of time $O(n)$, where $n$ is the number of data.  The
preprocessing requires extra memory to process the input data block,
or reads external files from disk.

We generalize our Catalan tree labeling technique for incrementally
added data by a {\em normalization}, which keeps the size of tree as a
constant $n$.  That is, suppose we want to encode a series of
Cartesian search trees of {\em increasing} sizes up to $n$, we can
append a path of $n-i$ {\em right-child-only} nodes to the rightmost
path of the existing tree of size $i$, so that the total number of
node is {\em always} $n$.  That is, when the computation starts
without any existing tree nodes, we will start with a path of $n$
right-child-only nodes.  Please refer to
Figure~\ref{fig:cartesianEncoding}(a) for an illustration.  After we
have added $i$ nodes, we will have a path of $n - i$ right-child-only
nodes attached to the rightmost path, as illustrated by
Figure~\ref{fig:cartesianEncoding}(b).  For ease of explanation we
will refer to this added path as the {\em virtual path}.

\begin{figure*}[!thb]
  \centering \subfigure[Initiation $t_{{\it root}_0}= {\cal C}_n - 1$]{
    \includegraphics[width=0.26\linewidth]{\GraphicPath/fig-cartesian-encoding.pdf}
    \label{fig:cartesianEncoding-init}
  } \subfigure[$t_{{\it root}_i}$]{
    \includegraphics[width=0.3\linewidth]{\GraphicPath/fig-cartesian-encoding-before.pdf}
    \label{fig:cartesianEncoding-before}
  } \subfigure[$t_{{\it root}_{i+1}} = t_{{\it root}_i} + t_x - t_A$]{
    \includegraphics[width=0.3\linewidth]{\GraphicPath/fig-cartesian-encoding-after.pdf}
    \label{fig:cartesianEncoding-after}
  }
  \caption{Normalization of Cartesian trees of increasing sizes by
    adding a virtual path.}
  \label{fig:cartesianEncoding}
\end{figure*}

There are two advantages of this tree normalization.  First, this
normalization simplifies the computation of Catalan index.  That is,
we can update the Catalan index with ease whenever we insert a new
data. Second, this normalization simply adds increasingly {\em
  larger} data to the end of a data block, so that the answers to the
interval maximum queries on actual data are {\em not} affected.  As a
result we can {\em dynamically} maintain a lookup table to obtain the
maximum value within a range, so as to answer a range query from this
dynamic Cartesian tree encoding.

%% Now, we provide the dynamic encoding method so that each operation is
%% amortized $O(1)$ time.

 
The dynamic Catalan index computation with tree normalization works as
follows.  We will maintain the Catalan index of the current Cartesian
tree (as $t^*$), and a stack of the rightmost path of the current tree
(as $D$).  Then we will {\em update} these information every time we add
a new data $x$ into the tree, by calling
Algorithm~\ref{alg:cartesian-encode-online}, which is based on
Algorithm~\ref{alg:parallel-LCA} and Equation~\ref{fun:tid}.  At any
given time $i$, where $i$ is the index of the data, we can compute the
Catalan index of the current tree, which has $i$ nodes, and use it to
answer a range maximum query.

For ease of notation, we use $t$ and $s$ to denote the Catalan index and
the size of the Cartesian tree {\em without} the virtual path. These
notations are similar to those in
Algorithm~\ref{alg:cartesian-encode-offline}.  We also use $t'$ and $s'$
to maintain the index and size of the tree {\em with} the virtual path.
Please refer to Figure~\ref{fig:cartesianEncoding} for an illustration,
where the tree enclosed by the red dotted line does include the virtual
path.

Algorithm~\ref{alg:cartesian-encode-online} first initializes $s$ and
$t$ as Algorithm~\ref{alg:cartesian-encode-offline} does, since it will
also work from the end of the rightmost path and start with an empty
tree.  Then Algorithm~\ref{alg:cartesian-encode-online} initializes $s'$
to $n - i$ since we will append the a virtual path of length $n - i$ to
the end of the rightmost path.  Consequently, $t'$ is ${\cal C}_{s'} -
1$ since it has the largest Catalan index of that size.

Algorithm~\ref{alg:cartesian-encode-online} then inserts the $i$-th
element $x$ in two stages.  The
line~\ref{alg:cartesian-encode-online-pop-c1} and
\ref{alg:cartesian-encode-online-pop-c2} of
Algorithm~\ref{alg:cartesian-encode-online} pops the elements smaller
than $x$ from the stack $D$, using $p$ as the stack pointer of $D$. Note
that $t$ and $s$ are the index and size of the last subtree {\em
without} the virtual path.  Consequently, we popped smaller elements
from the stack and update $s$ and $t$ just as in
Algorithm~\ref{alg:cartesian-encode-online}.  In
line~\ref{alg:cartesian-encode-online-pop-virtual-c1} and
\ref{alg:cartesian-encode-online-pop-virtual-c2} of the loop
Algorithm~\ref{alg:cartesian-encode-online} updates $s'$ and $t'$
similarly.  However it must maintain a virtual path of $n-s'$
right-child-only nodes to the right of the rightmost path.  Please refer
to the Algorithm~\ref{alg:cartesian-encode-online} and
Figure~\ref{fig:cartesianEncoding} for details.

\begin{algorithm}[!thb]
\SetAlgoNoLine
\KwIn{
  $S$: state of Cartesian Tree;  $v$: the added data;
  $D$: A stack where every element has $s$, $t$, and $v$ \;
}
\KwOut{
  $t$: The Catalan index of the input data block after adding $v$
}

$p \gets S.p$, $s \gets 0$, $t \gets 0$ \;
$s' \gets S.s - S.{i} + 1$ \;
$t' \gets C[s'] - 1$ \;

\While{$D[p].{value} < v$} {
  $t \gets t(D[p].s, D[p].t, s, t)$ \;
  $t' \gets t(D[p].s, D[p].t, s', t')$ \;
  $s \gets s + D[p].s+1$ \;
  $s' \gets s' + D[p].s+1$ \;
  $p \gets p - 1$ \;
}
$p \gets p + 1$ \;
$D[p] \gets \left \langle s, t, {v} \right \rangle$ \;
$S.p \gets p$ \;
$x.t \gets t(s, t, S.s-S.i, C[S.s-S.i]-1)$ \;
$S.t \gets S.t + t' - x.t$ \;
$S.i \gets S.i + 1$ \;
return $S.t$ \;

  \caption{Online Type of Cartesian Tree}
  \label{alg:cartesian-encode-online}
\end{algorithm}


Now we are ready to update the stack $D$ and return the overall Catalan
index $t^*$.  The new top of stack is now $x$, with left subtree of size
$s$ and index $t$, so we push them into the stack. Please refer to
Figure~\ref{fig:cartesianEncoding-after} for an illustration.  Next we
update the overall Catalan index $t^*$.  After we popped all elements
smaller than $x$, the original subtree of these popped nodes on the
rightmost path, should be inserted as the left subtree of $x$.  Please
compare Figure~\ref{fig:cartesianEncoding-before} (before) and
Figure~\ref{fig:cartesianEncoding-after} (after).

We observe that the only difference between
Figure~\ref{fig:cartesianEncoding-before} and
Figure~\ref{fig:cartesianEncoding-after} is the area enclosed by the red
dotted line.  If we compute the {\em difference} between the Catalan
indices of these two red dotted line area, we can compute the final
$t^*$ (Figure~\ref{fig:cartesianEncoding-after}) by {\em patching} the
Catalan index in Figure~\ref{fig:cartesianEncoding-before} with this
difference. Fortunately our tree lexicographical ordering considers left
subtree {\em before} right subtree, so the difference of the Catalan
indices between Figure~\ref{fig:cartesianEncoding-before} and
Figure~\ref{fig:cartesianEncoding-after} is $t' - {\cal T}(s, t, s - i,
{\cal C}_{s-i} - 1)$.  As a result, the new $t^*$ is the old value of
$t^*$, plus $t' - {\cal T}(s, t, s - i, {\cal C}_{s-i} - 1)$.

Note that Algorithm~\ref{alg:cartesian-encode-online} does {\em not}
increase the amortized time complexity to compute the Catalan index,
even when we dynamically add data into the tree.  As described earlier,
the time complexity of the static index computation
(Algorithm~\ref{alg:cartesian-encode-online}) is $O(n)$ because each
element is popped exactly once.  Similarly the total cost of calling
Algorithm~\ref{alg:cartesian-encode-online} $n$ times is also $O(n)$,
since each operation within the loop is $O(1)$.  As a result, the
amortized cost for an in-block query, including the computation of
Catalan index, is also $O(1)$.
