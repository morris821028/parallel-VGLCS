\section{Query on Incrementally Added Data} \label{sec:QIUD}

This section describes our approach to address the challenges in the
first stage of Algorithm~\ref{alg:parallel-VGLCS}, where we compute
the suffix maximum on every {\em column} of $V$ while new data is
added incrementally.  Here we generalize our technique so that we can
also answer incremental {\em ranged} maximum on incrementally added
data, so that our technique can be applied to other cases that require
ranged maximum query.

% not here, somewhere else
%\begin{table}
  %\tiny
  \centering
  \caption{   Our study shows in the bold front. We use the fixed size
$s=16$ on Cartesian tree. The small amortized constant will not
encounter serious load imbalance problem.   }

  \label{tlb:cmp-complexity}
  \begin{tabular}{ccc}
    \toprule
     & serial & parallel \\
    \midrule
    horizontal & \begin{tabular}{@{}c@{}}
              $\left \langle n \alpha(n) \right \rangle$ \cite{yunghsing} \\ 
              amortized $\left \langle n \right \rangle$\end{tabular}
              & \begin{tabular}{@{}c@{}}
                $\left \langle n \alpha(n)/p + \alpha(n) \right \rangle$ \cite{yunghsing} \\
                amortized $\left \langle n /p + o(1) \right \rangle$
                \end{tabular} \\
    vertical & \begin{tabular}{@{}c@{}}
                $\left \langle n \alpha(n) \right \rangle$ \\
                amortized $\left \langle n \right \rangle$
                \end{tabular}
            & \begin{tabular}{@{}c@{}}
                impossible \\
                amortized $\left \langle n /p + o(1) \right \rangle$
              \end{tabular}
              \\
    total & \begin{tabular}{@{}c@{}}
              $\left \langle n^2 \alpha(n) \right \rangle$ \cite{yunghsing} \\ 
              amortized $\left \langle n^2 \right \rangle$\end{tabular}
          & \begin{tabular}{@{}c@{}}
              impossible \\ 
              amortized $\left \langle n^2 /p + n \log n \right \rangle$\end{tabular} \\
    \bottomrule
  \end{tabular}
\end{table}

\subsection{Build Ancestor Table}

Recall from the discussion of Cartesian tree in
Section~\ref{sec:parallelRMQ} that finding the {\em lowest common
  ancestor} is important for answering ranged maximum queries.  Here
we need to address two issues -- how to map a binary tree into its
{\em Catalan index} and how to find the lowest common ancestor of two
nodes in a given tree.

%% In VGLCS problem, we could not use sorting to improve cache miss
%% because the number of elements and queries are similar.  

\subsubsection{Cartesian Tree Mapping}

Our Cartesian tree mapping lists {\em all} binary search trees in {\em
  lexicographical order} and label them from $0$ to the $n$-th Catalan
number minus 1.  The lexicographical order among binary search trees
of the {\em same} number of nodes is defined {\em recursively} as
follows.  A binary tree $a$ appears {\em before} another binary if $a$
has more nodes than $b$ in the left subtree, or $a$ and $b$ has the
same number of nodes in the left subtree, and $a$'s left subtree
appears before $b$'s left subtree in lexicographical order, or $a$ and
$b$ has the same left subtree, and $a$'s right subtree appears before
$b$'s right subtree in lexicographical order.
Figure~\ref{fig:labelingBST} shows our Catalan indexing of binary
search tree for 1, 2 and 3 nodes.

\begin{figure}[!thb]
  \centering
  \includegraphics[width=\linewidth]{\GraphicPath/fig-bst-encoding.pdf}
  \caption{The labeling of binary search trees}
  \label{fig:labelingBST}
\end{figure}

\subsubsection{Lowest Common Ancestor}

We also need to determine the lowest common ancestor {\em efficiently}
for answering ranged maximum queries on incrementally added data.  Let
$t$ be the Catalan index of the search tree, so $t$ is between 0 and
$s - 1$, where $s$ is the number of search trees.  Let ${\cal A}(s, t,
p, q)$ denote the {\em lowest common ancestor} of the node $p$ and $q$
within a binary search tree with $s$ nodes and Catalan index $t$.  For
example, ${\cal A}(3, 2, 0, 2) = 1$ from Figure~\ref{fig:labelingBST}.
Also we consider the tree of $s$ nodes with label $t$, and let $s_l$
denote the size of the left subtree, $s_r$ denote the size of the
right subtree, $t_l$ be the Catalan index of its left subtree, and
$t_r$ be the Catalan index of its right subtree.  With these notations
we can define the lowest common ancestor {\em recursively} as in
Equation~\ref{fun:LCA1} when $s_l \le p \le q < n$.  Other cases are
defined in Equation~\ref{fun:LCA2}.  A Pseudo code is given in
Algorithm~\ref{alg:parallel-LCA}.

\begin{equation*}
  \begin{split}
    &\mathit{LCA}(n, \mathit{tid}, p, q) \\
      &= \left\{\begin{matrix*}[l]
        \mathit{LCA}(\mathit{lsz}, \mathit{lid}, p, q) &&, p \le q < \mathit{lsz}\\ 
        \mathit{LCA}(\mathit{rsz}, \mathit{rid}, p-\mathit{lsz}-1, q-\mathit{lsz}-1)+\mathit{lsz}+1 &&, 
            \mathit{lsz} \le p \le q < n \\ 
        \mathit{lsz} && , 0 \le p \le \mathit{lsz}, \mathit{lsz} \le q \le i\\ 
        -1 && ,\mathit{otherwise}
      \end{matrix*}\right.
  \end{split}
\end{equation*}

\begin{algorithm}[!thb]
\SetAlgoNoLine
\KwIn{$s$: The maximum tree size}

\For{$n \gets 1$ to $s$} {
  \ForPar{$t \gets 0$ to $C_n - 1$} {
    \ForPar{$p \gets 0$ to $n-1$} {
      Compute $s_l$, $t_l$, $s_r$, and $t_r$ \;
      \For{$q \gets p$ to $n-1$} {
        Compute ${\cal A}[n][t][p][q]$ according to Equation~\ref{fun:LCA1} and \ref{fun:LCA2} \;
      }
    }
  }
}

\caption{A parallel algorithm that computes the least common ancesstor table}
\label{alg:parallel-LCA}
\end{algorithm}


We first analyze the space complexity of
Algorithm~\ref{alg:parallel-LCA}.  The lookup table records the all
the binary tree of sizes from 1 to $s$.  When tree size is $m$, the
number of different binary trees is the $n$-th Catalan number $C_m$,
which is $\frac{1}{m+1}\binom{2m}{m} = O(\frac{4^m}{m^{1.5}})$.  For
each binary tree of size $m$, we store the lowest common ancestor of
{\em every} pair of nodes into the table, so the size of the table is
$O(m^2)$.  Therefore, the space complexity is $O(s \times
\frac{1}{s+1}\binom{2s}{s} \times s^2)$, where $s$ is the number of
elements in a block.  When we set $s$ to $\frac{\log n}{4}$, the space
complexity is $O(\sqrt{n} \log ^{1.5} n)$.  Note that the query will
be on the tree size of $s$ {\em only}.  However, we do need the space
for tables of {\em smaller} tree sizes as intermiedate data to compute
the table of tree size $s$.  Also since the number of operations in
Equation~\ref{fun:LCA1} and \ref{fun:LCA2} is a constant, the time
complexity is also $O(\sqrt{n} \log ^{1.5} n)$ when we set $s$ to
$\frac{\log n}{4}$.

We now analyze the time complexity of the parallel version of
Algorithm~\ref{alg:parallel-LCA}.  As deescribed earlier, the
sequential time complexity of Algorithm~\ref{alg:parallel-LCA} is
$O(\frac{s^3}{s+1} \binom{2s}{s})$.  We observe that the computation
of the $C_m$ trees of size $m$ are independent, hence can be done in
parallel.  However, the time to find the sizes and ids of subtrees (in
line 4) is $O(m)$ for a tree of size $m$.  Since both line 4 and 5 are
in the same loop body, it is not necessary to parallel line 4 since
line 5 will dominate the time of the loop body.  As a result we can
only parallelize the double loops in line 2 and 3 in
Algortihm~\ref{alg:parallel-LCA}, and the time complexity of our
parallel algorithm is $O(\frac{s^3}{s+1} \binom{2s}{s} / p + s^2) =
O(\sqrt{n} (\log ^{1.5} n) / p + \log^2 n )$, where $p$ is the number
of processors.

%% % how to compute subtree information from t

%% Note that in line 4 of Algorithm~\ref{alg:parallel-LCA}, when given
%% the tree id $t$, we need to compute the sizes and ids of the left and
%% right subtrees in our encoding.  We can do this in $O(n)$ time, where
%% $n$ is the number of tree nodes.

\subsection{Catalan Index Computation}

Note that Algorithm~\ref{alg:parallel-LCA} requires Catalan index $t$,
so we need to determine $t$ efficiently when given a block of data.

% how to compute t from tree data structure

\subsubsection{Build the Tree}

In order to find the Catalan index of the block, we can build a
Cartesian tree corresponding to the elements of the block, and then
find the index of the Cartesian tree.  That is, we build the tree and
compute it from the the sizes and ids of the left and right subtrees.
This require a recursive traversal on the tree and has a $O(n)$ tgime
complexity, where $n$ is the number of tree nodes.  The conversion is
as in Equation~\ref{fun:tid}.  Recall that $s_l$ denotes the size of
the left subtree, $s_r$ denotes the size of the right subtree, $t_l$
is the Catalan index of the left subtree, and $t_r$ is the Catalan
index of the right subtree.

% \begin{algorithm}
  \caption{Get $tid$ from $\langle\mathit{lsz},\mathit{lid},\mathit{rsz},\mathit{rid}\rangle$ in $\theta(1)$ time}
  \label{alg:encode-tid}
  \begin{algorithmic}[1]
    \Require
      $\langle\mathit{lsz},\mathit{lid},\mathit{rsz},\mathit{rid}\rangle$: size and label in left/right subtree
    \Ensure
      $\mathit{tid}$: this label
    \If{$\mathit{rsz} = 0$}
      \State return $\mathit{lid}$
    \EndIf
    \State $n = \mathit{lsz}+\mathit{rsz}+1$
    \State $\mathit{base} = 0$
    \For{$i=0$ to $\mathit{lsz}-1$}
      \State $\mathit{base}$ = $\mathit{base} + C_i \cdot C_{n-i-1}$
    \EndFor
    \State $\mathit{offset}$ = $\mathit{lid} \cdot C_{\mathit{rsz}}$ + $\mathit{rid}$
    \State return $\mathit{base}$ + $\mathit{offset}$
  \end{algorithmic}
\end{algorithm}

\begin{equation}  \label{fun:tid}
  {\cal T}(s_l, t_l, s_r, t_r) = t_l \; {\cal C}_{s_r} + t_r +
  \sum_{i = 0}^{s_l - 1} {\cal C}_i \; {\cal C}_{s_l + s_r - i}
\end{equation}


We can further optimize Equation~\ref{fun:tid} by pre-computing the
{\em prefix sum} of the Catalan number products.  Then we store these
sums in memory, so that we can use them directly, instead of
recomputing them as in Equation~\ref{fun:tid}.  That is, we can
pre-compute these summation, and replace the summation in
Equation~\ref{fun:tid} with a table lookup.

% how to compute t with rightmost path of the tree

\subsubsection{Keep the Rightmost Path}

The previous computation of Catalan index requires building the tree
to obtain subtree information, and may not be efficient.  We propose a
method that detremines the Catalan index by keeps only the {\em
  rightmost path} in a {\em stack} without building the entire tree.
This technique is similar to the {\em compressed cartesian tree} in
Section~\ref{sec:cct}.  After knowing the Catalan index $t$ we can
compute LCA and answer queries with Algorithm~\ref{alg:parallel-LCA}.

We compute the Catalan index $t$ efficiently by the matintaining its
{\em rightmost path}.  The Cartesian tree for a sequence of data can
be constructed in linear time using a {\em stack} as follows.  The
stack maintains the Catalan indexes and sizes of every left subtree
along the right most path.  That is, we will {\em not} build these
left subtrees, but only keep their Catalan indexes and sizes.

\begin{figure}[!thb]
  \centering
  \includegraphics[width=0.6\linewidth]{\GraphicPath/fig-cartesian-encoding-static.pdf}
  \caption{Compute Catalan index for a tree.}
  \label{fig:fig-cartesian-encoding-static}
\end{figure}

The pseudo code of this Catalan index computation is in
Algorithm~\ref{alg:cartesian-encode-offline}.  This algorithm computes
the Catalan index for a given block of data.  Note that each node of
the stack $D$ has three members -- $v$ as the data, $s$ as the size of
its subtree, and $t$ as the index of its left subtree.  We also use a
pointer $p$ to point to the top of the stack.  In the first double
loop the outer loop goes through every input and the inner loop
inserts a data at the {\em end} of the right most path, which is at
the top of the stack $D[p]$, and traverse towards the root by popping
any {\em smaller} data out of the stack $D$.  When we rotate nodes
along the rightmost path to update the Cartesian tree, we compute the
new {\em index} $t$ and size $s$ of the new left subtree whenever the
newly inserted data replaces it.  As a result the new Catalan index
$t$ can be recomputed with Equation~\ref{fun:tid} by the indexes and
sizes of the left and right subtrees in the stack.  Please refer to
the first while loop of Algorithm~\ref{alg:cartesian-encode-offline}
and Figure~\ref{fig:fig-cartesian-encoding-static} for an
illustration.  After popping all smaller data in the stack the while
loop stops and the size, index, and input are pushed into the new top
of stack.  Finally we pop all data out of the stack and compute the
Catalan index for the entire block.

Algorithm~\ref{alg:cartesian-encode-offline} can compute any Catalan
index for Cartesian trees with the entire block of data and the block
size.  The algorithm runs in $O(s)$ time since an elelment is
pushed/popped at most {\em once}.

\begin{algorithm}
\SetAlgoNoLine
\KwIn{
  $A[1 \cdots s]$: storage array\;
  $s$: the number of elements\;
}
\KwOut{
  $\mathit{tid}$: this label
}
$\langle\mathit{lsz},\mathit{lid},\mathit{value}\rangle$ $D$[$s+1$] \;
$\textit{Dp} \gets 0$ \;
$D[0] \gets \langle 0,0,\infty \rangle$ \;
\For{$i \gets 1$ to $s$} {
  $v \gets A[i], \; \textit{lsz} \gets 0, \; \textit{lid} \gets 0$ \;
  \While{$D[\textit{Dp}].\textit{value} < v$} {
    $\textit{lid} \gets \textit{tid}(D[\textit{Dp}].\textit{lsz}, D[\textit{Dp}].\textit{lid}, \textit{lsz}, \textit{lid})$ \;
    $\textit{lsz} \gets \textit{lsz} + D[\textit{Dp}].\textit{lsz} + 1$ \;
    $\textit{Dp} \gets \textit{Dp} - 1$ \;
  }
  $\textit{Dp} \gets \textit{Dp} + 1$ \;
  $D[\textit{Dp}] \gets \langle\mathit{lsz},\mathit{lid},\mathit{v}\rangle$ \;
}

$\textit{lsz} \gets 0, \; \textit{lid} \gets 0$ \;
\While{$\textit{Dp} > 0$} {
  $\textit{lid} \gets \textit{tid}(D[\textit{Dp}].\textit{lsz}, D[\textit{Dp}].\textit{lid}, \textit{lsz}, \textit{lid})$ \;
  $\textit{lsz} \gets \textit{lsz} + D[\textit{Dp}].\textit{lsz} + 1$ \;
  $\textit{Dp} \gets \textit{Dp} - 1$ \;
}
return $\textit{lid}$ \;

\caption{Offline Type of Cartesian Tree}
\label{alg:cartesian-encode-offline}
\end{algorithm}

\subsection{Dynamic Catalan Index Computation}

Several encoding methods were proposed for indexing Cartesian search
trees.  Fischer~\cite{Fischer2006TheoreticalAP} introduced the first
encoding method and Masud~\cite{Hasan2010CacheOA} presents a new
encoding method to reduce the number of instructions.  Unfortunately
all these algorithm work in an off-line model, i.e., they assume all
data are given in advance, and cannot cope with incrementally added
data.  In addition, they require a preprocessing of time $O(n)$, where
$n$ is the number of data.  The preprocessing need more memory
transfer to find the information of the block of an input array, or
read external files from disk.

% Morris: Cartesian tree can be used to solve above problem

We can generalize our Catalan index computation technique for
incrementally added data.

It is easy to see that our stack-based
Algorithm~\ref{alg:cartesian-encode-offline} can easily support the
{\tt append} operation on Cartesian tree, as long as we provide a
dynamic encoding method.

That is, we can update all the Catalan index whenever we insert a new
data.  That is, we can {\em dynamically} maintain a lookup table to
obtain the maximum value in a range, so as to answer a ranged query
{\tt query(L, R)}, by our Cartesian tree encoding.


%% Now, we provide the dynamic encoding method so that each operation is
%% amortized $O(1)$ time.

We use five variables to record the state of a Cartesian tree, so as
to support dynamic encoding,

This dynamic encoding method is based on
Algorithm~\ref{alg:parallel-LCA} and Equation~\ref{fun:tid}.

Consider the step to insert the $i$-th element.  Let the index of the
current tree be $t$ and the rightmost path of the Cartesian tree is
presented by two variables -- stack pointer $Dp$ and a stack $D$.  We
first initialize a state set $i$ to be empty, $s$ to $\frac{\log
  n}{4}$, ${t}$ to $C_s - 1$, $Dp$ to 0, and the top of stack $D$ to
infinity.

The structure of state is as follows:

\iffalse
我們定義轉移狀態由 5 個變數來決定動態笛卡爾樹的編碼,當前插入第 $i$ 個
元素,最終填充 $s$ 個元素,當前的樹編號 $\mathit{tid}$,以及笛卡爾樹的
右鏈狀態指針 $Dp$ 與其堆疊 $D$,其結構如下:
\fi

% State(i = 0, s = n, tid = C[n]-1, Dp = 0, D[0].val = INF)

%\begin{minipage}{\linewidth}

\lstinputlisting[frame=single,basicstyle=\tt,caption=State of Cartesian Tree]{\CodePath/cartesian-state.h}

%\end{minipage}

% Morris: Idea for encode Cartesian tree dynamically: virtual node &
% propagation

In order to encode Cartesian tree dynamically, we initialize the $s$
number of virtual node on the rightmost path, and that is why the
default Catalan index ${\mathit t}$ is $C_s - 1$, which $C_s$ is the
$s$-th Catalan number.

Following the elements insertion, we assume the sequence of elements
which is not yet inserted are increasing.  In lexicographical order,
the rightmost path of Cartesian tree is belonged to the lower
dimension in the row-major like.

Simultaneously, building a Cartesian tree only modify the rightmost
path.  We use the propagation to get the Catalan index in the next
insertion.  Finally, we propose the difference
algorithm~\ref{alg:cartesian-encode-online} to satisfy above
requirement.

\iffalse 為了解決在線詢問操作,取 $s = \frac{\log n}{4}$。根據字典順序
的編碼性質,一開始建立虛設點 $s$ 個在右鏈上,其樹編號 $\mathit{tid} =
C_s - 1$ 。隨著插入元素的增加,尚未加入的元素都預設嚴格遞減,加上根據
編碼順序,我們藉由差值來維護在線編碼 (如圖
~\ref{fig:cartesianEncoding})。根據上述的編碼想法,我們得到算法
~\ref{alg:cartesian-encode-online}。\fi

We give an example of difference algorithm in the
figure~\ref{fig:cartesianEncoding}.  Each block has $s$ number of
elements.  We will build a Cartesian tree with $s$ number of nodes to
solve in-block query.  In initialization, it assume $s$ number of
nodes on the rightmost path and the default Catalan index
$\mathit{tid} = C_s - 1$.  When inserting $i$-th element, the Catalan
index is $\mathit{tid}_i$, and the Catalan index of the subtree $A$ is
$A.\mathit{tid}$.  If the value of $(i+1)$-th element is $x$, it will
rotate onto the node $A$.  After rotation, $A$ is a left subtree of
$A$, and we can compute the index of subtree $A$ during rotation.
Then, $x.\mathit{tid}$ can be computed by the $s-(i+1)$ number of
virtual nodes on the rightmost path and $A.\mathit{tid}$.  According
to the lexicographical order, we get $\mathit{tid}_{i+1} =
\mathit{tid}_i + (x.\mathit{tid} - A.\mathit{tid})$.

\begin{algorithm}[!thb]
\SetAlgoNoLine
\KwIn{
  $S$: state of Cartesian Tree;  $v$: the added data;
  $D$: A stack where every element has $s$, $t$, and $v$ \;
}
\KwOut{
  $t$: The Catalan index of the input data block after adding $v$
}

$p \gets S.p$, $s \gets 0$, $t \gets 0$ \;
$s' \gets S.s - S.{i} + 1$ \;
$t' \gets C[s'] - 1$ \;

\While{$D[p].{value} < v$} {
  $t \gets t(D[p].s, D[p].t, s, t)$ \;
  $t' \gets t(D[p].s, D[p].t, s', t')$ \;
  $s \gets s + D[p].s+1$ \;
  $s' \gets s' + D[p].s+1$ \;
  $p \gets p - 1$ \;
}
$p \gets p + 1$ \;
$D[p] \gets \left \langle s, t, {v} \right \rangle$ \;
$S.p \gets p$ \;
$x.t \gets t(s, t, S.s-S.i, C[S.s-S.i]-1)$ \;
$S.t \gets S.t + t' - x.t$ \;
$S.i \gets S.i + 1$ \;
return $S.t$ \;

  \caption{Online Type of Cartesian Tree}
  \label{alg:cartesian-encode-online}
\end{algorithm}


\begin{figure*}[!thb]
  \centering
  \includegraphics[width=\linewidth]{\GraphicPath/fig-cartesian-encoding.pdf}

  \caption{An example for difference algorithm to encode Cartesian tree.}

  \label{fig:cartesianEncoding}
\end{figure*}

Finally, we do not increase the time complexity of the building
Cartesian tree algorithm because each operation is $O(1)$.  For the in-
block query, we get the index of the Cartesian tree in amortized $O(1)$.

\iffalse
最後,我們不改變原本的建立笛卡爾樹算法,便能在過程中擭得樹的編號,
每一次的 in-block 詢問只需要一次記憶體存取,得到任一操作攤銷複雜度 $\theta(1)$。
\fi
