\section{平行 VGLCS 算法} \label{sec:parallelVGLCS}

\subsection{基礎動態規劃法}

首先,我們使用動態規劃法解決 VGLCS~\cite{Peng2011TheLC} 問題。
令 $A$ 和 $B$ 為兩個輸入長度分別為 $n$ 和 $m$ 的字串,
相應的 $G_A$ 和 $G_B$ 陣列為間隔約束,
定義 $V[i][j]$ 為子字串 $A[1, i]$ 和 $B[1, j]$ 的最長 VGLCS 的長度,
從上述的定義中,我們可以推出 $V[i][j]$ 來自於 $V[k][l]$,
其中 $k$ 介於 $i-G_A(i)-1$ 到 $i-1$、$l$ 介於 $j-G_B(j)-1$ 到 $j-1$,
意即陣列 $V$ 往左上延伸的一個矩形內最大值,
參照圖~\ref{fig:fig-VGLCS-dp-naive} 的說明。

\begin{figure}[!thb]
  \centering \subfigure[如何計算 $V$] {
    \includegraphics[width=0.45\linewidth]{\GraphicPath/fig-VGLCS-dp-naive.pdf}
    % \caption{How to compute $V$.}
    \label{fig:fig-VGLCS-dp-naive}
  } \subfigure[使用增長後綴最大值詢問計算 $V$] {
    \includegraphics[width=0.45\linewidth]{\GraphicPath/fig-VGLCS-dp.pdf}
    \label{fig:fig-VGLCS-dp}
  }
  \caption{VGLCS 的動態規劃方法}
  \label{fig:basic-dp-VGLCS}
\end{figure}

計算動態規劃中的 $V$ 的優化方法如下所述。
當計算相同的行 $i$ 上的 $V[i][j]$ 時,所有的矩形區域皆包含相同的約束 $G_A(i)$,
故將計算步驟拆分成兩個階段,第一階段計算每一列的後綴最大值放入陣列 $R$ 中,
接著在 $R$ 中找到後綴長度為 $G_B$ 的最大值恰好為 $V[i][j]$ 的結果,
請參照圖~\ref{fig:fig-VGLCS-dp} 的說明。

上述的優化需要「增長」後綴最大值詢問,如在第一階段中,每一個列的最大值,
當我們完成第 $i$ 行上的結果轉移到下一行 $V[i+1][*]$ 時,將會詢問每一列上,
後綴長度 $G_A(i+1)$ 的元素最大值,故每一列將會各自維護資料結構。同樣地,
在同一行的計算中,從 $V[i][j]$ 移動到 $V[i][j+1]$ 時,將從陣列 $R$ 中提取後綴最大值,
我們將這一系列的操作,稱呼為「增長後綴最大值詢問」(incremental suffix maximum query),
意即我們維護一個資料結構,允許插入一個新的元素至尾端,同時詢問任意位置到尾端的最大值。


\subsection{Peng 算法}

Peng~\cite{Peng2011TheLC} 提供 VGLCS 的循序算法~\ref{alg:serial-VGLCS},其
使用先前提到的優化方式完成。在最外層迴圈中,依序完成每一行,
而最裡層迴圈依序從左往右。我們假設資料結構 $C$ 可以回答任意列的後綴最大值,
我們定義 $C[j]$ 可以回答在 $V$ 的第 $j$ 列的後綴最大值。
從先前的觀察中,在同一行 $i$ 上有相同的約束長度 $G_A(i)$,
故我們將從 $C$ 中詢問從第 $i-1$ 行往前 $G_A(i)+1$ 個元素的後綴最大值,
接著,我們將從 $R$ 中提取後綴長度為 $G_B(j)+1$ 的最大值放入 $V[i]j[j]$,
請參照圖~\ref{fig:fig-VGLCS-dp}。

更新 $V$, $C$, $R$ 的方法如下述。當字串 $A$ 的第 $i$ 個字元匹配字串 $B$ 的第 $j$ 個字元,
我們將設 $V[i][j]$ 為左上矩形內部最大元素值加一,如圖~\ref{fig:fig-VGLCS-dp} 所示。
這一矩形內最大值可以從 $R$ 中,透過後綴 $G_B(j)$ 個元素找到其值。
而 $R$ 處理從前一行的結果,並且包含 $j-1$ 的元素,
我們會在這 $R$ 上詢問後綴長度為 $G_B(j)+1$ 的最大值結果,
並成為 $V[i][j]$ 同時加入到 $C[j]$ 中。相反地,如果字串位置沒有匹配,
直接將 $V[i][j]$ 設為 0,並更新相應的 $R$ 和 $C[j]$。
 
\begin{algorithm}[!thb]
\SetAlgoNoLine
\KwIn{$A, B$: the input string; $G_A, G_B$: the array of variable gapped constraints;}
\KwOut{Find the LCS with variable gap constraints.}

Create an array of $m$ data structures $C[m]$ that support ISMQ\;
Create an empty table $V[n][m]$\;

\For{$i \gets 1$ to $n$} {
  Create a data structure $R$ that supports ISMQ\;
  \For{$j \gets 1$ to $m$} {
    \uIf{$A[i] = B[j]$} {
        $t \gets$ Query $R$ for the maximum among the last $G_B(j)+1$ elements \;
        $V[i][j] \gets t + 1$\; 
        $t \gets$ Query $C[j]$ for the maximum among the last $G_A(i)+1$ elements \;
        Append $t$ to $R$ \;
    }
    \Else {
        $V[i][j] \gets 0$ \;
        $t \gets$ Query $C[j]$ for the maximum among the last $G_A(i)+1$ elements \;
        Append $t$ to $R$ \;
    }
    Append $V[i][j]$ into $C[j]$ \;
  }
}
Retrieve the VGLCS by tracing $V[n][m]$\;

  \caption{Peng's algorithm for finding VGLCS~\cite{Peng2011TheLC}}
  \label{alg:serial-VGLCS}
\end{algorithm}


\subsection{增長後綴最大值詢問}

從之前所討論的 Peng 算法,發現解決 VGLCS 問題需著重於「增長後綴最大值詢問」,
為了解決增長後綴最大值詢問,資料結構需要要三種類型的操作:
第一類操作 {\sc Make},建立空陣列 $A$;第二類操作 {\sc Append}$(V)$,
附加一個元素 $V$ 至陣列 $A$ 尾端:第三類操作 {\sc Query}$(x)$,
找到位置 $x$ 至陣列 $A$ 尾端的最大值。

Peng 算法中使用並查集解決所有的後綴最大值詢問,而並查集最早由 
Gabow~\cite{Gabow1983ALA} 和 Tarjan~\cite{Tarjan1975EfficiencyOA} 
提出解決聯集與查找問題 ({\em union-and-find problem})。
在這裡我們視並查集中的集合由左往右排成一列,每一集合的根節點表示該集合的最大值,
這些最大值排成一列呈現遞減順序。當我們加入一個元素 $x$ 到陣列尾端,
其 $x$ 視為一個集合,將會往左合併比它小的集合。很輕易地明白,
{\sc Query}$(x)$ 只需要查找 $x$ 所在的集合,透過並查集的 {\em find} 找到最大值。
任何合併和查找操作,我們得知攤銷複雜度為 $O(\alpha(n))$。

\subsection{使用稀疏表平行 VGLCS 算法}

Peng~\cite{Peng2011TheLC} 所提到的循序 VGLCS 算法~\ref{alg:serial-VGLCS} 
和其他 LCS 變形問題都難以使用一行接一行的平行方法,使用動態規劃的算法都依賴狀態的轉移,
這使得建構方法有大量的資料資賴性 ({\em heavy data dependency}),
這是難以使用一行接一行的平行方法的問題所在,因為每一個元素需要在同一行的元素結果。

Peng 算法需要使用波前平行方法,這使得需要「額外的」的空間維護每一行列的狀態,
這些反而在循序算法中視可以省略的的部分,在平行處理再次被需要而增加實作的硬體需求。
回到 VGLCS 需要找到 $V$ 矩行內部的最大值,我們每一次平行處理同一個對角線的資訊,
此時必須額外抄寫不同列的結果,因為每一列的約束限制都各有不同,
參照圖~\ref{fig:fig-VGLCS-dp-wavefront} 粗體說明表示平行處理的元素部分。
每一個操作都會保留相應的資料結構,額外的記憶體保存必然會帶來一些處理花費。

\begin{figure}[!thb]
  \centering \subfigure[第一階段]{
    \includegraphics[width=0.45\linewidth]{\GraphicPath/fig-VGLCS-dp-wavefront-second.pdf}
  } \subfigure[第二階段]{
    \includegraphics[width=0.45\linewidth]{\GraphicPath/fig-VGLCS-dp-wavefront-first.pdf}
  }
  \caption{波前平行方法額外抄寫的紀錄資料}
  \label{fig:fig-VGLCS-dp-wavefront}
\end{figure}

我們提出的列接列方法 (row-by-row approach) 將只需要一個資料結構維護行上的資訊,
這個資料結構收集每一列的後綴最大值,並且移除掉先前提及的繁重的資料依賴性,
不再依賴同一行上的處理資訊。

相較於波前平行方法,我們使用的列接列方法使用較少的空間、較好的工作負載平衡、
較低的執行緒同步開銷。同時,在波前平行方法的關鍵路徑 (critical path) 長度為
列接列方法的數倍。此外,我們移除掉同一行上的資料依賴性,將能「完全」和「均勻」地平行。
其結果造成較平衡的工作負載分佈和較容易的執行緒同步。

A sketch of our algorithm is as follows.  Our algorithm computes $V$ one
row at a time.  The computation of each row has two stages.  In the
first stage, the algorithm queries each data structure $C$ for every
column within the rectangle {\em in parallel}, so as to obtain the
maximums of suffix of length $G_A(i) + 1$ of every column, and place
them into an array $R$. Recall from Algorithm~\ref{alg:serial-VGLCS}
that every column of $V$ has a data structure $C$ that supports
incremental suffix maximum on $V$.  Please refer to
Figure~\ref{fig:fig-VGLCS-dp-rmq} for an illustration.

In the second stage, our algorithm issues {\em range maximum queries},
one for each column, on $R$ to compute all $m$ elements of the $i$-th
row of $V$ {\em in parallel}.  Note that unlike the sequential
algorithm, we compute all elements in the $i$-th row of $V$ in parallel,
so we cannot query the {\em suffix} of $R$.  Instead, we need to query a
{\em range} of $R$ for the maximum, where the range is the gap
constraint on that column.  Please refer to
Figure~\ref{fig:fig-VGLCS-dp-rmq} for an illustration.  Note that we
need to add the newly computed $V[i][j]$ into the $C$ of the $j$-th
column {\em incrementally}, so that they will contain the correct
information for the computation of the $(i+1)$-th row.  Also, since the
algorithm iterates in rows, these $C$'s only need to support suffix
maximum query.  No range query on them is required.  In contrast, we do
need to support range maximum query on $R$, and these queries will be in
parallel.

\begin{figure}[!thb]
  \centering \subfigure[The first stage]{
    \includegraphics[width=0.45\linewidth]{\GraphicPath/fig-VGLCS-dp-rmq-first.pdf}
  } \subfigure[The second stage]{
    \includegraphics[width=0.45\linewidth]{\GraphicPath/fig-VGLCS-dp-rmq-second.pdf}
  }
  \caption{Two stages of the computation of one row of $V$.}
  \label{fig:fig-VGLCS-dp-rmq}
\end{figure}

To resolve the data dependency, we need to consider a good data
structure that can handle incremental suffix/range maximum query {\em in
parallel}.  We note that it is {\em not} feasible to parallelize the
disjoint set implementation for three reasons.  First, a query for
disjoint set will change the data structure because a lookup will {\em
compress} the path to the root.  It is difficult to maintain a
consistent view of the data structure when multiple threads are
compressing the path {\em simultaneously}.  Second, when multiple
threads are compressing different paths, the load among them could be
very different, and this will incur load imbalance.  Third, there will
be a large number of threads working on different parts of the disjoint
set, therefore it will be difficult to synchronize them efficiently.

\subsubsection{Sparse Table} \label{sec:sparse-table}

Since the disjoint set cannot be implemented efficiently in parallel,
we use {\em sparse table}~\cite{Berkman1993RecursiveSP} to support
incremental suffix/range maximum queries in our VGLCS algorithm.
Sparse table~\cite{Berkman1993RecursiveSP} requires a $O(n \log n)$
preprocessing, and can support range maximum query in $O(1)$ time on
one dimensional data.  A sparse table is a two dimensional array.  The
element of a sparse table in the $j$-th row and $i$-th column is the
maximum among the $i$-th elements and its $2^j - 1$ predecessors in
the input array.

We give an example of the sparse table
(Figure~\ref{fig:interval-decomposition}).  The input is in array $A$.
We split array $A$ into five blocks so that each block has four
elements.  Then we build a sparse table $T$ on $A$ as described
earlier.  Now a ranged maximum query on $A$ can be answered by at most
{\em two} queries into the sparse table.  For example, if the query is
from 2 to 13, then the answer is the maximum of from 2 to 9
($T[3][9]$), and from 6 to 13 ($T[3][13]$).  Both are from the third
level of the table since each has the maximum of $2^3 = 8$ elements in
the input.

\begin{figure}[!thb]
  \centering \subfigure[Array]{
    \includegraphics[width=0.45\linewidth]{\GraphicPath/fig-interval-decomposition-origin.pdf}
    \label{fig:fig-interval-decomposition}
  } \subfigure[Sparse table]{
    \includegraphics[width=0.45\linewidth]{\GraphicPath/fig-sparse-table-origin.pdf}
    \label{fig:fig-sparse-table}
  }
  \caption{A sparse table example}
  \label{fig:interval-decomposition}
\end{figure}

It is easy to see that one can build a sparse table in parallel
efficiently.  Please refer to
Algorithm~\ref{alg:parallel-sparse-table} for details.
Algorithm~\ref{alg:parallel-sparse-table} builds sparse table in
parallel and in $O(n \log n / p + \log n)$ time, where $n$ is the
number of elements and $p$ is the number of processors.  This
algorithm is very easy to parallelize and implement.

\begin{algorithm}[H]
\SetAlgoNoLine
\LinesNumbered
\KwIn{$A[0 .. N-1]$: the input array}
\KwOut{$T$: a sparse table for the input array $A$}

Create a two-dimensional array $T[\log N][N]$ \;
Copy $A$ to $T[0]$ \;
\For{$j \gets 0$ to $\log N$} {
  \ForPar{$i \gets 2^j$ to $N$} {
    $T[j][i] \gets \max(T[j-1][i-2^{j}], T[j-1][i])$ \;
  }
}
  \caption{A Parallel Sparse Table Building Algorithm}
  \label{alg:parallel-sparse-table}
\end{algorithm}


\subsubsection{A Parallel VGLCS with Sparse Table}

The operations on a sparse table are much easily to parallelize than
those on a disjoint set, which is used within the inner loop of Peng's
sequential VGLCS algorithm.  The inner loop of Peng's algorithm
alternates between append and query operations on $R$.  Please refer
to Algorithm~\ref{alg:serial-VGLCS} for details.  This alternation
between appending and querying incurs heavy data dependency.  In
addition, the parallelism of operations on a disjoint tree is limited
by the length of path under compression.  The length is usually very
short and provides very limited parallelism.

The pseudo code of our parallel VGLCS algorithm with sparse table is
given in Algorithm~\ref{alg:parallel-VGLCS}.  The algorithm computes
$V$ one row at a time.  The computation of each row has two stages.
In the first stage, the algorithm queries the $C$'s {\em in parallel}
to obtain the maximums of suffix of length $G_A+1$ and place them
into an array $R$.  Then we build a sparse table $T$ with the data of
$R$.  Then in the second stage, the algorithm queries $T$ to find the
range maximum in $R$ to compute all elements in the $i$-th row of $V$
{\em in parallel}.

\begin{algorithm}[!thb]
\SetAlgoNoLine
\KwIn{$A, B$: the input string; $G_A, G_B$: the array of variable gapped constraints;}
\KwOut{Find the LCS with variable gapped constraints}
    
Create an array of $m$ data structure $C[m]$ to support ISMQ \;
Create an empty table $V[n][m]$ \;
\For{$i \gets 1$ to $n$} {
  \ForPar{$j \gets 1$ to $m$} {
    $M[j] \gets$ suffix maximum of length $G_A(i) + 1$ from $C[j]$ \;
  }
  Build a sparse table $T$ with data of $M$ in parallel with Algorithm~\ref{alg:parallel-sparse-table}\; 
  \ForPar{$j \gets 1$ to $m$} {
    \If{$A[i] = B[j]$} {
        $t \gets $ the range maximum among the $G_B(j) + 1$ elements before $M[j]$ by querying $T$ \;
        $V[i][j] \gets t + 1$ \;
        Append $V[i][j]$ to $C[j]$ \;
    }
  }
}
Retrieve the VGLCS by tracing $V[n][m]$ \;
  \caption{Parallel Algorithm for Finding VGLCS}
  \label{alg:parallel-VGLCS}
\end{algorithm}


The implementation of the two stages have different challenges.  The
first stage is easier to parallelize because the operations on
individual columns are {\em independent}.  However, it will insert new
data into $C$, and still needs to answer suffix queries efficiently in
order to build the $R$ array.  The second stage does {\em not}
requires insertion so it is more static.  However, since we compute
all $V$'s in the same row in parallel, it requires {\em ranged
  queries}, instead of suffix queries, on the sparse table $T$.  The
next two sections will describe our approaches to address these
challenges of two stages.  For ease of presentation we will describe
our approach for the second stage in Section~\ref{sec:parallelRMQ}
first.  Then we address the first stage in Section~\ref{sec:QIUD}.
