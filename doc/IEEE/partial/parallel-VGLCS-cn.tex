\section{平行 VGLCS 算法} \label{sec:parallelVGLCS}

\subsection{基礎動態規劃法}

首先,我們使用動態規劃法解決 VGLCS~\cite{Peng2011TheLC} 問題。
令 $A$ 和 $B$ 為兩個輸入長度分別為 $n$ 和 $m$ 的字串,
相應的 $G_A$ 和 $G_B$ 陣列為間隔約束,
定義 $V[i][j]$ 為子字串 $A[1, i]$ 和 $B[1, j]$ 的最長 VGLCS 的長度,
從上述的定義中,我們可以推出 $V[i][j]$ 來自於 $V[k][l]$,
其中 $k$ 介於 $i-G_A(i)-1$ 到 $i-1$、$l$ 介於 $j-G_B(j)-1$ 到 $j-1$,
意即陣列 $V$ 往左上延伸的一個矩形內最大值,
參照圖~\ref{fig:fig-VGLCS-dp-naive} 的說明。

\begin{figure}[!thb]
  \centering \subfigure[如何計算 $V$] {
    \includegraphics[width=0.45\linewidth]{\GraphicPath/fig-VGLCS-dp-naive.pdf}
    % \caption{How to compute $V$.}
    \label{fig:fig-VGLCS-dp-naive}
  } \subfigure[使用增長後綴最大值詢問計算 $V$] {
    \includegraphics[width=0.45\linewidth]{\GraphicPath/fig-VGLCS-dp.pdf}
    \label{fig:fig-VGLCS-dp}
  }
  \caption{VGLCS 的動態規劃方法}
  \label{fig:basic-dp-VGLCS}
\end{figure}

計算動態規劃中的 $V$ 的優化方法如下所述。
當計算相同的行 $i$ 上的 $V[i][j]$ 時,所有的矩形區域皆包含相同的約束 $G_A(i)$,
故將計算步驟拆分成兩個階段,第一階段計算每一列的後綴最大值放入陣列 $R$ 中,
接著在 $R$ 中找到後綴長度為 $G_B$ 的最大值恰好為 $V[i][j]$ 的結果,
請參照圖~\ref{fig:fig-VGLCS-dp} 的說明。

上述的優化需要「增長」後綴最大值詢問,如在第一階段中,每一個列的最大值,
當我們完成第 $i$ 行上的結果轉移到下一行 $V[i+1][*]$ 時,
將會詢問每一列上後綴長度 $G_A(i+1)$ 的元素最大值,故每一列將會各自維護資料結構。同樣地,
在同一行的計算中,從 $V[i][j]$ 移動到 $V[i][j+1]$ 時,將從陣列 $R$ 中提取後綴最大值,
我們將這一系列的操作,稱呼為「增長後綴最大值詢問」(incremental suffix maximum query),
意即我們維護一個資料結構,允許插入一個新的元素至尾端,同時詢問任意位置到尾端的最大值。


\subsection{Peng 算法}

Peng~\cite{Peng2011TheLC} 提供 VGLCS 的循序算法~\ref{alg:serial-VGLCS},其
使用先前提到的優化方式完成。在最外層迴圈中,依序完成每一行,
而最裡層迴圈依序從左往右。我們假設資料結構 $C$ 可以回答任意列的後綴最大值,
我們定義 $C[j]$ 可以回答在 $V$ 的第 $j$ 列的後綴最大值。
從先前的觀察中,在同一行 $i$ 上有相同的約束長度 $G_A(i)$,
故我們將從 $C$ 中詢問從第 $i-1$ 行往前 $G_A(i)+1$ 個元素的後綴最大值,
接著,我們將從 $R$ 中提取後綴長度為 $G_B(j)+1$ 的最大值放入 $V[i]j[j]$,
請參照圖~\ref{fig:fig-VGLCS-dp}。

更新 $V$, $C$, $R$ 的方法如下述。當字串 $A$ 的第 $i$ 個字元匹配字串 $B$ 的第 $j$ 個字元,
我們將設 $V[i][j]$ 為左上矩形內部最大元素值加一,如圖~\ref{fig:fig-VGLCS-dp} 所示。
這一矩形內最大值可以從 $R$ 中,透過後綴 $G_B(j)$ 個元素找到其值。
而 $R$ 處理從前一行的結果,並且包含 $j-1$ 的元素,
我們會在這 $R$ 上詢問後綴長度為 $G_B(j)+1$ 的最大值結果,
並成為 $V[i][j]$ 同時加入到 $C[j]$ 中。相反地,如果字串位置沒有匹配,
直接將 $V[i][j]$ 設為 0,並更新相應的 $R$ 和 $C[j]$。
 
\begin{algorithm}[!thb]
\SetAlgoNoLine
\KwIn{$A, B$: the input string; $G_A, G_B$: the array of variable gapped constraints;}
\KwOut{Find the LCS with variable gap constraints.}

Create an array of $m$ data structures $C[m]$ that support ISMQ\;
Create an empty table $V[n][m]$\;

\For{$i \gets 1$ to $n$} {
  Create a data structure $R$ that supports ISMQ\;
  \For{$j \gets 1$ to $m$} {
    \uIf{$A[i] = B[j]$} {
        $t \gets$ Query $R$ for the maximum among the last $G_B(j)+1$ elements \;
        $V[i][j] \gets t + 1$\; 
        $t \gets$ Query $C[j]$ for the maximum among the last $G_A(i)+1$ elements \;
        Append $t$ to $R$ \;
    }
    \Else {
        $V[i][j] \gets 0$ \;
        $t \gets$ Query $C[j]$ for the maximum among the last $G_A(i)+1$ elements \;
        Append $t$ to $R$ \;
    }
    Append $V[i][j]$ into $C[j]$ \;
  }
}
Retrieve the VGLCS by tracing $V[n][m]$\;

  \caption{Peng's algorithm for finding VGLCS~\cite{Peng2011TheLC}}
  \label{alg:serial-VGLCS}
\end{algorithm}


\subsection{增長後綴最大值詢問}

從之前所討論的 Peng 算法,發現解決 VGLCS 問題需著重於「增長後綴最大值詢問」,
為了解決增長後綴最大值詢問,資料結構需要要三種類型的操作:
第一類操作 {\sc Make},建立空陣列 $A$;第二類操作 {\sc Append}$(V)$,
附加一個元素 $V$ 至陣列 $A$ 尾端:第三類操作 {\sc Query}$(x)$,
找到位置 $x$ 至陣列 $A$ 尾端的最大值。

Peng 算法中使用並查集解決所有的後綴最大值詢問,而並查集最早由 
Gabow~\cite{Gabow1983ALA} 和 Tarjan~\cite{Tarjan1975EfficiencyOA} 
提出解決聯集與查找問題 ({\em union-and-find problem})。
在這裡我們視並查集中的集合由左往右排成一列,每一集合的根節點表示該集合的最大值,
這些最大值排成一列呈現遞減順序。當我們加入一個元素 $x$ 到陣列尾端,
其 $x$ 視為一個集合,將會往左合併比它小的集合。很輕易地明白,
{\sc Query}$(x)$ 只需要查找 $x$ 所在的集合,透過並查集的 {\em find} 找到最大值。
任何合併和查找操作,我們得知攤銷複雜度為 $O(\alpha(n))$。

\subsection{使用稀疏表平行 VGLCS 算法}

Peng~\cite{Peng2011TheLC} 所提到的循序 VGLCS 算法~\ref{alg:serial-VGLCS} 
和其他 LCS 變形問題都難以使用一行接一行的平行方法,使用動態規劃的算法都依賴狀態的轉移,
這使得建構方法有大量的資料資賴性 ({\em heavy data dependency}),
這是難以使用一行接一行的平行方法的問題所在,因為每一個元素需要在同一行的元素結果。

Peng 算法需要使用波前平行方法,這使得需要「額外的」的空間維護每一行列的狀態,
這些反而在循序算法中視可以省略的的部分,在平行處理再次被需要而增加實作的硬體需求。
回到 VGLCS 需要找到 $V$ 矩行內部的最大值,我們每一次平行處理同一個對角線的資訊,
此時必須額外抄寫不同列的結果,因為每一列的約束限制都各有不同,
參照圖~\ref{fig:fig-VGLCS-dp-wavefront} 粗體說明表示平行處理的元素部分。
每一個操作都會保留相應的資料結構,額外的記憶體保存必然會帶來一些處理花費。

\begin{figure}[!thb]
  \centering \subfigure[第一階段]{
    \includegraphics[width=0.45\linewidth]{\GraphicPath/fig-VGLCS-dp-wavefront-second.pdf}
  } \subfigure[第二階段]{
    \includegraphics[width=0.45\linewidth]{\GraphicPath/fig-VGLCS-dp-wavefront-first.pdf}
  }
  \caption{波前平行方法額外抄寫的紀錄資料}
  \label{fig:fig-VGLCS-dp-wavefront}
\end{figure}

我們提出的列接列方法 (row-by-row approach) 將只需要一個資料結構維護行上的資訊,
這個資料結構收集每一列的後綴最大值,並且移除掉先前提及的繁重的資料依賴性,
不再依賴同一行上的處理資訊。

相較於波前平行方法,我們使用的列接列方法使用較少的空間、較好的工作負載平衡、
較低的執行緒同步開銷。同時,在波前平行方法的關鍵路徑 (critical path) 長度為
列接列方法的數倍。此外,我們移除掉同一行上的資料依賴性,將能「完全」和「均勻」地平行。
其結果造成較平衡的工作負載分佈和較容易的執行緒同步。

我們提出的算法框架如下所述:每一次計算 $V$ 的同一列,計算同一列時分兩個階段,
第一階段從資料結構 $C$ 中平行詢問每一列的後綴長 $G_A(i)+1$ 最大值,並且儲存到陣列 $R$ 中,
接著繼續使用算法~\ref{alg:serial-VGLCS},將 $C$ 持續維護 $V$ 上的增長後綴最大值詢問。
請參考圖~\ref{fig:fig-VGLCS-dp-rmq} 的說明。

第二階段中,我們算法著重於「區間最大值詢問」(range maximum queries),陣列 $R$ 上紀錄 $m$ 個列的後綴最大值,
需要平行處理在第 $i$ 行上的 $m$ 的元素值。相較於循序算法,我們並不使用陣列 $R$ 的後綴詢問,
取而代之,我們需要詢問 $R$ 上的區間詢問,每一個詢問區間長度為每個列位置的間隔約束,
請參照圖~\ref{fig:fig-VGLCS-dp-rmq} 的說明。
當我們計算出 $V[i][j]$ 時,仍然需要將其加入 $C$ 的第 $j$ 列進行增長操作,以應付第 $i+1$ 行的計算結果。
然而,這 $C$ 只需要支持後綴詢問,並非區間詢問。相對地,我們對 $R$ 維護區間最大值詢問操作,
這些操作皆可以平行地去處理。

\begin{figure}[!thb]
  \centering \subfigure[第一階段]{
    \includegraphics[width=0.45\linewidth]{\GraphicPath/fig-VGLCS-dp-rmq-first.pdf}
  } \subfigure[第二階段]{
    \includegraphics[width=0.45\linewidth]{\GraphicPath/fig-VGLCS-dp-rmq-second.pdf}
  }
  \caption{計算 $V$ 的同一行,將分成兩個階段}
  \label{fig:fig-VGLCS-dp-rmq}
\end{figure}

為了解決資料依賴性,我們需要好的資料結構去解決「平行環境」的增長後綴/區間最大值詢問。
首先,我們發現並查集這一類的資料結構不易於平行,原因在於以下三點:
第一點,每一次詢問將會改變資料結構,因為通常都伴隨著「壓縮路徑」(compress path) 的操作,
同時壓縮路徑造成難以在執行緒之間得到一致的結構;第二點,由於執行緒各自壓縮不同路徑,
由於壓縮路徑長短不同,導致工作負載不平行;第三點,當有許多執行緒同時運行,
多次的同步操作將無法有效率地運行。

\subsubsection{稀疏表} \label{sec:sparse-table}

由於先前提到並查集在平行環境下的效能問題,我們使用「稀疏表」
({\em sparse table}~\cite{Berkman1993RecursiveSP}) 來支援 VGLCS 算法中的增長後綴/區間詢問。
稀疏表最原始的版本需要前處理,其花費 $O(n \log n)$ 時間,隨後對於任意一維的區間詢問只需要 $O(1)$ 時間。
稀疏表的儲存架構採用二維陣列,對於第 $j$ 行上的第 $i$ 列元素,為原輸入陣列的第 $i$ 的元素往前 $2^j-1$ 個元素的最大值。

我們以圖~\ref{fig:interval-decomposition} 簡單說明稀疏表如何運作。
對於原輸入陣列 $A$,接著根據先前提到的算法建造稀疏表 $T$。
當給定一個陣列 $A$ 上的區間詢問時,最多拆成 2 個稀疏表上的查找。
例如,詢問區間從位置 2 到 13,我們將拆成詢問 2 到 9 ($T[3][9]$)以及 6 到 13 ($T[3][13]$),
這兩者都是從第三層的稀疏表中提取連續 $2^3 = 8$ 個元素的最大值。


\begin{figure}[!thb]
  \centering \subfigure[原輸入陣列]{
    \includegraphics[width=0.45\linewidth]{\GraphicPath/fig-interval-decomposition-origin.pdf}
    \label{fig:fig-interval-decomposition}
  } \subfigure[稀疏表]{
    \includegraphics[width=0.45\linewidth]{\GraphicPath/fig-sparse-table-origin.pdf}
    \label{fig:fig-sparse-table}
  }
  \caption{稀疏表例子}
  \label{fig:interval-decomposition}
\end{figure}

平行建造稀疏表是相當容易的,請參照算法~\ref{alg:parallel-sparse-table} 的說明,
算法~\ref{alg:parallel-sparse-table} 建造稀疏表時間為 $O(n \log n / p + \log n)$,
其中 $n$ 為元素個數,而 $p$ 為處理器個數,這個算法相當容易實作。


\begin{algorithm}[H]
\SetAlgoNoLine
\LinesNumbered
\KwIn{$A[0 .. N-1]$: the input array}
\KwOut{$T$: a sparse table for the input array $A$}

Create a two-dimensional array $T[\log N][N]$ \;
Copy $A$ to $T[0]$ \;
\For{$j \gets 0$ to $\log N$} {
  \ForPar{$i \gets 2^j$ to $N$} {
    $T[j][i] \gets \max(T[j-1][i-2^{j}], T[j-1][i])$ \;
  }
}
  \caption{A Parallel Sparse Table Building Algorithm}
  \label{alg:parallel-sparse-table}
\end{algorithm}


\subsubsection{基於稀疏表之平行 VGLCS}

相較於並查集,稀疏表上的任何操作都較容易平行。因此,在 Peng 的循序算法中,
內層迴圈需要交替地附加和詢問操作於 $R$ 上,請參考算法~\ref{alg:serial-VGLCS} 提及的細節,
這些交替操作存在嚴重的資料依賴性。此外,平行度受到並查集的路徑長度影響,
而這些路徑長通常很短,故約束平行度的發展。

若我們使用稀疏表於平行 VGLCS 算法中,請參考算法~\ref{alg:parallel-VGLCS} 的說明。
算法中仍一次計算 $V$ 的每一行,計算單一行時分成兩個階段,
第一階段平行詢問 $C$ 上的後綴長 $G_A+1$,並將結果收集到陣列 $R$ 中;
接著根據陣列 $R$ 平行建造稀疏表 $T$。在第二階段使用 $T$ 平行詢問所有區間最大值。


\begin{algorithm}[!thb]
\SetAlgoNoLine
\KwIn{$A, B$: the input string; $G_A, G_B$: the array of variable gapped constraints;}
\KwOut{Find the LCS with variable gapped constraints}
    
Create an array of $m$ data structure $C[m]$ to support ISMQ \;
Create an empty table $V[n][m]$ \;
\For{$i \gets 1$ to $n$} {
  \ForPar{$j \gets 1$ to $m$} {
    $M[j] \gets$ suffix maximum of length $G_A(i) + 1$ from $C[j]$ \;
  }
  Build a sparse table $T$ with data of $M$ in parallel with Algorithm~\ref{alg:parallel-sparse-table}\; 
  \ForPar{$j \gets 1$ to $m$} {
    \If{$A[i] = B[j]$} {
        $t \gets $ the range maximum among the $G_B(j) + 1$ elements before $M[j]$ by querying $T$ \;
        $V[i][j] \gets t + 1$ \;
        Append $V[i][j]$ to $C[j]$ \;
    }
  }
}
Retrieve the VGLCS by tracing $V[n][m]$ \;
  \caption{Parallel Algorithm for Finding VGLCS}
  \label{alg:parallel-VGLCS}
\end{algorithm}


實作兩階段的部分時,有著各自不同的挑戰。第一階中由於列彼此之間獨立詢問,
故平行操作較為容易,然而,它仍需插入元素到 $C$ 的尾端,
同時要有效率地找到後綴最大值放入陣列 $R$。第二階段「不存在」強制的插入操作,
故可以靜態地處理之,因我們平行計算 $V$ 同一行上,需要的是區間詢問,而非後綴詢問於稀疏表 $T$ 上。
在接續的兩個章節中,我們將介紹如何克服這些挑戰。
章節~\ref{sec:parallelRMQ} 優先解決在第二階段的問題,
接著我們在章節~\ref{sec:QIUD} 著重於第一階段的挑戰。