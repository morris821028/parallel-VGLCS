\begin{abstract}

The longest common subsequence problem with variable gapped
constraints (VGLCS) is widely used in genes and molecular biology.  An
$O(nm)$ solution has been proposed in the previous study, by reduction
to the efficient incremental suffix maximum query (ISMQ) problem.
Algorithm for solving ISMQ supports appending a value to array and
querying the suffix maximum value in amortized $O(1)$ time. However,
we try to parallelize origin algorithm by wavefront method, but failed
to achieve better performance.  In this paper, our algorithm and data
structure can achieve a better theoretical time complexity on both
querying and appending.  The VGLCS problem can be solved in $O(nm / p
+ n \log n)$, where $p$ is the number of parallel running processors.
And also, the incremental range maximum query problem can be
solved by sparse table within $O(n)$ -- $O(1)$.

\end{abstract}

%\begin{IEEEkeywords}
%range minimum query,
%incremental range maximum query, incremental suffix maximum query,
%longest common sequence, parallel, cartesian tree
%\end{IEEEkeywords}
