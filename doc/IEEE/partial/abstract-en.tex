\begin{abstract}

The longest common subsequence problem with variable gapped
constraints (VGLCS) is widely used in genes and molecular biology.  An
$O(nm)$ solution has been proposed in the previous study, by reduction
to the efficient incremental suffix maximum query (ISMQ) problem.
Algorithm for solving ISMQ supports appending a value to array and
querying the suffix maximum value in amortized $O(1)$ time. However,
we try to parallelize origin algorithm by wavefront method, but failed
to achieve better performance.  In this paper, our algorithm and data
structure can achieve a better theoretical time complexity on both
querying and appending.  The VGLCS problem can be solved in $O(nm / p
+ n \log n)$, where $p$ is the number of parallel running processors.
And also, the incremental range maximum query problem can be
solved by sparse table within $O(n)$ -- $O(1)$.

\iffalse
可變的間隙限制最長同子序列應用於基因、分子生物學中。
在之前的研究已提出理論在 $O(nm)$ 的算法,
其使用特殊的 incremental tree set union 完成 $O(1)$ 的操作,
否則必須使用傳統并查集操作 $O(\alpha(n))$。
原先的序列算法無法以直觀的方式平行,我們修改了原本算法的資料結構,
可以在 $O(1)$ 時間內支援所有操作,其平行的運行時間為 $O(nm / p + n \log n)$。
同樣地,可解決遞增任意區間最大值詢問 (incremental RMQ) 藉由稀疏表於 $O(n)$ -- $O(1)$。
\fi

\end{abstract}

\begin{IEEEkeywords}
range minimum query,
incremental range maximum query, incremental suffix maximum query,
longest common sequence, parallel, cartesian tree
\end{IEEEkeywords}
