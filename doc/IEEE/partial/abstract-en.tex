\begin{abstract}

The longest common subsequence problem with variable gapped
constraints (VGLCS) is used in genes and molecular biology.  We can
find $O(nm)$ solution in the previous study, which use the efficient
incremental suffix maximum query (ISMQ).  ISMQ supports append a value
to array and get the suffix maximum value in amortized $O(1)$ time.
However, we can parallel origin algorithm by wavefront method, but not
have better performance.  In this paper, our algorithm and data
structure offer efficient operations and better theoretical time
complexity.  The VGLCS problem can be solved in $O(nm / p + n \log
n)$, which $p$ is the number of processors.  Simultaneously, we offer
that the incremental range minimum/maximum query problem is also be
solved in $O(n)$ -- $O(1)$ with sparse table.

\iffalse
可變的間隙限制最長同子序列應用於基因、分子生物學中。
在之前的研究中,已提出理論在 $O(nm)$ 的算法,
其使用特殊的 incremental tree set union 完成 $O(1)$ 的操作,
否則必須使用傳統并查集操作 $O(\alpha(n))$。
原先的序列算法無法以直觀的方式平行,我們修改了原本算法的資料結構,
且可解決遞增任意區間最大值詢問 (incremental RMQ),
可以在 $O(1)$ 時間內支援所有操作,其平行的運行時間為 $O(nm / p + n \log n)$。
在單一處理器上,理論時間複雜度仍保持 $O(nm)$,
其運行效能不輸 $O(nm\alpha(n))$ 實作版本。
\fi

\end{abstract}

\begin{IEEEkeywords}
range minimum query,
incremental range maximum query, incremental suffix maximum query,
longest common sequence, parallel, cartesian tree
\end{IEEEkeywords}
