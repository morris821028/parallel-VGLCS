\section{Experiment}
\label{sec:Experiment}

We conduct experiments on an Intel Xeon E5-2620 2.4 Ghz processor with
384K bytes of level 1 cache, 1536K bytes of level 2 cache, and 15M
bytes of shared level 3 cache.  The Intel CPU supports
hyper-threading, and each processor has six cores.  The opearating
system is Ubuntu 14.04.  We implemented all algorithms in C++ and
OpenMP and compiled them using gcc with {\tt -O2} and {\tt -fopenmp}
flag.

We implement {\em disjoint set} and {\em sparse table} in our
experiments and evaluate their performance.  Recall that there are three
data structures we can use to answer incremental suffix maximum query.
The first one is {\em disjoint set}, and it has an amortized time
$O(\alpha(n))$ for each query.  Its variant, {\em incremental   tree set
union}~\cite{Gabow1983ALA}, has an amortized time $O(1)$ for each query.
The second data structure is {\em sparse table}.  It takes $O(\log n)$
time to append a new value and $O(1)$ time to answer a query.  Although
the amortized complexity of answering a query with a disjoint set is
$O(\alpha(n))$ or $O(1)$, it causes inefficient synchronizations in a
parallel environment, therefore, we chose to implement the last data
structures and conduct experiments to evaluate their performance.

\subsection{Variable Gapped Longest Common Subsequence}

The strings of the experiment are generated randomly with alpha set
$\Sigma={\tt ATCG}$.   There are four kinds of the combinations of data
structure to implement VGLCS problem.  Among one kind of them is
implemented with disjoint set on both the first stage and second stage,
and it is origin serial Peng's algorithm.   The rest of the other kinds
can run in parallel environment.  The parallel-ST-disjoint version uses
sparse table in the second stage and disjoint set in the first stage.
The parallel-COST-disjoint version uses compressed sparse table in the
second stage and disjoint set in the first stage.  The parallel-COST
uses compressed sparse table in both first and second stage.

As a result, we get the better performance with compressed sparse table
compared to theoretical $O(1)$ sparse table.  The 
Figure~\ref{fig:fig-parallel} shows the different runtime in different 
parallelism, and the Figure~\ref{fig:fig-parallel-scala} shows the
scalability of our parallel algorithm.  On our server, the parallel
algorithm can get $8 \times$ faster than the serial algorithm.

\iffalse
我們運行優化策略中的空間壓縮版本,而非理論分析的 $\theta(1)$ 操作,
單次詢問落在 $O(s)$ 中,在實作上由於可以完全壓在暫存器上操作,效能表現較佳。
\fi

\begin{figure}
  \centering
  \subfigure[Runtime]{
    \includegraphics[width=0.45\linewidth]{\GraphicPath/fig-parallel-n.pdf}
    \label{fig:fig-parallel}
  }
  \subfigure[Scalability]{
    \includegraphics[width=0.45\linewidth]{\GraphicPath/fig-parallel-p.pdf}
    \label{fig:fig-parallel-scala}
  }
  \caption{Serial and Parallel Algorithm run on E5-2620}
\end{figure}

%\subsection{理論常數 VGLCS}

%尚未完成

\subsection{Incremental Suffix Maximum Query}

We use the random test cases without any limitation, e.g. the length of
range query has a uniform distribution.  After each append operation,
there is only one query operation in the experiment.  Then, we compare
the performance between the four data structure without parallel
environment as follows:

\iffalse
針對插入和詢問次數相同的 ISMQ 問題,運行以下四種數據結構:
\fi

\begin{itemize}
  \item 

Disjoint set: Average run time $o(\alpha(n))$ which is implemented by
path compression and rank.

  \item 

Sparse table: Insertion in $O(\log n)$ time, and query in $O(1)$ time.
We allocate $\tt{table}[\log N][N]$ which arranged in row-major to
reduce cache miss.  The Algorithm~\ref{alg:parallel-VGLCS} and
Figure~\ref{fig:interval-decomposition} illustrate the method of this
sparse table.

%   \item 

% Binary indexed tree: Both insert and query operation consume $O(\log
% n)$ time.

  \item 

Compressed sparse table: The insert operation consume amortized
$O(1)$ time.  The query operation consumes $O(s)$ time, which $s$ is the
size of the block.  Here, the program implement by the fixed size $s =
16$.  We can maintain extra the prefix and suffix maximum in block to
reduce the time complexity $O(s)$ to $O(1)$ in most queries.  It is introduced in the 

  \item

Amortized $O(1)$ sparse table:  The insert operation consume amortized
$O(1)$ time.  The query operation consume $O(1)$ time.  In the
implementation, we choose the fixed size $s = 8$.  The other
optimization is same as compressed sparse table.

\end{itemize}

\iffalse
\begin{itemize}
  \item 并查集 (Disjoint Set): 平均運行時間 $o(\alpha(n))$。只使用路徑壓縮技巧。
  \item 稀疏表 (Sparse Table): 插入 $O(\log n)$、詢問 $O(1)$。實作陣列宣告採用 $\tt{table}[\log N][N]$ 以減少快取未中。
  \item 樹狀數組 (Binary Indexed Tree): 插入、詢問均為 $O(\log n)$。
  \item 壓縮稀疏表 (Compressed Sparse Tree): 插入均攤 $O(1)$、詢問操作 $O(s)$,
  其中 $s$ 為拆分到區塊大小。實作時,維護區塊前綴和後綴最大值降低詢問複雜度至 $O(1)$,當發生 in-block 詢問再運行 $O(s)$ 算法。
\end{itemize}
\fi

The Figure~\ref{fig:fig-ISMQcmp} illustrate our compressed sparse table
run faster than the disjoint set when $n$ is greater than  $10^6$.  When
$n$ is greater than $10^7$, the compressed sparse table run $1.8 \times$
faster than the disjoint set in random test data.

\begin{figure}[!thb]
  \centering
  \includegraphics[width=0.45\linewidth]{\GraphicPath/fig-ISMQ.pdf}
  \caption{
  The performance of the ISMQ problem with different data structures on 
  E5-2620 server.
  }
  \label{fig:fig-ISMQcmp}
\end{figure}

However, we observe that our amortized $O(1)$ sparse table is more
slow than compressed sparse table.  In the dynamic programming, the
tendency of insertion value will inflict the performance of data
structure, so we experiment the different probability for each data
structure.  The number of the query is ten multiple of the number of
insertion.  The result is shown on table~\ref{tlb:ISMQcmp}.  The
table~\ref{tlb:ISMQcmp} show the performance of the ISMQ with sizes $N
= 10^7$, different maximum interval sizes $L$, the probability $p$ of
incremental elements and the probability $q$ of zero elements. There
are two strategies for each $(p, q, L)$, implemented by compressed
sparse table and amortized $O(1)$ sparse table.  The amortized $O(1)$
sparse table run $1.5 \times$ faster than the compressed table in
above situation.

\iffalse 當運行 $n > 10^6$ 時,我們提出的壓縮稀疏表的效能已經勝過并查
集的版本,其運行結果如圖表 ~\ref{fig:fig-ISMQcmp}。在 $n = 10^7$ 時,
加速 $1.25 \times$。然而,我們提供的 amortized $\theta(1)$ 的稀疏表慢
於并查集,我們做了深入的機率探討 (參照表 ~\ref{tlb:ISMQcmp}),由於大部
分的操作都被區塊後綴和前綴解決,沒有實際運用到內部詢問,約束區間詢問的
大小為 $L$,在 $N = 10^7$ 時,最多能加速 $1.26 \times$,其中插入和詢問
比例為 1:10,當詢問比重更大時,將有更明顯的加速。\fi

\begin{table*}
  \normalsize
  \caption{Total running time (second) for ISMQ with sizes $N = 10^7$, different maximum interval sizes $L$, the probability $p$ of incremental elements and the probility $q$ of zero elements. There are three strategies for each $(p, q, L)$, implemented by disjoint set, compressed sparse table, and O(1) sparse table.}
  \label{tlb:ISMQcmp}
  \centering
  \setlength\tabcolsep{0pt}
  \begin{tabular}{@{\extracolsep{4pt}}r c c c c c c c c c c c c c c c c}
    \firsthline
      & \multicolumn{5}{c}{$L=4$} & \multicolumn{5}{c}{$L=8$} & \multicolumn{5}{c}{$L=16$}\\
      \cline{2-6} \cline{7-11} \cline{12-16}
      $q$ & $0\%$ & $25\%$ & $50\%$ & $75\%$ & $100\%$ 
        & $0\%$ & $25\%$ & $50\%$ & $75\%$ & $100\%$ 
        & $0\%$ & $25\%$ & $50\%$ & $75\%$ & $100\%$ 
        & Speedup\\
      $p$ \\
      \hline
$0\%$ & \begin{tabular}{@{}r@{}} \textbf{1.72}\\1.95\\1.91 \end{tabular}& \begin{tabular}{@{}r@{}} \textbf{1.48}\\1.74\\1.78 \end{tabular}& \begin{tabular}{@{}r@{}} \textbf{1.48}\\1.74\\1.70 \end{tabular}& \begin{tabular}{@{}r@{}} \textbf{1.48}\\1.74\\1.75 \end{tabular}& \begin{tabular}{@{}r@{}} \textbf{1.48}\\1.74\\1.79 \end{tabular}& \begin{tabular}{@{}r@{}} \textbf{1.62}\\2.20\\1.78 \end{tabular}& \begin{tabular}{@{}r@{}} \textbf{1.62}\\2.20\\1.69 \end{tabular}& \begin{tabular}{@{}r@{}} \textbf{1.62}\\2.19\\1.72 \end{tabular}& \begin{tabular}{@{}r@{}} \textbf{1.62}\\2.20\\1.75 \end{tabular}& \begin{tabular}{@{}r@{}} \textbf{1.62}\\2.19\\1.71 \end{tabular}& \begin{tabular}{@{}r@{}} \textbf{1.83}\\2.23\\1.87 \end{tabular}& \begin{tabular}{@{}r@{}} \textbf{1.81}\\2.23\\1.88 \end{tabular}& \begin{tabular}{@{}r@{}} \textbf{1.81}\\2.23\\1.88 \end{tabular}& \begin{tabular}{@{}r@{}} \textbf{1.81}\\2.23\\1.87 \end{tabular}& \begin{tabular}{@{}r@{}} \textbf{1.81}\\2.23\\1.84 \end{tabular}& 0.98\\ \hline
$20\%$ & \begin{tabular}{@{}r@{}} \textbf{1.56}\\2.13\\2.04 \end{tabular}& \begin{tabular}{@{}r@{}} \textbf{1.57}\\2.14\\2.05 \end{tabular}& \begin{tabular}{@{}r@{}} \textbf{1.59}\\2.15\\2.06 \end{tabular}& \begin{tabular}{@{}r@{}} \textbf{1.57}\\2.16\\2.05 \end{tabular}& \begin{tabular}{@{}r@{}} \textbf{1.54}\\2.10\\1.99 \end{tabular}& \begin{tabular}{@{}r@{}} \textbf{1.93}\\3.31\\2.14 \end{tabular}& \begin{tabular}{@{}r@{}} \textbf{2.05}\\3.28\\2.15 \end{tabular}& \begin{tabular}{@{}r@{}}2.26\\3.21\\ \textbf{2.16} \end{tabular}& \begin{tabular}{@{}r@{}} \textbf{2.05}\\3.29\\2.16 \end{tabular}& \begin{tabular}{@{}r@{}} \textbf{1.78}\\3.32\\2.11 \end{tabular}& \begin{tabular}{@{}r@{}}2.66\\3.60\\ \textbf{2.31} \end{tabular}& \begin{tabular}{@{}r@{}}2.77\\3.55\\ \textbf{2.31} \end{tabular}& \begin{tabular}{@{}r@{}}2.92\\3.44\\ \textbf{2.32} \end{tabular}& \begin{tabular}{@{}r@{}}2.67\\3.57\\ \textbf{2.31} \end{tabular}& \begin{tabular}{@{}r@{}}2.45\\3.69\\ \textbf{2.28} \end{tabular}& \textbf{1.26}\\ \hline
$40\%$ & \begin{tabular}{@{}r@{}} \textbf{1.62}\\2.40\\2.12 \end{tabular}& \begin{tabular}{@{}r@{}} \textbf{1.65}\\2.46\\2.14 \end{tabular}& \begin{tabular}{@{}r@{}} \textbf{1.72}\\2.47\\2.16 \end{tabular}& \begin{tabular}{@{}r@{}} \textbf{1.65}\\2.46\\2.14 \end{tabular}& \begin{tabular}{@{}r@{}} \textbf{1.59}\\2.33\\2.00 \end{tabular}& \begin{tabular}{@{}r@{}} \textbf{2.21}\\4.05\\2.33 \end{tabular}& \begin{tabular}{@{}r@{}}2.46\\4.00\\ \textbf{2.34} \end{tabular}& \begin{tabular}{@{}r@{}}2.81\\3.79\\ \textbf{2.36} \end{tabular}& \begin{tabular}{@{}r@{}} \textbf{2.29}\\4.04\\2.35 \end{tabular}& \begin{tabular}{@{}r@{}} \textbf{1.81}\\4.04\\2.25 \end{tabular}& \begin{tabular}{@{}r@{}}2.81\\4.40\\ \textbf{2.50} \end{tabular}& \begin{tabular}{@{}r@{}}2.94\\4.22\\ \textbf{2.54} \end{tabular}& \begin{tabular}{@{}r@{}}3.04\\3.95\\ \textbf{2.53} \end{tabular}& \begin{tabular}{@{}r@{}}2.67\\4.37\\ \textbf{2.52} \end{tabular}& \begin{tabular}{@{}r@{}} \textbf{2.38}\\4.63\\2.46 \end{tabular}& \textbf{1.20}\\ \hline
$60\%$ & \begin{tabular}{@{}r@{}} \textbf{1.65}\\2.45\\2.15 \end{tabular}& \begin{tabular}{@{}r@{}} \textbf{1.71}\\2.58\\2.17 \end{tabular}& \begin{tabular}{@{}r@{}} \textbf{1.82}\\2.67\\2.19 \end{tabular}& \begin{tabular}{@{}r@{}} \textbf{1.69}\\2.52\\2.17 \end{tabular}& \begin{tabular}{@{}r@{}} \textbf{1.60}\\2.24\\2.03 \end{tabular}& \begin{tabular}{@{}r@{}} \textbf{2.29}\\4.30\\2.38 \end{tabular}& \begin{tabular}{@{}r@{}}2.60\\4.31\\ \textbf{2.41} \end{tabular}& \begin{tabular}{@{}r@{}}2.99\\4.08\\ \textbf{2.45} \end{tabular}& \begin{tabular}{@{}r@{}} \textbf{2.20}\\4.36\\2.44 \end{tabular}& \begin{tabular}{@{}r@{}} \textbf{1.75}\\3.98\\2.28 \end{tabular}& \begin{tabular}{@{}r@{}}2.97\\4.67\\ \textbf{2.55} \end{tabular}& \begin{tabular}{@{}r@{}}3.05\\4.51\\ \textbf{2.59} \end{tabular}& \begin{tabular}{@{}r@{}}3.12\\4.22\\ \textbf{2.63} \end{tabular}& \begin{tabular}{@{}r@{}}2.67\\4.81\\ \textbf{2.61} \end{tabular}& \begin{tabular}{@{}r@{}} \textbf{2.33}\\4.56\\2.47 \end{tabular}& \textbf{1.22}\\ \hline
$80\%$ & \begin{tabular}{@{}r@{}} \textbf{1.62}\\2.31\\2.12 \end{tabular}& \begin{tabular}{@{}r@{}} \textbf{1.68}\\2.53\\2.14 \end{tabular}& \begin{tabular}{@{}r@{}} \textbf{1.82}\\2.74\\2.19 \end{tabular}& \begin{tabular}{@{}r@{}} \textbf{1.64}\\2.38\\2.14 \end{tabular}& \begin{tabular}{@{}r@{}} \textbf{1.55}\\1.97\\2.01 \end{tabular}& \begin{tabular}{@{}r@{}} \textbf{2.19}\\4.00\\2.29 \end{tabular}& \begin{tabular}{@{}r@{}}2.57\\4.22\\ \textbf{2.35} \end{tabular}& \begin{tabular}{@{}r@{}}3.00\\4.15\\ \textbf{2.42} \end{tabular}& \begin{tabular}{@{}r@{}} \textbf{2.01}\\4.15\\2.38 \end{tabular}& \begin{tabular}{@{}r@{}} \textbf{1.69}\\3.13\\2.16 \end{tabular}& \begin{tabular}{@{}r@{}}2.99\\4.40\\ \textbf{2.46} \end{tabular}& \begin{tabular}{@{}r@{}}3.04\\4.42\\ \textbf{2.52} \end{tabular}& \begin{tabular}{@{}r@{}}3.15\\4.29\\ \textbf{2.62} \end{tabular}& \begin{tabular}{@{}r@{}}2.59\\4.65\\ \textbf{2.57} \end{tabular}& \begin{tabular}{@{}r@{}} \textbf{2.11}\\3.64\\2.34 \end{tabular}& \textbf{1.24}\\ \hline
$100\%$ & \begin{tabular}{@{}r@{}} \textbf{1.58}\\2.09\\2.03 \end{tabular}& \begin{tabular}{@{}r@{}} \textbf{1.61}\\2.33\\2.09 \end{tabular}& \begin{tabular}{@{}r@{}} \textbf{1.68}\\2.62\\2.16 \end{tabular}& \begin{tabular}{@{}r@{}} \textbf{1.57}\\2.07\\2.08 \end{tabular}& \begin{tabular}{@{}r@{}} \textbf{1.52}\\1.79\\1.79 \end{tabular}& \begin{tabular}{@{}r@{}} \textbf{1.68}\\3.29\\1.99 \end{tabular}& \begin{tabular}{@{}r@{}} \textbf{1.93}\\3.66\\2.04 \end{tabular}& \begin{tabular}{@{}r@{}}2.42\\3.90\\ \textbf{2.10} \end{tabular}& \begin{tabular}{@{}r@{}} \textbf{1.72}\\3.13\\2.04 \end{tabular}& \begin{tabular}{@{}r@{}} \textbf{1.68}\\2.19\\1.81 \end{tabular}& \begin{tabular}{@{}r@{}}2.22\\3.71\\ \textbf{2.19} \end{tabular}& \begin{tabular}{@{}r@{}}2.65\\3.80\\ \textbf{2.25} \end{tabular}& \begin{tabular}{@{}r@{}}2.87\\3.96\\ \textbf{2.33} \end{tabular}& \begin{tabular}{@{}r@{}} \textbf{2.06}\\3.50\\2.25 \end{tabular}& \begin{tabular}{@{}r@{}} \textbf{1.82}\\2.25\\2.02 \end{tabular}& \textbf{1.23}\\ \hline
Speedup & 0.90& 0.83& 0.87& 0.84& 0.85& 0.96& \textbf{1.09}& \textbf{1.24}& 0.98& 0.95& \textbf{1.21}& \textbf{1.21}& \textbf{1.26}& \textbf{1.16}& \textbf{1.07}\\

    \lasthline
  \end{tabular}
\end{table*}

\subsection{Parallel Range Query}

Each stage will have $n$ number of elements and the $n$ number of the
query.  In this special cases, we run the compressed sparse table to
improve the performance compared to the origin.  If we preprocess all
range size before building and , we get the optimized version which run
$2.35$ faster than origin sparse table.  The Table~\ref{tlb:CORMQ} is
shown the result of the strategy of the parallel range query.

\iffalse 每一次有 $n$ 個元素和 $n$ 組詢問,針對這種特殊性質的問題,我
們運行樸素的 \texttt{CORMQ} (compressed RMQ) 得到效能改善,搭配可預測
的分析降低運算量 (參照 \texttt{CORMQ-opt}),得到更好的改善。在
\texttt{CORMQ-opt} 策略中,得到 $2.35 \times$ 倍的加速,結果如表
~\ref{tlb:CORMQ}。\fi

\begin{table*}[!thb]
  %\tiny
  \caption{Total running time (ms) for finding RMQ of different sizes $N$ and maximum interval sizes $L$.}
  \label{tlb:CORMQ}
  \centering
  \begin{tabular}{l c c c c}
    \firsthline
      & \multicolumn{4}{c}{$N$} \\
      \cline{2-5}
        & \multicolumn{2}{c}{$30000$} & $50000$ & $100000$ \\
      $L$ & $2^{10}$ & $2^{15}$ & $2^{15}$ & $2^{15}$ \\
      \hline
      parallel-\tt{RMQ}     & $903$ & $1516$ & $1874$ & $4116$ \\
      parallel-\tt{CORMQ}   & $995$ & $1475$ & $1689$ & $2594$ \\
      parallel-\tt{CORMQ-opt} & $843$ & $1373$ & $1136$ & $1745$ \\
      \hline
      Speedup & $1.07\times$ & $1.10\times$ & $1.64\times$ & $2.35\times$\\
    \lasthline
  \end{tabular}
\end{table*}
