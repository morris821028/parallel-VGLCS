\section{Introduction} %Introduction
\label{sec:Introduction}

The \emph{longest common subsequence} (LCS) problem applied many
products and fields widely.  In multi-core platform, most studies
focus on the wavefront parallelism. Motivated by the definition of
recursion in LCS, Jiaoyun Yang introduced a new formula to exploit
more cache performance.  Here, we use the similar idea to improve LCS
with variable contraints, which refer in Iliopoulos' study.  For
example, \emph{fixed gap LCS} (FGLCS) require the distance between two
consecutive matches limited at most $k+1$.  It can be solved in
$O(nm)$, which $n, \; m$ is the length of input strings.

In many kinds of contraint, this paper foucus on \emph{variable gap
LCS} (VGLS). The VGLCS require the distance between two consecutive
matched limited at most $G_i$, which $G_i$ is the value of the
position $i$ in input array. For example, two protein sequences $A =
\tt{GCGCAATG}$ and $B = \tt{GCCCTAGCG}$, and two gap functions $G_A =
[3, 1, 1, 2, 0, 0, 2, 1]$ and $G_B = [2, 0, 3, 2, 0, 1, 2, 0, 1]$.
Figure ~\ref{fig:VGLCSex} presents an example to show that tow motifs
$\tt{G..C..C..A}$ and $\tt{G..C..C..T}$.  In previous study, Yung-
Hsing Peng ~\cite{yunghsing} introduced the $O(nm \alpha(n))$ solution
for the VGLCS problem.

The remainder of the paper is organized as follows. In the section 2,
we present the parallel algorithm to solve VGLCS problem. In the
section 3, we present a new algorithm, which parallel easily and time
complexity $O(nm)$ better than previous study. In the section 4 to 5,
we provide the optimized implementation and the result of experiments.
The last section has a brief conclusion in this paper.

\begin{figure}[!thb]
  \centering
  \includegraphics[width=\linewidth]{graphics/fig-VGLCSex.pdf}
  \includegraphics[width=\linewidth]{graphics/fig-VGLCSex2.pdf}

  \caption{   An example for illustrating the VGLCS, which has two
protein sequences $A = \tt{GCGCAATG}$ and $B = \tt{GCCCTAGCG}$, and
two gap functions $G_A = [3, 1, 1, 2, 0, 0, 2, 1]$ and $G_B = [2, 0,
3, 2, 0, 1, 2, 0, 1]$.   }
  
  \label{fig:VGLCSex}
\end{figure}