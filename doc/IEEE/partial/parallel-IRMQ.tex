\section{Range Maximum Query}
\label{sec:parallelRMQ}

\subsection{Background}

Even through we parallel origin serial algorithm successfully, the
theoretical time complexity of VGLCS algorithm is limited by the range
maximum/minimum query in parallel.  In VGLCS problem, each stage has
$n$ elements and $n$ numbers of range query. We should let pre-
processing and query time minimize. The most of tree structures cannot
show efficient performance in both pre-processing and query time. Some
offline algorithm is too hard to parallel. In this section, we provide
the solution to get better performance in both pre-processing and
query stage.

\subsection{Compressed Cartesian Tree}

In Fischer ~\cite{fischer} paper, the $O(n)$ -- $O(1)$ algorithm is
corrponding by Catalan number $\frac{1}{s+1}\binom{2s}{s} =
O(\frac{4^s}{s^{1.5}})$ to build look-up table. When we choose $s =
\frac{1}{4} \log n$ as block size, the space complexity is $O(s^2
\frac{4^s}{s^{1.5}}) = o(n)$, and time complexity is $o(n)$. Each
range query will be split into 4 parts, 2 super-block queries and 2
in-block queries. It need to 4 time memory access. We give a example
in the figure ~\ref{fig:interval-decomposition}.  In the offline RMQ
problem, it has the theoretical algorithm which run in $O(n)$ --
$O(1)$ time.

When $n$ is large, four time memory access caused serious cache miss.
In order to improve cache miss, Demaine introduced the cache-oblivious
algorithm in Cartesian tree ~\cite{demaine}.

In above technology, we parallel RMQ problem by Fischer's idea and get
time complexity $O(n / p + \log n)$ -- $O(1)$ algorithm. We also
combines compression skill from Demaine's paper. It reduces cache-miss
and run in ideal complexity.

We pick the fixed length $s = 16$, which can solve $n = 2^{64}$ one-
dimension range maximum query. When we insert $i$-th elements, the
number of $i$-th left rotation $l_i$ must satisfy $\sum_{i=1}^{n} l_i
< i$. Because all $l_i$ is small than 16, it can present in 4-bit
integer.  Due to above property of Cartesian tree, we merger 16 4-bit
integers into a 64-bit integer to present a Caartesian tree. The
compressed algorithm ~\ref{alg:cartesian-to-64bits} run in $O(s)$.

Finally, the appropriate size can compress the usage of space to
reduce cache-miss and also show better performance in modern 64-bit
register.  We modify Demaine's range query algorithm as the algorithm
~\ref{alg:cartesian64bits-query}.

\begin{algorithm*}
  \caption{Transfer Cartesian Tree to 64-bits with 8 integers}
  \label{alg:cartesian-to-64bits}
  \begin{algorithmic}[1]
    \Require
      $A[1 \cdots 16]$: 16 integers store in array
    \Ensure 
      $\textit{tmask}$: compress Cartesian tree into 64-bits integer
      \State $\tt{LOGS} = 4$
      \State $\tt{POWS} = 2^{\tt{LOGS}}$
      \State int $D[POWS+1]$, $Dp = 0$;
      \State uint64\_t $tmask$ = $0$
      \State $D[0]$ = SHRT\_MAX
      \For{$i = 1$ to $\tt{POWS}$} 
        \State $v = A[i]$;
        \State $cnt = 0$;
        \While{$D[Dp] < v$}
          \State $Dp = Dp-1$
          \State $cnt = cnt + 1$
        \EndWhile
        \State $Dp = Dp+1$
        \State $D[Dp] = v$
        \State $tmask = tmask | ((cnt)<<((i-1)<<2))$
      \EndFor
      \State return $tmask$
  \end{algorithmic}
\end{algorithm*}

\begin{algorithm*}
  \caption{Range Minimum Query in 64-bits Cartesian Tree}
  \label{alg:cartesian64bits-query}
  \begin{algorithmic}[1]
    \Require
      $\textit{tmask}$: 64-bits Cartesian tree;
      $[l, r]$: query interval between $l$ and $r$
    \Ensure 
      $\textit{minIdx}$: the index of the minimum value in interval
    \State int $\textit{minIdx}$ = $l$, $x$ = $0$;
    \For{$l = l+1$ to $r$}
      \State $x$ = $x+1 - ((tmask>>(l<<2))\&15)$
      \If{$x \le 0$}
        \State $\textit{minIdx}$ = $l$
        \State $x$ = $0$
      \EndIf
    \EndFor
    \State return $\textit{minIdx}$
  \end{algorithmic}
\end{algorithm*}

In VGLCS problem, above algorithm provide compression skill to reduce
cache-miss, but increase the time complexity. The pre-processing spend
$O(n)$ time, and single query spend $O(s)$ time in RMQ. Totally, time
complexity is $O(n^2 \; s / p + n \max(\log n, s))$.

\section{Incremental Range Maximum Query}

\subsection{Build Look-up Table}

\subsection{Dynamic Cartesian Tree}