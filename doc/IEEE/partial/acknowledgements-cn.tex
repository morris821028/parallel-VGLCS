\begin{acks}

能順利地完成這篇論文,首先,感謝劉邦鋒和吳真貞老師的指導,在論文架構和描
述手法的教導,才使得篇幅雜亂的初稿便得更加地易於理解,討論過程中更加地精
練專業知識,學生在此衷心感謝老師。

特別感謝中央大學的郭人維學長,拉拔在演算法及資料結構領域上的研究,其給予
的助力使得論文開花結果。更感謝在網路上許許多多來自各方的朋友分享研究心得
,讓彼此切磋茁壯。

在進入臺灣大學研究所的這兩年中,感謝實驗室學長們的鼓勵與支持,本對於學術
研究文化和自身能力的迷茫,接受學長們的啟示後,最終得以撐過第一學年。在第
二年中,感謝共同奮戰論文的古耕竹、古君葳、鄭以琳、吳軒衡等同學,彼此加油
打氣,使得在撰寫論文的路上並不孤單,督促進展、協助撰寫與驗證想法更加地順
利,願你們也能順利畢業、研究出滿意的成果。

在研究實驗上,感謝實驗室學弟張逸寧、林明璟、蔡慶源、朱清福的參與,被迫實
作出放置於批改娘系統上題目,這些題目原本為論文的一小部分,可透過不同資料
結構與算法解決,在本篇追求效能極致的路上貢獻了一份心力,為本實驗結果給予
更有信心的立論基礎。願你們在接下來的一年裡,經過老師指導與同學們相互引領
下順利畢業。

此外,特別感謝高中時期帶入門的溫健順老師,在選擇領域分組時,相信我在資訊
領域上的發展,拉近資訊組培養,經歷三年的教導後,才能順利走向這一條道路。


最後,感謝家人們一路相伴,進入資訊工程領域後,經歷大學轉學、延畢到研究所
的路上,對我的選擇給予支持。

感謝上述的各位與師長們一路上的支持與資助。

\end{acks}