\section{Implementation Optimization} \label{sec:Implementation}

This section describes optimization in our implementations of VGLCS
algorithms, both in sequential and parallel environments.  Note that
some of these techniques address hardware characteristics, e.g.,
memory cache behavior, and are not confined to asymptotically analysis
only.

\subsection{The Strategy of Disjoint Set Implementation}

We now describe the optimization in the implementations of disjoint
sets, whose applications include VGLCS.

%% There two techniques to efficiently merge two disjoint sets -- {\em
%%   path compression}, and {\em rank strategy}.

\subsubsection{Cache Performance}

Memory cache is essential to efficient implementation of disjoint sets.
Patwary, Blair, and Manne~\cite{Patwary2010ExperimentsOU} conducted
experiments on disjoint-set and showed that different implementations
have different impact on different levels of caches misses.  In
practice, the cache miss is strongly related to how we go from a child
to its ancestors through pointer chasing during path compression.
Usually, an algorithm with a lower time complexity, e.g., will have more
``long jumps'' than an algorithm with a higher time complexity.  Here
the long jump means the pointer chasing will go from a memory address to
a far away memory address.  Please refer to
Figure~\ref{fig:long-short-jump-disjoint} for an illustration.

\begin{figure}[!thb]
  \centering \subfigure[An algorithm with a lower time complexity] {
    \includegraphics[width=0.42\linewidth]{\GraphicPath/fig-rem-long-jump.pdf}
    \label{fig:long-short-jump-disjoint-long}
  } \subfigure[An algorithm with a higher time complexity] {
    \includegraphics[width=0.42\linewidth]{\GraphicPath/fig-rem-short-jump.pdf}
    \label{fig:long-short-jump-disjoint-short}
  }
  \caption{The parent jump in disjoint set}
  \label{fig:long-short-jump-disjoint}
\end{figure}

The Rem's algorithm~\cite{dijkstra1976a} achieves better cache
performance by a {\em merge-by-index} technique.  Traditional disjoint
set merging techniques are {\em merge-by-rank} and {\em
merge-by-size}~\cite{Tarjan1975EfficiencyOA}, which determine the root
of new tree by the ranks and the sizes of the two trees respectively.
Despite that they do have asymptotically better time complexity, their
performance in practice is not so impressive due to the previously
mentioned cache issue.  In contrast, Rem's algorithm assigns an {\em
index} to each node, and merges two disjoint sets according the index of
the roots of the two trees being merged.  That is, the root of the new
merged tree will be the root with the ``larger'' index.

In our experiments, we still use merge-by-rank to merge disjoint set
trees.  However, when we have two {\em equally ranked} disjoint set
trees to merge, we will apply the {\em merge-by-index} technique to
break the tie, so as to improve cache performance.  Experiments show
that the merge-by-index tie-breaking technique improves by 3\% over the
approach that breaks the tie randomly.  Please refer to
Figure~\ref{fig:fig-perf-cache-miss-rem} for an illustration.

\begin{figure}
  \centering
  \includegraphics[width=0.95\linewidth]{\GraphicPath/fig-perf-cache-miss-rem.pdf}
  \caption{The performance of the {\em merge-by-index tie-breaking} for
  the combination of different applications and merge technique in
  disjoint set data structure on an E5-2620 server, which is measured by
  the {\tt perf} analyzing tool in Linux.}
  \label{fig:fig-perf-cache-miss-rem}
\end{figure}

\subsubsection{Application on VGLCS}

% VGLCS

It is possible to further optimize the implementation of Peng's
sequential algorithm (Algorithm~\ref{alg:serial-VGLCS}) by a {\em lazy
insertion} technique.  Recall that in Algorithm~\ref{alg:serial-VGLCS},
when the $i$-th character from the first input string does not match the
$j$-th character of the second input string, we need to place a zero
into the dynamic programming table at position $V[i][j]$.  In practice,
this mismatch will happen {\em very frequently}, and will cause frequent
insertions of zeros into the disjoint sets.  Each insertion then links
the newly inserted singleton node of zero to the node of zero inserted
one iteration ago.

We implemented an optimization that resolves this repeated insertion,
and linking, of zeros.  Our implementation will scan through a series of
zeros, and locate the next non-zero $V$ element (denoted as $v$) in the
same column, and insert them as a batch.  Since all values in $V$ are
non-negative, we can link all zeros we have seen {\em directly} to $v$.
From the experiments, we observe that this technique causes less pointer
chasing and updates and does improve performance.

Note that we do {\em not} apply the lazy insertion optimization in our
{\em parallel} implementation.  For a multi-core platform, the
efficiency of thread synchronization is essential to the performance.
Since the threads of a parallel VGLCS algorithm needs to synchronize at
every column of the dynamic programming table in order to process a row,
the lazy insertion, which can ignore this dependency in a sequential
environment, is not beneficial, and is not adopted in our parallel
implementation.

\subsection{Parallel Range Query in VGLCS}

We further improve the performance of range query by maintaining two
extra tables in the blocked sparse table approach described in
Section~\ref{sec:blocked-sparse-table}.  There are now three tables -- a
sparse table on the block maximum $T_S$, a {\em prefix maximum table}
$P$, and a {\em suffix maximum table} $S$.  As described in
Section~\ref{sec:blocked-sparse-table}, the sparse table $T_S$ can
easily answer range query of consecutive blocks on the input $A$.  The
prefix maximum table $P$ maintains the maximum of the prefix {\em
within} a block, and similarly the suffix maximum table $S$ maintains
the maximum for prefix.  Please refer to
Figure~\ref{fig:compressed-sp-opt} for an illustration.

It is easy to see that any range query can be answered by {\em two}
queries into sparse table $T_S$, {\em one} query into prefix maximum
table $P$, and {\em one} into suffix maximum table $S$.  For example, in
Figure~\ref{fig:compressed-sp-opt}, the query from index 2 to 18 can be
answered by two queries into the sparse table $T_S$ -- one from block 1
to block 2, and one from 2 to 3, a query for the suffix of length 2 into
the first block of $S$, and a query for the prefix of length 3 into the
last block of $P$.  Please refer to Figure~\ref{fig:compressed-sp-opt}
for an illustration.

%% The in-block query is a very small probability event due to small $s =
%% \frac{\log n}{4}$.

%% We can use prefix and suffix maximum array to reduce the number of
%% lookup operation on LCA table.

\begin{figure}[!thb]
  \centering \subfigure[The prefix/suffix maximum tables for blocks] {
    \includegraphics[width=0.55\linewidth]{\GraphicPath/fig-compressed-sp-prefix-suffix.pdf}
    \label{fig:compressed-sp-opt-all}
  } \subfigure[A Sparse Table for $T_S$] {
    \includegraphics[width=0.35\linewidth]{\GraphicPath/fig-sparse-table-opt.pdf}
    \label{fig:compressed-sp-opt-ts}
  } \caption{Block maximum $T_S$, prefix maximum $P$, and suffix maximum $S$.}
  \label{fig:compressed-sp-opt}
\end{figure}

We argue that the {\em order} to access these maximum tables is
important.  Our implementation accesses the block maximum $T_S$ {\em
first}, then the prefix maximum $P$ {\em second}, then the suffix
maximum $S$ {\em last}.  The reasoning for this order is as follows.
Since we need to access two elements in the sparse table $T_S$ in {\em
the same level}, it is very likely that they will be in the same cache
line, so access the first will bring in the other automatically by the
hardware caching mechanism.  For example, in
Figure~\ref{fig:compressed-sp-opt-ts}, we will access both $T_{S}[1][2]$
and $T_{S}[1][3]$ in order to find the maximum from block 1 to 3.

Our implementation improves performance by ``peeking'' into two
neighboring elements in the $T_{S}$ table.  To be more specific, we
peek the element {\em before} the first element, and the one {\em
  after} the second element, which we just accessed from $T_S$.

For example, in Figure~\ref{fig:compressed-sp-opt-ts}, we will peek into
$T_{S}[1][1]$ and $T_{S}[1][4]$, which are very likely also present in
cache because they are in the same level of the sparse table as the
previously accessed elements $T_{S}[1][2]$ and $T_{S}[1][3]$.  The idea
is that if the maximum from the block maximum, i.e., the maximum of
$T_{S}[1][2]$ and $T_{S}[1][3]$, is already {\em larger} than the
overall block maximum where the prefix belongs, i.e., in
$T_{S}[1][4]$, then we do {\em not} need to check the $P$ table.
Similarly, if the maximum from the block maximum {\em and} the prefix
maximum is larger than the block maximum where the suffix belongs,
i.e., in $T_{S}[1][1]$, then we do not need to check the $S$ table
either.  This peeking is efficient since the data being peeked are
most likely in the same cache line, and can save unnecessary access to
$P$ and $S$ tables.

Algorithm~\ref{alg:rmq-access-order-2e} shows the access order for the
range maximum query to reduce the cache miss.  Note that we check the
prefix maximum $P$ {\em before} the suffix maximum $S$ since the value
of the dynamic programming table in the same row is {\em increasing},
so it is more likely that a prefix maximum, which has a larger column
index, can save a check into the suffix table, which has a smaller row
index.

Our experiments show that despite that these maximum arrays require
additional $O(n)$ memory, the peeking technique improves overall
performance by up to 8\% when $n = 10^4$, and it improves the overall
performance of query operations by up to 35\% when $n = 10^7$.  Please
refer to Figure~\ref{fig:fig-perf-cache-miss-peek} for an illustration.

\begin{figure}
  \centering
  \includegraphics[width=0.95\linewidth]{\GraphicPath/fig-perf-cache-miss-peek.pdf}
  \caption{The performance of the ``peeking'' for
  the combination of different applications and environments in blocked
  sparse table on an E5-2620 server, which is measured by the {\tt perf}
  analyzing tool in Linux.}
  \label{fig:fig-perf-cache-miss-peek}
\end{figure}

% One can consider the
% first level of the compressed sparse table $T_S[0]$ as a small cache,
% which may prevent access to the prefix/suffix maximum array, as
% described earlier.

\begin{algorithm}
\SetAlgoNoLine
\KwIn{
  $A$: the input array\; 
  $P, \; S$: the prefix/suffix maximum array for each block of the compressed sparse table \;
  ${\cal T}$: the Catalan index array for each block of the compressed sparse table \;
  $T_s$: the compressed sparse table \; 
  $[l, r]$: ranged query \;
}
\KwOut{
  $v$: the maximum value in $A[l .. r]$ \;
}

\If{$l$ and $r$ in the same block} {
  $i \gets$ Query the ranged maximum query $[l, r]$ on ${\cal T}_\text{block(l)}$ \;
  return $A[i]$ \;
}
$v \gets - \infty$ \;
$l' \gets$ The lower bound of the next block by the position $l$ \;
$r' \gets$ The upper bound of the previous block by the position $r$ \;
\If{$l' \le r'$} {
  $t \gets \lfloor \log_2 (r'-l'+1) \rfloor$ \;
  $SQ_L \gets T_s[t][l' + 2^t - 1]$ \;
  $SQ_R \gets T_s[t][r']$ \;
  $v \gets \max(SQ_L, SQ_R)$ \;
}

\If(\tcc*[f]{$T_s[0]$ is a cache to avoid impossible event}){$T_s[0][\text{block}(r)] > v$} {
  $v \gets \max(v, P[r])$ \;
}

\If(\tcc*[f]{Place very low possibility event to the end}){$T_s[0][\text{block}(l)] > v$} {
  $v \gets \max(v, S[l])$ \;
}

return $v$ \;

\caption{Access order of ranged maximum query}
\label{alg:rmq-access-order-2e}
\end{algorithm}


