\section{實驗結果} \label{sec:Experiment}

實驗主要分成以下三種:其一,測試分塊與未分塊之間的稀疏表,測試平行區間詢問的效能;
其二,增長後綴最大值詢問於不同的資料結構上的效能差異;最後,
計算不同資料結構應用在 VGLCS 計算上的效能差異。

我們實驗於 Intel Xeon E5-2620 2.4 Ghz 處理器,其包含 384K bytes 的 L1 快取、
1536K bytes 的 L2 快取、15M bytes 的 L3 快取。而 Intel CPU 支持「超執行緒」(hyper-threading),
每一處理器包含 6 個核心。使用的作業系統為 Ubuntu 14.04,程序撰寫使用 C++ 和 OpenMP,
編譯器使用 gcc 以及相應的編譯參數 {\tt -O2} 和 {\tt -fopenmp} 進行實驗。

\subsection{分塊與未分塊稀疏表}

首先,我們比較未分塊稀疏表 (章節~\ref{sec:sparse-table} 所述) 
和分塊稀疏表 (章節~\ref{sec:blocked-sparse-table} 所述) 於平行區間詢問的效能。
在分塊稀疏表中,我們使用「右側棧出」編碼,而非查找 LCA 表,
因我們發現「右側棧出」編碼比 LCA 表來得更有效率。接續的實驗中,
我們實驗不同區間長度與不同元素個數 $N$ 的關係。

參照表~\ref{tlb:CORMQ} 比較分塊與未分塊的稀疏表在平行區間詢問的效率。
從實驗中,發現到使用右側棧出的分塊稀疏表比未分塊的版本來得快,
效能隨著 $N$ 增加而更加明顯。如在 $N$ 達到 $10^5$ 時,
分塊稀疏表相交於未分塊快上 $1.4$ 倍。

我們相信限制最大區間詢問的長度 (在表~\ref{tlb:CORMQ} 中的 $L$) 影響存取稀疏表的快取效能。
此外,當固定元素個數為 $N$ 時,隨著 $L$ 增加,加速比例越明顯,其原因可能為以下幾點所致:
當詢問的區間長度較大時,分塊稀疏表只需要在 $\log{\nicefrac{N}{s}}$ 層之間跳躍存取,
2 次超塊詢問以及 2 次 $O(s)$ 的塊內詢問。
相反地,未分塊稀疏表需要在 $\log{N}$ 層之間存取,因此在不同層數之間存取稀疏表的影響下,
易造成應答區間詢問時,大部分存取與先前不同的層,導致資料局部性下降而產生快取未中。
然而,上述的問題在 $N$ 小不會發生,唯有 $N$ 大到影響快取效能才使得分塊效益更為顯著。

\begin{table*}[!thb]
  %\tiny
  \caption{Total running time (ms) for finding RMQ of different sizes $N$ and maximum interval sizes $L$.}
  \label{tlb:CORMQ}
  \centering
  \begin{tabular}{l c c c c}
    \firsthline
      & \multicolumn{4}{c}{$N$} \\
      \cline{2-5}
        & \multicolumn{2}{c}{$30000$} & $50000$ & $100000$ \\
      $L$ & $2^{10}$ & $2^{15}$ & $2^{15}$ & $2^{15}$ \\
      \hline
      parallel-\tt{RMQ}     & $903$ & $1516$ & $1874$ & $4116$ \\
      parallel-\tt{CORMQ}   & $995$ & $1475$ & $1689$ & $2594$ \\
      parallel-\tt{CORMQ-opt} & $843$ & $1373$ & $1136$ & $1745$ \\
      \hline
      Speedup & $1.07\times$ & $1.10\times$ & $1.64\times$ & $2.35\times$\\
    \lasthline
  \end{tabular}
\end{table*}

\subsection{「右側棧出」與「LCA 表」}

這一小節,我們比較四種資料結構解決「增長後綴詢問」({\em incremental suffix} query) 的能力,
這四個資料結構分別為並查集、未分塊稀疏表、使用右側棧出編碼的塊狀稀疏表以及使用 LCA 表的塊狀稀疏表。
使用並查集解決增長後綴詢問的細節請參照章節~\ref{sec:parallelRMQ}。
對於並查集的實作細節,使用「路徑壓縮」({\em path compression}) 和「秩合併」({\em merge-by-rank}) 的策略,
每個操作的攤銷時間複雜度為 $O(\alpha(n))$。在分塊稀疏表中,我們選用塊大小 $s$ 為 $8$,
所有稀疏表建造時皆以「行為主」(row-major manner) 的管理來減少快取未中,
請參照算法~\ref{alg:parallel-VGLCS} 和圖~\ref{fig:interval-decomposition} 的說明。

% Here we need a table of all complexity.

接下來的簡化案例中,我們交替使用「附加」({\em appending}) 
和「詢問區間」({\em range querying}) 操作,實驗中的長度和詢問位置皆採用均勻分布,
如圖~\ref{fig:fig-ISMQcmp} 說明四個不同資料結構在增長後綴詢問的效能。
其中,右側棧出編碼的分塊稀疏表快於其他的資料結構。
當 $n$ 達 $10^6$ 時,分塊稀疏表皆快於並查集。
而當 $n$ 達 $10^7$ 時,右側棧出編碼的稀疏表比並查集快上 $1.8$ 倍。

\begin{figure}[!thb]
  \centering
  \includegraphics[width=0.75\linewidth]{\GraphicPath/fig-ISMQ.pdf}
  \caption{
    不同資料結構在增長後綴最大值詢問的效能差異,實驗環境於 E5-2620 機器上}
  \label{fig:fig-ISMQcmp}
\end{figure}

在另一個複雜的案例中,我們討論在動態規劃上的一些變因如何影響效能,
這些變因包含插入值分布、最大區間詢問的分布以及插入與詢問之間的比例。
為方便討論,我們使用 $p$ 表示插入比前一個更大的元素機率、
$q$ 表示插入一個零元素的機率和 $L$ 表示詢問區間的最大長度。
在我們的實驗中,我們固定元素個數 $N$ 為 $10^7$,詢問區間最大長度 $L$ 為 $4$ 到 $16$ 之間,
$p$ 和 $q$ 之間的機率於 $0 \%$ 到 $100 \%$ 且詢問次數為插入次數的 10 倍。

如圖~\ref{fig:fig-ISMQcmp},我們觀察到分塊稀疏表比並查集或者未分塊都來得好,
所以我們著重要塊狀稀疏表的實作,意即我們先前討論的右側棧出和 LCA 表的實作方式,
而右側棧出編碼對於增長詢問很容易地想到,只需要對最後一個塊直接添加資訊即可。

表~\ref{tlb:ISMQcmp} 比較「右側棧出編碼」與「LCA 表」解決增長後綴區間詢問的運行時間,
特別注意到 LCA 表的方法提供理論上攤銷時間 $O(1)$ 的時間,而先前的實驗中,
右側棧出編碼明顯地快於LCA 表 $1.5$ 之多,我們相信有以下兩個原因導致這個現象:
其一,LCA 表需要更多的指令來計算出卡塔蘭索引值,相反地,右側棧出編碼只需要維護推出次數;
其二,根據理論~\cite{Fischer2006TheoreticalAP} 選擇塊大小 $s$ 為 $\frac{\log n}{4}$,
這影響到無法使用太大 $s$,因為 LCA 表所造成的空間過大,越大則造成更多的快取未中問題發生。

\begin{table}[htbp]
  \caption{The timing (in seconds) of answering incremental suffix
    maximum query using rightmost-pops sparse table and the theoretically
    better LCA table sparse table (in bold
    font).} \label{tlb:ISMQcmp} \tiny
  \begin{tabular}{|r|rrrrr|rrrrr|rrrrr|r|} 
    \hline
      & \multicolumn{5}{c|}{$L = 4$} & \multicolumn{5}{c|}{$L=8$} & \multicolumn{5}{c|}{$L=16$} &  \\ 
      \hline 
      \diagbox{$q$}{$p$} & 0\% & 25\% & 50\% & 75\% & 100\% & 0\% & 25\% & 50\% & 75\% & 100\% & 0\% & 25\% & 50\% & 75\% & 100\% & speedup\\
      \hline
      $0\%$ &
            {\bf 1.15} & {\bf 0.89} & {\bf 0.86} & {\bf 0.88} & {\bf 0.91}
          & {\bf 0.88} & {\bf 0.87} & {\bf 0.87} & {\bf 0.85} & {\bf 0.87}   
          & {\bf 1.02} & {\bf 1.00} & {\bf 0.99} & {\bf 1.00} & {\bf 1.02} & 1.56 \\
        & 1.30 & 1.05 & 1.05 & 1.05 & 1.05   & 1.32 & 1.32 & 1.32 & 1.32 & 1.32   & 1.35 & 1.34 & 1.34 & 1.34 & 1.34 & \\ \hline
      $20\%$ & 
            {\bf 0.98} & {\bf 0.95} & {\bf 0.98} & {\bf 0.99} & {\bf 0.96}   
          & {\bf 1.16} & {\bf 1.16} & {\bf 1.19} & {\bf 1.19} & {\bf 1.18}   
          & {\bf 1.24} & {\bf 1.28} & {\bf 1.31} & {\bf 1.25} & {\bf 1.21} & 1.26 \\
        & 1.09 & 1.09 & 1.09 & 1.09 & 1.09   & 1.40 & 1.40 & 1.40 & 1.40 & 1.40   & 1.53 & 1.53 & 1.53 & 1.53 & 1.53 & \\ \hline
      $40\%$ & 
            {\bf 1.01} & {\bf 1.01} & {\bf 1.02} & {\bf 1.02} & {\bf 0.99}   
          & {\bf 1.23} & {\bf 1.24} & {\bf 1.25} & {\bf 1.24} & {\bf 1.21}   
          & {\bf 1.39} & {\bf 1.43} & {\bf 1.45} & {\bf 1.31} & {\bf 1.26} & 1.28 \\
        & 1.13 & 1.13 & 1.13 & 1.13 & 1.12   & 1.46 & 1.46 & 1.47 & 1.47 & 1.45   & 1.62 & 1.62 & 1.62 & 1.62 & 1.61 & \\ \hline
      $60\%$ & 
            {\bf 1.01} & {\bf 1.02} & {\bf 1.04} & {\bf 1.02} & {\bf 0.99}   
          & {\bf 1.23} & {\bf 1.25} & {\bf 1.27} & {\bf 1.26} & {\bf 1.20}   
          & {\bf 1.44} & {\bf 1.48} & {\bf 1.51} & {\bf 1.34} & {\bf 1.26} & 1.28 \\
        & 1.13 & 1.14 & 1.15 & 1.14 & 1.12   & 1.47 & 1.48 & 1.50 & 1.49 & 1.45   & 1.63 & 1.64 & 1.66 & 1.65 & 1.61 & \\ \hline
      $80\%$ & 
            {\bf 0.99} & {\bf 1.01} & {\bf 1.03} & {\bf 1.01} & {\bf 0.97}   
          & {\bf 1.20} & {\bf 1.22} & {\bf 1.25} & {\bf 1.23} & {\bf 1.15}   
          & {\bf 1.38} & {\bf 1.43} & {\bf 1.49} & {\bf 1.31} & {\bf 1.19} & 1.31 \\
        & 1.11 & 1.12 & 1.15 & 1.12 & 1.10   & 1.44 & 1.46 & 1.49 & 1.47 & 1.41   & 1.59 & 1.61 & 1.65 & 1.63 & 1.56 & \\ \hline
      $100\%$ &
            {\bf 0.94} & {\bf 0.96} & {\bf 1.00} & {\bf 0.97} & {\bf 0.91}   
          & {\bf 0.96} & {\bf 1.01} & {\bf 1.01} & {\bf 0.99} & {\bf 1.03}   
          & {\bf 1.09} & {\bf 1.12} & {\bf 1.16} & {\bf 1.34} & {\bf 1.20} & 1.39 \\
        & 1.04 & 1.06 & 1.10 & 1.07 & 1.03   & 1.34 & 1.36 & 1.39 & 1.36 & 1.33   & 1.39 & 1.41 & 1.44 & 1.41 & 1.39 & \\ \hline
  \end{tabular}
\end{table}


我們觀察到機率 $p$ 將影響「偷看」技術中的改善效能比例。
當 $p$ 越接近 $1$,則偷看技術越能帶來更多的效能改善。
偷看操作也依賴塊的大小,而我們也知道右側棧出編碼能使用的塊大於 LCA 表的版本,
故改善情況的程度幅度不盡相同。

我們觀察到機率 $q$ 越接近 $1$,則右側棧出編碼與 LCA 表的效能差意越不明顯,
這可以明白 LCA 表在計算卡塔蘭索引值上的問題。
當 $q$ 越接近 $1$,卡塔蘭索引值將會重複計算更多次,
因此效能會越接近右側棧出編碼的版本。

在區間詢問的長度 (表~\ref{tlb:ISMQcmp} 中的 $L$) 影響到右側棧出的稀疏表,
卻不影響 LCA 表的效能,其原因在於算法~\ref{alg:cartesian64bits-query} 的內層迴圈,
將運行最多 $s$ 次,而實作採用 $s$ 為 $16$,將造成詢問長度增加而效能下降。
在另一方面查找 LCA 表皆為 $O(1)$ 操作,效能不受 $L$ 增長而改變。

\subsection{可變的間隙限制最長共同子序列}

我們接下來比較四種資料結構的組合於 VGLCS 的問題上,並統計它們的運行效能。
第一種採用 Peng 算法的循序版本,其採用並查集於兩個階段,
由於並查集只支持增長後綴最大值詢問,因此兩階段接使用後綴最大值詢問。
除了第一種外,另外三種組合皆以平行方式運行。
第二種 {\em DS-ST} 採用並查集和未分塊稀疏表,分別於算法的第一階段和第二階段。
而第三種 {\em DS-BST} 採用並查集和塊狀稀疏表,分別於算法的第一階段和第二階段。
同理,第四種 {\em BST2} 在兩階段皆使用塊狀稀疏表。
特別注意到,上述的塊狀稀疏表皆使用「右側棧出」編碼。

圖~\ref{fig:fig-parallel} 展示四種不同組合的資料結構對於不同問題大小的效能,
輸入字串使用字母集 $\{A, T, C, G\}$ 隨機產生,因為這是生物學常用的幾個字母。
我們從實驗中發現 {\em BST2} 遠比其他平行版本來得好。
圖~\ref{fig:fig-parallel-scala} 展示我們提出最好的 {\em BST2} 資料結構,
在 6 核心且支持超執行緒的環境下,提供至少 8 倍快的效能改善。

\begin{figure}[!thb]
  \centering
  \subfigure[四種不同組合的運行時間]{
    \includegraphics[width=0.45\linewidth]{\GraphicPath/fig-parallel-n.pdf}
    \label{fig:fig-parallel}
  }
  \subfigure[右側棧出編碼的規模伸縮性]{
    \includegraphics[width=0.45\linewidth]{\GraphicPath/fig-parallel-p.pdf}
    \label{fig:fig-parallel-scala}
  }
  \caption{實驗於 E5-2620 主機,其包含 2 個 6 核心處理器且支持超執行緒技術}
\end{figure}
