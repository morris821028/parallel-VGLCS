\section{Introduction} %Introduction
\label{sec:Introduction}

% This is a topic sentence by itself.
% say something about sequence alignment

The {\em longest common subsequence} (LCS)~\cite{Hirschberg1975ALS} is
a famous problem in string processing.  For example, the {\tt diff}
utility show the difference between texts by LCS, and the revison
control systems such as SVN/Git use LCS to reconciling multiple
changes.  In bioinformatics, the best-known application of the LCS
problem is the sequence alignment~\cite{mount2001bioinformatics,
Ann2010EfficientAF}, which identify the region of similarity between
the sequences of DNA, RND, or protein.


\iffalse 最長共同子序列 (\emph{longest common subsequence}, LCS) 廣泛
地使用在各個應用上。在多核心平台下,大多數的研究專注於如何高效率地在波
前平行 (wavefront parallelism),而 Jiaoyun Yang ~\cite{jiaoyun} 提出的
論文中改變一般的 LCS 遞迴定義以得到更好快取使用率。在這篇論文中,針對
在 Iliopoulos 和 Rahman ~\cite{iliopoulos} 提及的約束條件下的 LCS 問題
使用相關的想法來改善效能。\fi

% give citation

Iliopoulos and Rahman~\cite{Rahman2006AlgorithmsFC} introduced many
contrainted versions of LCS.  For example, a {\em fixed gap LCS}
(FGLCS) requires that the distance between consecutive characers in
the LCS is at most $k + 1$ characters away.  Fixed gap LCS can be
solved in $O(nm)$, where $n$ and $m$ are the lengths of the two input
strings~\cite{citation}.  On the other hand, a {\em variable   gap
LCS} (VGLCS) requires that each charachter has a {\em gap} value and
two consecutive characters in LCS must be with distance of the gap of
the latter character plus 1.  One can think of the FGLCS as a special
case of VGLCS in which the gap values of all characters are $k$.

\begin{figure}[!thb]
  \centering
  \includegraphics[width=0.8\linewidth]{graphics/fig-VGLCSex.pdf}
  \includegraphics[width=0.8\linewidth]{graphics/fig-VGLCSex2.pdf}
  \caption{An example VGLCS example}    \label{fig:VGLCSex}
\end{figure}

We use an exmaple to illustrte the gap function and VGLCS.  Let string
$A$ be {\tt GCGCAATG} with gap values $(3, 1, 1, 2, 0, 0, 2, 1)$, and
let string $B$ be {\tt GCCCTAGCG} with gap values $(2, 0, 3, 2, 0, 1,
2, 0, 1)$.  Please refer to Figure~\ref{fig:VGLCSex} for an
illustration.  Now the LCS $GCCT$ is a VGLCS because every character
in the LCS can find its predecessor in the LCS with distance at most
its gap value plus 1.

This paper focuses on finding efficient parallel algorithm to solve
VGLS.  Peng~\cite{Peng2011TheLC} gives a $O(nm \alpha(n))$ algorithm
that is easy to implement and a asymptotically better $O(nm)$
algorithm.  Then, we propose our $O(nm)$ algorithm which is easy to
implement and run efficiently in parallel environment.

% then we???

The parallelization of LCS on most multi-core platforms focuses on {\em
wavefront} parallelism.  The wavefront parallelism is motivated by the
recursive solution of LCS.  For example, Jiaoyun
Yang~\cite{Yang2010AnEP} introduced a new formula to exploit more
cache performance.  Our algorithm use more powerful sparse table
instead of the disjoint set in the Peng's serial algorithm and get
better cache performance.

% Then we will ???

\iffalse 在約束條件下的 LCS 中,如 \emph{fixed gap LCS } (FGLCS)要求任
兩個挑選的距離在相對應的另一個字串中相等,同時距離最大為 $k+1$,可在時
間複雜度在 $O(nm)$ 內解決,其中 $n$, $m$ 分別為兩個輸入的字串長度。我
們將在這篇論文針對 \emph{variable gap LCS} (VGLCS) 進行探討。在 VGLCS
中,對各個不同的位置提供約束限制,如目前給定兩個字串 $A =
\tt{GCGCAATG}$, $B = \tt{GCCCTAGCG}$,各自的約束限制為 $G_A = [3, 1,
  1, 2, 0, 0, 2, 1]$ 和 $G_B = [2, 0, 3, 2, 0, 1, 2, 0, 1]$,其中
$G_A(i)$ 表示當挑選第 $i$ 個位置時,與前一個挑選的位置最多差
$G_A(i)+1$,同理 $G_B(i)$;我們可以得到兩組 VGLCS 的解
$\tt{G..C..C..A}$ 和 $\tt{G..C..C..T}$,挑選的方式如圖
~\ref{fig:VGLCSex}。在 Yung-Hsing Peng ~\cite{yunghsing} 的論文已對
VGLCS 提出易於實作的 $O(nm \alpha(n))$ 和理論 $O(nm)$ 的解法。\fi

% Use reference in all section numbering

The remainder of the paper is organized as follows.  In Section
\ref{sec:parallelVGLCS}, we present the parallel algorithm to solve
VGLCS problem.  In Section \ref{sec:parallelRMQ} and
\ref{sec:parallelIRMQ}, we present a new algorithm that be easily
parallelized with a time complexity $O(nm)$, which is better than
previous works.  In Section \ref{sec:Implementation} and
\ref{sec:Experiment}, we describe our optimized implementation and the
result of our experiments. Section xx conclude this paper with lessons
learned and posiible future works.

\iffalse 這一篇論文,我們將在第二 \ref{sec:parallelVGLCS} 節部分將
Yung-Hsing Peng ~\cite{yunghsing} 提出的算法進行平行化。在第三節
~\ref{sec:parallelRMQ},在理論分析上提供易平行且時間複雜度 $O(nm)$ 的
設計。在第四節 ~\ref{sec:Implementation},我們將藉由快取忘卻
(cache-oblivious) 技術,在實作上提供更好的效能。最後,我們總結實驗結果
與理論實務上的差異。\fi

