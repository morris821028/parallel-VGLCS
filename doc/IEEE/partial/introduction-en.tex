\section{Introduction} %Introduction
\label{sec:Introduction}

The \emph{longest common subsequence} (LCS) problem applied many
products and fields widely.  In the multi-core platform, most studies
focus on the wavefront parallelism. Motivated by the definition of
recursion in LCS, Jiaoyun Yang introduced a new formula to exploit
more cache performance.  Here, we use the similar idea to improve LCS
with variable constraints, which refer in Iliopoulos' study. 

\iffalse
最長共同子序列 (\emph{longest common subsequence}, LCS)廣泛地使用在各個應用上。
在多核心平台下,大多數的研究專注於如何高效率地在波前平行 (wavefront parallelism),
而 Jiaoyun Yang ~\cite{jiaoyun} 提出的論文中改變一般的 LCS 遞迴定義以得到更好快取使用率。
在這篇論文中,針對在 Iliopoulos ~\cite{iliopoulos} 提及的約束條件下的 LCS 問題使用相關的想法來改善效能。
\fi

There are many kinds of contraint.  For example, \emph{fixed gap LCS}
(FGLCS) require the distance between two consecutive matches limited
at most $k+1$.  It can be solved in $O(nm)$, which $n, \; m$ is the
length of input strings. This paper foucus on \emph{variable gap LCS}
(VGLS).  The VGLCS require the distance between two consecutive
matched limited at most $G_i$, which $G_i$ is the value of the
position $i$ in input array. For example, two protein sequences $A =
\tt{GCGCAATG}$ and $B = \tt{GCCCTAGCG}$, and two gap functions $G_A =
[3, 1, 1, 2, 0, 0, 2, 1]$ and $G_B = [2, 0, 3, 2, 0, 1, 2, 0, 1]$.
Figure ~\ref{fig:VGLCSex} presents an example to show that tow motifs
$\tt{G..C..C..A}$ and $\tt{G..C..C..T}$.  In previous study, Yung-
Hsing Peng \cite{Yang2010AnEP} introduced the $O(nm)$ solution for the
VGLCS problem.

\iffalse
在約束條件下的 LCS 中,如 \emph{fixed gap LCS } (FGLCS)要求任兩個挑選的距離在相對應的另一個字串中相等,
同時距離最大為 $k+1$,可在時間複雜度在 $O(nm)$ 內解決,其中 $n$, $m$ 分別為兩個輸入的字串長度。
我們將在這篇論文針對 \emph{variable gap LCS} (VGLCS) 進行探討。
在 VGLCS 中,對各個不同的位置提供約束限制,如目前給定兩個字串 $A = \tt{GCGCAATG}$, 
$B = \tt{GCCCTAGCG}$,各自的約束限制為 $G_A = [3, 1, 1, 2, 0, 0, 2, 1]$ 
和 $G_B = [2, 0, 3, 2, 0, 1, 2, 0, 1]$,其中 $G_A(i)$ 表示當挑選第 $i$ 個位置時,與前一個挑選的位置最多差 $G_A(i)+1$,同理 $G_B(i)$;
我們可以得到兩組 VGLCS 的解 $\tt{GCCA}$ 和 $\tt{GCCT}$,挑選的方式如圖 ~\ref{fig:VGLCSex}。
在 Yung-Hsing Peng ~\cite{yunghsing} 的論文已對 VGLCS 提出易於實作的 $O(nm \alpha(n))$ 和理論 $O(nm)$ 的解法。
\fi

The remainder of the paper is organized as follows. In section 2, we
present the parallel algorithm to solve VGLCS problem. In section 3,
we present a new algorithm, which parallels easily and time complexity
$O(nm)$ better than the previous study. In section 4 to 5, we provide
the optimized implementation and the result of experiments. The last
section has a brief conclusion in this paper.

\iffalse
這一篇論文,我們將在第二 \ref{sec:parallelSerial} 節部分將 Yung-Hsing Peng ~\cite{yunghsing} 提出的算法進行平行化。
在第三節 ~\ref{sec:parallelRMQ},在理論分析上提供易平行且時間複雜度 $O(nm)$ 的設計。
在第四節 ~\ref{sec:Implementation},我們將藉由快取忘卻 (cache-oblivious) 技術,在實作上提供更好的效能。最後,我們總結實驗結果與理論實務上的差異。
\fi

\begin{figure}[!thb]
  \centering
  \includegraphics[width=0.8\linewidth]{graphics/fig-VGLCSex.pdf}
  \includegraphics[width=0.8\linewidth]{graphics/fig-VGLCSex2.pdf}

  \caption{   An example for illustrating the VGLCS, which has two
protein sequences $A = \tt{GCGCAATG}$ and $B = \tt{GCCCTAGCG}$, and
two gap functions $G_A = [3, 1, 1, 2, 0, 0, 2, 1]$ and $G_B = [2, 0,
3, 2, 0, 1, 2, 0, 1]$.   }
  
  \label{fig:VGLCSex}
\end{figure}