\section{Introduction} %Introduction
\label{sec:Introduction}

The {\em longest common subsequence} (LCS)~\cite{Hirschberg1975ALS} is a
famous problem in string processing.  For example, the {\tt diff}
utility show the difference between texts by finding their LCS. Revision
control systems like SVN and Git use LCS to reconciling multiple
changes.  In bioinformatics, the best-known application of the LCS
problem is the sequence alignment~\cite{mount2001bioinformatics,
Ann2010EfficientAF}, which identifies the region of similarity between
the sequences of DNA, RND, or protein.

Iliopoulos and Rahman~\cite{Rahman2006AlgorithmsFC} introduced many
constrained versions of LCS.  For example, a {\em fixed gap LCS} (FGLCS)
requires that the distance between consecutive characters in the LCS is
{\em at most} $k + 1$ characters away.  A fixed gap LCS can be found in
time $O(nm)$, where $n$ and $m$ are the lengths of the two input
strings~\cite{Rahman2006AlgorithmsFC}.  On the other hand, a {\em
variable gap LCS} (VGLCS) requires that each character has a {\em gap}
value and two consecutive characters in LCS must be with distance of the
gap of the {\em latter} character {\em plus} 1.  One can think of the
fixed gap LCS as a special case of variable gap LCS in which the gap
values of all characters are $k$.

\begin{figure}[!thb]
  \centering
  \includegraphics[width=0.45\linewidth]{\GraphicPath/fig-VGLCSex.pdf}
  \includegraphics[width=0.45\linewidth]{\GraphicPath/fig-VGLCSex2.pdf}
  \caption{A VGLCS example} \label{fig:VGLCSex}
\end{figure}

We use an example to illustrate the gap function and VGLCS.  Let string
$A$ be {\tt GCGCAATG} with gap values $(3, 1, 1, 2, 0, 0, 2, 1)$, and
let string $B$ be {\tt GCCCTAGCG} with gap values $(2, 0, 3, 2, 0, 1, 2,
0, 1)$.  Please refer to Figure~\ref{fig:VGLCSex} for an illustration.
Now, the LCS $GCCT$ is a VGLCS because every character in the LCS can
find its predecessor in the LCS with distance at most its gap value plus
1.

This paper focuses on {\em efficient parallel algorithms} that find
VGLCS.  Peng~\cite{Peng2011TheLC} gives a $O(nm \alpha(n))$ algorithm
that is easy to implement and an asymptotically better $O(nm)$
algorithm, where $\alpha$ is the inverse of Ackermainn's
function~\cite{Banachowski1980ACT}.  In this paper, we propose our
$O(nm)$ algorithm which is {\em easy} to implement and runs {\em
efficiently} in a {\em parallel} environment.

The parallelization of LCS on most multi-core platforms focuses on
{\em wavefront} parallelism.  The wavefront parallelism is motivated
by the recursive solution of LCS.  For example,
Yang~\cite{Yang2010AnEP} introduced a new formulation to exploit more
cache performance.  Our algorithm uses a more powerful {\em sparse
  table} instead of the disjoint set in the Peng's serial algorithm
and achieves better cache performance.

The remainder of the paper is organized as follows.  In
Section~\ref{sec:parallelVGLCS}, we present previous parallel algorithms
for finding VGLCS.  In Section~\ref{sec:parallelRMQ} and \ref{sec:QIUD},
we present our algorithm that is easy to parallelize, and has a time
complexity $O(nm)$, which is better than previous works.  In
Section~\ref{sec:Implementation} and \ref{sec:Experiment}, we describe
our optimized implementation and the results of our experiments.
Section~\ref{sec:Conclusion} conclude this paper with lessons learned
and possible future works.

