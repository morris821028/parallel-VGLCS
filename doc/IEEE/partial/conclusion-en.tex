\section{Conclusion}
\label{sec:Conclusion}

Our parallel VGLCS algorithm run in $\theta(n \log n)$ time, and we use
sparse table to solve incremental suffix maximum query problem.  We
provide the amortized sparse table which supports incremental range
maximum query problem.  Finally, we can use them to solve VGLCS problem
in $\theta(nm)$ time, and also use $\theta(nm)$ time to solve variable
range gapped LCS problem which is hard than variable gapped LCS problem.

In practice, we presented easy-to-implements data structure for
constant-time IRMQ-retrieval and provide the extra dynamic programming to
reduce the computing boundary, compressed the space to reduce cache
miss, and the auxiliary prefix/suffix array to avoid building
Cartesian trees. Finally, the CORMQ-opt run $2.35 \times$ faster than
the origin the algorithm in the parallel environment.

In the incremental range maximum query, we provide the theoretical
$\theta(n)$ - amortized $O(1)$ algorithm by lexicographical order
encoding.

\iffalse
我們修改 VGLCS 的序列算法,將其平行化於 $\theta(n \log n)$ 時間內,
並以稀疏表實作 ISMQ 問題。提出的稀疏表能解決比 VGLCS 更困難的 Variable Range Gapped LCS,
致使 VIGLCS 可在時間複雜度 $\theta(nm)$ 被解決。

在實務上,我們提供以動態規劃減少計算量,以及使用空間壓縮降低快取未中的策略,
最終平行 RMQ 獲取 $2.35 \times$ 倍的加速;在增長區間最大值詢問 (IRMQ) 問題中,
以字典順序的編碼策略,提出理論 $\theta(n)$ -- amortized $\theta(1)$  的算法。
\fi