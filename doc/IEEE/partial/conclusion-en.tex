\ifdefined\MasterThesis
\chapter{Conclusion}
\else
\section{Conclusion}
\fi
\label{sec:Conclusion}

In this paper, we propose two stages parallel algorithm instead of the
wavefront method, then it shrinks the length of the critical path of
intuitive parallel algorithm.  In the two stages, we introduce the
sparse table as the main data structure of the variable gapped longest
common subsequence.  It can run efficiently because it improves the
thread synchronize and workload imbalance compare to disjoint set
version.  Simultaneously, we presented east-to-implements linear sparse
table as compressed sparse table for our VGLCS problem which run in
$O(n^2 s / p + n \times \max(\log n, s))$.

In the parallel environment, we provide the parallel building Catalan
index algorithm for the linear time Fischer's sparse table algorithm.
The parallel building Catalan index algorithm use lexicographical order
to encode each binary search tree.  Then, we adjust the Cartesian tree
building algorithm to satisfy our encoding with amortized $O(1)$ insert
operation.  Therefore, the VGLCS problem can be solved in theorem $O(n^2
/ p + n \log n)$ efficiently.

The incremental ranged maximum query run in amortized $O(1)$ by our
sparse table, and it is more powerful than the incremental suffix
maximum query. Therefore, the time complexity of the variable ranged
gapped longest common subsequence problem is $O(n^2 / p + n \log n)$ as
same as VGLCS problem.
