\section{Parallel Range Maximum Query}
\label{sec:parallelRMQ}

\subsection{Background}

Even though we parallel origin serial algorithm successfully, the
theoretical time complexity of VGLCS algorithm is limited by the range
maximum/minimum query in parallel.  In VGLCS problem, each stage has
$n$ elements and $n$ numbers of the range query.  We should let pre-
processing and query time minimize.  The most of the tree structures
cannot show efficient performance in both pre-processing and query
time.  Some offline algorithm is too hard to parallel.  In this
section, we provide the solution to get better performance in both
pre-processing and query stage.

\iffalse
縱使我們已能很好地平行化原本的序列算法,在理論複雜度上受限於平行下的區間極值查詢 (Range Minimum/Maximum Query, RMQ)。
在這個應用中,每一階段有 $n$ 個元素和 $n$ 個區間詢問。
這樣的條件下,大部分樹狀結構難以在前處理過程和每次詢問皆達到最好效能。
對於 $O(n)$ -- $O(1)$ 操作的離線區間詢問無法提供平行。
再接續的小節中,我們將提出在兼顧建表、插入和查找的數據結構與算法。
\fi

\subsection{Compressed Cartesian Tree}

In Fischer ~\cite{Fischer2006TheoreticalAP} paper, the $O(n)$ --
$O(1)$ algorithm is corrponding by $C_s$, where $C_s$ is the $s$-th
Catalan number  and $C_s = \frac{1}{s+1}\binom{2s}{s} =
O(\frac{4^s}{s^{1.5}})$ to build look-up table.  When we c hoose $s =
\frac{1}{4} \log n$ as block size, the space complexity is $O(s^2
\frac{4^s}{s^{1.5}}) = O(n)$, and time complexity is $O(n)$.  Each
range query will be split into 4 parts, 2 super-block queries and 2
in-block queries.  It need to 4 time memory access.  We give a example
in the figure ~\ref{fig:interval-decomposition}.  In the offline RMQ
problem, it has the theoretical algorithm which run in $O(n)$ --
$O(1)$ time.  When $n$ is large, four-time memory access caused the
serious cache miss. In order to improve cache miss, Demaine introduced
the cache-oblivious the algorithm \cite{Demaine2009OnCT} in Cartesian
tree \cite{Vuillemin1980AUL}.

\iffalse
在 Fischer \cite{fischer} 的論文中,
根據卡塔蘭數 $\frac{1}{s+1}\binom{2s}{s} = O(\frac{4^s}{s^{1.5}})$ 建立查找表 (lookup-table),
其中選擇 $s = \frac{1}{4} \log n$ 時,空間複雜度 $O(s^2 \frac{4^s}{s^{1.5}}) = o(n)$ 且建表複雜度 $o(n)$。
每一個區間詢問將會拆成 2 個 super-block 和 2 個 in-block 詢問,共計需要 4 次的記憶體存取。
在理論分析上,離線 RMQ 問題可在 $\theta(n)$ -- $\theta(1)$ 時間內解決任一詢問。
當 $n$ 越大時,這 4 次的記憶體存取會遭遇到嚴重的快取未中 (cache miss),
在 Demaine ~\cite{demaine} 的論文中,發展出快取忘卻 (cache oblivious) 形式的查找方案,
降低在離線版本中的 in-block 詢問產生的快取未中。
\fi

In above technology, we parallel RMQ problem by Fischer's idea and get
time complexity $O(n / p + \log n)$ -- $O(1)$ algorithm.  We also
combine compression skill from Demaine's paper. It reduces cache-miss
and run in ideal complexity.

\iffalse
在上述的技術中,我們可以藉由 Fischer 提出的方案平行化 RMQ 至 $O(n / p + \log n)$ -- $O(1)$,使用 Demaine 提供的技巧壓縮空間使用量,降低快取未中以提升運行效能。這裡我們挑選固定長度的壓縮方案 $s = 16$,其能解決序列長度為 $n = 2^{64}$ 的區間查找,將 16 個整數壓縮成一棵笛卡爾樹。在第 $i$ 次插入時,左旋的次數 $l_i$,每次操作皆符合 $\sum_{i=1}^{n} l_i < i$。
\fi

We pick the fixed length $s = 16$, which can solve $n = 2^{64}$ one-
dimension range maximum query.  When inserting $i$-th elements, the
number of $i$-th insertion satisfy $\sum_{i=1}^{n} l_i < i$ where $0
\le l_i < s$ is the number of nodes remove from the rightmost path of
Cartesian tree.  Because all $l_i$ is small than 16, it can present in
4-bit integer.  Due to above property of Cartesian tree, we merger 16
4-bit integers into a 64-bit integer to present a Caartesian tree.
The compressed algorithm ~\ref{alg:cartesian-to-64bits} run in $O(s)$.

Finally, the appropriate size can compress the usage of space to
reduce cache miss and also show better performance in modern 64-bit
register.  We modify Demaine's range query algorithm as the algorithm
~\ref{alg:cartesian64bits-query}.

\iffalse
因所有 $l_i < 16$,使得每個 $l_i$ 可用 4-bit 表示之,
整體便可用 64-bit 長整數表示一棵笛卡爾樹的狀態。
為了現在常見的 64-byte 快取列 (cache line) 和 64-bit 暫存器 (register) 考量,
我們選用合適的大小進行測試,不僅壓縮空間使用量,同時也減少快取未中的問題。
最後,我們得到壓縮算法 \ref{alg:cartesian-to-64bits},其相對應的區間查找算法,
根據 Demaine \cite{demaine} 進行修改,得到壓縮下的詢問算法 \ref{alg:cartesian64bits-query}。
\fi

\begin{algorithm*}
  \caption{Transfer Cartesian Tree to 64-bits with 8 integers}
  \label{alg:cartesian-to-64bits}
  \begin{algorithmic}[1]
    \Require
      $A[1 \cdots 16]$: 16 integers store in array
    \Ensure 
      $\textit{tmask}$: compress Cartesian tree into 64-bits integer
      \State $\tt{LOGS} = 4$
      \State $\tt{POWS} = 2^{\tt{LOGS}}$
      \State int $D[POWS+1]$, $Dp = 0$;
      \State uint64\_t $tmask$ = $0$
      \State $D[0]$ = SHRT\_MAX
      \For{$i = 1$ to $\tt{POWS}$} 
        \State $v = A[i]$;
        \State $cnt = 0$;
        \While{$D[Dp] < v$}
          \State $Dp = Dp-1$
          \State $cnt = cnt + 1$
        \EndWhile
        \State $Dp = Dp+1$
        \State $D[Dp] = v$
        \State $tmask = tmask | ((cnt)<<((i-1)<<2))$
      \EndFor
      \State return $tmask$
  \end{algorithmic}
\end{algorithm*}

\begin{algorithm}[!thb]
  \caption{Range Minimum Query in 64-bits Cartesian Tree}
  \label{alg:cartesian64bits-query}
  \begin{algorithmic}[1]
    \Require
      $\textit{tmask}$: 64-bits Cartesian tree;
      $[l, r]$: query range
    \Ensure 
      $\textit{minIdx}$: the index of the minimum value in interval
    \State $\textit{minIdx} \gets l, \; x \gets 0$;
    \For{$l \gets l+1$ to $r$}
      \State $x \gets x+1 - ((\textit{tmask} \gg (l \ll 2)) \mathrel{\&} 15)$
      \If{$x \le 0$}
        \State $\textit{minIdx} \gets l$
        \State $x \gets 0$
      \EndIf
    \EndFor
    \State return $\textit{minIdx}$
  \end{algorithmic}
\end{algorithm}

In VGLCS problem, above algorithm provide compression skill to reduce
cache miss, but increase the time complexity.  The pre-processing
spend $O(n)$ time, and single query spends $O(s)$ time in RMQ.
Totally, time complexity is $O(n^2 \; s / p + n \max(\log n, s))$.

\iffalse
回到 VGLCS 的應用中,上述算法使用壓縮方式降低快取未中。
我們可以使用上述的算法取代原先的并查集,建表的時間複雜度為 $O(n)$,
單一查詢的時間複雜度為 $O(s)$。
整體的時間複雜度為 $O(n^2 \; s / p + n \max(\log n, s))$。
\fi