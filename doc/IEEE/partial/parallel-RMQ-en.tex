\section{Ranged Maximum Query} \label{sec:parallelRMQ}

In this section we will describe our approach to address the
challenges in the second stage of Algorithm~\ref{alg:parallel-VGLCS},
i.e., an efficient {\em ranged maximum query} on the sparse table $T$.
The ranged maximum query problem is more complicated than the previous
incremental suffix maximum query problem.  Again a {\tt make}
operation creates an empty array $A$, an {\tt append(V)} operation
appends a value $V$ to the end of an array $A$.  Finally, a {\tt
  query(L, R)} operation finds the {\em maximum} value among the
$L$-th value to the $R$-th value of an array $A$.  One can think of
the suffix maximum query as a special case of the ranged maximum
query.

%% We can use a parallel sparse table implementation to answer ranged
%% maximum queries with $p$ processors, so that it requires $O(n \log n /
%% p + \log n)$ in preprocessing, and takes only $O(1)$ time to answer a
%% query.  On the other hand, it is difficult to efficiently parallelize
%% the querying on tree-like data structures, e.g., disjoint sets.

\subsection{Blocked Sparse Table}

\begin{figure}[!thb]
  \centering 
    \includegraphics[width=0.9\linewidth]{\GraphicPath/fig-interval-decomposition.pdf}
    \includegraphics[width=0.9\linewidth]{\GraphicPath/fig-sparse-table.pdf}
  \caption{A Sparse Table}
  \label{fig:block-interval-decomposition}
\end{figure}

We improve the efficiency of our parallel VGLCS algorithm by a {\em
  blocked} sparse table proposed by
Fischer~\cite{Fischer2006TheoreticalAP}.  The blocked approach first
partition the data into $s$ blocks, then it computes the maximum of
each block, and compute a sparse table $T_s$ for these maximums.

The blocked approach answers a ranged maximum query as follows.  We
consider two types of queries -- {\em super block} query and {\em
  in-block} query.  A super block query queries the answer for
consecutive blocks, and a in-block query queries a segment {\em
  within} a block.  It is easy to see that we can answer a super block
query by querying $T_s$ at most {\em twice}.  We can also answer an
in-block queries answered by a {\em single} lookup into an {\em least
  common ancestor table}.  We will provide more details on this later.
Since we can split {\em any} ranged query into at most {\em two}
queries into $T_s$ and {\em two} in-block queries, we need at most
{\em four} memory access to answer any ranged maximum query.

Fischer's algorithm~\cite{Fischer2006TheoreticalAP} scans through the
data within a block, and places the data into a {\em Cartesian tree}.
Each node of the Cartesian tree has a data and the index of this data
in the block.  One can think of this Cartesian tree as a heap where
the data are in {\em heap order}, and the indexes of the data are in
{\em sorted binary search tree order}.  Please refer to
Figure~\ref{fig:ancesstor-cartesian} for an illustration.  As a result
the answer to the in-block ranged maximum query from the $i$-th to the
$j$-th element of a block is located at their {\em least common
  ancestor} in the tree.

\begin{figure}[htbp]   
  \centering
  \includegraphics[width=\linewidth]{\GraphicPath/fig-interval-cartesian.pdf}
  \caption{Least common ancestor tables}
  \label{fig:ancesstor-cartesian}
\end{figure}

Fischer's algorithm computes a {\em least common ancestor table} for
every block.  After scanning the data in a block, the algorithm will
build a Cartesian tree, and is able to answer the in-block ranged
maximum query from the $i$-th to the $j$-th element.  Let the index of
this least common ancestor be $k$. then the algorithm add the key
value pairs $((i, j), k)$ into the least common ancestor table of this
block.  Please refer to Figure~\ref{fig:ancesstor-cartesian} for an
illustration.  That is, one can think of the least common ancestor
table as a mapping table from the in-block query $(i, j)$ to its
answer $k$.  Note that the algorithm does {\em not} maintain the value
the $k$-th element.  Instead it keeps the {\em index}, i.e., $k$, of
the least common ancestor so that two blocks with the {\em same
  relative key order} can {\em share} a least common ancestor table.
For example, the first three blocks in
Figure~\ref{fig:ancesstor-cartesian} can share the same least common
ancestor table because they have their same Cartesian tree.
Consequently an in-block ranged maximum query $(1, 3)$ to {\em any} of
these three blocks will return the {\em same} answer $2$.

The main idea of Fischer's algorithm is to compute an ancestor table
for each block, and answer an in-block query directly by looking into
its ancestor table.  It is easy to see that there are $C_s$ different
Cartesian trees, where $C_s$ is number of different binary trees of
$s$ nodes.  It is also easy to see that each block can be identified
by the shape of its Cartesian tree, so it can be represented by an
index.  For ease of notation we will refer to this index as its {\em
  Catalan index}.  By knowing the Catalan index of a block, we can
answer a in-block ranged maximum query by looking into its
corresponding ancestor table.  Please refer to
Figure~\ref{fig:ancesstor-cartesian} for an illustration.

Fischer's algorithm~\cite{Fischer2006TheoreticalAP} builds ancestor
tables by choosing $s = \frac{1}{4} \log n$ as the block size for
performance reason.  Recall that $C_s = \frac{1}{s+1}\binom{2s}{s} =
O(\frac{4^s}{s^{1.5}})$.  As a result the preprocessing time is
$O(n)$, and the space complexity is $O(s^2 \frac{4^s}{s^{1.5}}) =
O(n)$.  That is, a sequential version requires $O(n)$ time in
preprocessing, and $O(1)$ time in answering a query, and both
preprocessing time and query answering are the best.

Fischer's algorithm causes {\em serious cache miss} when the number of
data is large.  Fischer's algorithm will construct least common
ancestor tables for blocks in an on-demand manner.  When the algorithm
finds that the corresponding least common ancestor table is {\em not}
present in memory, it will build it in memory, which will be cached.
This process will repeat the number of blocks times, and causes
serious cache misses.

In order to reduce cache miss, Demaine~\cite{Demaine2009OnCT} proposed
{\em cache-aware} operations on Cartesian
tree~\cite{Vuillemin1980AUL}.  Demaine's algorithm skips the checking
for existing least common ancestor table, and builds an least common
ancestor table for {\em every} block.  In addition, Demaine's
algorithm uses a binary string of length $2s$ as an identifier of a
block to its Cartesian tree.  The binary string is encoded in such a
way that one can answer an in-block ranged maximum query by examining
this binary string.  However, this operation requires counting the
number of 1's {\em between} the last two 0's, which is {\em hard} to
implement efficiently in a modern computer.

% We replace lookup operation to naive operation.  The naive operation
% is find the maximum value by comparing each element on the compressed
% data. On the other hand, the lookup operation find the index of
% maximum value from the ancestor table. When loading a element from index
% table, it also bring some useless data to caches.   In order to use
% cache efficiently, the naive operation is better than the lookup
% operation because the access pattern is almost one by one in our VGLCS
% algorithm.

\subsection{Compressed Cartesian Tree} \label{sec:cct}

We propose a new encoding for blocks called {\em compressed Cartesian
  tree}, instead of the binary string by
Demaine~\cite{Demaine2009OnCT}, in order to improve performance.  The
compressed sparse table is inspired by Fischer's sparse table and
Cartesian tree.

The compressed sparse table works by keeping only the {\rm rightmost}
path of the Cartesian tree in a {\em stack}.  Please refer to
Figure~\ref{fig:interval-cartesian} for an illustration.  It is easy
to see that when we add the $i$-th data $a_i$ into the Cartesian tree,
we need to {\em pop} the data in the stack, which stores the rightmost
path of the Cartesian tree, that are {\em smaller} than $a_i$.  We
keep popping data until the top of stack is no less than $a_i$, then
we push $a_i$ into the stack.  Let $t_i$ be the number of nodes that
need to be popped, and it is easy to see that $0 \le t_i < s$, where
$s$ is the block size.  

Consider the example in Figure~\ref{fig:interval-cartesian}.  When we
insert $a_1 = 0$, we just insert it into the stack since $a_0 = 1$,
and no data is popped, so $t_1$ is $0$.  When we insert $a_2 = 4$ we
need to pop both $a_0$ and $a_1$ out of the stack, since they are
smaller than $a_2$, so $t_2$ is $2$.  It is easy to see that the
contents of the stack is exactly the rightmost path of the Cartesian
tree.

\begin{figure}[!thb]
  \centering
  \includegraphics[width=\linewidth]{\GraphicPath/fig-cartesian-encoding-stack.pdf}
  \caption{Right most path in stack}
  \label{fig:interval-cartesian}
\end{figure}


The key observation of compressed Cartesian tree is that we can use
$t_i$'s to {\em implicitly identify} the Cartesian tree of this block
of data, so that we can answer in-block ranged queries simply by
examining these $t_i$'s.  More details on how to answer queries with
these $t_i$'s will be described later.  Now since all $t_i$'s are
small than 16, we can represent each of them as a 4-bit integer.  We
then merge 16 of these 4-bit integers into a 64-bit integer to present
a Cartesian tree for a block.  The pseudo code on how to build a 64
bit integer to represent a block of 16 data is in
Algorithm~\ref{alg:cartesian-to-64bits}, which runs in time
$O(s)$. where $s$ is the block size.  Note that all the operations,
e.g., shift, addition, subtraction, in
Algorithm~\ref{alg:cartesian-to-64bits} map directly to machine
instructions and are straightforward to implement.

\begin{algorithm}
\SetAlgoNoLine
\KwIn{$A[1 .. 16]$: input data block}
\KwOut{$t$: a 64 bit right-most-pops encoding of $A$}

$\tt{LOGS} \gets 4$, $\tt{POWS} \gets 2^{\tt{LOGS}}$ \;
Create an array $D$ of size $\tt{POWS}+1$ \;
$p \gets 0$, $D[0] \gets \infty$ \;

$t \gets 0$ \;
\For{$i \gets 1$ to $\tt{POWS}$} {
  $v \gets A[i], \; c \gets 0$\;
  \While{$D[p] < v$} {
    $p \gets p - 1$, $c \gets c + 1$ \;
  }
  $p \gets p + 1$ \;
  $D[p] \gets v$ \;
  $t \gets t \mathrel{|} (c \ll ((i-1) \ll 2))$ \;
}
return $t$ \;

  \caption{Encode a data block of sixteen data with right-most-pops
    encoding into a 64-bits integer.}
  \label{alg:cartesian-to-64bits}
\end{algorithm}


We answer an in-block maximum query that ranges from $l$ to $r$ with
these $t_i$'s as follows.  We maintain the number of times data are
{\em popped} from the stack in a variable $x$, and initialize $x$ to
0.  We then loop through $t_l$ to $t_r$ and let the index run from $l$
to $r$.  Every iteration adds 1 to $x$ then subtracts $t_j$ from $x$.
We need to remember the index when $x$ becomes smaller or equal to 0.
We report the index $j$, the {\em last} index when $x$ becomes smaller
or equal to 0, as the answer.  The pseudo code of the query answering
algorithm is in Algorithm~\ref{alg:cartesian64bits-query}.  The
correctness proof of Algorithm~\ref{alg:cartesian64bits-query} is in
Theorem~\ref{thm:correctness}.  Again all the operations of
Algorithm~\ref{alg:cartesian64bits-query} map directly to machine
instructions so that unlike Demaine's algorithm,
Algorithm~\ref{alg:cartesian64bits-query} is extremely intuitive to
implement.

%\begin{eqnarray}  \label{fun:rmq-stack-pos}
	\text{stack position: } \textit{pos}_i = i - \sum_{j=0}^{i} t_j && t_i \text{ is \#pop operation in $i$-th insertion}
\end{eqnarray}

\begin{eqnarray}  \label{fun:rmq-reduction}
	\max(A[l, r]) = \max(A[k, r])
		&& \text{if } k \in [l, r]\text{ and } \forall \alpha \in [l, r): a_\alpha \le a_k
\end{eqnarray}

\begin{eqnarray}  \label{fun:rmq-reduction-compute}
	\textit{pos}_k \le \textit{pos}_l 
		&\Rightarrow& k - \sum_{j=0}^{k} t_j \le l - \sum_{j=0}^{l} t_j \\
		&\Rightarrow& (k-l) - \sum_{j=l+1}^{k} t_j \le 0
\end{eqnarray}



\begin{algorithm}[H]
\SetAlgoNoLine
\LinesNumbered
\KwIn{$\textit{tmask}$: 64-bits Cartesian tree; $[l, r]$: query range}
\KwOut{$\textit{minIdx}$: the index of the minimum value in interval}
    
$\textit{minIdx} \gets l, \; x \gets 0$ \;
\For{$l \gets l+1$ to $r$} {
  $x \gets x+1 - ((\textit{tmask} \gg (l \ll 2)) \mathrel{\&} 15)$ \;
  \If{$x \le 0$} {
    $\textit{minIdx} \gets l$, $x \gets 0$ \;
  }
}
return $\textit{minIdx}$ \;

  \caption{Range Minimum Query in 64-bits Cartesian Tree}
  \label{alg:cartesian64bits-query}
\end{algorithm}

\begin{theorem} \label{thm:correctness}
  Algorithm~\ref{alg:cartesian64bits-query} correctly answers a
  in-block ranged maximum query.
\end{theorem}
\begin{proof}
The algorithm will correctly answer a in-block ranged maximum query.
When the $x$, the number of poppings, is {\em smaller than or equal}
to 0, it means the added data has become the {\em root} of the subtree
of the queried interval.  As a result when we report that the added
data became the root of the subtree for the {\em last} time, the
reported root is indeed the maximum among this interval, because
according to the heap property, the root is the maximum among the
nodes within this subtree.
\end{proof}

We choose the block size as $s = {{\frac{\log n}{4}}} = 16$ for
performance reason.  Modern CPUs support 64 bit register and fast
operation on them.  When we pack 16 $t_i$ in to a 64 integer, we can
leverage fast 64 bit instructions and improve performance.  In
addition, we do {\em not} build least common ancestor tables {\em
  explicitly} since we implicitly maintain the Cartesian tree
information within these 64 bit $t_i$.  This approach reduces memory
usage and improve cache performance, it also can efficiently answer
one-dimension ranged maximum query for up to $n = 2^{64}$ data.

Note that our compressed Cartesian table approach does improve cache
performance, but will increase the time complexity.
Algorithm~\ref{alg:cartesian64bits-query} access data in a very
regular manner and has a better data locality than Fischer's
algorithm.  The preprocessing time is $O(n)$, as same as in Fischer's
algorithm.  However, a single query now needs $O(s)$ time, where $s$
is the block size.  This is acceptable in practice since we use $s =
16$, which is a small constant.  The space complexity is $O(n)$ as in
Fischer's algorithm.  The overall time complexity of the parallel
version of our VGLCS algorithm becomes $O(n^2 s / p + n \times
\max(\log n, s))$.
