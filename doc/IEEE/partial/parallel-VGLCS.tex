\section{Parallel VGLCS Algorithm} %
\label{sec:parallelSerial}

In the serial algorithm which was designed by Yung-Hsing Peng, we
observed that variants of LCS use many status to decide a new status.
Its definition makes lots of data dependency, so we cannot parallel
row by row intuitively.  Even though it's hard to parallel by several
modifications, we can parallel the serial algorithm by using wavefront
method.  Because the usage of cache  memory is hard to use efficiently
in it, so Saeed Maleki~\cite{saeed} developed a technique that uses
the property of rank convergence to exploit more parallism to solve
dynamic programming problems.

\begin{algorithm*}[!thb]
  \caption{Algorithm for Finding VGLCS}
  \label{alg:serial-VGLCS}
  \begin{algorithmic}[1]
    \Require
      $A, B$: the input string;
      $G_A, G_B$: the array of variable gapped constraints;
    \Ensure Find the LCS with variable gapped constraints
    \State Create $m$ number of data structure $Q[m]$ to support ISMQ problem.
    \State Create an empty table $V[n][m]$.
    \For{$i \gets 1$ to $n$}
      \State Create a data structure $RQ$ to support ISMQ problem.
      \State $r \gets i - (GA[i]+1)$
      \For{$j \gets 1$ to $m$}
        \If{$A[i] = B[j]$}
            \State $t \gets $ query suffix maximum value from position $j - (GB[j]+1)$ to tail in $RQ$.
            \State $V[i][j] \gets t + 1$
            \State $t \gets $ get the suffix maximum value from position $r$ to $i$ in $Q[j]$
            \State Append value $t$ into $RQ$.
            \State Append value $V[i][j]$ into $Q[j]$.
        \Else
            \State $V[i][j] \gets 0$
            \State $t \gets $ get the suffix maximum value from position $r$ to $i$ in $Q[j]$
            \State Append value $t$ into $RQ$.
        \EndIf
      \EndFor
    \EndFor
    \State Retrieve the VGLCS by tracing $V[n][m]$
  \end{algorithmic}
\end{algorithm*}

In the algrithm ~\ref{alg:serial-VGLCS}, it run in $O(nm \alpha)$ time
and $O(nm)$ space.  If we use wavefront method to parallel, it must
use extra storage space to record all row status compared to origin
algorithm.  If use the rank convergence technique, it also use extra
storage space to reserve the informations of state translation, and
spends more time to merge split parts.

In this paper, we tend to design the algorithm which has less space
and be more cache-friendly.  We define two stages in the serial
algorithm, row and column stage.  In the row stage, it use disjoint
set to maintain \emph{incremental suffix maximum query} (ISMQ).
However, ISMQ change itself state of data structure when it append a
new value each time, so it makes more hard to parallel.  We study some
alternative plans as follows:

\begin{itemize}
  \item 

Binary Indexed Tree -- $O(\log n)$: It supports prefix sum query and
update single value in $O(\log n)$.  In our application, it also
supports update maximum value and query suffix maximum value.  It run
faster than segment tree.

  \item 

Segment Tree -- $O(\log n)$: It supports query maximum value and
update values in multi-dimension.  It can use $O(\log n)$ time to
solve range maximum query in one-dimension.

  \item 

Sparse Table -- $O(n \log n)$ -- $O(1)$: We use $ST[j][i]$ to present
the maximum value in array $(i-2^j,i]$.  It spend $O(n \log n)$ time
to build table and require $O(1)$ time to get the range maximum query.
In order to support append value to tail, it cannot run $O(n)$ --
$O(1)$ solution which Fischer ~\cite{fischer} introduced.

\end{itemize}

\begin{figure*}[!thb]
  \centering
  \includegraphics[width=\linewidth]{graphics/fig-interval-decomposition.pdf}
  \includegraphics[width=\linewidth]{graphics/fig-sparse-table.pdf}

  \caption{  
An example for illustrating the sparse table, which has
a array $A$. $A$ is splited into 5 blocks, each block has 4 elements.
If query range maximum value in $[2, 18]$, it will merge four maximum
result $B1$, $Q_L$, $B5$, and $Q_R$.   
}

  \label{fig:interval-decomposition}
\end{figure*}

Among of them, we consider that the sparse table is the best
alternative plan.  We present the parallel algorithm ~\ref{alg
:parallel-VGLCS} for the VGLCS problem, and its time complexity is
$O(n^2 \alpha(n) / p + n \log n)$, which $p$ is the number of
processors. In the next chapter, we provide a new data structure to
instead of disjoint set and reduce $O(n^2 \alpha(n))$ time to $O(n^2)$
time in theoretically.  In the parallel environment, it can run in
$O(n^2 /p + n \log n)$ time better than $O(n^2 \alpha(n)/p + n \log
n)$.

\begin{algorithm*}[!thb]
  \caption{Parallel Algorithm for Finding VGLCS}
  \label{alg:parallel-VGLCS}
  \begin{algorithmic}[1]
    \Require
      $A, B$: the input string;
      $G_A, G_B$: the array of variable gapped constraints;
    \Ensure Find the LCS with variable gapped constraints
    \State Create $m$ number of data structure $Q[m]$ to support ISMQ problem.
    \State Create an empty table $V[n][m]$.
    \For{$i \gets 1$ to $n$}
      \State Create a sparse table data structure $\textit{sp}$, and initialize $\textit{sp}$ to zero.
      \ParFor{$j \gets 1$ to $m$}
        \State $\textit{sp}[j] \gets$ query suffix maximum value from position $r$ to tail in $Q[j]$.
      \EndParFor
      \State Build sparse table $\textit{sp}$ with $m$ elements in parallel $O(n/p \log n + \log n)$ time.
      \ParFor{$j \gets 1$ to $m$}
        \If{$A[i] = B[j]$}
            \State $t \gets $ query suffix maximum value from position $j - (GB[j] + 1)$ to $j-1$ in $\textit{sp}$
            \State $V[i][j] \gets t + 1$
            \State Append value $V[i][j]$ into $Q[j]$.
        \EndIf
      \EndParFor
    \EndFor
    \State Retrieve the VGLCS by tracing $V[n][m]$
  \end{algorithmic}
\end{algorithm*}