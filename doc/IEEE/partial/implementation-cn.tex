\section{實作與優化} \label{sec:Implementation}

這一章節將描述如何優化 VGLCS 算法,不論是循序或者平行環境,
我們將討論實作上的優化策略。特別注意到,有些技術將根據硬體特性進行優化,
如快取行為等,這些並不會影響到漸近分析。

\subsection{並查集實作策略}

在 VGLCS 算法中,循序版本使用的並查集,將在這一章節探討如何在實作上優化。

\subsubsection{快取效能}

並查集的運行效能易受到快取行為影響。
在 Patwary,Blair 和 Manne~\cite{Patwary2010ExperimentsOU} 等人的研究中,
探討不同的實作策略導致不同程度的快取未中問題。在實作中,
快取未中將會發生在如何從子節點找到父節點的部分,
這些伴隨著路徑壓縮 (path compression) 中使用指針找到父節點。
然而,一個複雜度較低的算法,通常會有很多的「遠跳」(long jumps),
而這些遠跳將會存取距離較遠的記憶體位址而導致快取未中,
請參照圖~\ref{fig:long-short-jump-disjoint} 說明。

\begin{figure}[!thb]
  \centering \subfigure[低複雜度的算法] {
    \includegraphics[width=0.42\linewidth]{\GraphicPath/fig-rem-long-jump.pdf}
    \label{fig:long-short-jump-disjoint-long}
  } \subfigure[高複雜度的算法] {
    \includegraphics[width=0.42\linewidth]{\GraphicPath/fig-rem-short-jump.pdf}
    \label{fig:long-short-jump-disjoint-short}
  }
  \caption{並查集所需的父節點跳躍}
  \label{fig:long-short-jump-disjoint}
\end{figure}

「雷姆算法」(Rem's algorithm~\cite{dijkstra1976a}) 
使用「索引值合併」({\em merge-by-index}) 策略以得到較好的快取效能。
傳統的並查集使用「秩合併」({\em merge-by-rank}) 或者「權重合併」({\em
merge-by-size}~\cite{Tarjan1975EfficiencyOA}),這些方法分別使用根的秩和節點個數決定合併方向。
儘管它們有漸近最好的理論時間複雜度,但它們在實作上並不是這麼出色,原因在於先前的快取未中問題。
相比之下,雷姆算法在合併操作中,將索引值「低」的集合指向索引值「高」的集合,將提供更容易預測的快取算法。

在我們的實驗中,我們仍使用秩合併為主要操作,然而當秩相同時,我們偏向索引值合併,
打破等價情況並改善快取效能。從實驗中,
我們得到「索引值合併破壞等價」(merge-by-index tie-breaking) 提供 3 \% 的效能改善相較於隨機合併,
請參照圖~\ref{fig:fig-perf-cache-miss-rem} 說明。

\begin{figure}
  \centering
  \includegraphics[width=0.95\linewidth]{\GraphicPath/fig-perf-cache-miss-rem.pdf}
  \caption{「索引值合併破壞等價」應用在不同合併策略與不同應用問題上的效能,
  透過 Linux {\tt perf} 分析工具實驗於 E5-2620 伺服器。}
  \label{fig:fig-perf-cache-miss-rem}
\end{figure}

\subsubsection{應用於 VGLCS 算法}

% VGLCS

更進一步優化 Peng 算法~\ref{alg:serial-VGLCS},
將採用「懶插入」({\em lazy insertion}) 技巧。在算法~\ref{alg:serial-VGLCS} 中,
在匹配第一個字串的第 $i$ 個字元與第二個字串的第 $j$ 個字元時,
大部分的情況由於沒有匹配而填入元素 $0$ 至 $V[i][j]$,
存在大量的 $0$ 插入到並查集中,每一次操作頻繁地與前一個 $0$ 一同合併。
因此,懶插入將會改善快取效能。

在懶插入操作中,我們優化許多次的插入、聯結許多個零元素。
我們直到第一個非 $0$ 元素 $v$,才將前大段的 $0$ 元素一同合併,
直接將每一個元素的父節點指向到 $v$ 所在的集合。從實驗中,我們觀察到這減少指針鏈以及更新路徑效能。

特別注意到我們不使用「懶插入」於平行環境下,
由於多核心平台下的執行緒同步也是重要的效能影響,
因此,在 VGLCS 算法中,懶插入降低同步效率,故將不應用於平行實驗中。

\subsection{平行區間詢問於 VGLCS 算法中}

We further improve the performance of range query by maintaining two
extra tables in the blocked sparse table approach described in
Section~\ref{sec:blocked-sparse-table}.  There are now three tables -- a
sparse table on the block maximum $T_S$, a {\em prefix maximum table}
$P$, and a {\em suffix maximum table} $S$.  As described in
Section~\ref{sec:blocked-sparse-table}, the sparse table $T_S$ can
easily answer range query of consecutive blocks on the input $A$.  The
prefix maximum table $P$ maintains the maximum of the prefix {\em
within} a block, and similarly the suffix maximum table $S$ maintains
the maximum for prefix.  Please refer to
Figure~\ref{fig:compressed-sp-opt} for an illustration.

It is easy to see that any range query can be answered by {\em two}
queries into sparse table $T_S$, {\em one} query into prefix maximum
table $P$, and {\em one} into suffix maximum table $S$.  For example, in
Figure~\ref{fig:compressed-sp-opt}, the query from index 2 to 18 can be
answered by two queries into the sparse table $T_S$ -- one from block 1
to block 2, and one from 2 to 3, a query for the suffix of length 2 into
the first block of $S$, and a query for the prefix of length 3 into the
last block of $P$.  Please refer to Figure~\ref{fig:compressed-sp-opt}
for an illustration.

%% The in-block query is a very small probability event due to small $s =
%% \frac{\log n}{4}$.

%% We can use prefix and suffix maximum array to reduce the number of
%% lookup operation on LCA table.

\begin{figure}[!thb]
  \centering \subfigure[The prefix/suffix maximum tables for blocks] {
    \includegraphics[width=0.55\linewidth]{\GraphicPath/fig-compressed-sp-prefix-suffix.pdf}
    \label{fig:compressed-sp-opt-all}
  } \subfigure[A Sparse Table for $T_S$] {
    \includegraphics[width=0.35\linewidth]{\GraphicPath/fig-sparse-table-opt.pdf}
    \label{fig:compressed-sp-opt-ts}
  } \caption{Block maximum $T_S$, prefix maximum $P$, and suffix maximum $S$.}
  \label{fig:compressed-sp-opt}
\end{figure}

We argue that the {\em order} to access these maximum tables is
important.  Our implementation accesses the block maximum $T_S$ {\em
first}, then the prefix maximum $P$ {\em second}, then the suffix
maximum $S$ {\em last}.  The reasoning for this order is as follows.
Since we need to access two elements in the sparse table $T_S$ in {\em
the same level}, it is very likely that they will be in the same cache
line, so access the first will bring in the other automatically by the
hardware caching mechanism.  For example, in
Figure~\ref{fig:compressed-sp-opt-ts}, we will access both $T_{S}[1][2]$
and $T_{S}[1][3]$ in order to find the maximum from block 1 to 3.

Our implementation improves performance by ``peeking'' into two
neighboring elements in the $T_{S}$ table.  To be more specific, we
peek the element {\em before} the first element, and the one {\em
  after} the second element, which we just accessed from $T_S$.

For example, in Figure~\ref{fig:compressed-sp-opt-ts}, we will peek into
$T_{S}[1][1]$ and $T_{S}[1][4]$, which are very likely also present in
cache because they are in the same level of the sparse table as the
previously accessed elements $T_{S}[1][2]$ and $T_{S}[1][3]$.  The idea
is that if the maximum from the block maximum, i.e., the maximum of
$T_{S}[1][2]$ and $T_{S}[1][3]$, is already {\em larger} than the
overall block maximum where the prefix belongs, i.e., in
$T_{S}[1][4]$, then we do {\em not} need to check the $P$ table.
Similarly, if the maximum from the block maximum {\em and} the prefix
maximum is larger than the block maximum where the suffix belongs,
i.e., in $T_{S}[1][1]$, then we do not need to check the $S$ table
either.  This peeking is efficient since the data being peeked are
most likely in the same cache line, and can save unnecessary access to
$P$ and $S$ tables.

Algorithm~\ref{alg:rmq-access-order-2e} shows the access order for the
range maximum query to reduce the cache miss.  Note that we check the
prefix maximum $P$ {\em before} the suffix maximum $S$ since the value
of the dynamic programming table in the same row is {\em increasing},
so it is more likely that a prefix maximum, which has a larger column
index, can save a check into the suffix table, which has a smaller row
index.

Our experiments show that despite that these maximum arrays require
additional $O(n)$ memory, the peeking technique improves overall
performance by up to 8\% when $n = 10^4$, and it improves the overall
performance of query operations by up to 35\% when $n = 10^7$.  Please
refer to Figure~\ref{fig:fig-perf-cache-miss-peek} for an illustration.

\begin{figure}
  \centering
  \includegraphics[width=0.95\linewidth]{\GraphicPath/fig-perf-cache-miss-peek.pdf}
  \caption{The performance of the ``peeking'' for
  the combination of different applications and environments in blocked
  sparse table on an E5-2620 server, which is measured by the {\tt perf}
  analyzing tool in Linux.}
  \label{fig:fig-perf-cache-miss-peek}
\end{figure}

% One can consider the
% first level of the compressed sparse table $T_S[0]$ as a small cache,
% which may prevent access to the prefix/suffix maximum array, as
% described earlier.

\begin{algorithm}
\SetAlgoNoLine
\KwIn{
  $A$: the input array\; 
  $P, \; S$: the prefix/suffix maximum array for each block of the compressed sparse table \;
  ${\cal T}$: the Catalan index array for each block of the compressed sparse table \;
  $T_s$: the compressed sparse table \; 
  $[l, r]$: ranged query \;
}
\KwOut{
  $v$: the maximum value in $A[l .. r]$ \;
}

\If{$l$ and $r$ in the same block} {
  $i \gets$ Query the ranged maximum query $[l, r]$ on ${\cal T}_\text{block(l)}$ \;
  return $A[i]$ \;
}
$v \gets - \infty$ \;
$l' \gets$ The lower bound of the next block by the position $l$ \;
$r' \gets$ The upper bound of the previous block by the position $r$ \;
\If{$l' \le r'$} {
  $t \gets \lfloor \log_2 (r'-l'+1) \rfloor$ \;
  $SQ_L \gets T_s[t][l' + 2^t - 1]$, $SQ_R \gets T_s[t][r']$ \;
  $v \gets \max(SQ_L, SQ_R)$ \;
}

\If(\tcc*[f]{$T_s[0]$ is a cache to avoid impossible event}){$T_s[0][\text{block}(r)] > v$} {
  $v \gets \max(v, P[r])$ \;
}

\If(\tcc*[f]{Place very low possibility event to the end}){$T_s[0][\text{block}(l)] > v$} {
  $v \gets \max(v, S[l])$ \;
}

return $v$ \;

\caption{Access order of ranged maximum query}
\label{alg:rmq-access-order-2e}
\end{algorithm}


