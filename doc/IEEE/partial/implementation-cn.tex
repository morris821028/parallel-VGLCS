\section{實作與優化} \label{sec:Implementation}

這一章節將描述如何優化 VGLCS 算法,不論是循序或者平行環境,
我們將討論實作上的優化策略。特別注意到,有些技術將根據硬體特性進行優化,
如快取行為等,這些並不會影響到漸近分析。

\subsection{並查集實作策略}

在 VGLCS 算法中,循序版本使用的並查集,將在這一章節探討如何在實作上優化。

\subsubsection{快取效能}

並查集的運行效能易受到快取行為影響。
在 Patwary,Blair 和 Manne~\cite{Patwary2010ExperimentsOU} 等人的研究中,
探討不同的實作策略導致不同程度的快取未中問題。在實作中,
快取未中將會發生在如何從子節點找到父節點的部分,
這些伴隨著路徑壓縮 (path compression) 中使用指針找到父節點。
然而,一個複雜度較低的算法,通常會有很多的「遠跳」(long jumps),
而這些遠跳將會存取距離較遠的記憶體位址而導致快取未中,
請參照圖~\ref{fig:long-short-jump-disjoint} 說明。

\begin{figure}[!thb]
  \centering \subfigure[低複雜度的算法] {
    \includegraphics[width=0.42\linewidth]{\GraphicPath/fig-rem-long-jump.pdf}
    \label{fig:long-short-jump-disjoint-long}
  } \subfigure[高複雜度的算法] {
    \includegraphics[width=0.42\linewidth]{\GraphicPath/fig-rem-short-jump.pdf}
    \label{fig:long-short-jump-disjoint-short}
  }
  \caption{並查集所需的父節點跳躍}
  \label{fig:long-short-jump-disjoint}
\end{figure}

「雷姆算法」(Rem's algorithm~\cite{dijkstra1976a}) 
使用「索引值合併」({\em merge-by-index}) 策略以得到較好的快取效能。
傳統的並查集使用「秩合併」({\em merge-by-rank}) 或者「權重合併」({\em
merge-by-size}~\cite{Tarjan1975EfficiencyOA}),這些方法分別使用根的秩和節點個數決定合併方向。
儘管它們有漸近最好的理論時間複雜度,但它們在實作上並不是這麼出色,原因在於先前的快取未中問題。
相比之下,雷姆算法在合併操作中,將索引值「低」的集合指向索引值「高」的集合,將提供更容易預測的快取算法。

在我們的實驗中,我們仍使用秩合併為主要操作,然而當秩相同時,我們偏向索引值合併,
打破等價情況並改善快取效能。從實驗中,
我們得到「索引值合併破壞等價」(merge-by-index tie-breaking) 提供 3 \% 的效能改善相較於隨機合併,
請參照圖~\ref{fig:fig-perf-cache-miss-rem} 說明。

\begin{figure}
  \centering
  \includegraphics[width=0.95\linewidth]{\GraphicPath/fig-perf-cache-miss-rem.pdf}
  \caption{「索引值合併破壞等價」應用在不同合併策略與不同應用問題上的效能,
  透過 Linux {\tt perf} 分析工具實驗於 E5-2620 伺服器}
  \label{fig:fig-perf-cache-miss-rem}
\end{figure}

\subsubsection{應用於 VGLCS 算法}

% VGLCS

更進一步優化 Peng 算法~\ref{alg:serial-VGLCS},
將採用「懶插入」({\em lazy insertion}) 技巧。在算法~\ref{alg:serial-VGLCS} 中,
在匹配第一個字串的第 $i$ 個字元與第二個字串的第 $j$ 個字元時,
大部分的情況由於沒有匹配而填入元素 $0$ 至 $V[i][j]$,
存在大量的 $0$ 插入到並查集中,每一次操作頻繁地與前一個 $0$ 一同合併。
因此,懶插入將會改善快取效能。

在懶插入操作中,我們優化許多次的插入、聯結許多個零元素。
我們直到第一個非 $0$ 元素 $v$,才將前大段的 $0$ 元素一同合併,
直接將每一個元素的父節點指向到 $v$ 所在的集合。從實驗中,我們觀察到這減少指針鏈以及更新路徑效能。

特別注意到我們不使用「懶插入」於平行環境下,
那是因為多核心平台下的執行緒同步也是重要的效能指標。
在平行 VGLCS 算法中,懶插入降低同步效率,故將不應用於平行實驗中。

\subsection{平行區間詢問於 VGLCS 算法中}

關於章節~\ref{sec:blocked-sparse-table} 所提及的塊狀稀疏表方法,
我們額外維護兩個表以增進其效能,因此共包含三張表:其一,塊狀稀疏表 $T_S$,
其二,前綴最大值表 ({\em prefix maximum table}) $P$,其三,
後綴最大值表 ({\em suffix maximum table}) $S$。
如同先前章節~\ref{sec:blocked-sparse-table} 的描述,
表 $T_S$ 解決 $A$ 中連續完整塊的區間詢問,
而前綴最大值表 $P$ 維護各自塊內的前綴最大值,同理後綴最大值表 $S$,
請參照圖~\ref{fig:compressed-sp-opt} 的說明。

對於任一區間詢問,將會被拆成 2 次的 $T_S$ 上的詢問以及各 1 次的 $P$、$S$ 上的詢問。
例如圖~\ref{fig:compressed-sp-opt} 詢問區間 2 到 18 的最大值,拆分 2 次詢問於 $T_S$ 中,
分別為塊 1 到塊 2 的最大值、塊 2 到塊 3 的最大值,在塊 0 內的後綴最大值和塊 4 的前綴最大值。
請參照圖請參照圖~\ref{fig:compressed-sp-opt} 的藍色圈選的部分說明。

\begin{figure}[!thb]
  \centering \subfigure[塊內前綴、後綴最大值表] {
    \includegraphics[width=0.55\linewidth]{\GraphicPath/fig-compressed-sp-prefix-suffix.pdf}
    \label{fig:compressed-sp-opt-all}
  } \subfigure[稀疏表 $T_S$] {
    \includegraphics[width=0.35\linewidth]{\GraphicPath/fig-sparse-table-opt.pdf}
    \label{fig:compressed-sp-opt-ts}
  } \caption{塊狀稀疏表 $T_S$、前綴最大值表 $P$、後綴最大值表 $S$}
  \label{fig:compressed-sp-opt}
\end{figure}

我們認為存取表格的「順序」相當重要。在實作中,首先存取稀疏表 $T_S$,
接著存取前綴最大值表 $P$,最後才存取後綴最大值表 $S$,其原因在於以下幾點:
由於我們存取 $T_S$ 時會在二維陣列的同一層,對於快取存取時容易一同帶入。
如圖~\ref{fig:compressed-sp-opt-ts} 所描述,我們依序存取 $T_{S}[1][2]$ 時,
硬體快取容易一同帶入 $T_{S}[1][3]$。

接著,我們使用「偷看」({\em peeking}) 的方式,將存取 $T_S$ 時的左右鄰居紀錄下來,
在前半部的存取偷看前一個元素,在後半部存取偷看後一個元素。

如圖~\ref{fig:compressed-sp-opt-ts} 的說明,將偷看 $T_{S}[1][1]$、$T_{S}[1][4]$ 兩個元素,
由於它們分別表示該塊內的最大值,必然會大於等於任意的塊內的前綴或後綴最大值。
意即如果 $T_{S}[1][4]$ 已經小於目前的最大值,這使得我們無須到表 $P$ 中存取,
同理 $T_{S}[1][1]$ 也類似防止到表 $S$ 中存取。「偷看」將會改善不少的效能,
這些偷看的原因大部分落在同一段快取中,減少存取 $P$ 和 $S$ 的機會,
便大幅度減少快取未中的可能。

算法~\ref{alg:rmq-access-order-2e} 呈現我們在實作中的存取順序,以減少快取未中的方法。
特別注意,存取前綴最大值表 $P$ 先於後綴最大值表 $S$,
原因在於動態規劃法的運作模式使得索引值大的元素值有大於索引值小的趨勢。

實驗結果仍需要額外的 $O(n)$ 空間,而偷看技巧可以應用上 $n=10^4$ 時,提供 8\% 的效能改善。
若單看詢問操作,當 $n=10^7$ 時可以改善 35\% 的效能。
請參照圖~\ref{fig:fig-perf-cache-miss-peek} 的說明。

\begin{figure}
  \centering
  \includegraphics[width=0.95\linewidth]{\GraphicPath/fig-perf-cache-miss-peek.pdf}
  \caption{
  「偷看」於不同應用與環境上的效能,
  透過 Linux {\tt perf} 分析工具實驗於 E5-2620 伺服器}
  \label{fig:fig-perf-cache-miss-peek}
\end{figure}

\begin{algorithm}
\SetAlgoNoLine
\KwIn{
  $A$: the input array\; 
  $P, \; S$: the prefix/suffix maximum array for each block of the compressed sparse table \;
  ${\cal T}$: the Catalan index array for each block of the compressed sparse table \;
  $T_s$: the compressed sparse table \; 
  $[l, r]$: ranged query \;
}
\KwOut{
  $v$: the maximum value in $A[l .. r]$ \;
}

\If{$l$ and $r$ in the same block} {
  $i \gets$ Query the ranged maximum query $[l, r]$ on ${\cal T}_\text{block(l)}$ \;
  return $A[i]$ \;
}
$v \gets - \infty$ \;
$l' \gets$ The lower bound of the next block by the position $l$ \;
$r' \gets$ The upper bound of the previous block by the position $r$ \;
\If{$l' \le r'$} {
  $t \gets \lfloor \log_2 (r'-l'+1) \rfloor$ \;
  $SQ_L \gets T_s[t][l' + 2^t - 1]$ \;
  $SQ_R \gets T_s[t][r']$ \;
  $v \gets \max(SQ_L, SQ_R)$ \;
}

\If(\tcc*[f]{$T_s[0]$ is a cache to avoid impossible event}){$T_s[0][\text{block}(r)] > v$} {
  $v \gets \max(v, P[r])$ \;
}

\If(\tcc*[f]{Place very low possibility event to the end}){$T_s[0][\text{block}(l)] > v$} {
  $v \gets \max(v, S[l])$ \;
}

return $v$ \;

\caption{Access order of ranged maximum query}
\label{alg:rmq-access-order-2e}
\end{algorithm}


