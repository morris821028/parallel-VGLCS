\section{平行區間查詢}
\label{sec:parallelRMQ}

\subsection{背景}

縱使我們已能很好地平行化原本的序列算法,在理論複雜度上受限於平行下的區間極值查詢 (Range Minimum/Maximum Query, RMQ)。在這個應用中,每一階段有 $n$ 個元素和 $n$ 個區間詢問。這樣的條件下,大部分樹狀結構難以在前處理過程和每次詢問皆達到最好效能。對於 $O(n)$ -- $O(1)$ 操作的離線區間詢問無法提供平行。再接續的小節中,我們將提出在兼顧建表、插入和查找的數據結構與算法。

\subsection{壓縮笛卡爾樹}

在 Fischer ~\cite{fischer} 的論文中,根據卡塔蘭數 $\frac{1}{s+1}\binom{2s}{s} = O(\frac{4^s}{s^{1.5}})$ 建立查找表 (lookup-table),其中選擇 $s = \frac{1}{4} \log n$ 時,空間複雜度 $O(s^2 \frac{4^s}{s^{1.5}}) = o(n)$ 且建表複雜度 $o(n)$。每一個區間詢問將會拆成 2 個 super-block 和 2 個 in-block 詢問 (參照圖 ~\ref{fig:interval-decomposition}),共計需要 4 次的記憶體存取。在理論分析上,離線 RMQ 問題可在 $\theta(n)$ -- $\theta(1)$ 時間內解決任一詢問。

當 $n$ 越大時,這 4 次的記憶體存取會遭遇到嚴重的快取未中 (cache miss),在 Demaine ~\cite{demaine} 的論文中,發展出快取忘卻 (cache oblivious) 形式的查找方案,降低在離線版本中的 in-block 詢問產生的快取未中。

在上述的技術中,我們可以藉由 Fischer 提出的方案平行化 RMQ 至 $O(n / p + \log n)$ -- $O(1)$,使用 Demaine 提供的技巧壓縮空間使用量,降低快取未中以提升運行效能。這裡我們挑選固定長度的壓縮方案 $s = 16$,其能解決序列長度為 $n = 2^{64}$ 的區間查找,將 16 個整數壓縮成一棵笛卡爾樹。在第 $i$ 次插入時,左旋的次數 $l_i$,每次操作皆符合 $\sum_{i=1}^{n} l_i < i$。

因所有 $l_i < 16$,使得每個 $l_i$ 可用 4-bit 表示之,整體便可用 64-bit 長整數表示一棵笛卡爾樹的狀態。為了現在常見的 64-byte 快取列 (cache line) 和 64-bit 暫存器 (register) 考量,我們選用合適的大小進行測試,不僅壓縮空間使用量,同時也減少快取未中的問題。壓縮流程如算法 ~\ref{alg:cartesian-to-64bits},對應的區間查找算法,根據 Demaine ~\cite{demaine} 進行修改,流程如算法 ~\ref{alg:cartesian64bits-query}。

\begin{algorithm*}
  \caption{Transfer Cartesian Tree to 64-bits with 8 integers}
  \label{alg:cartesian-to-64bits}
  \begin{algorithmic}[1]
    \Require
      $A[1 \cdots 16]$: 16 integers store in array
    \Ensure 
      $\textit{tmask}$: compress Cartesian tree into 64-bits integer
      \State $\tt{LOGS} = 4$
      \State $\tt{POWS} = 2^{\tt{LOGS}}$
      \State int $D[POWS+1]$, $Dp = 0$;
      \State uint64\_t $tmask$ = $0$
      \State $D[0]$ = SHRT\_MAX
      \For{$i = 1$ to $\tt{POWS}$} 
        \State $v = A[i]$;
        \State $cnt = 0$;
        \While{$D[Dp] < v$}
          \State $Dp = Dp-1$
          \State $cnt = cnt + 1$
        \EndWhile
        \State $Dp = Dp+1$
        \State $D[Dp] = v$
        \State $tmask = tmask | ((cnt)<<((i-1)<<2))$
      \EndFor
      \State return $tmask$
  \end{algorithmic}
\end{algorithm*}

\begin{algorithm}[!thb]
  \caption{Range Minimum Query in 64-bits Cartesian Tree}
  \label{alg:cartesian64bits-query}
  \begin{algorithmic}[1]
    \Require
      $\textit{tmask}$: 64-bits Cartesian tree;
      $[l, r]$: query range
    \Ensure 
      $\textit{minIdx}$: the index of the minimum value in interval
    \State $\textit{minIdx} \gets l, \; x \gets 0$;
    \For{$l \gets l+1$ to $r$}
      \State $x \gets x+1 - ((\textit{tmask} \gg (l \ll 2)) \mathrel{\&} 15)$
      \If{$x \le 0$}
        \State $\textit{minIdx} \gets l$
        \State $x \gets 0$
      \EndIf
    \EndFor
    \State return $\textit{minIdx}$
  \end{algorithmic}
\end{algorithm}

在 VGLCS 的應用中,上述算法使用壓縮方式降低快取未中。我們可以使用上述的算法取代原先的并查集,建表的時間複雜度為 $O(n)$,單一查詢的時間複雜度為 $O(s)$。整體的時間複雜度為 $O(n^2 \; s / p + n \max(\log n, s))$。

\section{動態區間詢問}

在 VGLCS 問題中,主要分成縱向和橫向兩階段,縱向處理每一列的區間極值查找,橫向處理每一行的區間極值查找,兩者合併構成區域極值查找。在縱向方面為數個獨立的數據結構,這部分易於平行;相反地,在橫向方面,需要共同協作一個數據結構。綜觀這兩者的差異,縱向需要動態的後綴插入和區間查詢,而橫向可以離線完成區間查找。在上一節中,我們提出在橫向處理的實作,若限制上述的實作方案在單一處理器上,時間複雜度的瓶頸在於縱向的動態更新與查找。

在這個章節中,我們提出支持動態插入和區間查找的數據結構,最後的成果如表 ~\ref{tlb:cmp-complexity}。

\begin{table*}
  %\tiny
  \centering
  \begin{table}
  %\tiny
  \centering
  \caption{   Our study shows in the bold front. We use the fixed size
$s=16$ on Cartesian tree. The small amortized constant will not
encounter serious load imbalance problem.   }

  \label{tlb:cmp-complexity}
  \begin{tabular}{ccc}
    \toprule
     & serial & parallel \\
    \midrule
    horizontal & \begin{tabular}{@{}c@{}}
              $\left \langle n \alpha(n) \right \rangle$ \cite{yunghsing} \\ 
              amortized $\left \langle n \right \rangle$\end{tabular}
              & \begin{tabular}{@{}c@{}}
                $\left \langle n \alpha(n)/p + \alpha(n) \right \rangle$ \cite{yunghsing} \\
                amortized $\left \langle n /p + o(1) \right \rangle$
                \end{tabular} \\
    vertical & \begin{tabular}{@{}c@{}}
                $\left \langle n \alpha(n) \right \rangle$ \\
                amortized $\left \langle n \right \rangle$
                \end{tabular}
            & \begin{tabular}{@{}c@{}}
                impossible \\
                amortized $\left \langle n /p + o(1) \right \rangle$
              \end{tabular}
              \\
    total & \begin{tabular}{@{}c@{}}
              $\left \langle n^2 \alpha(n) \right \rangle$ \cite{yunghsing} \\ 
              amortized $\left \langle n^2 \right \rangle$\end{tabular}
          & \begin{tabular}{@{}c@{}}
              impossible \\ 
              amortized $\left \langle n^2 /p + n \log n \right \rangle$\end{tabular} \\
    \bottomrule
  \end{tabular}
\end{table}
  \caption{我們的研究成果如粗體字所述,均攤部分將在 $s=16$ 的笛卡爾樹上,整體影響的常數很小,不易遇到負載平衡上的問題。}
  \label{tlb:cmp-complexity}
\end{table*}

\subsection{平行建立查找表}

關於在線操作,我們從 Fischer \cite{fischer} 和 Masud \cite{masud} 的研究中,分別得到關於笛卡爾的編碼與快取改善的技術,而這些技術都著手於離線操作,即一開始給訂 $n$ 個元素值,並且在 $O(n)$ 時間內編碼一棵樹;接著,再利用前處理的查找表完成極值查找。

關於快取效能,因詢問次數與元素個數相當,故無法像 Masud \cite{masud} 的研究藉由排序編碼以減少快取未中的問題,只能依賴數據本身的分佈和編碼之間的關聯來減少快取未中的情況。

從上述幾點觀點出發,我們提出動態的編碼方式。算法採用字典順序的方式編碼一棵樹,優先增長左子樹,當相同左子樹時,增長右子樹的方式進行編號,其編碼方式可參考圖 ~\ref{fig:lablingBST}。

\begin{figure}[!thb]
  \centering
  \includegraphics[width=\linewidth]{graphics/fig-bst-encoding.pdf}
  \caption{編碼二元搜尋樹}
  \label{fig:lablingBST}
\end{figure}

對於 $s$ 個節點的二元搜尋樹,其權重從 $0$ 開始至 $s-1$、樹編號為 $\mathit{tid}$。定義 $\mathit{LCA}(s, \mathit{tid}, p, q)$ 為其樹上兩點 $p$ 和 $q$ 的最小共同祖先,如 $\mathit{LCA}(3, 2, 0, 2) = 1$。

定義四個變數 $\langle\mathit{lsz},\mathit{lid},\mathit{rsz},\mathit{rid}\rangle$ 分別為左右子樹的大小和其編號,這四個變數根據指定的 $\mathit{tid}$ 可在 $O(n)$ 時間內得到。最後,推導得遞迴公式 ~\ref{fun:LCA}。 

\begin{figure*}[!thb]
  \begin{equation*}
  \begin{split}
    &\mathit{LCA}(n, \mathit{tid}, p, q) \\
      &= \left\{\begin{matrix*}[l]
        \mathit{LCA}(\mathit{lsz}, \mathit{lid}, p, q) &&, p \le q < \mathit{lsz}\\ 
        \mathit{LCA}(\mathit{rsz}, \mathit{rid}, p-\mathit{lsz}-1, q-\mathit{lsz}-1)+\mathit{lsz}+1 &&, 
            \mathit{lsz} \le p \le q < n \\ 
        \mathit{lsz} && , 0 \le p \le \mathit{lsz}, \mathit{lsz} \le q \le i\\ 
        -1 && ,\mathit{otherwise}
      \end{matrix*}\right.
  \end{split}
\end{equation*}
  \caption{建立所有二元搜尋樹的最小共同祖先}
  \label{fun:LCA}
\end{figure*}

為記錄所有的二元搜尋樹的 LCA,空間消耗 $\theta\left(\frac{s^2}{s+1} \binom{2s}{s}\right) = \theta\left(n\right)$;其平行算法 \ref{alg:parallel-LCA} 的時間複雜度如下:

\begin{equation}
\theta\left(\frac{s^3}{s+1} \binom{2s}{s} \bigg/ p + s^2 \right)
\end{equation}

\begin{algorithm}[!thb]
  \caption{Parallel Algorithm for building LCA}
  \label{alg:parallel-LCA}
  \begin{algorithmic}[1]
    \Require
      $s$: Maximum size for required the number of BST
    \For{$n \gets 1$ to $s$}
      \ParFor{$\mathit{tid} \gets 0$ to $C_n - 1$}
        \ParFor{$p \gets 0$ to $n-1$}
          \State compute $\langle\mathit{lsz},\mathit{lid},\mathit{rsz},\mathit{rid}\rangle$
          \For{$q \gets p$ to $n-1$}
            \State $\textit{LCA}[n][\mathit{tid}][p][q] \gets$ Equation~\ref{fun:LCA1} and \ref{fun:LCA2}
          \EndFor
        \EndParFor
      \EndParFor
    \EndFor
  \end{algorithmic}
\end{algorithm}

\subsection{動態編碼笛卡爾樹}

ISMQ 已知解法有二,其一使用並查集在 $O(\alpha(n))$ 解決單一操作,其二使用樸素的稀疏表在 $O(\log n)$完成插入操作、$O(1)$ 完成詢問操作。其二,Fischer \cite{fischer} 提出的 $\theta(n)$ -- $\theta(1)$ 無法應用在此,其原因在於插入元素時,無法動態決定 in-block 的最大值,必須等到整個 in-block 塞滿至預設值才可解決。

在我們的應用中維護後綴最大值,拓展其操作成為增長區間最大值查找 (\emph{incremental range maximum query}, IRMQ),其支援兩項操作:

\begin{itemize}
  \item \texttt{Append V} : 插入元素 $V$ 至陣列 $A$ 的尾端
  \item \texttt{Query L R} : 詢問 $A[L .. R]$ 中的最大值
\end{itemize}


接下來的幾段中,我們提供動態的編碼方式使得每一操作皆均攤 $\theta(1)$ 完成。首先,我們需要充分認知編碼相互轉換的公式,藉以在算法中達到完成每一步的要求。

在上一節提出對於任意編號 $\mathit{tid}$ 可以在 $O(n)$ 時間內得到 $\langle\mathit{lsz},\mathit{lid},\mathit{rsz},\mathit{rid}\rangle$;相反地,可以在 $\theta(1)$ 時間內逆推得到 $\mathit{tid}$,如算法 ~\ref{alg:encode-tid}。透過預處理,事先將所有前綴和保存下來,在算法中的迴圈可視為一次內存存取,使得時間複雜度 $\theta(1)$。

\begin{algorithm}
  \caption{Get $tid$ from $\langle\mathit{lsz},\mathit{lid},\mathit{rsz},\mathit{rid}\rangle$ in $\theta(1)$ time}
  \label{alg:encode-tid}
  \begin{algorithmic}[1]
    \Require
      $\langle\mathit{lsz},\mathit{lid},\mathit{rsz},\mathit{rid}\rangle$: size and label in left/right subtree
    \Ensure
      $\mathit{tid}$: this label
    \If{$\mathit{rsz} = 0$}
      \State return $\mathit{lid}$
    \EndIf
    \State $n = \mathit{lsz}+\mathit{rsz}+1$
    \State $\mathit{base} = 0$
    \For{$i=0$ to $\mathit{lsz}-1$}
      \State $\mathit{base}$ = $\mathit{base} + C_i \cdot C_{n-i-1}$
    \EndFor
    \State $\mathit{offset}$ = $\mathit{lid} \cdot C_{\mathit{rsz}}$ + $\mathit{rid}$
    \State return $\mathit{base}$ + $\mathit{offset}$
  \end{algorithmic}
\end{algorithm}

根據先前的字典順序編碼,只需要維護笛卡爾樹的右鏈,實作上與堆疊結構相同。基於 row-major 順序和遞迴定義 ~\ref{fun:LCA},修改之前論文對於的離線編碼,其對應方案如算法 ~\ref{alg:cartesian-encode-offline}。

\begin{algorithm*}
  \caption{Offline Type of Cartesian Tree}
  \label{alg:cartesian-encode-of}
  \begin{algorithmic}[1]
    \Require
      $A[1 \cdots s]$: storage array;
      $s$: the number of elements;
    \Ensure
      $\mathit{tid}$: this label
    \State $\langle\mathit{lsz},\mathit{lid},\mathit{value}\rangle$ $D$[$s+1$]
    \State $\textit{Dp} \gets 0$
    \State $D[0] \gets \langle 0,0,\infty \rangle$
    \For{$i \gets 1$ to $s$}
      \State $v \gets A[i], \; \textit{lsz} \gets 0, \; \textit{lid} \gets 0$
      \While{$D[\textit{Dp}].\textit{value} < v$}
        \State $\textit{lid} \gets \textit{tid}(D[\textit{Dp}].\textit{lsz}, D[\textit{Dp}].\textit{lid}, \textit{lsz}, \textit{lid})$
        \State $\textit{lsz} \gets \textit{lsz} + D[\textit{Dp}].\textit{lsz} + 1$
        \State $\textit{Dp} \gets \textit{Dp} - 1$
      \EndWhile
      \State $\textit{Dp} \gets \textit{Dp} + 1$
      \State $D[\textit{Dp}] \gets \langle\mathit{lsz},\mathit{lid},\mathit{v}\rangle$
    \EndFor
    \State $\textit{lsz} \gets 0, \; \textit{lid} \gets 0$
    \While{$\textit{Dp} > 0$} \Comment{pop all elements}
      \State $\textit{lid} \gets \textit{tid}(D[\textit{Dp}].\textit{lsz}, D[\textit{Dp}].\textit{lid}, \textit{lsz}, \textit{lid})$
      \State $\textit{lsz} \gets \textit{lsz} + D[\textit{Dp}].\textit{lsz} + 1$
      \State $\textit{Dp} \gets \textit{Dp} - 1$
    \EndWhile
    \State return $\textit{lid}$
  \end{algorithmic}
\end{algorithm*}

我們定義轉移狀態由 5 個變數來決定動態笛卡爾樹的編碼,當前插入第 $i$ 個元素,最終填充 $s$ 個元素,當前的樹編號 $\mathit{tid}$,以及笛卡爾樹的右鏈狀態指針 $Dp$ 與其堆疊 $D$,其結構如下:

\begin{figure}[!thb]
  \begin{lstlisting}[frame=single,caption=State of Cartesian Tree]
struct Node {
  int lsz, lid, val;
};
struct State {
  int i, s, tid, Dp;
  struct Node D[s+1];
  State(i = 0, s = n, 
          tid = C[n]-1, Dp = 0,
           D[0].val = INF)
};
  \end{lstlisting}
\end{figure}

為了解決在線詢問操作,取 $s = \frac{\log n}{4}$。根據字典順序的編碼性質,一開始建立虛設點 $s$ 個在右鏈上,其樹編號 $\mathit{tid} = C_n - 1$ 。隨著插入元素的增加,尚未加入的元素都預設嚴格遞減,加上根據編碼順序,我們藉由差值來維護在線編碼 (如圖 ~\ref{fig:cartesianEncoding})。根據上述的編碼想法,我們得到算法 ~\ref{alg:cartesian-encode-online}。

\begin{algorithm}[!thb]
  \caption{Online Type of Cartesian Tree}
  \label{alg:cartesian-encode-online}
  \begin{algorithmic}[1]
  \Require
      $\mathit{state}$: state of Cartesian Tree;
      $v$: the value which append to array
  \Ensure
      $\mathit{tid}$: this label
  \State $\textit{Dp} \gets \textit{state}.\textit{Dp}$, $\textit{lsz} \gets 0$, $\textit{lid} \gets 0$
  \State $\textit{bsz} \gets \textit{state}.\textit{s} - \textit{state}.\textit{i} + 1$
  \State $\textit{bid} \gets C[\textit{bsz}] - 1$
  \While{$\textit{state}.D[\textit{Dp}].\textit{value} < v$}
    \State $\textit{lid} \gets \textit{tid}(\textit{state}.D[\textit{Dp}].\textit{lsz}, \textit{state}.D[\textit{Dp}].\textit{lid}, \textit{lsz}, \textit{lid})$
    \State $\textit{bid} \gets \textit{tid}(\textit{state}.D[\textit{Dp}].\textit{lsz}, \textit{state}.D[\textit{Dp}].\textit{lid}, \textit{bsz}, \textit{bid})$
    \State $\textit{lsz} \gets \textit{lsz} + \textit{state}.D[\textit{Dp}].\textit{lsz}+1$
    \State $\textit{bsz} \gets \textit{bsz} + \textit{state}.D[\textit{Dp}].\textit{lsz}+1$
    \State $\textit{Dp} \gets \textit{Dp} - 1$
  \EndWhile
  \State $\textit{Dp} \gets \textit{Dp} + 1$
  \State $\textit{state}.D[\textit{Dp}] \gets \left \langle \textit{lsz}, \textit{lid}, \textit{v} \right \rangle$
  \State $\textit{state}.\textit{Dp} \gets \textit{Dp}$
  \State $x.\textit{tid} \gets \textit{tid}(\textit{lsz}, \textit{lid}, \textit{state}.s-\textit{state}.i, C[\textit{state}.s-\textit{state}.i]-1)$
  \State $\textit{state}.\textit{tid} \gets \textit{state}.\textit{tid} + \textit{bid} - x.\textit{tid}$
  \State $\textit{state}.i \gets \textit{state}.i + 1$
  \State return $\textit{state}.\textit{tid}$
  \end{algorithmic}
\end{algorithm}

\begin{figure*}[!thb]
  \centering
  \includegraphics[width=\linewidth]{graphics/fig-cartesian-encoding.pdf}
  \caption{每個區塊有 $s$ 個元素,初始情況虛設 $s$ 個點在右鏈,則具有 $s$ 個節點的 BST,其編號 $\text{tid}_0 = C_s - 1$。當插入第 $i$ 個元素時,當前編號為 $\text{tid}_i$,以節點 $A$ 為根的樹編號為 $A.\text{tid}$,若第 $i+1$ 個元素值為 $x$,其將會翻轉到 $A$ 之上,而 $A$ 成為 $x$ 的左子節點,翻轉過程中計算得到 $A.\text{tid}$,而以 $x$ 為根的樹將虛設 $s-(i+1)$ 個節點在其右鏈,最後得到 $x.\text{tid}$。根據字典順序,我們將得到 $\text{tid}_{i+1} = \text{tid}_{i} + (x.\text{tid} - A.\text{tid})$。}
  \label{fig:cartesianEncoding}
\end{figure*}

最後,我們不改變原本的建立笛卡爾樹算法,便能在過程中擭得樹的編號,每一次的 in-block 詢問只需要一次記憶體存取,得到任一操作攤銷複雜度 $\theta(1)$。