\section{實驗結果}
\label{sec:Experiment}

我們運行在 Intel Xeon E5-2620 2.40 GHz 主機上,其擁有 L1 cache 384 KiB、L2 cache 1536 KiB 和 L3 cache 15 MiB。

\subsection{空間壓縮 VGLCS}

我們運行優化策略中的空間壓縮版本,而非理論分析的 $\theta(1)$ 操作,單次詢問落在 $O(s)$ 中,在實作上由於可以完全壓在暫存器上操作,效能表現較佳。

\begin{figure*}[!thb]
  \centering
  \subfigure[Runtime]{
    \begin{tikzpicture}[scale=0.4,font=\sffamily]
	\begin{axis}[
			xlabel={Length $n$},
			ylabel={Time (second)},
			xmin=0, xmax=10000,
			ymin=0, ymax=4.5,
			scaled ticks = false,
			tick label style={/pgf/number format/fixed},
			xtick={0, 1000, 2000, 3000, 4000, 5000, 6000, 7000, 8000, 9000, 10000},
			ytick={0, 0.5, 1, 1.5, 2, 2.5, 3, 3.5, 4, 4.5},
			legend pos=north east,
			legend cell align=left,
			ymajorgrids=true,
			grid style=dashed,
			,height=10cm,width=15cm,
		]
		\addplot [mark=*] file{figure/serial-n.dat};
		\addplot [mark=square*, mark options={fill=white}] file{figure/parallel-n.dat};
		\legend{serial, parallel}
	\end{axis}
\end{tikzpicture}

    \label{fig:fig-parallel}
  }
  \subfigure[Scalability]{
    \begin{tikzpicture}[scale=0.4,font=\sffamily]
	\begin{axis}[
			xlabel={Processor $p$},
			ylabel={Time (second)},
			xmin=1, xmax=16,
			ymin=0, ymax=1.2,
			scaled ticks = false,
			tick label style={/pgf/number format/fixed},
			xtick={1, 2, 4, 8, 16},
			ytick={0, 0.2, 0.4, 0.6, 0.8, 1, 1.2},
			legend pos=north east,
			legend cell align=left,
			ymajorgrids=true,
			grid style=dashed,
			,height=10cm,width=15cm,
		]
		\addplot [mark=square*, mark options={fill=white}] file{figure/parallel-p.dat};
		\legend{parallel $n=5000$}
	\end{axis}
\end{tikzpicture}

    \label{fig:fig-parallel-scala}
  }
  \caption{Serial and Parallel Algorithm run on E5-2620}
\end{figure*}

從圖 ~\ref{fig:fig-parallel-scala} 看出,平行 VGLCS 的縮放性不佳,從 profiler 中得到快取未中的問題需要解決。

%\subsection{理論常數 VGLCS}

%尚未完成

\subsection{增加後綴區間查找 ISMQ}

針對插入和詢問次數相同的 ISMQ 問題,運行以下四種數據結構:

\begin{itemize}
  \item 并查集 (Disjoint Set): 平均運行時間 $o(\alpha(n))$。只使用路徑壓縮技巧。
  \item 稀疏表 (Sparse Table): 插入 $O(\log n)$、詢問 $O(1)$。實作陣列宣告採用 $\tt{table}[\log N][N]$ 以減少快取未中。
  \item 樹狀數組 (Binary Indexed Tree): 插入、詢問均為 $O(\log n)$。
  \item 壓縮稀疏表 (Compressed Sparse Tree): 插入均攤 $O(1)$、詢問操作 $O(s)$,其中 $s$ 為拆分到區塊大小。實作時,維護區塊前綴和後綴最大值降低詢問複雜度至 $O(1)$,當發生 in-block 詢問再運行 $O(s)$ 算法。
\end{itemize}

當運行 $n > 10^6$ 時,我們提出的壓縮稀疏表的效能已經勝過并查集的版本,其運行結果如圖表 ~\ref{fig:fig-ISMQcmp}。在 $n = 10^7$ 時,加速 $1.25 \times$。然而,我們提供的 amortized $\theta(1)$ 的稀疏表慢於并查集,我們做了深入的機率探討 (參照表 ~\ref{tlb:ISMQcmp}),由於大部分的操作都被區塊後綴和前綴解決,沒有實際運用到內部詢問,約束區間詢問的大小為 $L$,在 $N = 10^7$ 時,最多能加速 $1.26 \times$,其中插入和詢問比例為 1:10,當詢問比重更大時,將有更明顯的加速。

\begin{figure*}[!thb]
  \centering
  \documentclass[border=2pt]{standalone}
\usepackage{tikz}
\usetikzlibrary{calc} \usetikzlibrary{positioning} \usetikzlibrary{shapes,arrows} \usetikzlibrary{plotmarks}
\usepackage{pgfplots}

\begin{document}
	\begin{tikzpicture}
		\begin{axis}[
				xlabel={Length $n$},
				ylabel={Time (ms)},
				xmin=0, xmax=10000000,
				ymin=0, ymax=1600,
				scaled ticks = false,
				tick label style={/pgf/number format/fixed},
				xtick={20000, 1000000, 5000000, 1000000, 2500000, 5000000, 10000000},
				ytick={0, 200, 400, 600, 800, 1000, 1200, 1400, 1600},
				legend pos=north west,
				legend cell align=left,
				ymajorgrids=true,
				grid style=dashed,
				line width=1pt,
				mark size=3pt,
				height=10cm,width=20cm,
			]
			\addplot [mark=*, mark options={fill=white}] file{./data/ismq-disjoint.dat};
			\addplot [mark=square*, mark options={fill=white}] file{./data/ismq-st.dat};
			% binary indexed tree
			% \addplot [mark=diamond*, mark options={fill=white}] file{./data/ismq-bit.dat};
			\addplot [mark=triangle*, mark options={fill=white}] file{./data/ismq-cost.dat};
			\addplot [mark=triangle*, mark options={fill=black}] file{./data/ismq-onest.dat};
			\legend{Disjoint Set, Standard Sparse Table, Rightmost-pops Blocked Sparse Table, LCA Table Blocked Sparse Table}
		\end{axis}
	\end{tikzpicture}
\end{document}
  \caption{ISMQ runs on E5-2620 with different data structures. We use the random test cases without any limitation, e.g. the length of interval query has a uniform distribution.}
  \label{fig:fig-ISMQcmp}
\end{figure*}

\begin{table*}
  \normalsize
  \caption{Total running time (second) for ISMQ with sizes $N = 10^7$, different maximum interval sizes $L$, the probability $p$ of incremental elements and the probility $q$ of zero elements. There are three strategies for each $(p, q, L)$, implemented by disjoint set, compressed sparse table, and O(1) sparse table.}
  \label{tlb:ISMQcmp}
  \centering
  \setlength\tabcolsep{0pt}
  \begin{tabular}{@{\extracolsep{4pt}}r c c c c c c c c c c c c c c c c}
    \firsthline
      & \multicolumn{5}{c}{$L=4$} & \multicolumn{5}{c}{$L=8$} & \multicolumn{5}{c}{$L=16$}\\
      \cline{2-6} \cline{7-11} \cline{12-16}
      $q$ & $0\%$ & $25\%$ & $50\%$ & $75\%$ & $100\%$ 
        & $0\%$ & $25\%$ & $50\%$ & $75\%$ & $100\%$ 
        & $0\%$ & $25\%$ & $50\%$ & $75\%$ & $100\%$ 
        & Speedup\\
      $p$ \\
      \hline
$0\%$ & \begin{tabular}{@{}r@{}}0.36\\ \textbf{0.33}\\0.46 \end{tabular}& \begin{tabular}{@{}r@{}}0.22\\ \textbf{0.18}\\0.29 \end{tabular}& \begin{tabular}{@{}r@{}}0.22\\ \textbf{0.18}\\0.29 \end{tabular}& \begin{tabular}{@{}r@{}}0.22\\ \textbf{0.18}\\0.31 \end{tabular}& \begin{tabular}{@{}r@{}}0.22\\ \textbf{0.18}\\0.30 \end{tabular}& \begin{tabular}{@{}r@{}}0.29\\ \textbf{0.28}\\0.38 \end{tabular}& \begin{tabular}{@{}r@{}}0.22\\ \textbf{0.21}\\0.31 \end{tabular}& \begin{tabular}{@{}r@{}}0.22\\ \textbf{0.21}\\0.31 \end{tabular}& \begin{tabular}{@{}r@{}}0.22\\ \textbf{0.21}\\0.31 \end{tabular}& \begin{tabular}{@{}r@{}}0.22\\ \textbf{0.21}\\0.29 \end{tabular}& \begin{tabular}{@{}r@{}} \textbf{0.25}\\0.26\\0.31 \end{tabular}& \begin{tabular}{@{}r@{}}0.22\\ \textbf{0.21}\\0.31 \end{tabular}& \begin{tabular}{@{}r@{}}0.22\\ \textbf{0.21}\\0.30 \end{tabular}& \begin{tabular}{@{}r@{}}0.22\\ \textbf{0.21}\\0.31 \end{tabular}& \begin{tabular}{@{}r@{}}0.22\\ \textbf{0.21}\\0.31 \end{tabular}& \begin{tabular}{@{}r@{}}1.00\\ \textbf{1.20}\\ 0.81 \end{tabular}\\ \hline
$20\%$ & \begin{tabular}{@{}r@{}}0.28\\ \textbf{0.23}\\0.40 \end{tabular}& \begin{tabular}{@{}r@{}}0.29\\ \textbf{0.23}\\0.40 \end{tabular}& \begin{tabular}{@{}r@{}}0.31\\ \textbf{0.24}\\0.41 \end{tabular}& \begin{tabular}{@{}r@{}}0.28\\ \textbf{0.23}\\0.40 \end{tabular}& \begin{tabular}{@{}r@{}}0.26\\ \textbf{0.23}\\0.38 \end{tabular}& \begin{tabular}{@{}r@{}}0.30\\ \textbf{0.27}\\0.40 \end{tabular}& \begin{tabular}{@{}r@{}}0.32\\ \textbf{0.27}\\0.40 \end{tabular}& \begin{tabular}{@{}r@{}}0.33\\ \textbf{0.27}\\0.40 \end{tabular}& \begin{tabular}{@{}r@{}}0.31\\ \textbf{0.27}\\0.40 \end{tabular}& \begin{tabular}{@{}r@{}}0.28\\ \textbf{0.27}\\0.39 \end{tabular}& \begin{tabular}{@{}r@{}}0.32\\ \textbf{0.29}\\0.43 \end{tabular}& \begin{tabular}{@{}r@{}}0.33\\ \textbf{0.29}\\0.43 \end{tabular}& \begin{tabular}{@{}r@{}}0.34\\ \textbf{0.29}\\0.44 \end{tabular}& \begin{tabular}{@{}r@{}}0.33\\ \textbf{0.29}\\0.43 \end{tabular}& \begin{tabular}{@{}r@{}}0.31\\ \textbf{0.29}\\0.42 \end{tabular}& \begin{tabular}{@{}r@{}}1.00\\ \textbf{1.33}\\ 0.82 \end{tabular}\\ \hline
$40\%$ & \begin{tabular}{@{}r@{}}0.32\\ \textbf{0.26}\\0.44 \end{tabular}& \begin{tabular}{@{}r@{}}0.35\\ \textbf{0.27}\\0.45 \end{tabular}& \begin{tabular}{@{}r@{}}0.39\\ \textbf{0.28}\\0.46 \end{tabular}& \begin{tabular}{@{}r@{}}0.33\\ \textbf{0.27}\\0.45 \end{tabular}& \begin{tabular}{@{}r@{}}0.27\\ \textbf{0.26}\\0.41 \end{tabular}& \begin{tabular}{@{}r@{}}0.35\\ \textbf{0.31}\\0.44 \end{tabular}& \begin{tabular}{@{}r@{}}0.37\\ \textbf{0.31}\\0.44 \end{tabular}& \begin{tabular}{@{}r@{}}0.41\\ \textbf{0.32}\\0.45 \end{tabular}& \begin{tabular}{@{}r@{}}0.35\\ \textbf{0.32}\\0.45 \end{tabular}& \begin{tabular}{@{}r@{}} \textbf{0.29}\\0.30\\0.42 \end{tabular}& \begin{tabular}{@{}r@{}}0.35\\ \textbf{0.33}\\0.48 \end{tabular}& \begin{tabular}{@{}r@{}}0.38\\ \textbf{0.34}\\0.48 \end{tabular}& \begin{tabular}{@{}r@{}}0.42\\ \textbf{0.34}\\0.49 \end{tabular}& \begin{tabular}{@{}r@{}}0.36\\ \textbf{0.34}\\0.48 \end{tabular}& \begin{tabular}{@{}r@{}} \textbf{0.31}\\0.33\\0.46 \end{tabular}& \begin{tabular}{@{}r@{}}1.00\\ \textbf{1.41}\\ 0.89 \end{tabular}\\ \hline
$60\%$ & \begin{tabular}{@{}r@{}}0.34\\ \textbf{0.27}\\0.45 \end{tabular}& \begin{tabular}{@{}r@{}}0.38\\ \textbf{0.28}\\0.46 \end{tabular}& \begin{tabular}{@{}r@{}}0.44\\ \textbf{0.30}\\0.47 \end{tabular}& \begin{tabular}{@{}r@{}}0.35\\ \textbf{0.28}\\0.46 \end{tabular}& \begin{tabular}{@{}r@{}}0.26\\ \textbf{0.26}\\0.42 \end{tabular}& \begin{tabular}{@{}r@{}}0.35\\ \textbf{0.31}\\0.45 \end{tabular}& \begin{tabular}{@{}r@{}}0.39\\ \textbf{0.33}\\0.46 \end{tabular}& \begin{tabular}{@{}r@{}}0.46\\ \textbf{0.34}\\0.47 \end{tabular}& \begin{tabular}{@{}r@{}}0.35\\ \textbf{0.33}\\0.46 \end{tabular}& \begin{tabular}{@{}r@{}} \textbf{0.28}\\0.30\\0.42 \end{tabular}& \begin{tabular}{@{}r@{}}0.36\\ \textbf{0.34}\\0.48 \end{tabular}& \begin{tabular}{@{}r@{}}0.40\\ \textbf{0.35}\\0.50 \end{tabular}& \begin{tabular}{@{}r@{}}0.47\\ \textbf{0.37}\\0.52 \end{tabular}& \begin{tabular}{@{}r@{}} \textbf{0.35}\\0.36\\0.49 \end{tabular}& \begin{tabular}{@{}r@{}} \textbf{0.28}\\0.33\\0.45 \end{tabular}& \begin{tabular}{@{}r@{}}1.00\\ \textbf{1.47}\\ 0.96 \end{tabular}\\ \hline
$80\%$ & \begin{tabular}{@{}r@{}}0.32\\ \textbf{0.26}\\0.44 \end{tabular}& \begin{tabular}{@{}r@{}}0.37\\ \textbf{0.27}\\0.46 \end{tabular}& \begin{tabular}{@{}r@{}}0.44\\ \textbf{0.29}\\0.47 \end{tabular}& \begin{tabular}{@{}r@{}}0.33\\ \textbf{0.27}\\0.46 \end{tabular}& \begin{tabular}{@{}r@{}}0.25\\ \textbf{0.24}\\0.40 \end{tabular}& \begin{tabular}{@{}r@{}}0.33\\ \textbf{0.29}\\0.43 \end{tabular}& \begin{tabular}{@{}r@{}}0.38\\ \textbf{0.31}\\0.45 \end{tabular}& \begin{tabular}{@{}r@{}}0.46\\ \textbf{0.34}\\0.47 \end{tabular}& \begin{tabular}{@{}r@{}}0.33\\ \textbf{0.32}\\0.45 \end{tabular}& \begin{tabular}{@{}r@{}} \textbf{0.25}\\0.27\\0.40 \end{tabular}& \begin{tabular}{@{}r@{}}0.34\\ \textbf{0.32}\\0.46 \end{tabular}& \begin{tabular}{@{}r@{}}0.38\\ \textbf{0.34}\\0.48 \end{tabular}& \begin{tabular}{@{}r@{}}0.47\\ \textbf{0.37}\\0.51 \end{tabular}& \begin{tabular}{@{}r@{}} \textbf{0.33}\\0.34\\0.47 \end{tabular}& \begin{tabular}{@{}r@{}} \textbf{0.25}\\0.29\\0.42 \end{tabular}& \begin{tabular}{@{}r@{}}1.00\\ \textbf{1.50}\\ 0.97 \end{tabular}\\ \hline
$100\%$ & \begin{tabular}{@{}r@{}}0.30\\ \textbf{0.24}\\0.40 \end{tabular}& \begin{tabular}{@{}r@{}}0.35\\ \textbf{0.25}\\0.44 \end{tabular}& \begin{tabular}{@{}r@{}}0.43\\ \textbf{0.28}\\0.47 \end{tabular}& \begin{tabular}{@{}r@{}}0.30\\ \textbf{0.25}\\0.44 \end{tabular}& \begin{tabular}{@{}r@{}}0.23\\ \textbf{0.22}\\0.35 \end{tabular}& \begin{tabular}{@{}r@{}}0.30\\ \textbf{0.26}\\0.38 \end{tabular}& \begin{tabular}{@{}r@{}}0.35\\ \textbf{0.28}\\0.42 \end{tabular}& \begin{tabular}{@{}r@{}}0.43\\ \textbf{0.31}\\0.45 \end{tabular}& \begin{tabular}{@{}r@{}} \textbf{0.27}\\0.28\\0.41 \end{tabular}& \begin{tabular}{@{}r@{}} \textbf{0.23}\\0.25\\0.32 \end{tabular}& \begin{tabular}{@{}r@{}}0.31\\ \textbf{0.24}\\0.41 \end{tabular}& \begin{tabular}{@{}r@{}}0.35\\ \textbf{0.26}\\0.43 \end{tabular}& \begin{tabular}{@{}r@{}}0.44\\ \textbf{0.30}\\0.47 \end{tabular}& \begin{tabular}{@{}r@{}}0.30\\ \textbf{0.26}\\0.43 \end{tabular}& \begin{tabular}{@{}r@{}} \textbf{0.23}\\0.23\\0.36 \end{tabular}& \begin{tabular}{@{}r@{}}1.00\\ \textbf{1.49}\\ 0.96 \end{tabular}\\ \hline
Speedup & \begin{tabular}{@{}r@{}}1.00\\ \textbf{1.27}\\ 0.77 \end{tabular}& \begin{tabular}{@{}r@{}}1.00\\ \textbf{1.38}\\ 0.82 \end{tabular}& \begin{tabular}{@{}r@{}}1.00\\ \textbf{1.50}\\ 0.93 \end{tabular}& \begin{tabular}{@{}r@{}}1.00\\ \textbf{1.24}\\ 0.76 \end{tabular}& \begin{tabular}{@{}r@{}}1.00\\ \textbf{1.20}\\ 0.74 \end{tabular}& \begin{tabular}{@{}r@{}}1.00\\ \textbf{1.16}\\ 0.79 \end{tabular}& \begin{tabular}{@{}r@{}}1.00\\ \textbf{1.26}\\ 0.85 \end{tabular}& \begin{tabular}{@{}r@{}}1.00\\ \textbf{1.40}\\ 0.97 \end{tabular}& \begin{tabular}{@{}r@{}}1.00\\ \textbf{1.13}\\ 0.78 \end{tabular}& \begin{tabular}{@{}r@{}}1.00\\ \textbf{1.05}\\ 0.76 \end{tabular}& \begin{tabular}{@{}r@{}}1.00\\ \textbf{1.25}\\ 0.81 \end{tabular}& \begin{tabular}{@{}r@{}}1.00\\ \textbf{1.34}\\ 0.81 \end{tabular}& \begin{tabular}{@{}r@{}}1.00\\ \textbf{1.47}\\ 0.94 \end{tabular}& \begin{tabular}{@{}r@{}}1.00\\ \textbf{1.15}\\ 0.77 \end{tabular}& \begin{tabular}{@{}r@{}}1.00\\ \textbf{1.07}\\ 0.74 \end{tabular}\\
    \lasthline
  \end{tabular}
\end{table*}

\subsection{平行區間查找}

每一次有 $n$ 個元素和 $n$ 組詢問,針對這種特殊性質的問題,我們運行樸素的 \texttt{CORMQ} (cache-oblivious RMQ) 得到效能改善,搭配可預測的分析降低運算量 (參照 \texttt{CORMQ-opt}),得到更好的改善。在 \texttt{CORMQ-opt} 策略中,得到 $2.35 \times$ 倍的加速,結果如表 ~\ref{tlb:CORMQ}。

\begin{table*}[!thb]
  %\tiny
  \centering
  \begin{tabular}{l c c c c}
    \firsthline
      & \multicolumn{4}{c}{$N$} \\
      \cline{2-5}
        & \multicolumn{2}{c}{$30000$} & $50000$ & $100000$ \\
      $L$ & $2^{10}$ & $2^{15}$ & $2^{15}$ & $2^{15}$ \\
      \hline
      parallel-\tt{RMQ}     & $903$ & $1516$ & $1874$ & $4116$ \\
      parallel-\tt{CORMQ}   & $995$ & $1475$ & $1689$ & $2594$ \\
      parallel-\tt{CORMQ-opt} & $843$ & $1373$ & $1136$ & $1745$ \\
      \hline
      Speedup & $1.07\times$ & $1.10\times$ & $1.64\times$ & $2.35\times$\\
    \lasthline
  \end{tabular}
  \caption{Total running time (ms) for finding RMQ of different sizes $N$ and maximum interval sizes $L$.}
  \label{tlb:CORMQ}
\end{table*}