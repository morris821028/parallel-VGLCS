\section{實驗結果}
\label{sec:Experiment}

我們運行在 Intel Xeon E5-2620 2.40 GHz 主機上,其擁有 L1 cache 384 KiB、L2 cache 1536 KiB 和 L3 cache 15 MiB。

\subsection{空間壓縮 VGLCS}

我們運行優化策略中的空間壓縮版本,而非理論分析的 $\theta(1)$ 操作,單次詢問落在 $O(s)$ 中,在實作上由於可以完全壓在暫存器上操作,效能表現較佳。

\begin{figure*}[!thb]
  \centering
  \subfigure[Runtime]{
    \includegraphics[width=0.45\linewidth]{graphics/fig-parallel-n.pdf}
    \label{fig:fig-parallel}
  }
  \subfigure[Scalability]{
    \includegraphics[width=0.45\linewidth]{graphics/fig-parallel-p.pdf}
    \label{fig:fig-parallel-scala}
  }
  \caption{Serial and Parallel Algorithm run on E5-2620}
\end{figure*}

從圖 ~\ref{fig:fig-parallel-scala} 看出,平行 VGLCS 的縮放性不佳,從 profiler 中得到快取未中的問題需要解決。

%\subsection{理論常數 VGLCS}

%尚未完成

\subsection{增加後綴區間查找 ISMQ}

針對插入和詢問次數相同的 ISMQ 問題,運行以下四種數據結構:

\begin{itemize}
  \item 并查集 (Disjoint Set): 平均運行時間 $o(\alpha(n))$。只使用路徑壓縮技巧。
  \item 稀疏表 (Sparse Table): 插入 $O(\log n)$、詢問 $O(1)$。實作陣列宣告採用 $\tt{table}[\log N][N]$ 以減少快取未中。
  \item 樹狀數組 (Binary Indexed Tree): 插入、詢問均為 $O(\log n)$。
  \item 壓縮稀疏表 (Compressed Sparse Tree): 插入均攤 $O(1)$、詢問操作 $O(s)$,其中 $s$ 為拆分到區塊大小。實作時,維護區塊前綴和後綴最大值降低詢問複雜度至 $O(1)$,當發生 in-block 詢問再運行 $O(s)$ 算法。
\end{itemize}

當運行 $n > 10^6$ 時,我們提出的壓縮稀疏表的效能已經勝過并查集的版本,其運行結果如圖表 ~\ref{fig:fig-ISMQcmp}。在 $n = 10^7$ 時,加速 $1.25 \times$。然而,我們提供的 amortized $\theta(1)$ 的稀疏表慢於并查集,我們做了深入的機率探討 (參照表 ~\ref{tlb:ISMQcmp}),由於大部分的操作都被區塊後綴和前綴解決,沒有實際運用到內部詢問,約束區間詢問的大小為 $L$,在 $N = 10^7$ 時,最多能加速 $1.26 \times$,其中插入和詢問比例為 1:10,當詢問比重更大時,將有更明顯的加速。

\begin{figure*}[!thb]
  \centering
  \includegraphics[width=0.90\linewidth]{graphics/fig-ISMQ.pdf}
  \caption{ISMQ runs on E5-2620 with different data structures. We use the random test cases without any limitation, e.g. the length of interval query has a uniform distribution.}
  \label{fig:fig-ISMQcmp}
\end{figure*}

\begin{table*}
  \normalsize
  \caption{Total running time (second) for ISMQ with sizes $N = 10^7$, different maximum interval sizes $L$, the probability $p$ of incremental elements and the probility $q$ of zero elements. There are three strategies for each $(p, q, L)$, implemented by disjoint set, compressed sparse table, and O(1) sparse table.}
  \label{tlb:ISMQcmp}
  \centering
  \setlength\tabcolsep{0pt}
  \begin{tabular}{@{\extracolsep{4pt}}r c c c c c c c c c c c c c c c c}
    \firsthline
      & \multicolumn{5}{c}{$L=4$} & \multicolumn{5}{c}{$L=8$} & \multicolumn{5}{c}{$L=16$}\\
      \cline{2-6} \cline{7-11} \cline{12-16}
      $q$ & $0\%$ & $25\%$ & $50\%$ & $75\%$ & $100\%$ 
        & $0\%$ & $25\%$ & $50\%$ & $75\%$ & $100\%$ 
        & $0\%$ & $25\%$ & $50\%$ & $75\%$ & $100\%$ 
        & Speedup\\
      $p$ \\
      \hline
$0\%$ & \begin{tabular}{@{}r@{}} \textbf{1.72}\\1.95\\1.91 \end{tabular}& \begin{tabular}{@{}r@{}} \textbf{1.48}\\1.74\\1.78 \end{tabular}& \begin{tabular}{@{}r@{}} \textbf{1.48}\\1.74\\1.70 \end{tabular}& \begin{tabular}{@{}r@{}} \textbf{1.48}\\1.74\\1.75 \end{tabular}& \begin{tabular}{@{}r@{}} \textbf{1.48}\\1.74\\1.79 \end{tabular}& \begin{tabular}{@{}r@{}} \textbf{1.62}\\2.20\\1.78 \end{tabular}& \begin{tabular}{@{}r@{}} \textbf{1.62}\\2.20\\1.69 \end{tabular}& \begin{tabular}{@{}r@{}} \textbf{1.62}\\2.19\\1.72 \end{tabular}& \begin{tabular}{@{}r@{}} \textbf{1.62}\\2.20\\1.75 \end{tabular}& \begin{tabular}{@{}r@{}} \textbf{1.62}\\2.19\\1.71 \end{tabular}& \begin{tabular}{@{}r@{}} \textbf{1.83}\\2.23\\1.87 \end{tabular}& \begin{tabular}{@{}r@{}} \textbf{1.81}\\2.23\\1.88 \end{tabular}& \begin{tabular}{@{}r@{}} \textbf{1.81}\\2.23\\1.88 \end{tabular}& \begin{tabular}{@{}r@{}} \textbf{1.81}\\2.23\\1.87 \end{tabular}& \begin{tabular}{@{}r@{}} \textbf{1.81}\\2.23\\1.84 \end{tabular}& 0.98\\ \hline
$20\%$ & \begin{tabular}{@{}r@{}} \textbf{1.56}\\2.13\\2.04 \end{tabular}& \begin{tabular}{@{}r@{}} \textbf{1.57}\\2.14\\2.05 \end{tabular}& \begin{tabular}{@{}r@{}} \textbf{1.59}\\2.15\\2.06 \end{tabular}& \begin{tabular}{@{}r@{}} \textbf{1.57}\\2.16\\2.05 \end{tabular}& \begin{tabular}{@{}r@{}} \textbf{1.54}\\2.10\\1.99 \end{tabular}& \begin{tabular}{@{}r@{}} \textbf{1.93}\\3.31\\2.14 \end{tabular}& \begin{tabular}{@{}r@{}} \textbf{2.05}\\3.28\\2.15 \end{tabular}& \begin{tabular}{@{}r@{}}2.26\\3.21\\ \textbf{2.16} \end{tabular}& \begin{tabular}{@{}r@{}} \textbf{2.05}\\3.29\\2.16 \end{tabular}& \begin{tabular}{@{}r@{}} \textbf{1.78}\\3.32\\2.11 \end{tabular}& \begin{tabular}{@{}r@{}}2.66\\3.60\\ \textbf{2.31} \end{tabular}& \begin{tabular}{@{}r@{}}2.77\\3.55\\ \textbf{2.31} \end{tabular}& \begin{tabular}{@{}r@{}}2.92\\3.44\\ \textbf{2.32} \end{tabular}& \begin{tabular}{@{}r@{}}2.67\\3.57\\ \textbf{2.31} \end{tabular}& \begin{tabular}{@{}r@{}}2.45\\3.69\\ \textbf{2.28} \end{tabular}& \textbf{1.26}\\ \hline
$40\%$ & \begin{tabular}{@{}r@{}} \textbf{1.62}\\2.40\\2.12 \end{tabular}& \begin{tabular}{@{}r@{}} \textbf{1.65}\\2.46\\2.14 \end{tabular}& \begin{tabular}{@{}r@{}} \textbf{1.72}\\2.47\\2.16 \end{tabular}& \begin{tabular}{@{}r@{}} \textbf{1.65}\\2.46\\2.14 \end{tabular}& \begin{tabular}{@{}r@{}} \textbf{1.59}\\2.33\\2.00 \end{tabular}& \begin{tabular}{@{}r@{}} \textbf{2.21}\\4.05\\2.33 \end{tabular}& \begin{tabular}{@{}r@{}}2.46\\4.00\\ \textbf{2.34} \end{tabular}& \begin{tabular}{@{}r@{}}2.81\\3.79\\ \textbf{2.36} \end{tabular}& \begin{tabular}{@{}r@{}} \textbf{2.29}\\4.04\\2.35 \end{tabular}& \begin{tabular}{@{}r@{}} \textbf{1.81}\\4.04\\2.25 \end{tabular}& \begin{tabular}{@{}r@{}}2.81\\4.40\\ \textbf{2.50} \end{tabular}& \begin{tabular}{@{}r@{}}2.94\\4.22\\ \textbf{2.54} \end{tabular}& \begin{tabular}{@{}r@{}}3.04\\3.95\\ \textbf{2.53} \end{tabular}& \begin{tabular}{@{}r@{}}2.67\\4.37\\ \textbf{2.52} \end{tabular}& \begin{tabular}{@{}r@{}} \textbf{2.38}\\4.63\\2.46 \end{tabular}& \textbf{1.20}\\ \hline
$60\%$ & \begin{tabular}{@{}r@{}} \textbf{1.65}\\2.45\\2.15 \end{tabular}& \begin{tabular}{@{}r@{}} \textbf{1.71}\\2.58\\2.17 \end{tabular}& \begin{tabular}{@{}r@{}} \textbf{1.82}\\2.67\\2.19 \end{tabular}& \begin{tabular}{@{}r@{}} \textbf{1.69}\\2.52\\2.17 \end{tabular}& \begin{tabular}{@{}r@{}} \textbf{1.60}\\2.24\\2.03 \end{tabular}& \begin{tabular}{@{}r@{}} \textbf{2.29}\\4.30\\2.38 \end{tabular}& \begin{tabular}{@{}r@{}}2.60\\4.31\\ \textbf{2.41} \end{tabular}& \begin{tabular}{@{}r@{}}2.99\\4.08\\ \textbf{2.45} \end{tabular}& \begin{tabular}{@{}r@{}} \textbf{2.20}\\4.36\\2.44 \end{tabular}& \begin{tabular}{@{}r@{}} \textbf{1.75}\\3.98\\2.28 \end{tabular}& \begin{tabular}{@{}r@{}}2.97\\4.67\\ \textbf{2.55} \end{tabular}& \begin{tabular}{@{}r@{}}3.05\\4.51\\ \textbf{2.59} \end{tabular}& \begin{tabular}{@{}r@{}}3.12\\4.22\\ \textbf{2.63} \end{tabular}& \begin{tabular}{@{}r@{}}2.67\\4.81\\ \textbf{2.61} \end{tabular}& \begin{tabular}{@{}r@{}} \textbf{2.33}\\4.56\\2.47 \end{tabular}& \textbf{1.22}\\ \hline
$80\%$ & \begin{tabular}{@{}r@{}} \textbf{1.62}\\2.31\\2.12 \end{tabular}& \begin{tabular}{@{}r@{}} \textbf{1.68}\\2.53\\2.14 \end{tabular}& \begin{tabular}{@{}r@{}} \textbf{1.82}\\2.74\\2.19 \end{tabular}& \begin{tabular}{@{}r@{}} \textbf{1.64}\\2.38\\2.14 \end{tabular}& \begin{tabular}{@{}r@{}} \textbf{1.55}\\1.97\\2.01 \end{tabular}& \begin{tabular}{@{}r@{}} \textbf{2.19}\\4.00\\2.29 \end{tabular}& \begin{tabular}{@{}r@{}}2.57\\4.22\\ \textbf{2.35} \end{tabular}& \begin{tabular}{@{}r@{}}3.00\\4.15\\ \textbf{2.42} \end{tabular}& \begin{tabular}{@{}r@{}} \textbf{2.01}\\4.15\\2.38 \end{tabular}& \begin{tabular}{@{}r@{}} \textbf{1.69}\\3.13\\2.16 \end{tabular}& \begin{tabular}{@{}r@{}}2.99\\4.40\\ \textbf{2.46} \end{tabular}& \begin{tabular}{@{}r@{}}3.04\\4.42\\ \textbf{2.52} \end{tabular}& \begin{tabular}{@{}r@{}}3.15\\4.29\\ \textbf{2.62} \end{tabular}& \begin{tabular}{@{}r@{}}2.59\\4.65\\ \textbf{2.57} \end{tabular}& \begin{tabular}{@{}r@{}} \textbf{2.11}\\3.64\\2.34 \end{tabular}& \textbf{1.24}\\ \hline
$100\%$ & \begin{tabular}{@{}r@{}} \textbf{1.58}\\2.09\\2.03 \end{tabular}& \begin{tabular}{@{}r@{}} \textbf{1.61}\\2.33\\2.09 \end{tabular}& \begin{tabular}{@{}r@{}} \textbf{1.68}\\2.62\\2.16 \end{tabular}& \begin{tabular}{@{}r@{}} \textbf{1.57}\\2.07\\2.08 \end{tabular}& \begin{tabular}{@{}r@{}} \textbf{1.52}\\1.79\\1.79 \end{tabular}& \begin{tabular}{@{}r@{}} \textbf{1.68}\\3.29\\1.99 \end{tabular}& \begin{tabular}{@{}r@{}} \textbf{1.93}\\3.66\\2.04 \end{tabular}& \begin{tabular}{@{}r@{}}2.42\\3.90\\ \textbf{2.10} \end{tabular}& \begin{tabular}{@{}r@{}} \textbf{1.72}\\3.13\\2.04 \end{tabular}& \begin{tabular}{@{}r@{}} \textbf{1.68}\\2.19\\1.81 \end{tabular}& \begin{tabular}{@{}r@{}}2.22\\3.71\\ \textbf{2.19} \end{tabular}& \begin{tabular}{@{}r@{}}2.65\\3.80\\ \textbf{2.25} \end{tabular}& \begin{tabular}{@{}r@{}}2.87\\3.96\\ \textbf{2.33} \end{tabular}& \begin{tabular}{@{}r@{}} \textbf{2.06}\\3.50\\2.25 \end{tabular}& \begin{tabular}{@{}r@{}} \textbf{1.82}\\2.25\\2.02 \end{tabular}& \textbf{1.23}\\ \hline
Speedup & 0.90& 0.83& 0.87& 0.84& 0.85& 0.96& \textbf{1.09}& \textbf{1.24}& 0.98& 0.95& \textbf{1.21}& \textbf{1.21}& \textbf{1.26}& \textbf{1.16}& \textbf{1.07}\\

    \lasthline
  \end{tabular}
\end{table*}

\subsection{平行區間查找}

每一次有 $n$ 個元素和 $n$ 組詢問,針對這種特殊性質的問題,我們運行樸素的 \texttt{CORMQ} (cache-oblivious RMQ) 得到效能改善,搭配可預測的分析降低運算量 (參照 \texttt{CORMQ-opt}),得到更好的改善。在 \texttt{CORMQ-opt} 策略中,得到 $2.35 \times$ 倍的加速,結果如表 ~\ref{tlb:CORMQ}。

\begin{table*}[!thb]
  %\tiny
  \centering
  \begin{tabular}{l c c c c}
    \firsthline
      & \multicolumn{4}{c}{$N$} \\
      \cline{2-5}
        & \multicolumn{2}{c}{$30000$} & $50000$ & $100000$ \\
      $L$ & $2^{10}$ & $2^{15}$ & $2^{15}$ & $2^{15}$ \\
      \hline
      parallel-\tt{RMQ}     & $903$ & $1516$ & $1874$ & $4116$ \\
      parallel-\tt{CORMQ}   & $995$ & $1475$ & $1689$ & $2594$ \\
      parallel-\tt{CORMQ-opt} & $843$ & $1373$ & $1136$ & $1745$ \\
      \hline
      Speedup & $1.07\times$ & $1.10\times$ & $1.64\times$ & $2.35\times$\\
    \lasthline
  \end{tabular}
  \caption{Total running time (ms) for finding RMQ of different sizes $N$ and maximum interval sizes $L$.}
  \label{tlb:CORMQ}
\end{table*}