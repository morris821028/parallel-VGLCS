\section{結論}
\label{sec:Conclusion}

我們修改 VGLCS 的序列算法,將其平行化於 $\theta(n \log n)$ 時間內,並以稀疏表實作 ISMQ 問題。提出的稀疏表能解決比 VGLCS 更困難的 Variable Interval Gapped LCS,致使 VIGLCS 可在時間複雜度 $\theta(nm)$ 被解決。

在實務上,我們提供以動態規劃減少計算量,以及使用空間壓縮降低快取未中的策略,最終平行 RMQ 獲取 $2.35 \times$ 倍的加速;在增長區間最大值詢問 (IRMQ) 問題中,以字典順序的編碼策略,提出理論 $\theta(n)$ -- amortized $\theta(1)$  的算法。