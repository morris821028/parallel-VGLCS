\section{平行化序列算法} %
\label{sec:parallelSerial}

在 $O(nm \alpha(n))$ 的序列算法 \ref{alg:serial-VGLCS} 中,
我們發現算法如大多數的變型 LCS 相同,依賴數個狀態以轉移當前狀態,
大量的資料依賴性不易於細粒度平行。使用波前運行平行是一種常見的解決方案,
由於這種平行對於運行時的快取不友善 (cache-unfriendly),
在 Saeed Maleki ~\cite{saeed} 論文中提到如何使用 Rank Convergence 的特殊性質,
拓展出更高平行度來解決動態規劃的相關問題。

\begin{algorithm*}[!thb]
  \caption{Algorithm for Finding VGLCS}
  \label{alg:serial-VGLCS}
  \begin{algorithmic}[1]
    \Require
      $A, B$: the input string;
      $G_A, G_B$: the array of variable gapped constraints;
    \Ensure Find the LCS with variable gapped constraints
    \State Create $m$ number of data structure $Q[m]$ to support ISMQ problem.
    \State Create an empty table $V[n][m]$.
    \For{$i \gets 1$ to $n$}
      \State Create a data structure $RQ$ to support ISMQ problem.
      \State $r \gets i - (GA[i]+1)$
      \For{$j \gets 1$ to $m$}
        \If{$A[i] = B[j]$}
            \State $t \gets $ query suffix maximum value from position $j - (GB[j]+1)$ to tail in $RQ$.
            \State $V[i][j] \gets t + 1$
            \State $t \gets $ get the suffix maximum value from position $r$ to $i$ in $Q[j]$
            \State Append value $t$ into $RQ$.
            \State Append value $V[i][j]$ into $Q[j]$.
        \Else
            \State $V[i][j] \gets 0$
            \State $t \gets $ get the suffix maximum value from position $r$ to $i$ in $Q[j]$
            \State Append value $t$ into $RQ$.
        \EndIf
      \EndFor
    \EndFor
    \State Retrieve the VGLCS by tracing $V[n][m]$
  \end{algorithmic}
\end{algorithm*}

序列算法的空間複雜度為 $O(nm)$。若使用波前平行,需要同時維護橫向的所有狀態,
需要多付出一倍的空間量。若加入 Rank Convergence 的想法拓展出,
勢必要記錄轉移的狀態,需要耗費更多的記憶體空間,用以在最後階段合併所用。

這裡我們傾向空間複雜度常數小且針對快取友善設計算法。
平行算法主要分成兩個階段-縱向和橫向階段,縱向階段為數個列的後綴極值查找,
橫向階段在行上運行 $n$ 個元素和 $n$ 組詢問。
在橫向階段,我們需要解決增長後綴最大值查找 (\emph{incremental suffix maximum query}, ISMQ)
易於實作的并查集支持單一操作 $O(\alpha(n))$。
然而,在過程中每插入一個元素便改動數據結構以支持下一個後綴詢問,
這部分使得查詢難以平行化。為消除資料相依性,我們找到幾種區間詢問的替代方案。如:

\begin{itemize}
  \item 樹狀數組 (Binary Indexed Tree) -- $O(\log n)$: 對於任意前綴查找極值和更新元素,單一操作的時間複雜度為 $O(\log n)$,其運行常數比線段樹低。
  \item 線段樹 (Segment Tree) -- $O(\log n)$: 支持更高維度的正交區塊搜索,而我們用在區間極值查找需要 $O(\log n)$ 的時間完成所有區間查詢操作。
  \item 稀疏表 (Sparse Table) -- $O(n)$ -- $O(1)$:
    建立表格 $ST[j][i]$ 表示區間 $(i-2^j,i]$ 之間的極值。建表時間複雜度需 $O(n)$
    ,對於任意區間詢問可以拆分 2 個 super-block 檢索和 2 個 in-block 檢索,
    如圖 \ref{fig:interval-decomposition} 的說明,轉換過程和存取時間皆需要 $O(1)$。
\end{itemize}

\begin{figure*}[!thb]
  \centering
  \includegraphics[width=\linewidth]{graphics/fig-interval-decomposition.pdf}
  \includegraphics[width=\linewidth]{graphics/fig-sparse-table.pdf}
  \caption{給定一陣列 $A$ 如上圖所述,並且拆成 5 個區塊,每個區塊皆有 4 個元素,
  若詢問區間 $[2, 18]$ 的最大值,將分成 $B1$ 的內部詢問 (in-block query) $Q_L$、
  $B5$ 的內部詢問 $Q_R$ 和兩個跨區間詢問 (super-block query) $SQ_L$、$SQ_R$}
  \label{fig:interval-decomposition}
\end{figure*}

稀疏表是我們認為最好的替代方案,其整合後為 VGLCS 平行算法 \ref{alg:parallel-VGLCS},
算法的時間複雜度為 $O(n^2 / p + n \log n)$,其中 $p$ 為處理器個數。
在後續的章節,我們將提出新的數據結構取代并查集操作,
且能在平行算法達到理想複雜度 $O(n^2 / p + n \log n)$。

\begin{algorithm*}[!thb]
  \caption{Parallel Algorithm for Finding VGLCS}
  \label{alg:parallel-VGLCS}
  \begin{algorithmic}[1]
    \Require
      $A, B$: the input string;
      $G_A, G_B$: the array of variable gapped constraints;
    \Ensure Find the LCS with variable gapped constraints
    \State Create $m$ number of data structure $Q[m]$ to support ISMQ problem.
    \State Create an empty table $V[n][m]$.
    \For{$i \gets 1$ to $n$}
      \State Create a sparse table data structure $\textit{sp}$, and initialize $\textit{sp}$ to zero.
      \ParFor{$j \gets 1$ to $m$}
        \State $\textit{sp}[j] \gets$ query suffix maximum value from position $r$ to tail in $Q[j]$.
      \EndParFor
      \State Build sparse table $\textit{sp}$ with $m$ elements in parallel $O(n/p \log n + \log n)$ time.
      \ParFor{$j \gets 1$ to $m$}
        \If{$A[i] = B[j]$}
            \State $t \gets $ query suffix maximum value from position $j - (GB[j] + 1)$ to $j-1$ in $\textit{sp}$
            \State $V[i][j] \gets t + 1$
            \State Append value $V[i][j]$ into $Q[j]$.
        \EndIf
      \EndParFor
    \EndFor
    \State Retrieve the VGLCS by tracing $V[n][m]$
  \end{algorithmic}
\end{algorithm*}