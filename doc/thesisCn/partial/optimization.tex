\section{實作細節}
\label{sec:Implementation}

\subsection{平行并查集操作策略}

運行 VGLCS 時,將耗費 $\theta(n^2)$ 的內存空間。使用遞增後綴最大值 (ISMQ) 時,採用並查集實作將會遭遇到很多不平衡的工作負載,其原因在於合併的策略,常見的有路徑壓縮和啟發式合併兩種策略,這間接影響到不同次數的分枝判斷。實務上須考慮到快取未中,故兩種策略只能擇其一,兩者皆用將引發更多的快取未中而導致效能下滑。

每個執行緒負責數個完整的并查集,操作時應偏向延遲標記操作,儘早合併的策略易造成快取未中。由於動態規劃的傾向中,插入值的趨勢有兩種情況,其一為連續不合定義的零元素插入,其二為遞增元素的插入,在這兩者穿插的趨勢中,我們發現延遲操作將會帶來較能改善快取未中問題。

\subsection{平行區間查找策略}

運行區間查找時,一般依賴內建函數在 $O(1)$ 時間完成對數取整,然而,在 VGLCS 這類型的動態規劃中,區間查找的對數結果是可以被預測的,預先將每一組詢問的區段對數結果儲存在陣列中,便可降低指令次數。

由於已知所有詢問區間,建立稀疏表時,可藉由動態規劃在 $O(n \log n)$ 排除掉不可能的計算 (參照算法 ~\ref{alg:reduceBoundary}),降低過程中的計算量。由於 VGLCS 在平行操作需要 $O(n \log n)$,故使用動態規劃不影響我們的最終結果。

\begin{algorithm*}
  \caption{Reduce Boundary Dynamic Programming}
  \label{alg:reduceBoundary}
  \begin{algorithmic}[1]
  \Require
      $G[1 \cdots n]$: the variabled gap contraints;
  \Ensure
      $\text{limD}$: boundary for doubling algorithm;
  \State int8\_t limD[1 ... n]
  \For{$i = n$ to $1$}
    \State limD[i] = max(limD[i], floor(log2(min(G[i]+1, i))))
    \For{$k = 1$ to limD[i]}
      \State limD[$i-2^{k-1}$] = max(limD[$i-2^{k-1}$], $k-1$);
    \EndFor
  \EndFor
  \State return limD
  \end{algorithmic}
\end{algorithm*}

從機率分佈的角度來看,因 $s = \frac{1}{4} \log n$ 過小,區間詢問完全落於 block 的機率低,故額外維護區段前綴和後綴最大值 (prefix/suffix maximum value in block) 取代笛卡爾樹的建立。藉這兩個額外儲存空間,將會增加空間複雜度的常數,卻能有效地降低整體的指令次數。