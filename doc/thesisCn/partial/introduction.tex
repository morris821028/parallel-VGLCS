\section{介紹} %Introduction
\label{sec:Introduction}

最長共同子序列 (\emph{longest common subsequence}, LCS)廣泛地使用在各個應用上。在多核心平台下,大多數的研究專注於如何高效率地在波前平行 (wavefront parallelism),而 Jiaoyun Yang ~\cite{jiaoyun} 提出的論文中改變一般的 LCS 遞迴定義以得到更好快取使用率。這裡,我們使用相關的理論來改善在 Iliopoulos ~\cite{iliopoulos} 提及的約束條件下的 LCS,如 \emph{fixed gap LCS } (FGLCS)要求任兩個挑選的距離在相對應的另一個字串中相等,同時距離最大為 $k+1$,可在時間複雜度在 $O(nm)$ 內解決,其中 $n$, $m$ 分別為兩個輸入的字串長度。

在眾多的約束條件類型中,我們將在這篇論文針對 \emph{variable gap LCS} (VGLCS) 進行探討。在 VGLCS 中,對各個不同的位置提供約束限制,如目前給定兩個字串 $A = \tt{RCLPCRR}$, $B = \tt{RPPLCPLRC}$,各自的約束限制為 $G_A = [2, 3, 0, 0, 3, 2, 2]$ 和 $G_B = [2, 0, 0, 0, 3, 0, 0, 2, 3]$,其中 $G_A(i)$ 表示當挑選第 $i$ 個位置時,與前一個挑選的位置最多差 $G_A(i)+1$,挑選的方式如圖 ~\ref{fig:VGLCSex}。這個問題已在 Yung-Hsing Peng ~\cite{yunghsing} 的論文針對 VGLCS 提出 $O(nm \alpha(n))$ 的解法。

這一篇論文,我們將在第二 \ref{sec:parallelSerial} 節部分將 Yung-Hsing Peng ~\cite{yunghsing} 提出的算法進行平行化。在第三節 ~\ref{sec:parallelRMQ},在理論分析上提供易平行且時間複雜度 $O(nm)$ 的設計。在第四節 ~\ref{sec:Implementation},我們將藉由快取忘卻 (cache-oblivious) 技術,在實作上提供更好的效能。最後,我們總結實驗結果與理論實務上的差異。

\begin{figure}[!thb]
  \centering
  \includegraphics[width=\linewidth]{graphics/fig-VGLCSex.pdf}
  \includegraphics[width=\linewidth]{graphics/fig-VGLCSex2.pdf}
  \caption{VGLCS 於兩個序列 $A = \tt{RCLPCRR}$, $B = \tt{RPPLCPLRC}$,各自的約束限制為 $G_A = [2, 3, 0, 0, 3, 2, 2]$ 和 $G_B = [2, 0, 0, 0, 3, 0, 0, 2, 3]$,的其中幾個可挑選的方案}
  \label{fig:VGLCSex}
\end{figure}