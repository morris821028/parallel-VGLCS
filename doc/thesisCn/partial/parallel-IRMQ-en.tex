\section{Range Maximum Query}
\label{sec:parallelRMQ}

\subsection{Background}

Even through we parallel origin serial algorithm successfully, the
theoretical time complexity of VGLCS algorithm is limited by the range
maximum/minimum query in parallel.  In VGLCS problem, each stage has
$n$ elements and $n$ numbers of range query.  We should let pre-
processing and query time minimize.  The most of tree structures
cannot show efficient performance in both pre-processing and query
time.  Some offline algorithm is too hard to parallel.  In this
section, we provide the solution to get better performance in both
pre-processing and query stage.

\subsection{Compressed Cartesian Tree}

In Fischer ~\cite{fischer} paper, the $O(n)$ -- $O(1)$ algorithm is
corrponding by Catalan number $\frac{1}{s+1}\binom{2s}{s} =
O(\frac{4^s}{s^{1.5}})$ to build look-up table.  When we choose $s =
\frac{1}{4} \log n$ as block size, the space complexity is $O(s^2
\frac{4^s}{s^{1.5}}) = o(n)$, and time complexity is $o(n)$.  Each
range query will be split into 4 parts, 2 super-block queries and 2
in-block queries. It need to 4 time memory access.  We give a example
in the figure ~\ref{fig:interval-decomposition}.  In the offline RMQ
problem, it has the theoretical algorithm which run in $O(n)$ --
$O(1)$ time.

When $n$ is large, four time memory access caused serious cache miss.
In order to improve cache miss, Demaine introduced the cache-oblivious
algorithm in Cartesian tree ~\cite{demaine}.

In above technology, we parallel RMQ problem by Fischer's idea and get
time complexity $O(n / p + \log n)$ -- $O(1)$ algorithm.  We also
combines compression skill from Demaine's paper. It reduces cache-miss
and run in ideal complexity.

We pick the fixed length $s = 16$, which can solve $n = 2^{64}$ one-
dimension range maximum query.  When inserting $i$-th elements, the
number of $i$-th left rotation $l_i$ must satisfy $\sum_{i=1}^{n} l_i
< i$.  Because all $l_i$ is small than 16, it can present in 4-bit
integer.  Due to above property of Cartesian tree, we merger 16 4-bit
integers into a 64-bit integer to present a Caartesian tree.  The
compressed algorithm ~\ref{alg:cartesian-to-64bits} run in $O(s)$.

Finally, the appropriate size can compress the usage of space to
reduce cache-miss and also show better performance in modern 64-bit
register.  We modify Demaine's range query algorithm as the algorithm
~\ref{alg:cartesian64bits-query}.

\begin{algorithm*}
  \caption{Transfer Cartesian Tree to 64-bits with 8 integers}
  \label{alg:cartesian-to-64bits}
  \begin{algorithmic}[1]
    \Require
      $A[1 \cdots 16]$: 16 integers store in array
    \Ensure 
      $\textit{tmask}$: compress Cartesian tree into 64-bits integer
      \State $\tt{LOGS} = 4$
      \State $\tt{POWS} = 2^{\tt{LOGS}}$
      \State int $D[POWS+1]$, $Dp = 0$;
      \State uint64\_t $tmask$ = $0$
      \State $D[0]$ = SHRT\_MAX
      \For{$i = 1$ to $\tt{POWS}$} 
        \State $v = A[i]$;
        \State $cnt = 0$;
        \While{$D[Dp] < v$}
          \State $Dp = Dp-1$
          \State $cnt = cnt + 1$
        \EndWhile
        \State $Dp = Dp+1$
        \State $D[Dp] = v$
        \State $tmask = tmask | ((cnt)<<((i-1)<<2))$
      \EndFor
      \State return $tmask$
  \end{algorithmic}
\end{algorithm*}

\begin{algorithm}[!thb]
  \caption{Range Minimum Query in 64-bits Cartesian Tree}
  \label{alg:cartesian64bits-query}
  \begin{algorithmic}[1]
    \Require
      $\textit{tmask}$: 64-bits Cartesian tree;
      $[l, r]$: query range
    \Ensure 
      $\textit{minIdx}$: the index of the minimum value in interval
    \State $\textit{minIdx} \gets l, \; x \gets 0$;
    \For{$l \gets l+1$ to $r$}
      \State $x \gets x+1 - ((\textit{tmask} \gg (l \ll 2)) \mathrel{\&} 15)$
      \If{$x \le 0$}
        \State $\textit{minIdx} \gets l$
        \State $x \gets 0$
      \EndIf
    \EndFor
    \State return $\textit{minIdx}$
  \end{algorithmic}
\end{algorithm}

In VGLCS problem, above algorithm provide compression skill to reduce
cache-miss, but increase the time complexity.  The pre-processing
spend $O(n)$ time, and single query spend $O(s)$ time in RMQ.
Totally, time complexity is $O(n^2 \; s / p + n \max(\log n, s))$.

%%%%%%%%%%%%%%

\section{Incremental Range Maximum Query}

In the VGLCS problem, we divide the algorithm into two stage, row and
colume stage.  In the row stage, it maintains $m$ number of data
structure to support ISMQ problem.  After row stage, the colume stage
use one ISMQ data structure to answer each query.

In parallel environment, the column stage has a linear algorithm which
use Fischer's sparse table instead of disjoint set, and we also
parallel its algorithm in theoretical $O(n / p + \log n)$ time, and
better performance algorithm in $O(n s / p + \log n)$ in section
~\ref{sec:parallelRMQ}.  Oppositely, the row stage has $m$ number of
independent data structure and run in amortized $O(n \alpha(n) / p +
\alpha(n))$ time.  In this section, we provide the new data structure
to make row stage run in amortized $O(n / p + 1)$ time.  The final
result is presented on table ~\ref{tlb:cmp-complexity}.

\begin{table*}
  %\tiny
  \centering
  \begin{table}
  %\tiny
  \centering
  \caption{   Our study shows in the bold front. We use the fixed size
$s=16$ on Cartesian tree. The small amortized constant will not
encounter serious load imbalance problem.   }

  \label{tlb:cmp-complexity}
  \begin{tabular}{ccc}
    \toprule
     & serial & parallel \\
    \midrule
    horizontal & \begin{tabular}{@{}c@{}}
              $\left \langle n \alpha(n) \right \rangle$ \cite{yunghsing} \\ 
              amortized $\left \langle n \right \rangle$\end{tabular}
              & \begin{tabular}{@{}c@{}}
                $\left \langle n \alpha(n)/p + \alpha(n) \right \rangle$ \cite{yunghsing} \\
                amortized $\left \langle n /p + o(1) \right \rangle$
                \end{tabular} \\
    vertical & \begin{tabular}{@{}c@{}}
                $\left \langle n \alpha(n) \right \rangle$ \\
                amortized $\left \langle n \right \rangle$
                \end{tabular}
            & \begin{tabular}{@{}c@{}}
                impossible \\
                amortized $\left \langle n /p + o(1) \right \rangle$
              \end{tabular}
              \\
    total & \begin{tabular}{@{}c@{}}
              $\left \langle n^2 \alpha(n) \right \rangle$ \cite{yunghsing} \\ 
              amortized $\left \langle n^2 \right \rangle$\end{tabular}
          & \begin{tabular}{@{}c@{}}
              impossible \\ 
              amortized $\left \langle n^2 /p + n \log n \right \rangle$\end{tabular} \\
    \bottomrule
  \end{tabular}
\end{table}

  \caption{   Our study shows in the bold front. We use the fixed size
$s=16$ on Cartesian tree. The small amortized constant will not
encounter serious load imbalance problem.   }

  \label{tlb:cmp-complexity}
\end{table*}

\subsection{Build Look-up Table}

We have two main subjects: cartesian tree encoding and cache
performance.  Fischer introduced the first encoding method, and the
Masud presents the new encoding method and processing step to reduce
the number of instructions.  However, most of them are foucs on
offline algorithm.  It means that there are given $n$ elements, and
then has $m$ queries.  During any queries, the $n$ elements will not
be modified by any operation.

We could not use sorting to improve cache miss because the number of
elements and queries are similar.  However, we can still improve the
cache miss by design the special encoding method according to value
distribution.

In our application, we provide the online encoding algorithm.  We list
all binary search tree by lexicographical order, and label them from
$0$ to Catlan $n$-th number.  The lexicographical order for binary
search tree is defined by the left subtree high priority and then
right subtree.  The figure \ref{fig:labelingBST} shows label of binary
search tree for the $n=1,2,3$ number of nodes.

\begin{figure}[!thb]
  \centering
  \includegraphics[width=\linewidth]{graphics/fig-bst-encoding.pdf}
  \caption{The label of each binary search tree}
  \label{fig:labelingBST}
\end{figure}

For the $s$ nodes binary search tree, we label node from $0$ to $s-1$
and the identify $\mathit{tid}$.  We define $\mathit{LCA}(s,
\mathit{tid}, p, q)$ as the lowest common ancestor of the node $p$ and
$q$ on a binary search tree which has $s$ nodes and labeling
$\mathit{tid}$, such as $\mathit{LCA}(3, 2, 0, 2) = 1$.

We define four variable
$\langle\mathit{lsz},\mathit{lid},\mathit{rsz},\mathit{rid}\rangle$,
which $\mathit{lsz}$ is the size of the left subtree, $\mathit{rsz}$
is the size of the right subtree, $\mathit{lid}$ is the identify of
the left subtree, and $\mathit{rid}$ is the identify of the right
subtree.  These four variables
$\langle\mathit{lsz},\mathit{lid},\mathit{rsz},\mathit{rid}\rangle$ is
decided by the identify of binary search tree $\mathit{tid}$ in
$O(s)$.  Finally, the formula \ref{fun:LCA} is shown for the lowest
common ancestor.

In order to store all binary search trees, the space complexity is 

\begin{equation}
\theta\left(\frac{s^2}{s+1} \binom{2s}{s}\right) = \theta\left(n\right)
\end{equation}

, and the time complexity of parallel algorithm \ref{alg:parallel-LCA}
is

\begin{equation}
\theta\left(\frac{s^3}{s+1} \binom{2s}{s} \bigg/ p + s^2 \right)
\end{equation}.

\begin{figure*}[!thb]
  \begin{equation*}
  \begin{split}
    &\mathit{LCA}(n, \mathit{tid}, p, q) \\
      &= \left\{\begin{matrix*}[l]
        \mathit{LCA}(\mathit{lsz}, \mathit{lid}, p, q) &&, p \le q < \mathit{lsz}\\ 
        \mathit{LCA}(\mathit{rsz}, \mathit{rid}, p-\mathit{lsz}-1, q-\mathit{lsz}-1)+\mathit{lsz}+1 &&, 
            \mathit{lsz} \le p \le q < n \\ 
        \mathit{lsz} && , 0 \le p \le \mathit{lsz}, \mathit{lsz} \le q \le i\\ 
        -1 && ,\mathit{otherwise}
      \end{matrix*}\right.
  \end{split}
\end{equation*}
  \caption{The formula of lowest common ancestor}
  \label{fun:LCA}
\end{figure*}

\begin{algorithm}[!thb]
  \caption{Parallel Algorithm for building LCA}
  \label{alg:parallel-LCA}
  \begin{algorithmic}[1]
    \Require
      $s$: Maximum size for required the number of BST
    \For{$n \gets 1$ to $s$}
      \ParFor{$\mathit{tid} \gets 0$ to $C_n - 1$}
        \ParFor{$p \gets 0$ to $n-1$}
          \State compute $\langle\mathit{lsz},\mathit{lid},\mathit{rsz},\mathit{rid}\rangle$
          \For{$q \gets p$ to $n-1$}
            \State $\textit{LCA}[n][\mathit{tid}][p][q] \gets$ Equation~\ref{fun:LCA1} and \ref{fun:LCA2}
          \EndFor
        \EndParFor
      \EndParFor
    \EndFor
  \end{algorithmic}
\end{algorithm}

\subsection{Dynamic Cartesian Tree}

There are three intuitive solutions for ISMQ.  The first one use
disjoint set which use $O(\alpha(n))$ for each query.  The seoncd one
use sparse table which use $O(\log n)$ time to append a new value to
the tail of array and $O(1)$ time to response a query.  The third one
is binary indexed tree which used $O(\log n)$ time for each operation. 

In the sparse table, Fischer provide the $\theta(n)$ -- $\theta(1)$
could not support append operation.  However, the sparse table support
RMQ problem is more powerful than suffix maximum value problem.  In
this section, we provide a solution to make sparse table to support
append operation.  The problem which is more powerful than ISQM
problem is named incremental range maximum query (IRMQ).  IRMQ support
two kinds of operation as follows: 

\begin{itemize}
  \item 
  	\texttt{Append V} : Append a new value $V$ to the tail of array.

  \item
    \texttt{Query L R} : Query the maximum value between position $L$
and $R$. 

\end{itemize}

Now, we provide the dynamic encoding method so that each operation is
amortized $\theta(1)$ time.  First, we need to fully recognize the
formulas encoder and decoder, so that each step in the algorithm is
$\theta(1)$.

In the previous section, for any identify $\mathit{tid}$ of BST, we
can get
$\langle\mathit{lsz},\mathit{lid},\mathit{rsz},\mathit{rid}\rangle$ in
$O(n)$ time.  Oppositely, we get $\mathit{tid}$ from
$\langle\mathit{lsz},\mathit{lid},\mathit{rsz},\mathit{rid}\rangle$ in
$\theta(1)$.  The algorithm \ref{alg:encode-tid} show the inverse
function run in $\theta(1)$.  By the pre-processing, all the prefix
sum store in the memory, so the for loop can be replaced by one time
memory access in the algorithm \ref{alg:encode-tid}.

\begin{algorithm}
  \caption{Get $tid$ from $\langle\mathit{lsz},\mathit{lid},\mathit{rsz},\mathit{rid}\rangle$ in $\theta(1)$ time}
  \label{alg:encode-tid}
  \begin{algorithmic}[1]
    \Require
      $\langle\mathit{lsz},\mathit{lid},\mathit{rsz},\mathit{rid}\rangle$: size and label in left/right subtree
    \Ensure
      $\mathit{tid}$: this label
    \If{$\mathit{rsz} = 0$}
      \State return $\mathit{lid}$
    \EndIf
    \State $n = \mathit{lsz}+\mathit{rsz}+1$
    \State $\mathit{base} = 0$
    \For{$i=0$ to $\mathit{lsz}-1$}
      \State $\mathit{base}$ = $\mathit{base} + C_i \cdot C_{n-i-1}$
    \EndFor
    \State $\mathit{offset}$ = $\mathit{lid} \cdot C_{\mathit{rsz}}$ + $\mathit{rid}$
    \State return $\mathit{base}$ + $\mathit{offset}$
  \end{algorithmic}
\end{algorithm}

By the recursive formula,  we could implement without storing full
cartesian tree.  The right chain is compsited of root, right child of
root, right child of right child of root, ..., and so on.  We maintain
the right chain of cartesian tree as a stack.  The algorithm \ref{alg
:cartesian-encode-offline} is shown for offline encoding any cartesian
tree.

\begin{algorithm*}
  \caption{Offline Type of Cartesian Tree}
  \label{alg:cartesian-encode-of}
  \begin{algorithmic}[1]
    \Require
      $A[1 \cdots s]$: storage array;
      $s$: the number of elements;
    \Ensure
      $\mathit{tid}$: this label
    \State $\langle\mathit{lsz},\mathit{lid},\mathit{value}\rangle$ $D$[$s+1$]
    \State $\textit{Dp} \gets 0$
    \State $D[0] \gets \langle 0,0,\infty \rangle$
    \For{$i \gets 1$ to $s$}
      \State $v \gets A[i], \; \textit{lsz} \gets 0, \; \textit{lid} \gets 0$
      \While{$D[\textit{Dp}].\textit{value} < v$}
        \State $\textit{lid} \gets \textit{tid}(D[\textit{Dp}].\textit{lsz}, D[\textit{Dp}].\textit{lid}, \textit{lsz}, \textit{lid})$
        \State $\textit{lsz} \gets \textit{lsz} + D[\textit{Dp}].\textit{lsz} + 1$
        \State $\textit{Dp} \gets \textit{Dp} - 1$
      \EndWhile
      \State $\textit{Dp} \gets \textit{Dp} + 1$
      \State $D[\textit{Dp}] \gets \langle\mathit{lsz},\mathit{lid},\mathit{v}\rangle$
    \EndFor
    \State $\textit{lsz} \gets 0, \; \textit{lid} \gets 0$
    \While{$\textit{Dp} > 0$} \Comment{pop all elements}
      \State $\textit{lid} \gets \textit{tid}(D[\textit{Dp}].\textit{lsz}, D[\textit{Dp}].\textit{lid}, \textit{lsz}, \textit{lid})$
      \State $\textit{lsz} \gets \textit{lsz} + D[\textit{Dp}].\textit{lsz} + 1$
      \State $\textit{Dp} \gets \textit{Dp} - 1$
    \EndWhile
    \State return $\textit{lid}$
  \end{algorithmic}
\end{algorithm*}

In order to support online encoding, we use 5 variable to present the
state of cartesian tree.  The next step will insert $i$-th elements,
and final stage fill $s$ number of elements.  The current tree label
is $\mathit{tid}$ and the right chain of the cartesian tree is
presented by two variable, stack pointer $\mathit{Dp}$ and the stack
$\mathit{D}$. The structure of state is as follows:

\begin{figure}[!thb]
  \begin{lstlisting}[frame=single,caption=State of Cartesian Tree]
struct Node {
  int lsz, lid, val;
};
struct State {
  int i, s, tid, Dp;
  struct Node D[s+1];
  State(i = 0, s = n, 
          tid = C[n]-1, Dp = 0,
           D[0].val = INF)
};
  \end{lstlisting}
\end{figure}

For the online query, we choose $s=\frac{\log n}{4}$ by the Fischer's
RMQ.  In our encoding method, we initialize the $s$ number of virtual
node on the right chain, so the default tree label $\mathit{tid}$ is
$C_s - 1$, which $C_s$ is the $s$-th Catalan number.  Following the
elements insertion, we assume the sequence of elements which is not
yet inserted are increasing.

Because of lexicographical order, the right chain of cartesian is
belong to the lower dimension in the row-majar like.  Simultaneously,
building a cartesian tree only modify the right chain.  We can use the
propagation characteristics to get the identify of tree.  Finally, we
design the difference algorithm \ref{alg:cartesian-encode-online} to
satisfy above requirement.

\begin{algorithm}[!thb]
  \caption{Online Type of Cartesian Tree}
  \label{alg:cartesian-encode-online}
  \begin{algorithmic}[1]
  \Require
      $\mathit{state}$: state of Cartesian Tree;
      $v$: the value which append to array
  \Ensure
      $\mathit{tid}$: this label
  \State $\textit{Dp} \gets \textit{state}.\textit{Dp}$, $\textit{lsz} \gets 0$, $\textit{lid} \gets 0$
  \State $\textit{bsz} \gets \textit{state}.\textit{s} - \textit{state}.\textit{i} + 1$
  \State $\textit{bid} \gets C[\textit{bsz}] - 1$
  \While{$\textit{state}.D[\textit{Dp}].\textit{value} < v$}
    \State $\textit{lid} \gets \textit{tid}(\textit{state}.D[\textit{Dp}].\textit{lsz}, \textit{state}.D[\textit{Dp}].\textit{lid}, \textit{lsz}, \textit{lid})$
    \State $\textit{bid} \gets \textit{tid}(\textit{state}.D[\textit{Dp}].\textit{lsz}, \textit{state}.D[\textit{Dp}].\textit{lid}, \textit{bsz}, \textit{bid})$
    \State $\textit{lsz} \gets \textit{lsz} + \textit{state}.D[\textit{Dp}].\textit{lsz}+1$
    \State $\textit{bsz} \gets \textit{bsz} + \textit{state}.D[\textit{Dp}].\textit{lsz}+1$
    \State $\textit{Dp} \gets \textit{Dp} - 1$
  \EndWhile
  \State $\textit{Dp} \gets \textit{Dp} + 1$
  \State $\textit{state}.D[\textit{Dp}] \gets \left \langle \textit{lsz}, \textit{lid}, \textit{v} \right \rangle$
  \State $\textit{state}.\textit{Dp} \gets \textit{Dp}$
  \State $x.\textit{tid} \gets \textit{tid}(\textit{lsz}, \textit{lid}, \textit{state}.s-\textit{state}.i, C[\textit{state}.s-\textit{state}.i]-1)$
  \State $\textit{state}.\textit{tid} \gets \textit{state}.\textit{tid} + \textit{bid} - x.\textit{tid}$
  \State $\textit{state}.i \gets \textit{state}.i + 1$
  \State return $\textit{state}.\textit{tid}$
  \end{algorithmic}
\end{algorithm}

\begin{figure*}[!thb]
  \centering
  \includegraphics[width=\linewidth]{graphics/fig-cartesian-encoding.pdf}

  \caption{
Each block has $s$ number of elements.  We will build a cartesian tree with $s$ number of nodes to solve in-block query.  In initialization, it assume $s$ number of nodes on the right chain and the default tree label $\mathit{tid} = C_s - 1$.
}
  \label{fig:cartesianEncoding}
\end{figure*}





