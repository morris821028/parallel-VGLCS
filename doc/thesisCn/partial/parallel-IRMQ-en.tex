\section{Range Maximum Query}
\label{sec:parallelRMQ}

\subsection{Background}

Even through we parallel origin serial algorithm successfully, the
theoretical time complexity of VGLCS algorithm is limited by the range
maximum/minimum query in parallel.  In VGLCS problem, each stage has
$n$ elements and $n$ numbers of range query. We should let pre-
processing and query time minimize. The most of tree structures cannot
show efficient performance in both pre-processing and query time. Some
offline algorithm is too hard to parallel. In this section, we provide
the solution to get better performance in both pre-processing and
query stage.

\subsection{Compressed Cartesian Tree}

In Fischer ~\cite{fischer} paper, the $O(n)$ -- $O(1)$ algorithm is
corrponding by Catalan number $\frac{1}{s+1}\binom{2s}{s} =
O(\frac{4^s}{s^{1.5}})$ to build look-up table. When we choose $s =
\frac{1}{4} \log n$ as block size, the space complexity is $O(s^2
\frac{4^s}{s^{1.5}}) = o(n)$, and time complexity is $o(n)$. Each
range query will be split into 4 parts, 2 super-block queries and 2
in-block queries. It need to 4 time memory access. We give a example
in the figure ~\ref{fig:interval-decomposition}.  In the offline RMQ
problem, it has the theoretical algorithm which run in $O(n)$ --
$O(1)$ time.

When $n$ is large, four time memory access caused serious cache miss.
In order to improve cache miss, Demaine introduced the cache-oblivious
algorithm in Cartesian tree ~\cite{demaine}.

In above technology, we parallel RMQ problem by Fischer's idea and get
time complexity $O(n / p + \log n)$ -- $O(1)$ algorithm. We also
combines compression skill from Demaine's paper. It reduces cache-miss
and run in ideal complexity.

We pick the fixed length $s = 16$, which can solve $n = 2^{64}$ one-
dimension range maximum query. When we insert $i$-th elements, the
number of $i$-th left rotation $l_i$ must satisfy $\sum_{i=1}^{n} l_i
< i$. Because all $l_i$ is small than 16, it can present in 4-bit
integer.  Due to above property of Cartesian tree, we merger 16 4-bit
integers into a 64-bit integer to present a Caartesian tree. The
compressed algorithm ~\ref{alg:cartesian-to-64bits} run in $O(s)$.

Finally, the appropriate size can compress the usage of space to
reduce cache-miss and also show better performance in modern 64-bit
register.  We modify Demaine's range query algorithm as the algorithm
~\ref{alg:cartesian64bits-query}.

\begin{algorithm*}
  \caption{Transfer Cartesian Tree to 64-bits with 8 integers}
  \label{alg:cartesian-to-64bits}
  \begin{algorithmic}[1]
    \Require
      $A[1 \cdots 16]$: 16 integers store in array
    \Ensure 
      $\textit{tmask}$: compress Cartesian tree into 64-bits integer
      \State $\tt{LOGS} = 4$
      \State $\tt{POWS} = 2^{\tt{LOGS}}$
      \State int $D[POWS+1]$, $Dp = 0$;
      \State uint64\_t $tmask$ = $0$
      \State $D[0]$ = SHRT\_MAX
      \For{$i = 1$ to $\tt{POWS}$} 
        \State $v = A[i]$;
        \State $cnt = 0$;
        \While{$D[Dp] < v$}
          \State $Dp = Dp-1$
          \State $cnt = cnt + 1$
        \EndWhile
        \State $Dp = Dp+1$
        \State $D[Dp] = v$
        \State $tmask = tmask | ((cnt)<<((i-1)<<2))$
      \EndFor
      \State return $tmask$
  \end{algorithmic}
\end{algorithm*}

\begin{algorithm}[!thb]
  \caption{Range Minimum Query in 64-bits Cartesian Tree}
  \label{alg:cartesian64bits-query}
  \begin{algorithmic}[1]
    \Require
      $\textit{tmask}$: 64-bits Cartesian tree;
      $[l, r]$: query range
    \Ensure 
      $\textit{minIdx}$: the index of the minimum value in interval
    \State $\textit{minIdx} \gets l, \; x \gets 0$;
    \For{$l \gets l+1$ to $r$}
      \State $x \gets x+1 - ((\textit{tmask} \gg (l \ll 2)) \mathrel{\&} 15)$
      \If{$x \le 0$}
        \State $\textit{minIdx} \gets l$
        \State $x \gets 0$
      \EndIf
    \EndFor
    \State return $\textit{minIdx}$
  \end{algorithmic}
\end{algorithm}

In VGLCS problem, above algorithm provide compression skill to reduce
cache-miss, but increase the time complexity. The pre-processing spend
$O(n)$ time, and single query spend $O(s)$ time in RMQ. Totally, time
complexity is $O(n^2 \; s / p + n \max(\log n, s))$.

%%%%%%%%%%%%%%

\section{Incremental Range Maximum Query}

In the VGLCS problem, we divide the algorithm into two stage, row and
colume stage. In the row stage, it maintains $m$ number of data
structure to support ISMQ problem. After row stage, the colume stage
use one ISMQ data structure to answer each query.

In parallel environment, the column stage has a linear algorithm which
use Fischer's sparse table instead of disjoint set, and we also
parallel its algorithm in theoretical $O(n / p + \log n)$ time, and
better performance algorithm in $O(n s / p + \log n)$ in section
~\ref{sec:parallelRMQ}.  Oppositely, the row stage has $m$ number of
independent data structure and run in amortized $O(n \alpha(n) / p +
\alpha(n))$ time.  In this section, we provide the new data structure
to make row stage run in amortized $O(n / p + 1)$ time. The final
result is presented on table ~\ref{tlb:cmp-complexity}.

\begin{table*}
  %\tiny
  \centering
  \begin{table}
  %\tiny
  \centering
  \caption{   Our study shows in the bold front. We use the fixed size
$s=16$ on Cartesian tree. The small amortized constant will not
encounter serious load imbalance problem.   }

  \label{tlb:cmp-complexity}
  \begin{tabular}{ccc}
    \toprule
     & serial & parallel \\
    \midrule
    horizontal & \begin{tabular}{@{}c@{}}
              $\left \langle n \alpha(n) \right \rangle$ \cite{yunghsing} \\ 
              amortized $\left \langle n \right \rangle$\end{tabular}
              & \begin{tabular}{@{}c@{}}
                $\left \langle n \alpha(n)/p + \alpha(n) \right \rangle$ \cite{yunghsing} \\
                amortized $\left \langle n /p + o(1) \right \rangle$
                \end{tabular} \\
    vertical & \begin{tabular}{@{}c@{}}
                $\left \langle n \alpha(n) \right \rangle$ \\
                amortized $\left \langle n \right \rangle$
                \end{tabular}
            & \begin{tabular}{@{}c@{}}
                impossible \\
                amortized $\left \langle n /p + o(1) \right \rangle$
              \end{tabular}
              \\
    total & \begin{tabular}{@{}c@{}}
              $\left \langle n^2 \alpha(n) \right \rangle$ \cite{yunghsing} \\ 
              amortized $\left \langle n^2 \right \rangle$\end{tabular}
          & \begin{tabular}{@{}c@{}}
              impossible \\ 
              amortized $\left \langle n^2 /p + n \log n \right \rangle$\end{tabular} \\
    \bottomrule
  \end{tabular}
\end{table}

  \caption{   Our study shows in the bold front. We use the fixed size
$s=16$ on Cartesian tree. The small amortized constant will not
encounter serious load imbalance problem.   }

  \label{tlb:cmp-complexity}
\end{table*}

\subsection{Build Look-up Table}

We have two main subjects: cartesian tree encoding and cache
performance.   Fischer introduced the first encoding method, and the
Masud presents the new encoding method and processing step to reduce
the number of instructions. However, most of them are foucs on offline
algorithm. It means that there are given $n$ elements, and then has
$m$ queries. During any queries, the $n$ elements will not be modified
by any operation.

We could not use sorting to improve cache miss because the number of
elements and queries are similar. However, we can still improve the
cache miss by design the special encoding method according to value
distribution.

In our application, we provide the online encoding algorithm. We list
all binary search tree by lexicographical order, and label them from
$0$ to Catlan $n$-th number. The lexicographical order for binary
search tree is defined by the left subtree high priority and then
right subtree. The figure \ref{fig:labelingBST} shows label of binary
search tree for the $n=1,2,3$ number of nodes.

\begin{figure}[!thb]
  \centering
  \includegraphics[width=\linewidth]{graphics/fig-bst-encoding.pdf}
  \caption{The label of each binary search tree}
  \label{fig:lablingBST}
\end{figure}

For the $s$ nodes binary search tree, we label node from $0$ to $s-1$
and the identify $\textit{tid}$.  We define $\mathit{LCA}(s,
\mathit{tid}, p, q)$ as the lowest common ancestor of the node $p$ and
$q$ on a binary search tree which has $s$ nodes and labeling
$\mathit{tid}$, such as $\mathit{LCA}(3, 2, 0, 2) = 1$.

\begin{figure*}[!thb]
  \begin{equation*}
  \begin{split}
    &\mathit{LCA}(n, \mathit{tid}, p, q) \\
      &= \left\{\begin{matrix*}[l]
        \mathit{LCA}(\mathit{lsz}, \mathit{lid}, p, q) &&, p \le q < \mathit{lsz}\\ 
        \mathit{LCA}(\mathit{rsz}, \mathit{rid}, p-\mathit{lsz}-1, q-\mathit{lsz}-1)+\mathit{lsz}+1 &&, 
            \mathit{lsz} \le p \le q < n \\ 
        \mathit{lsz} && , 0 \le p \le \mathit{lsz}, \mathit{lsz} \le q \le i\\ 
        -1 && ,\mathit{otherwise}
      \end{matrix*}\right.
  \end{split}
\end{equation*}
  \caption{The formula of lowest common ancestor}
  \label{fun:LCA}
\end{figure*}

\subsection{Dynamic Cartesian Tree}